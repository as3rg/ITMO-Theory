\documentclass[12pt]{article}
\usepackage{bbold}
\usepackage{amsfonts}
\usepackage{amsmath}
\usepackage{amssymb}
\usepackage{color}
\setlength{\columnseprule}{1pt}
\usepackage[utf8]{inputenc}
\usepackage[T2A]{fontenc}
\usepackage[english, russian]{babel}
\usepackage{graphicx}
\usepackage{hyperref}
\usepackage{mathdots}
\usepackage{xfrac}


\def\columnseprulecolor{\color{black}}

\graphicspath{ {./resources/} }


\usepackage{listings}
\usepackage{xcolor}
\definecolor{codegreen}{rgb}{0,0.6,0}
\definecolor{codegray}{rgb}{0.5,0.5,0.5}
\definecolor{codepurple}{rgb}{0.58,0,0.82}
\definecolor{backcolour}{rgb}{0.95,0.95,0.92}
\lstdefinestyle{mystyle}{
    backgroundcolor=\color{backcolour},   
    commentstyle=\color{codegreen},
    keywordstyle=\color{magenta},
    numberstyle=\tiny\color{codegray},
    stringstyle=\color{codepurple},
    basicstyle=\ttfamily\footnotesize,
    breakatwhitespace=false,         
    breaklines=true,                 
    captionpos=b,                    
    keepspaces=true,                 
    numbers=left,                    
    numbersep=5pt,                  
    showspaces=false,                
    showstringspaces=false,
    showtabs=false,                  
    tabsize=2
}

\lstset{extendedchars=\true}
\lstset{style=mystyle}

\newcommand\0{\mathbb{0}}
\newcommand{\eps}{\varepsilon}
\newcommand\overdot{\overset{\bullet}}
\DeclareMathOperator{\sign}{sign}
\DeclareMathOperator{\re}{Re}
\DeclareMathOperator{\im}{Im}
\DeclareMathOperator{\Arg}{Arg}
\DeclareMathOperator{\const}{const}
\DeclareMathOperator{\rg}{rg}
\DeclareMathOperator{\Span}{span}
\DeclareMathOperator{\alt}{alt}
\DeclareMathOperator{\Sim}{sim}
\DeclareMathOperator{\inv}{inv}
\DeclareMathOperator{\dist}{dist}
\newcommand\1{\mathbb{1}}
\newcommand\ul{\underline}
\renewcommand{\bf}{\textbf}
\renewcommand{\it}{\textit}
\newcommand\vect{\overrightarrow}
\newcommand{\nm}{\operatorname}
\DeclareMathOperator{\df}{d}
\DeclareMathOperator{\tr}{tr}
\newcommand{\bb}{\mathbb}
\newcommand{\lan}{\langle}
\newcommand{\ran}{\rangle}
\newcommand{\an}[2]{\lan #1, #2 \ran}
\newcommand{\fall}{\forall\,}
\newcommand{\ex}{\exists\,}
\newcommand{\lto}{\leftarrow}
\newcommand{\xlto}{\xleftarrow}
\newcommand{\rto}{\rightarrow}
\newcommand{\xrto}{\xrightarrow}
\newcommand{\uto}{\uparrow}
\newcommand{\dto}{\downarrow}
\newcommand{\lrto}{\leftrightarrow}
\newcommand{\llto}{\leftleftarrows}
\newcommand{\rrto}{\rightrightarrows}
\newcommand{\Lto}{\Leftarrow}
\newcommand{\Rto}{\Rightarrow}
\newcommand{\Uto}{\Uparrow}
\newcommand{\Dto}{\Downarrow}
\newcommand{\LRto}{\Leftrightarrow}
\newcommand{\Rset}{\bb{R}}
\newcommand{\Rex}{\overline{\bb{R}}}
\newcommand{\Cset}{\bb{C}}
\newcommand{\Nset}{\bb{N}}
\newcommand{\Qset}{\bb{Q}}
\newcommand{\Zset}{\bb{Z}}
\newcommand{\Bset}{\bb{B}}
\renewcommand{\ker}{\nm{Ker}}
\renewcommand{\span}{\nm{span}}
\newcommand{\Def}{\nm{def}}
\newcommand{\mc}{\mathcal}
\newcommand{\mcA}{\mc{A}}
\newcommand{\mcB}{\mc{B}}
\newcommand{\mcC}{\mc{C}}
\newcommand{\mcD}{\mc{D}}
\newcommand{\mcJ}{\mc{J}}
\newcommand{\mcT}{\mc{T}}
\newcommand{\us}{\underset}
\newcommand{\os}{\overset}
\newcommand{\ol}{\overline}
\newcommand{\ot}{\widetilde}
\newcommand{\vl}{\Biggr|}
\newcommand{\ub}[2]{\underbrace{#2}_{#1}}

\def\letus{%
    \mathord{\setbox0=\hbox{$\exists$}%
             \hbox{\kern 0.125\wd0%
                   \vbox to \ht0{%
                      \hrule width 0.75\wd0%
                      \vfill%
                      \hrule width 0.75\wd0}%
                   \vrule height \ht0%
                   \kern 0.125\wd0}%
           }%
}
\DeclareMathOperator*\dlim{\underline{lim}}
\DeclareMathOperator*\ulim{\overline{lim}}

\everymath{\displaystyle}

% Grath
\usepackage{tikz}
\usetikzlibrary{positioning}
\usetikzlibrary{decorations.pathmorphing}
\tikzset{snake/.style={decorate, decoration=snake}}
\tikzset{node/.style={circle, draw=black!60, fill=white!5, very thick, minimum size=7mm}}

\title{Математическая Логика. Теория}
\author{Александр Сергеев}
\date{}

\begin{document}
\maketitle
\section{Введение}
\textbf{Силлогизмы}\\
\textit{Modus Ponendo Ponens}: Если $A$ и $A \rto B$, то $B$\\
\textbf{Парадокс Рассела}\\
$X = \{x: x \not\in x\}$\\
$(X \in X)?$\\
\textbf{Определение}\\
\textit{Номинализм} -- учение о том, что существуют лишь единичные вещи, а общие понятия -- лишь имена\\
\textit{Реализм} -- учение о том, что общие понятия объективно существуют\\
Номинализм в вопросе решения парадокса Рассела: надо придумать понятие множества\\
Реализм в вопросе решения парадокса Рассела: необходимо понять, что такое множество на самом деле, докопаться до сути\\
\textit{Программа Гильберта} -- мы должны формализовать математику и избавиться от произвола, который может вносить сторонний исследователь\\
Формализация должна быть проверена, и ее непротиворечивость должна быть доказана\\
Программма Гильберта -- реализм: мы верим, что мир устроен некоторым образом, а значит существует некоторая идеальная математика, удовлетворяющая этим свойствам. Нам нужно лишь прислушаться к миру и найти ее
\section{Исчисление высказываний}
\textbf{Определение}\\
\textit{Высказывание} -- строка, сформулированная по следующим правилам\\
\textit{Предметный язык} -- язык, который мы изучаем (язык математической логики)\\
\textit{Метаязык} -- соглачения о записи. Из метаязыка можно получить предметный язык некоторыми неформализованными действиями\\
$A, B, \ldots$ -- Пропозиционная переменная\\
$\alpha, \beta, \ldots$ -- метапеременные (высказывания)\\
$\alpha \land \beta$ -- Конъюнкция\\
$\alpha \lor \beta$ -- Дизъюнкция\\
$\lnot \alpha$ -- Отрицание\\
$\alpha \rto \beta$ -- Импликация\\
$X, Y, Z$ -- метапеременные для пропозиционных переменных\\
Приоритет: отрицание, конъюнкция, дизъюнкция, импликация\\
Дизъюнкция и конъюнкция -- левоассоциативные,импликация -- правоассоциативная\\
Выражение на предметном языке -- пропозиционные переменные, 4 вида \textit{связкок} и полностью расставленные скобки. Все остальные формы записи -- метаязык\\
\textit{Схема} -- строка, строящаяся по правилам предметного языка, где вместо пропозиционных переменных могут стоять маленькие греческие буквы(метапеременные)\\
\textbf{Определение}\\
Оценка высказывания $f: P \rto V$, где $V=\{T, F\}, P$ -- множество пропозиционных переменных\\
$[[\alpha]] = T$ -- оценка высказывания (значение $\alpha$ -- истина)\\
$[[\alpha]]^{X_1:=v_1, \ldots, X_n:=v_n}$ -- оценка высказывания\\
\textbf{Определение}\\
Если $[[\alpha]] = T$ при любой оценке переменных, то она \textit{общезначима (тавтология)}: $\models \alpha$\\
Иначе \textit{опровержима}\\
Если $[[\alpha]] = T$ при любой оценке переменных, при которой $[[\gamma_1]] = \ldots = [[\gamma_n]] = T$, то $\alpha$ -- следствие этих высказываний: $\gamma_1,\ldots, \gamma_n \models \alpha$\\
Если $[[\alpha]] = T$ при некоторой оценке, то она \textit{выполнима}, иначе \textit{невыполнима}\\
\textbf{Аксиомы исчисления высказываний}
\begin{enumerate}
    \item $\alpha \rto \beta \rto \alpha$
    \item $(\alpha \rto \beta) \rto (\alpha \rto \beta \rto \gamma) \rto (\alpha \rto \gamma)$
    \item $\alpha \rto \beta \rto \alpha \land \beta$
    \item $\alpha \land \beta \rto \alpha$
    \item $\alpha \land \beta \rto \beta$
    \item $\alpha \rto \alpha \lor \beta$
    \item $\beta \rto \alpha \lor \beta$
    \item $(\alpha \rto \gamma) \rto (\beta \rto \gamma) \rto (\alpha \lor \beta \rto \gamma)$
    \item $(\alpha \rto \beta) \rto (\alpha \rto \lnot \beta) \rto \lnot \alpha$
    \item $\lnot\lnot \alpha \rto \alpha$
\end{enumerate}
\textbf{Определение}\\
\textit{Доказательством} назовем последовательность высказываний $\delta_1, \ldots, \delta_n$, где каждое высказывание $\delta_i$ либо:
\begin{itemize}
    \item является аксиомой (существует замена метапеременных для какой-либо схемы аксоим, позволяющая получить схему $\delta_i$)
    \item получается из $\delta_1, \ldots, \delta_{i-1}$ по правилу Modus Ponens: существуют такие $j, k < i: \delta_k \equiv \delta_j \rto \delta_i$
\end{itemize}
Формула \textit{выводима/доказуема}, если существует ее доказательство\\
\textbf{Пример}\\
$A \rto (A \rto A)$\\
$(A \rto (A\rto A)) \rto (A \rto ((A\rto A) \rto A)) \rto (A \rto A)$\\
$(A \rto ((A\rto A)\rto A)) \rto (A \rto A)$\\
$A \rto ((A\rto A)\rto A)$\\
$A \rto A$\\
\textbf{Определение}\\
\textit{Вывод формулы $\alpha$ из гипотез $\gamma_1,\ldots, \gamma_k$} -- такая последовательность $\sigma_1, \ldots, \sigma_n$, что $\sigma_i$ является (одним из следующих):
\begin{itemize}
    \item аксиомой
    \item одной из гипотез $\gamma_t$
    \item получена по правилу Modus Ponens из предыдущих
\end{itemize}
Формула \textit{выводима из гипотез}, если существует ее вывод\\
Обозначение: $\gamma_1, \ldots, \gamma_k \vdash \alpha$\\
\textbf{Определение (корректность теории)}\\
Теория \textit{корректна}, если любое доказуемое в ней утверждение общезначимо\\
То есть, $\vdash \alpha$ влечет $\models \alpha$\\
\textbf{Определение (полнота теории)}\\
Теория семантически полна, если любое общезначимое в ней утверждение доказуемо. То есть, $\models \alpha$ влечет $\vdash \alpha$\\\\
\textbf{Теорема (корректность вычисления высказываний)}\\
\textbf{Доказательство}\\
Докажем, что любая строка доказательства является общезначимой\\
Докажем индукцией по количеству строк\\
База: $n=1$ -- аксиома. В ней нет правила Modus Ponens. Она общезначима\\ 
Переход: Пусть для любого доказательства длины $n$ формула $\delta_n$ общезначима. Рассмотрим $\delta_{n+1}$
\begin{enumerate}
    \item Аксиома. Убедимся, что аксиома общезначима
    \item Modus Ponens j, k -- убедимся, что если $\models \delta_j$ и $\models \delta_k, \delta_k = \delta_j \rto \delta_{n+1}$, то $\models \delta_{n+1}$
\end{enumerate}
Аксиому проверим через таблицу истинности\\
Докажем правило Modus Ponens\\
По индукционному предположению $\models \delta_j, \models \delta_k$\\
Зафиксируем произвольную оценку\\
Из общезначимости $[[\delta_j]] = T, [[\delta_k]] = T$\\
Тогда из таблицы истинности $[[\delta_j]] = [[\delta_k]] = T$ только при $[[\delta_{n+1}]] = T$\\
Отсюда $\models \delta_{n+1}$\\
\textbf{Определение}\\
\textit{Контекст} -- совокупность всех гипотез. Обозначается большой греческой буквой\\
Пример записи:\\
$\Gamma = \{ \gamma_1,\ldots, \gamma_n\}$\\
$\Delta = \{ \delta_1,\ldots, \delta_m \}$\\
$\Gamma,\Delta \vdash \alpha$\\
\textbf{Теорема о дедукции}\\
$\Gamma, \alpha \vdash \beta \LRto \Gamma\vdash \alpha \rto \beta$\\
\textbf{Доказательство $\Lto$}\\
Пусть $\Gamma\vdash \alpha \rto \beta$\\
Т.е. существует вывод $\delta_1,\ldots, \delta_{n-1}, \ub{\delta_n}{\alpha \rto \beta}$\\
Дополним вывод: добавим туда $\alpha$\\
По правилу Modus Ponens добавим туда $\beta$\\
Отсюда $\Gamma, \alpha \vdash \beta$\\
\textbf{Определение}\\
\textit{Конечная последовательность} -- функция $\delta: \{1,\ldots, n\} \rto \mc F$\\
\textit{Конечная последовательность, индексированная дробными числами} -- функция $\zeta: I \rto \mc F, I\subseteq \Qset^+, |I|\in \Nset$\\
\textbf{Доказательство $\Rto$}\\
Пусть $\Gamma, \alpha \vdash \beta$\\
Пусть дан некоторый вывод: $\delta_1, \ldots, \delta_{n-1}, \ub{\delta_n}{\beta}$\\
Тогда рассмотрим последовательность: $\alpha \rto \delta_1, \ldots, \alpha \rto \delta_{n-1}, \alpha \rto \beta$\\
Заметим, что выводом эта формула не является, т.к. в ней нет аксиом\\
Докажем по индукции по длине вывода\\
Если $\delta_1, \ldots, \delta_n$ -- вывод $\Gamma,\alpha\vdash \beta$, то найдется $\zeta_k$ для $\Gamma\vdash \alpha\rto \beta$, причем $\zeta_1 =\alpha\rto \delta_1, \ldots, \zeta_n=\alpha\rto \delta_n$
\begin{enumerate}
    \item $n = 1$ -- ч.с. перехода без Modus Ponens
    \item Пусть $\delta_1, \ldots \delta_{n+1}$ -- исходный вывод\\
    По индукционному предположению по $\delta_1, \ldots, \delta_n$ построен вывод $\zeta_k$ утверждения $\Gamma\vdash \alpha\rto \delta_n$\\
    Достроим его для $\delta_{n+1}$
    \begin{itemize}
        \item $\delta_{n+1}$ -- аксиома или $\delta_{n+1} \in \Gamma$:\\
        $\zeta_{n+\sfrac13} = \delta_{n+1} \rto \alpha \rto \delta_{n+1}$\\
        $\zeta_{n+\sfrac23} = \delta_{n+1}$\\
        $\zeta_{n + 1} = \alpha \rto \delta_{n+1}$ 
        \item $\delta_{n+1} = \alpha$:\\
        $\zeta_{n+\sfrac15} = a \rto a \rto a$\\
        $\zeta_{n+\sfrac25} = (a \rto a \rto a) \rto (a \rto (a \rto a) \rto a) \rto (a \rto a)$\\
        $\zeta_{n+\sfrac35} = (a \rto (a \rto a) \rto a) \rto (a \rto a)$\\
        $\zeta_{n+\sfrac45} = a \rto (a \rto a) \rto a$\\
        $\zeta_{n+1} = a \rto a$
        \item $\delta_{n+1}$ -- Modus Ponens из $\delta_j$ и $\delta_k = \delta_j \rto \delta_{n+1}$:\\
        $\zeta_{n+\sfrac15} = \alpha\rto \delta_j$\\
        $\zeta_{n+\sfrac25} = \alpha \rto \delta_j \rto \delta_{n+1}$\\
        $\zeta_{n+\sfrac35} = (\alpha \rto \delta_j) \rto (\alpha \rto \delta_j \rto \delta_{n+1}) \rto (\alpha \rto \delta_{n+1})$\\
        $\zeta_{n+\sfrac45} = (\alpha \rto \delta_j \rto \delta_{n+1}) \rto (\alpha \rto \delta_{n+1})$\\
        $\zeta_{n+1} = \alpha \rto \delta_{n+1}$
    \end{itemize}
\end{enumerate}
\textbf{Лемма (правило контрапозиции)}\\
Каково бы ни были формулы $\alpha, \beta$, справедливо, что $\vdash (\alpha \rto \beta) \rto (\lnot \beta \rto \lnot \alpha)$\\
\textbf{Лемма (правило исключенного третьего)}\\
Какова бы ни была формула $\alpha$, справедливо, что $\vdash \alpha \lor \lnot \alpha$\\
\textbf{Лемма (правило исключенного допущения)}\\
Пусть справедливо $\Gamma, \rho \vdash \alpha$ и $\Gamma, \lnot \rho \vdash \alpha$\\
Тогда $\Gamma \vdash \alpha$\\
\textbf{Теорема}\\
Если $\models \alpha$, то $\vdash \alpha$\\
\textbf{Определение}\\
Зададим некоторую оценку, что $[[\alpha]] = x$\\
Тогда \textit{условным отрицанием} формулы $\alpha$ называется формула $(|\alpha|) = \left\{\begin{array}{cc}
    \alpha, & x = T\\
    \lnot \alpha, & x = F
\end{array}\right.$\\
Если $\Gamma=\gamma_1, \ldots, \gamma_n$, то $(|\Gamma|) = (|\gamma_1|),\ldots, (|\gamma_n|)$\\
Пример: $(|A|), (|B|) \vdash (|A\rto B|)$ позволяет записать таблицу истинности в одну строку, перебрав ее для всех оценок\\
\textbf{Доказательство теоремы}\\
Для каждой возможной связки $\star$ докажем формулы $(|\phi|), (|\psi|) \vdash (|\phi \star \psi)$\\
Теперь построим таблицу истинности для $\alpha$ и докажем в ней каждую строку:\\
$(|\Xi|) \vdash (|\alpha|), \Xi$ -- контекст(все переменные в таблице)\\
Если формула общезначима, то в ней все строки будут иметь вид $(|\Xi|) \vdash \alpha$. От гипотез можно избавиться индукционно по теореме об исплючении допущения и получить требуемое $\vdash \alpha$\\
\textbf{Лемма (Условное отрицание формул)}\\
Пусть пропозиционные переменные $\Xi = X_1, \ldots, X_n$ -- все переменные, которые используются в $\alpha$\\
Пусть задана некоторая оценка переменных\\
Тогда $(|\Xi|) \vdash (|\alpha|)$\\
\textbf{Доказательство}\\
Докажем по индукции по длине формулы $\alpha$
\begin{itemize}
    \item База: формула атомарная, т.е. $\alpha = X_i$\\
    Тогда при любом $\Xi$ выполнено $(|\Xi|)^{X_i=T}\vdash X_i$ и $(|\Xi|)^{X_i=F}\vdash \lnot X_i$
    \item Переход:\\
    $\alpha = \phi \star \psi, (|\Xi|) \vdash (|\phi|)$ и $(|\Xi|) \vdash (|\psi|)$\\
    Тогда построим вывод\\
    Сначала запишем доказательство $(|\phi|)$\\
    Потом припишем доказательство $(|\psi|)$\\
    Потом припишем доказательство леммы о связках
\end{itemize}
\section{Интуиционистская логика}
Примеры:\\
\textbf{Теорема Брауэра о неподвижной точке}\\
Любое непрерывное отображение $f$ шара $\Rset^n$ на себя имеет неподвижную точку\\
\textbf{Замечание}\\
Заметим, что теорема (и доказательство) не говорит ничего о том, как эту точку найти\\
\textbf{Теорема}\\
$\ex a,b$ -- иррациональные $: a^b$ -- рациональное\\
\textbf{Доказательство}\\
Пусть $a=b=\sqrt{2}$
\begin{itemize}
    \item $\sqrt{2}^{\sqrt{2}}$ -- рациональное
    \item $\sqrt{2}^{\sqrt{2}}$ -- иррациональное\\
    Тогда $(\sqrt{2}^{\sqrt{2}})^{\sqrt{2}} = \sqrt{2}^2 = 2$ -- рациональное
\end{itemize}
\textbf{Замечание}\\
% Заметим, что мы не знаем, является ли $\sqrt{2}^\sqrt{2}$ рациональным или нет\\
Т.о. мы доказали теорему, не предоставив пример. Наше знание о рациональных и иррациональных числах от этого не увеличилось\\
\textbf{Определение}\\
\textit{Доказательство чистого существования} -- доказательство существования объекта без приведения реального примера/рецепта создания этого объекта\\
(Неконструктивное доказательство существования объекта)\\
\textbf{Замечание}\\
Парадокс брадобрея -- результат работы с чистым существованием. Мы предполагаем существование абстрактного объекта, не приводя рецепта для его создания\\\\
Может ли быть, что, работая с чистым существованием, мы сможем получить парадоксальные объекты и в других областях математики?\\
Давайте запретим доказательства чистого существования\\
\textbf{Интуиционизм}
\begin{itemize}
    \item Математика не формальна\\
    (не надо ограничивать математику формальностями)
    \item Математика независима от окружающего мира
    \item Математика не зависит от логики -- это логика зависит от математики\\
    (если мы сможем придумать более удобную логику для математики, мы можем ее использовать)
\end{itemize}
\textbf{BHK-интерпретация логических связок}\\
(сокращение от: Брауэр, Гейтинг, Колмогоров)\\
Пусть $\alpha, \beta$ -- некоторая конструкция (что угодно -- физическая конструкция, логическое построение, программа, доказательство)\\
\begin{itemize}
    \item $\alpha \& \beta$ построено, если построены $\alpha$ и $\beta$
    \item $\alpha \lor \beta$ построено, если построено $\alpha$ или $\beta$, \underline{и мы знаем, что именно}
    \item $\alpha \rto \beta$ построено, если есть способ перестроения $\alpha$ в $\beta$
    \item $\perp$ -- конструкция, не имеющая построения
    \item $\lnot \alpha$ -- построено, если построено $\alpha \rto \perp$
\end{itemize}
\textbf{Дизъюнкция}\\
$\alpha \lor \lnot \alpha$ не может быть построено в общем виде, потому что мы не знаем, что именно было построено\\
\textbf{Пример}\\
Пусть $\alpha$ -- это задача $P = NP$\\
Тогда $\alpha \lor \lnot \alpha$ не может быть построено, т.к. мы не знаем, $P=NP$ или $P\neq NP$\\
\textbf{Импликация}\\
Пусть:
$A$ -- сегодня в СПб идет дождь\\
$B$ -- сегодня в СПб светит солнце\\
$C$ -- сегодня я получил <<отлично>> по матлогу\\
Рассмотрим $(A \rto B) \lor (B \rto C) \lor (C \rto A)$\\
Заметим это выражение не может быть построено, в отличие от классической логики\\
Отсюда: импликацию можно понимать как <<формальную>> и <<материальную>>\\
\textbf{Формализация}\\
Заметим, что формализация интуиционистской логики возможна, но интуитивное понимание -- основное\\
Аксиоматика интуиционистского исчисления высказываний в гильбертовском стиле: аксиоматика КИВ, в которой 10 схема аксиом $\lnot\lnot \alpha\rto \alpha$ заменена на $\alpha \rto \lnot \alpha \rto \beta$
\section{Топология}
\textbf{Обозначение}\\
$\mc P(X)$ -- множество всех подмножеств $X$\\
\textbf{Определение}\\
\textit{Топологическое пространство} -- упорядоченная пара $\an X\Omega, X$ -- множество (носитель), $\Omega \subseteq \mc P(X)$ -- \textit{топология}, причем
\begin{enumerate}
    \item $\varnothing, X \in \Omega$
    \item если $A_1, \ldots, A_n \in \Omega$, то $A_1 \cap A_2 \cap \ldots \cap A_n \in \Omega$ (конечное пересечение)
    \item если $\{A_\alpha\}$ -- семейство множеств из $\Omega$, то $\bigcup_\alpha A_\alpha \in \Omega$ (произвольное объединение)
\end{enumerate}
Элементы $\Omega$ -- \textit{открытые множества}\\
\textbf{Определение}\\
$\mcB$ -- \textit{база} топологического пространства $\an X\Omega (\mcB \subseteq \Omega)$, если всевозможные объединения элементов (в т.ч. пустые) из $\mcB$ дают $\Omega$\\
\textbf{Определение}\\
Дискретная топология -- $\an X{\mc P(X)}$\\
Топология стрелки -- $\an \Rset{\{(x, +\infty):x\in \Rset\} \cup \{\varnothing, \Rset\}}$\\
\textbf{Примеры}\\
$\{(x,y): x,y \in \Rset\}$ -- база евклидовой топологии на $\Rset$\\
$\{\{x\}: x \in X\}$ -- база дискретной топологии\\
\textbf{Определение}\\
Метрикой на $X$ назовем множество, на котором определена функция расстрояния $d: X^2 \rto \Rset^+:$
\begin{enumerate}
    \item $d(x,y) = 0 \LRto x = y$
    \item $d(x,y) = d(y,x)$
    \item $d(x,z) \leq d(x,y) + d(y,z)$
\end{enumerate}
\textbf{Определение}\\
Открытым $\eps$-шаром с центром в $x\in X$ назовем $\{t \in X: d(x,y) < \eps\}$\\
\textbf{Определение}\\
Если $X$ -- некоторое множество и $d$ -- метрика на $X$, то будем говорить, что топологическое пространство, задаваемое базой $\mcB = \{O_\eps(x): \eps \in \Rset^+, x \in X\}$ порождено метрикой $d$\\
\textbf{Определение}\\
Функция $f: X \rto Y$ непрерывна, если прообраз любого открытого множества открыт\\
\textbf{Определение}\\
Будем говорить, что множество компактно, если из любого его открытого покрытия можно выбрать конечное подпокрытие\\
\textbf{Определение}\\
Пространство $\an{X_1}{\Omega_1}$ -- подпространство пространства $\an X\Omega$, если $X_1\subseteq X$ и $\Omega_1 = \{A \cap X_1, A \in \Omega\}$\\
\textbf{Определение}\\
Пространство $\an X\Omega$ связно, если нет $A, B\in \Omega$, что $A \cup B = X, A \cap B = \0$ и $A, B \neq \varnothing$\\
\textbf{Определение (топология на деревьях)}\\
Пусть некоторый лес задан конечным множеством вершин $V$ и отношением $\prec: a \prec b \LRto a$ -- предок $b$\\
Тогда подмножество вершин $X \subseteq V$ назовем открытым, если из $a \in X, a\preceq b$ следует, что $b \in X$ (множество вершин и всех их потомков)\\
\textbf{Теорема}\\
Лес связен (как граф) тогда и только тогда, когда соответствующее ему топологическое пространство связно\\
\textbf{Доказательство $\Rto$}\\
Пусть лес связен, но топологически не связен. Тогда найдутся непустые $A, B$, что $A \cup B = V, A \cap B = \varnothing$\\
Пусть $v \in V$ -- корень дерева $V$ и $v \in A$\\
Тогда $A = \{x: v \preceq x\} = V, B = \varnothing$\\
\textbf{Доказательство $\Lto$}\\
Пусть лес топологически связен, но есть несколько корней $v_1, \ldots, v_k$\\
Возьмем $A_i = \{x: v_i \preceq x\}$. Тогда все $A_i$ открыты, непусты и дизъюнктны\\
$V = \bigcup_i A_i$\\
\textbf{Определение}\\
\textit{Линейная связность} -- любые точки соединены путем\\
\textbf{Определение}\\
Множество нижних границ ($\nm{lwb}_\Omega$) -- ...\\
Множество верхних границ ($\nm{uwb}_\Omega$) -- ...\\
Минимальный элемент ($m \in X$) -- Нет элементов, что $x \prec m$\\
Максимальный элемент ($m \in X$) -- ...\\
Наименьший элемент ($m \in X$) -- При всех $y \in X$ выполнено $m \preceq y$\\
Наибольший элемент ($m \in X$) -- ...\\
Инфинут -- наибольшая нижняя граница\\
Супремум -- наименьшая верхняя граница\\
\textbf{Определение}\\
Рассмотрим $\an X\Omega$ и возьмем $\subseteq$ как отношение частичного порядка на $\mc P(X)$\\
Тогда $A^\circ := \inf_\Omega(\{A\})$ -- внутренность множества\\
\textbf{Теорема}\\
$A^\circ$ определена для любого $A$\\
\textbf{Доказательство}\\
Пусть $V = \nm{lwb}_\Omega\{A\} = \{Q \in \Omega: Q \subseteq A\}$\\
Тогда $\inf_\Omega\{A\} = \bigcup_{v \in V} v$\\
Напомним, что $\inf_{\mc U} T = \text{наиб}(\nm{lwb}_{\mc U}T)$
\begin{enumerate}
    \item Покажем принадлежность: $\bigcup_{v \in V} v \subseteq A, \in \Omega$
    \item Покажем, что все из $V$ меньше или равны: пусть $X \in V$. Тогда $X \subseteq \bigcup_{v \in V} v$
\end{enumerate}
\textbf{Определение}\\
\textit{Решеткой} называется упорядоченная пара $\an X{(\preceq)}$, где $X$ -- некоторое множество, $(\preceq)$ -- частичный порядок на $X$, такой, что для любых $a, b \in X$ определены $a+b = \sup \{a, b\}, a\cdot b = \inf\{a,b\}$\\
То есть $a+b$ -- наименьший элемент $c$, что $a,b\preceq c$\\
\textbf{Определение}\\
Псевдодополнение $a \rto b$ -- наибольший из $\{x: a\cdot x \preceq b\}$\\
\textbf{Определение}\\
Дистрибутивной решеткой называется такая, что $\fall a,b,c\ a\cdot(b+c) = a\cdot b + a \cdot c$\\
Импликативная решетка -- такая, что для любых элементов есть псевдодополнение\\
\textbf{Лемма}\\
Любая импликативная решетка -- дистрибутивна\\
\textbf{Определение}\\
0 -- наименьший элемент решетки, 1 -- наибольший элемент решетки\\
\textbf{Лемма}\\
В любой импликативной решетке $\an X{(\preceq)}$ есть 1\\
\textbf{Доказательство}\\
Рассмотрим $a \rto a$, тогда $a \rto a = \text{наиб}\{c: ac \preceq a\} = \text{наиб }X = 1$\\
\textbf{Определение}\\
Импликативная решетка с 0 -- псевдобулева алгебра (алгебра Гейтинга)\\
В такой решетке определено $\sim a := a \rto 0$\\
\textbf{Определение}\\
Булева алгебра -- псевдобулева алгебра, в которой $a + \sim a \equiv 1$\\
\textbf{Замечание}\\
Известная нам булева <<алгебра>> -- булева алгебра\\
\textbf{Лемма}\\
$\an {\mc P(X)}{(\subseteq)}$ -- булева алгебра\\
\textbf{Лемма}\\
$\an \Omega{(\subseteq)}$ -- псевдобулева алгебра\\
\textbf{Определение}\\
Пусть некоторое вычисление высказываний оценивается значениями из некоторой решетки\\
Назовем оценку согласованной с исчислением, если\\
$[[\alpha \& \beta]] = [[\alpha]]\cdot[[\beta]]$\\
$[[\alpha \lor \beta]] = [[\alpha]] + [[\beta]]$\\
$[[\alpha \rto \beta]] = [[\alpha]] \rto [[\beta]]$\\
$[[\lnot \alpha]] = \sim[[\alpha]]$\\
$[[A\&\lnot A]] = 0$\\
$[[A \rto A]] = 1$\\
\textbf{Теорема}\\
Любая псевдобулева алгебра, являющаяся согласованной оценкой интуиционистского исчисления высказываний, является его корректной моделью: если $\vdash \alpha$, то $[[\alpha]] = 1$\\
\textbf{Теорема}\\
Любая булева алгебра, являющаяся согласованной оценкой классического исчисления высказываний, является его корректной моделью:  если $\vdash \alpha$, то $[[\alpha]] = 1$
\section{Интуиционистское исчисление высказываний (+ алгебра Гейтинга)}
\textbf{Определение}\\
Язык \textit{разрешим}, если существует программа, позволяющая определить, относится ли слово к языку или нет\\
Язык исчислений разрешим, если для каждой формулы мы можем проверить, истинна она или ложна\\\\
Язык И.И.В. корректен (задание в д.з.) и непротиворечив (т.к. является упрощением К.И.В., которая непротиворечива)\\
\textbf{Определение}\\
Определим предпорядок на высказываниях $\alpha \preceq \beta := \alpha \vdash \beta$ -- в интуиционистском исчислении высказываний\\
Также $\alpha \approx \beta$, если $\alpha \preceq \beta$ и $\beta \preceq \alpha$\\
\textbf{Определение}\\
Пусть $L$ -- множество всех высказываний\\
Тогда алгебра Линденбаума $\mc L = L / \approx$\\
\textbf{Теорема}\\
$\mc L$ -- псевдобулева алгебра\\
\textbf{Схема доказательства}\\
$[\alpha]_{\mc L}$ -- класс эквивалентности в алгебре Линденбаума\\
Надо показать, что $\preceq$ -- отношение порядка на $\mc L, [\alpha \lor \beta]_{\mc L} = [\alpha]_{\mc L} + [\beta]_{\mc L}, [\alpha \& \beta]_{\mc L} = [\alpha]_{\mc L} \cdot [\beta]_{\mc L}$, что импликация есть псевдодополнение, $[A \& \lnot A]_{\mc L} = 0, [\alpha]_{\mc L} \rto 0 = [\lnot \alpha]_{\mc L}$\\
\textbf{Теорема}\\
Пусть $[[\alpha]] = [\alpha]_{\mc L}$. Такая оценка высказываний интуиционистского исчисления высказываний алгеброй Линденбаума является согласованной\\
\textbf{Теорема}\\
Интуиционистское исчисление высказываний полно в псевдобулевых алгебрах: если $\models \alpha$ во всех псевдобулевых алгебрах, то $\vdash \alpha$\\
\textbf{Доказательство}\\
Возьмем в качестве модели исчисления алгебру Линденбаума: $[[\alpha]] = [\alpha]_{\mc L}$\\
Пусть $\models \alpha$\\
Тогда $[[\alpha]] = 1$ во всех псевдобулевых алгебрах, в том числе и $[[\alpha]] = 1_{\mc L}$\\
То есть $[\alpha]_{\mc L} = [A\rto A]_{\mc L}$\\
То есть $A\rto A \approx \alpha$\\
Значит в частности $A \rto A \vdash \alpha$\\
Значит $\vdash \alpha$\\
\textbf{Определение}\\
Модель Крипке(Шкала) $\lan W, \preceq, \Vdash\ran$:\\
Представим, что существует множество альтернативных миров, в которых верны или не верны различные утверждения\\
$\mc W$ -- множество миров\\
$\preceq$ -- нестрогий частичный порядок на $\mc W$\\
$(\Vdash)\subseteq \mc W \times P$ -- отношения \textit{вынуждения} между мирами и переменными, которые выполнены в этих мирах, причем если $W_i \preceq W_j$ и $W_i \Vdash X$, то $W_j \Vdash X$\\
Доопределим вынужденность:\\
$W \Vdash \alpha \& \beta$, если $W \Vdash \alpha$ и $W \Vdash \beta$\\
$W \Vdash \alpha \lor \beta$, если $W \Vdash \alpha$ или $W \Vdash \beta$\\
$W \Vdash \alpha \rto \beta$, если всегда при $W \preceq W_1$ и $W_1 \Vdash \alpha$ выполнено $W_1 \Vdash \beta$\\
$W \Vdash \lnot \alpha$, если всегда при $W \preceq W_1$ выполнено $W_1 \not\Vdash \alpha$\\
Будем говорить, что $\Vdash \alpha$, если $W \Vdash \alpha$ при всех $W$\\
Будем говорить, что $\models \alpha$, если $\Vdash \alpha$ во всех моделях Крипке\\
\textbf{Пример}\\
$\not\models A \lor \lnot A$\\
\textbf{Доказательство}\\
$W_1 \preceq W_2, W_3$\\
$W_2 \Vdash A$\\
$W_3 \Vdash \lnot A$\\
Тогда $W_1 \not\Vdash A, W_1 \not\Vdash \lnot A$\\
Отсюда $W_1 \not\Vdash A \lor \lnot A$\\
\textbf{Лемма}\\
Если $W_1 \Vdash \alpha$ и $W_1 \preceq W_2$, то $W_2 \Vdash \alpha$\\
\textbf{Теорема}\\
Пусть $\lan W, \preceq, \Vdash\ran$ -- некоторая модель Крипке\\
Тогда она корректная модель интуиционистского исчисления высказываний\\
\textbf{Доказательство}\\
Докажем для всех древовидных $\preceq$\\
Обобщение на произвольный порядок несложное(так утверждается)\\
Заметим, что $V(\alpha) := \{w \in \mc W: W \Vdash \alpha\}$ -- открыто в топологии для дерева\\
Значит, положив $V = \{S: S \subset \mc W \& S \text{ -- открыто}\}$ и $[[\alpha]] = V(\alpha)$, получим алгебру Гейтинга\\
\textbf{Определение}\\
Пусть задано множество значений $V$, $T \in V$ -- истина, функция $f_P:P \rto V$, функции $f_{\&}, f_\lor, f_\rto: V\times V \rto V$, функция $f_\lnot: V \rto V$\\
Тогда оценка $[[X]] = f_P(X), [[\alpha\star \beta]] = f_\star([[\alpha]], [[\beta]]), [[\lnot \alpha]] = f_\lnot([[\alpha]])$ -- табличная\\
Если $\vdash \alpha$ влечет $[[\alpha]] = T$ при всех оценках пропозиционных переменных $f_P$, то $\mc M:= \lan V, T, f_\&, f_\lor, f_\rto, f_\lnot \ran$ -- \textit{табличная модель}\\
\textbf{Определение}\\
Табличная модель конечна, если $V$ -- конечно\\
\textbf{Теорема}\\
Не существует полной конечной табличной модели для интуиционистского исчисления высказываний\\
\textbf{Доказательство}\\
Пусть существует полная конечная табличная модель $\mc M, V = \{ v_1, \ldots, v_n \}$\\
То есть, если $\models_{\mc M} \alpha$, то $\vdash \alpha$\\
Рассмотрим $\alpha_n = \bigvee_{1\leq p < q \leq n+1} A_p \rto A_q$\\
Рассмотрим оценку $f_P: \{A_1,\ldots, A_{n+1}\} \rto \{v_1,\ldots, v_n\}$\\
По принципу Дирихле существуют $p\neq q: [[A_p]] = [[A_q]]$\\
Значит $[[A_p \rto A_q]] = f_{\rto}([[A_p]],[[A_q]]) = f_\rto(v,v)$\\
С другой стороны, $\vdash X \rto X$ (из полноты) -- поэтому $f_\rto([[X]], [[X]]) = T$\\
Значит $[[A_p\rto A_q]] = f_\rto(v,v) = f_\rto([[X]], [[X]]) = T$\\
Аналогично $\vdash \sigma\lor (X\rto X) \lor \tau$\\
Отсюда $[[\alpha_n]] = [[\sigma\lor (X\rto X)\lor \tau]] = T$. Т.е. $\models \alpha_n$ в табличной модели\\
Однако, в такой модели $\not \Vdash \alpha_n$\\
Пусть $W_R \preceq W_i, i = 1\ldots n$\\
$W_i \Vdash A_i$\\
Если $q > 1$, то $W_1 \not\Vdash A_q$ и $W_1 \not\Vdash A_1 \rto A_q$\\
Если $q > 2$, то $W_2 \not\Vdash A_q$ и $W_2 \not\Vdash A_2 \rto A_q$\\
\ldots\\
Если $q < p$, то $W_p \not\Vdash A_q$ и $W_p \not\Vdash A_p \rto A_q$\\
Т.е. $W_R \not\Vdash A_p \rto A_q$\\
Отсюла $W_R \not\Vdash \bigvee_{p<q} A_p \rto A_q$\\
Т.е. $W_R \not\Vdash \alpha_n$, потому $\not\Vdash \alpha_n$, а значит $\not\vdash \alpha_n$\\
Отсюда если мы проверили формулу в некоторой конечной модели, то из этого не следует, что она инстинная вообще\\
\textbf{Определение}\\
Исчисление дизъюнктно, если при любых $\alpha,\beta$ из $\vdash \alpha\lor \beta$ следует $\vdash \alpha$ или $\vdash \beta$\\
\textbf{Определение}\\
Решетка геделева, если $a+b = 1$ влечет $a=1$ или $b=1$\\
\textbf{Теорема}\\
Интуиционистское исчисление высказываний дизъюнктно\\
\textbf{Определение}\\
По определению\\
\textbf{Определение}\\
Для алгебры Гейтинга $\mcA = \an{A}{\preceq}$ определим операцию геделевизации\\
$\Gamma(\mcA) = \an{A \cup \{\omega\}}{\preceq_{\Gamma(\mc A)}}$\\
Где $\preceq_{\Gamma(\mc A)}$ -- минимальное отношение порядка, удовлетворяющее условиям:
\begin{itemize}
    \item $a\preceq_{\Gamma(\mc A)} b$, если $a\preceq b$ и $a,b \not\in \{\omega, 1\}$
    \item $a\preceq_{\Gamma(\mc A)} \omega$, если $a\neq 1$
    \item $\omega\preceq_{\Gamma(\mc A)} 1$
\end{itemize}
\textbf{Теорема}\\
$\Gamma(\mcA)$ -- геделева алгебра\\
\textbf{Доказательство}\\
Проверка определения алгебры Гейтинга и наблюдение: если $a \preceq \omega$ и $b \preceq \omega$, то $a+b\preceq \omega$\\
(т.к. $\omega \in $ множество верхних граней $\{a, b\}$)\\
\textbf{Теорема}\\
Рассмотрим оценку $[[\alpha]]_{\Gamma(\mc L)} = [[\alpha]]_{\mc A}$\\
Тогда она является согласованной с ИИВ\\
Индукция по структуре формулы и перебор операций\\
Рассмотрим $\&$\\
Неформально: почти везде $[[\alpha]]_{\Gamma(\mc L)}\cdot [[\beta]]_{\Gamma(\mc L)} = [[\alpha]]_{\mc L}\cdot [[\beta]]_{\mc L}$, посколькую $[[\sigma]]_{\Gamma(\mc L)} \neq \omega$\\
Но нет ли случаев, когда $\omega = \text{наиб}\{x: x \preceq [[\alpha]]_{\Gamma(\mc L)}\&x \preceq [[\beta]]_{\Gamma(\mc L)}\}$\\
Чтобы убедиться, что всегда $[[\alpha\&\beta]]_{\Gamma(\mc L)} = [[\alpha]]_{\Gamma(\mc L)}\cdot[[\beta]]_{\Gamma(\mc L)}$, надо показать:
\begin{itemize}
    \item $[\alpha\&\beta]$ -- из множества нижних граней: $\alpha\&\beta\vdash \alpha$ и $\alpha\&\beta \vdash \beta$
    \item $[\alpha\&\beta]$ -- наибольшная нижняя грань: $x \preceq [\alpha]$ и $x\preceq [\beta]$ влечет $x\preceq [\alpha\&\beta]$\\
    Разбор случаев: $(x \in \mc L, x = \omega)$. $\omega\preceq[\alpha]$ и $\omega \preceq [\beta]$ влечет $[\alpha] = [\beta] = 1$, отсюда $[\alpha\&\beta] = [\alpha]\cdot[\beta] = 1$
\end{itemize}
\textbf{Определение}\\
$\mcA, \mcB$ -- алгебры Гейтинга\\
Тогда $g: \mcA\rto\mcB$ -- гомоморфизм, если $\gamma(a\star b) = g(a)\star g(b), g(0_\mcA) = 0_\mcB, g(1_\mcA)=1_\mcB$\\
\textbf{Определение}\\
Будем говорить, что оценка $[[\cdot]]_\mcA$ согласована с $[[\cdot]]_\mcB$ и гомоморфизмом $g$, если $g(\mcA) = \mcB, g([[\alpha]]_\mcA) = [[\alpha]]_\mcB$\\
\textbf{Определение($\mc G: \Gamma(\mc L)\rto \mc L$)}\\
$\mc G(a) = \left\{\begin{array}{cc}
    a, & a \neq \omega\\
    1, & a = \omega
\end{array}\right.$\\
\textbf{Лемма}\\
$\mc G$ -- гомоморфизм $\Gamma(\mc L)$ и $\mc L$, причем $[[\cdot]]_{\Gamma(\mc L)}$ согласована с $\mc G$ и $[[\cdot]]_{\mc L}$\\
\textbf{Теорема}\\
Если $\vdash \alpha \lor \beta$, то либо $\vdash \alpha$, либо $\vdash \beta$\\ 

\end{document}
