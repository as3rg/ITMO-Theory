\documentclass[12pt]{article}
\usepackage{bbold}
\usepackage{amsfonts}
\usepackage{amsmath}
\usepackage{amssymb}
\usepackage{color}
\setlength{\columnseprule}{1pt}
\usepackage[utf8]{inputenc}
\usepackage[T2A]{fontenc}
\usepackage[english, russian]{babel}
\usepackage{graphicx}
\usepackage{hyperref}
\usepackage{mathdots}
\usepackage{xfrac}


\def\columnseprulecolor{\color{black}}

\graphicspath{ {./resources/} }


\usepackage{listings}
\usepackage{xcolor}
\definecolor{codegreen}{rgb}{0,0.6,0}
\definecolor{codegray}{rgb}{0.5,0.5,0.5}
\definecolor{codepurple}{rgb}{0.58,0,0.82}
\definecolor{backcolour}{rgb}{0.95,0.95,0.92}
\lstdefinestyle{mystyle}{
    backgroundcolor=\color{backcolour},   
    commentstyle=\color{codegreen},
    keywordstyle=\color{magenta},
    numberstyle=\tiny\color{codegray},
    stringstyle=\color{codepurple},
    basicstyle=\ttfamily\footnotesize,
    breakatwhitespace=false,         
    breaklines=true,                 
    captionpos=b,                    
    keepspaces=true,                 
    numbers=left,                    
    numbersep=5pt,                  
    showspaces=false,                
    showstringspaces=false,
    showtabs=false,                  
    tabsize=2
}

\lstset{extendedchars=\true}
\lstset{style=mystyle}

\newcommand\0{\mathbb{0}}
\newcommand{\eps}{\varepsilon}
\newcommand\overdot{\overset{\bullet}}
\DeclareMathOperator{\sign}{sign}
\DeclareMathOperator{\re}{Re}
\DeclareMathOperator{\im}{Im}
\DeclareMathOperator{\Arg}{Arg}
\DeclareMathOperator{\const}{const}
\DeclareMathOperator{\rg}{rg}
\DeclareMathOperator{\Span}{span}
\DeclareMathOperator{\alt}{alt}
\DeclareMathOperator{\Sim}{sim}
\DeclareMathOperator{\inv}{inv}
\DeclareMathOperator{\dist}{dist}
\newcommand\1{\mathbb{1}}
\newcommand\ul{\underline}
\renewcommand{\bf}{\textbf}
\renewcommand{\it}{\textit}
\newcommand\vect{\overrightarrow}
\newcommand{\nm}{\operatorname}
\DeclareMathOperator{\df}{d}
\DeclareMathOperator{\tr}{tr}
\newcommand{\bb}{\mathbb}
\newcommand{\lan}{\langle}
\newcommand{\ran}{\rangle}
\newcommand{\an}[2]{\lan #1, #2 \ran}
\newcommand{\fall}{\forall\,}
\newcommand{\ex}{\exists\,}
\newcommand{\lto}{\leftarrow}
\newcommand{\xlto}{\xleftarrow}
\newcommand{\rto}{\rightarrow}
\newcommand{\xrto}{\xrightarrow}
\newcommand{\uto}{\uparrow}
\newcommand{\dto}{\downarrow}
\newcommand{\lrto}{\leftrightarrow}
\newcommand{\llto}{\leftleftarrows}
\newcommand{\rrto}{\rightrightarrows}
\newcommand{\Lto}{\Leftarrow}
\newcommand{\Rto}{\Rightarrow}
\newcommand{\Uto}{\Uparrow}
\newcommand{\Dto}{\Downarrow}
\newcommand{\LRto}{\Leftrightarrow}
\newcommand{\Rset}{\bb{R}}
\newcommand{\Rex}{\overline{\bb{R}}}
\newcommand{\Cset}{\bb{C}}
\newcommand{\Nset}{\bb{N}}
\newcommand{\Qset}{\bb{Q}}
\newcommand{\Zset}{\bb{Z}}
\newcommand{\Bset}{\bb{B}}
\renewcommand{\ker}{\nm{Ker}}
\renewcommand{\span}{\nm{span}}
\newcommand{\Def}{\nm{def}}
\newcommand{\mc}{\mathcal}
\newcommand{\mcA}{\mc{A}}
\newcommand{\mcB}{\mc{B}}
\newcommand{\mcC}{\mc{C}}
\newcommand{\mcD}{\mc{D}}
\newcommand{\mcJ}{\mc{J}}
\newcommand{\mcT}{\mc{T}}
\newcommand{\us}{\underset}
\newcommand{\os}{\overset}
\newcommand{\ol}{\overline}
\newcommand{\ot}{\widetilde}
\newcommand{\vl}{\Biggr|}
\newcommand{\ub}[2]{\underbrace{#2}_{#1}}

\def\letus{%
    \mathord{\setbox0=\hbox{$\exists$}%
             \hbox{\kern 0.125\wd0%
                   \vbox to \ht0{%
                      \hrule width 0.75\wd0%
                      \vfill%
                      \hrule width 0.75\wd0}%
                   \vrule height \ht0%
                   \kern 0.125\wd0}%
           }%
}
\DeclareMathOperator*\dlim{\underline{lim}}
\DeclareMathOperator*\ulim{\overline{lim}}

\everymath{\displaystyle}

% Grath
\usepackage{tikz}
\usetikzlibrary{positioning}
\usetikzlibrary{decorations.pathmorphing}
\tikzset{snake/.style={decorate, decoration=snake}}
\tikzset{node/.style={circle, draw=black!60, fill=white!5, very thick, minimum size=7mm}}

\title{Дифференциальные уравнения. Теория}
\author{Александр Сергеев}
\date{}

\begin{document}
\maketitle
\section{Уравнения первого порядка}
\subsection{Дифференциальное уравнение первого порядка и его решение}
\textbf{Определение}\\
$F(x, y, y') = 0$ -- обыкновенное д/у первого порядка\\
($F$ -- функция от трех параметров)\\
\textbf{Определение}\\
$\phi$ -- решение д/у на $\an ab$, если $\phi \in C^1\an ab$ и $F(x, \phi(x), \phi'(x)) \equiv 0$ на $\an ab$\\
(п. 1 необязательный, но нам будет удобнее работать только с такими функциями)\\
\textbf{Определение}\\
Интегральная кривая уравнения -- график его решения\\
\textbf{Определение}\\
Общее решение -- множество всех его решений\\
\textbf{Определение}\\
Общий интеграл уравнения -- уравнение вида $\Phi(x, y, C) = 0$, определяющее некоторые решения при некоторых значениях $C$\\
\textbf{Первый метод решения} -- подбор\\
\textbf{Второй метод решения} -- интегрирование (для уравнений вида $y'=Cx$)
\subsection{Уравнения в нормальной форме}
\textbf{Определение}\\
$y' = f(x, y)$ -- уравнение, выраженное относительно производной/уравнение в нормальной форме/нормальное уравнение\\
\textbf{Определение}\\
Область определения нормального уравнения -- область определния $f$\\
(множество точек, через которые могут проходить интегральные кривые)\\
\textbf{Определение}\\
Ломаная Эйлера -- ломаная с вершинами $\{(x_k, y_k)\}$, где $x_{k+1} = x_k + h, y_{k+1}=y_k+h\cdot f(x_k, y_k)$\\
\textbf{Третий метод решения (метод Эйлера)} -- построение ломаной Эйлера. Метод приближенный
\subsection{Уравнение в дифференциалах}
\textbf{Определение}\\
$P(x,y)\df x + Q(x,y)\df y = 0$ -- уравнение в дифференциалах\\
\textbf{Определение}\\
Решением $y=\phi(x)$ -- решение уравнения в дифференциалах на $\an ab$, если $\phi \in C^1\an ab$ и $P(x, \phi(x)) + Q(x, \phi(x))\phi'(x) \equiv 0$ на $\an ab$\\
Также решениями будут функции $x=\psi(y)$ (аналогично)\\
\textbf{Определение}\\
Область определения уравнения в дифференциалах = $D_P \cap D_Q$\\
\textbf{Определение}\\
$P(x)\df x + Q(y)\df y = 0$ -- уравнение с разделенными переменными\\
\textbf{Замечание}\\
Общий интеграл уравнения с разделенными переменными имеет вид $\int P(x)\df x + \int Q(y)\df y = 0$\\
\textbf{Определение}\\
Вектор-функция $(\phi, \psi): \an \alpha, \beta \rto \Rset^2$ -- параметрическое решение у.д., если $\phi, \psi \in C^1\an\alpha\beta, (\phi', \psi') \neq (0, 0)$ (кривая гладкая)\\
и $P(\phi(t), \psi(t))\phi'(t) + Q(\phi(t), \psi(t))\psi'(t) \equiv 0$ на $\an \alpha\beta$\\
\textbf{Определение}\\
Интегральная кривая уравнения в дифференциалах -- годограф ее параметрического решения\\
\textbf{Определение}\\
$\gamma = \{ r(t) | t \in \an\alpha\beta\}$ -- годограф функции $r(t) = (\phi(t), \psi(t))$\\
\textbf{Утверждение}\\
Если $y=\phi(x)$ -- решение уравнения в дифференциалах, то $(t, \phi(t))$ -- параметрическое решение\\
Если $(\phi(t), \psi(t))$ -- параметрическое решение на $\an \alpha\beta$, то $\fall t_0\in \an \alpha\beta\ \ex U(t_0):$ годограф функции $(\phi, \psi)$ -- график некоторого решения $y = g(x)$ или $x = h(y)$\\
\textbf{Геометрический смысл}\\
Пусть $(\phi, \psi)$ -- параметрическое решение на $\an \alpha\beta$\\
Тогда $P(\phi(t_0), \psi(t_0))\phi'(t_0) + Q(\phi(t_0), \psi(t_0))\psi'(t_0) = 0$ при $t_0 \in \an\alpha\beta$\\
$\letus F = \begin{vmatrix}
    P(x,y)\\
    Q(x,y)
\end{vmatrix}, r(t) = \begin{vmatrix}
    \phi(t)\\
    \psi(t)
\end{vmatrix}$\\
$F(\phi(t_0), \psi(t_0))\cdot r'(t_0) = 0$\\
$F(\phi(t_0), \psi(t_0))\perp r'(t_0)$\\
$r'(t_0)$ -- вектор касательной к интегральным кривым\\
Тогда $F$ -- поле перпендикуляров\\
\textbf{Определение}\\
Поле на плоскости -- это отображение $F: D \subset \Rset^2 \rto \Rset^2$\\
\textbf{Определение}\\
Дифференциальные уравнения называют эквивалентными/равносильными на множестве $D$, если на этом множестве они имеют одинаковое множество интегральных кривых\\
\textbf{Утверждение}\\
$y'=f(x,y)$ равносильно $\df y = f(x,y)\df x$\\
\textbf{Замечание}\\
Уравнение в дифференциалах равносильно $y'_x = -\frac{P(x,y)}{Q(x,y)}$ в областях, где $Q(x,y) \neq 0$\\
и $x_y' = -\frac{Q(x,y)}{P(x,y)}$ в областях, где $P(x,y)\neq 0$\\
\textbf{Определение}\\
Если $P(x_0, y_0) = Q(x_0, y_0) = 0$, то $(x_0, y_0)$ -- особая точка уравнения в дифференциалах\\
\subsection{Уравнения с разделяющимися переменными}
\textbf{Определение}\\
$P(x)\df x + Q(y)\df y = 0$ -- Уравнение с разделенными переменными\\
\textbf{Определение}\\
Функция $y=\phi(x)$ задана неявно уравнением $F(x,y) = 0$ при $x\in E$, если $F(x, \phi(x)) \equiv 0$ при $x \in E$\\
\textbf{Теорема (общие решение уравнения с разделенными переменными)}\\
Пусть $P \in C\an ab, Q \in C\an cd$\\
$P^{(-1)}, Q^{(-1)}$ -- некоторые первообразные $P, Q$\\
Тогда $y = \phi(x)$ -- решение уравнения на $\an \alpha\beta \LRto$  
\begin{itemize}
    \item $\phi \in C^1\an\alpha\beta$
    \item $\ex C \in \Rset: y=\phi(x)$ неявно задана уравнением $P^{(-1)}(x) + Q^{(-1)}(y) = C$
\end{itemize}
\textbf{Доказательство $\Rto$}\\
Пусть $y = \phi(x)$ -- решение на $\an \alpha\beta$\\
1 -- по определению\\
Выберем $x_0 \in \an \alpha\beta, y_0 = \phi(x_0)$\\
Заметим, что $\ex A:\ P^{(-1)}(x) = \int_{x_0}^x P(t)\df t + A$\\
$\ex A_2:\ Q^{(-1)}(y) = \int_{y_0}^y Q(t)\df t + A_2$\\
$\int_{x_0}^x P(t)\df t + A + \int_{y_0}^y Q(t)\df t + A_2 \equiv C$\\
Сделаем замену $t \rto \phi(t)$ справа\\
$\int_{x_0}^x P(t)\df t + A + \int_{x_0}^x Q(\phi(t))\phi'(t)\df t + A_2 \equiv C$\\
$\int_{x_0}^x P(t)\df t + \int_{x_0}^x Q(\phi(t))\phi'(t)\df t \equiv C - A - A_2$\\
$\int_{x_0}^x \ub{0 \text{ по определению решения}}{(P(t) + Q(\phi(t))\phi'(t))}\df t \equiv C - A - A_2$\\
Отсюда $C := A + A_2$\\
Т.о. 2 доказано\\
\textbf{Доказательство $\Lto$}\\
Проверим $P(x) + Q(\phi(x))\phi'(x) = 0$ на $\an \alpha\beta$\\
$P^{(-1)}(x) + Q^{(-1)}(\phi(x)) \equiv C$\\
Продифференцируем\\
$P(x) = Q(\phi(x))\phi'(x) = 0$\\
\textbf{Определение}\\
$p_1(x)q_1(y)\df x + p_2(x)q_2(y)\df y = 0$ -- уравнение с разделяющимися переменными
\subsection{Задача Коши}
Рассмотрим $y'=f(x,y)$\\
\textbf{Определение}\\
Задачей Коши (ЗК) для нормального уравнения называют задачу нахождения его решения, удовлетворяющего начальному условию $y(x_0) = y_0$\\
\textbf{Теорема Пиано (частный случай) -- существование решения задачи Коши для нормального уравнения 1 порядка}\\
Пусть $f \in C(G), G$ -- область (открытое связное множество)\\
Возьмем $(x_0,y_0) \in G$\\
Тогда $\ex E = \an ab, x_0 \in E, \ex \phi: E \rto \Rset$ -- решение для задачи Коши\\
$y'=f(x,y), y(x_0) = y_0$\\
\textbf{Теорема Пикара о единственности решения задачи Коши для нормального уравнения 1 порядка}\\
$f, f'_y \in C(G), G$ -- область, $(x_0, y_0) \in G$\\
Пусть $\psi,\phi$ -- решения задачи Коши\\
Тогда $\phi = \psi$ на $D_\phi \cap D_\psi$
\subsection{Линейное уравнение 1-ого порядка}
\textbf{Определение}\\
$y'=p(x)y+q(x)$ -- линейное уравнение\\
$y'= p(x)y$ -- однородное линейное уравнение\\
\textbf{Теорема (общее решение линейного уравнения первого порядка)}\\
$E = \an ab, p,q \in C(E),\mu = e^{-\int p}$\\
Тогда $y=\frac{C+\int (q\mu)}{\mu}, C \in \Rset, D_y =E$ -- общее решение ЛУ\\
\textbf{Доказательство}\\
Пусть $S$ -- множество всех решений ЛУ\\
$F := \{\phi: \ub{\text{промежуток}}{\ot E} \subset E \rto \Rset\}, \phi = \frac{C + \int (q\mu)}{\mu}, C \in \Rset$\\
Докажем, что $F=S$\\
Возьмем $\phi \in S$\\
Тогда $\phi' \equiv p\phi + q$ на $\ot E$\\
$\phi'\mu = p\phi\kappa + q\mu$\\
$\phi'\mu-p\phi\mu=q\mu$\\
$\phi'e^{-\int p} - p\phi e^{-\int p} = (\phi e^{-\int p})' = (\phi\mu)'$\\
$(\phi\mu)'=q\mu$\\
$\phi\mu=\int q\mu + C$\\
$\phi = \frac{\int (\phi\mu) + C}\mu$\\
Отсюда $\phi \in F$\\
Возьмем $\phi \in F$\\
$\phi = \frac{C+\int (\mu q)}\mu$ на $\ot E$\\
$\phi \in C^1$\\
Подставим в уравнение\\
$\phi'=p\phi+q$\\
$\frac{\mu q + \mu'(C+\int (\mu q))}{\mu^2} = \frac{p(C+\int (\mu q))}\mu + q$\\
Л.ч.: $\frac{\mu q + \mu'(C+\int (\mu q))}{\mu^2} = \frac{\mu^2 q - (-\pi)\mu(C+\int (pq))}{\mu^2} = q + \frac p\mu(C+\int pq)$\\
П.ч. = Л.ч.\\
Ч.Т.Д.\\
\textbf{Следствие (общее решение ЛОУ)}\\
$p \in C^1(E), E = \an ab$\\
Тогда $y = Ce^{\int p}, C \in \Rset, D_y = E$\\
\textbf{Доказательство}\\
$q = 0$\\
\textbf{Метод Лагранжа (метод вариации постоянной)}
\begin{enumerate}
    \item Для ЛУ $y' = p(x)y + q(x)$ запишем соответствующее ЛОУ\\
    $y_2'=p(x)y_2$\\
    $y_2=Ce^{\int p}$
    \item Заменим $C$ на $C(x)$ и подставим в исходное уравнение\\
    $y = C(x)e^{\int p}$\\
    $p(x)(C(x)e^{\int p})+q(x) = (C(x)e^{\int p})'$
    \item Находим $C(x)$ из полученного уравнения
    \item Запишем общее решение $y=C(x)e^{\int p}$
\end{enumerate}
\textbf{Доказательство}\\
В общем виде мы получим ту же формулу, что и в предыдущем методе
\subsection{Уравнение в полных дифференциалах}
\textbf{Определение}\\
$p(x,y)\df x + Q(x,y)\df y = 0, \ex u: u_x'=P, u_y' = Q$ -- уравнение в дифференциалах\\
Его решение имеет вид $\df u = 0$\\
$\df u = u_x'\df x + u_y'\df y$\\
Тогда $u = \const$\\
Признак уравнения в полных дифференциалах: $P,Q \in C^1(G), G$ -- область, $P'_y = Q_x'$
\end{document}
