\documentclass[12pt]{article}
\usepackage{bbold}
\usepackage{amsfonts}
\usepackage{amsmath}
\usepackage{amssymb}
\usepackage{color}
\setlength{\columnseprule}{1pt}
\usepackage[utf8]{inputenc}
\usepackage[T2A]{fontenc}
\usepackage[english, russian]{babel}
\usepackage{graphicx}
\usepackage{hyperref}
\usepackage{mathdots}
\usepackage{xfrac}


\def\columnseprulecolor{\color{black}}

\graphicspath{ {./resources/} }


\usepackage{listings}
\usepackage{xcolor}
\definecolor{codegreen}{rgb}{0,0.6,0}
\definecolor{codegray}{rgb}{0.5,0.5,0.5}
\definecolor{codepurple}{rgb}{0.58,0,0.82}
\definecolor{backcolour}{rgb}{0.95,0.95,0.92}
\lstdefinestyle{mystyle}{
    backgroundcolor=\color{backcolour},   
    commentstyle=\color{codegreen},
    keywordstyle=\color{magenta},
    numberstyle=\tiny\color{codegray},
    stringstyle=\color{codepurple},
    basicstyle=\ttfamily\footnotesize,
    breakatwhitespace=false,         
    breaklines=true,                 
    captionpos=b,                    
    keepspaces=true,                 
    numbers=left,                    
    numbersep=5pt,                  
    showspaces=false,                
    showstringspaces=false,
    showtabs=false,                  
    tabsize=2
}

\lstset{extendedchars=\true}
\lstset{style=mystyle}

\newcommand\0{\mathbb{0}}
\newcommand{\eps}{\varepsilon}
\newcommand\overdot{\overset{\bullet}}
\DeclareMathOperator{\sign}{sign}
\DeclareMathOperator{\re}{Re}
\DeclareMathOperator{\im}{Im}
\DeclareMathOperator{\Arg}{Arg}
\DeclareMathOperator{\const}{const}
\DeclareMathOperator{\rg}{rg}
\DeclareMathOperator{\Span}{span}
\DeclareMathOperator{\alt}{alt}
\DeclareMathOperator{\Sim}{sim}
\DeclareMathOperator{\inv}{inv}
\DeclareMathOperator{\dist}{dist}
\newcommand\1{\mathbb{1}}
\newcommand\ul{\underline}
\renewcommand{\bf}{\textbf}
\renewcommand{\it}{\textit}
\newcommand\vect{\overrightarrow}
\newcommand{\nm}{\operatorname}
\DeclareMathOperator{\df}{d}
\DeclareMathOperator{\tr}{tr}
\newcommand{\bb}{\mathbb}
\newcommand{\lan}{\langle}
\newcommand{\ran}{\rangle}
\newcommand{\an}[2]{\lan #1, #2 \ran}
\newcommand{\fall}{\forall\,}
\newcommand{\ex}{\exists\,}
\newcommand{\lto}{\leftarrow}
\newcommand{\xlto}{\xleftarrow}
\newcommand{\rto}{\rightarrow}
\newcommand{\xrto}{\xrightarrow}
\newcommand{\uto}{\uparrow}
\newcommand{\dto}{\downarrow}
\newcommand{\lrto}{\leftrightarrow}
\newcommand{\llto}{\leftleftarrows}
\newcommand{\rrto}{\rightrightarrows}
\newcommand{\Lto}{\Leftarrow}
\newcommand{\Rto}{\Rightarrow}
\newcommand{\Uto}{\Uparrow}
\newcommand{\Dto}{\Downarrow}
\newcommand{\LRto}{\Leftrightarrow}
\newcommand{\Rset}{\bb{R}}
\newcommand{\Rex}{\overline{\bb{R}}}
\newcommand{\Cset}{\bb{C}}
\newcommand{\Nset}{\bb{N}}
\newcommand{\Qset}{\bb{Q}}
\newcommand{\Zset}{\bb{Z}}
\newcommand{\Bset}{\bb{B}}
\renewcommand{\ker}{\nm{Ker}}
\renewcommand{\span}{\nm{span}}
\newcommand{\Def}{\nm{def}}
\newcommand{\mc}{\mathcal}
\newcommand{\mcA}{\mc{A}}
\newcommand{\mcB}{\mc{B}}
\newcommand{\mcC}{\mc{C}}
\newcommand{\mcD}{\mc{D}}
\newcommand{\mcJ}{\mc{J}}
\newcommand{\mcT}{\mc{T}}
\newcommand{\us}{\underset}
\newcommand{\os}{\overset}
\newcommand{\ol}{\overline}
\newcommand{\ot}{\widetilde}
\newcommand{\vl}{\Biggr|}
\newcommand{\ub}[2]{\underbrace{#2}_{#1}}

\def\letus{%
    \mathord{\setbox0=\hbox{$\exists$}%
             \hbox{\kern 0.125\wd0%
                   \vbox to \ht0{%
                      \hrule width 0.75\wd0%
                      \vfill%
                      \hrule width 0.75\wd0}%
                   \vrule height \ht0%
                   \kern 0.125\wd0}%
           }%
}
\DeclareMathOperator*\dlim{\underline{lim}}
\DeclareMathOperator*\ulim{\overline{lim}}

\everymath{\displaystyle}

% Grath
\usepackage{tikz}
\usetikzlibrary{positioning}
\usetikzlibrary{decorations.pathmorphing}
\tikzset{snake/.style={decorate, decoration=snake}}
\tikzset{node/.style={circle, draw=black!60, fill=white!5, very thick, minimum size=7mm}}

\DeclareMathOperator{\Lin}{Lin}
\DeclareMathOperator{\Int}{Int}
\DeclareMathOperator{\grad}{grad}
\newcommand{\ppart}[2]{\frac{\partial #1}{\partial #2}}

\title{История науки и техники}
\author{Александр Сергеев}
\date{}
\begin{document}
\maketitle
\section{Лекция 05.10}
Технологии надо внедрять грамотно\\
Цели истории науки и техники:
\begin{itemize}
    \item Обеспечить постоянное повышение качества научно-технического потенциала человечества путем внедрения новых знаний
    \item Служить основой для интеграции естественно-научной, технической и гуманитарной форм знаний
    \item Постоянный ввод в оборот нового фактического и концептуального научно-технического знания\\
    (понять, почему знания появились в этот период)
    \item Создать фактологическую и концептуальную основу для моделирования будущего прогресса
\end{itemize}
Задачи:
\begin{itemize}
    \item Поиск, систематизация, анализ и обобщение историко-научных фактов
    \item Расширение базы источников для исследований
    \item Выявление и обоснование законов научно-технического развития
    \item Анализ роли и значения научно-технического развития в истории
    \item Совершенствование методологического инструментария
    \item Рассмотрение вопросов приоритета различных новшеств
\end{itemize}
Источники:
\begin{itemize}
    \item Письменные
    \item Визуальные
    \item Материальные
    \item Устные
    \item Этнографические
    \item Цифровые
    \item Археологические
\end{itemize}
\textbf{Определение}\\
Наука -- это непрерывно развивающаяся система знаний объективных законов природы, общества и мышления, получаемых и превращаемых в непосредственную производительную силу общества в результате социально-экономической деятельности\\
\textbf{Определение}\\
Техника -- это
\begin{enumerate}
    \item совокупность технических устройств (артефактов)
    \item совокупность различных видов технической деятельности по созданию технических артефактов
    \item совокупность технических знаний
\end{enumerate}
Можешь выбрать наиболее понравившееся или доказать эквивалентность\\
\textbf{Определение}\\
Технология -- совокупность процессов получения и обработки сырья и материалов\\
Существует несколько точек зрения на связь между наукой и технологией\\
\begin{itemize}
    \item Наука -- теоретическая часть, технология -- прикладная часть
    \item Развитие техники обгоняет развитие науки
    \item Развитие науки всегда обгоняет развитие техники
    \item Наука и техника -- автономные процессы, дополняющие друг друга
\end{itemize}
Наука декультурна и денациональна\\
Наукой может заниматься любой индивида\\
Философы науки:
\begin{itemize}
    \item Томас Кун 
    \item Жан Бодрийяр
    \item Маршалл Маклюэн
    \item Поль Вирильо
\end{itemize}
Все они имеют негативный взгляд на технологии\\
Люди становятся зависимыми от них\\
Они окружают себя <<протезами>>, усиливающими наши способности\\
Периоды историиЖ
\begin{itemize}
    \item Первобытное общество -- от возникновения человека до возникновения первых древних цивилизаций -- Древняя Индия, Древний Китай (6-5 век до нашей эры)
    \item Эпоха древности -- от возникновения первых цивилизаций до 476 г.н.э. (падение Западной Римской Империи)
    \item Средние века -- от 476 г. н.э. до эпохи Ренесансва (в каждой стране она случилась в разное время) (либо до 16 века, либо до Великой Французской революции в конце 18 века)
    \item Эпоха Ренесанса -- от момента, когда она началась (большинство считает, что в 16 веке) до Нового времени (модерна)
    \item Новое время -- от момента начала до 1914 (Первая Мировая Война)
    \item Новешая история -- от конца ПМВ до нашего времени
\end{itemize}
Этапы периодизации историии:
\begin{itemize}
    \item Пранаука (величайшее изобретение -- речь)
    \item Античная наука (Древняя Греция, Рим) -- период формирования первых научных теорий, трактатов
    \item Период средневековой магии (деградация, первые эксперименты)
    \item Классическая наука (с начала Ренесанса) -- Ньютон, Галилей, Коперник
    \item Постклассическая наука -- после мировых войн
\end{itemize}
Альтернативный взгляд:
\begin{itemize}
    \item Доклассическая наука 
    \item Классическая наука
    \item Постклассическая наука
\end{itemize}
\section{Лекция 19.10}
Никакая цивилизаций не могла быть построена без земледелия\\
Земледелие появляется в результате \textit{неолитической} революции\\
В результате этой революции народы смогли осесть, что стало началом цивилизации\\
Земледелие дает кратно больше продукта, чем собирательство и охота\\
Древняя Греция строилась по системе полисов (городов-государств), которые были независимы друг от друга\\
Экономика полисов была основана на рабстве, а свободные жители было освобождены от тяжелого труда\\
Античная наука делилась на следующие этапы
\begin{enumerate}
    \item Период единой науки о природе -- философия\\
    Самый известный деятель -- Аристотель (4 век до н.э.)
    \item Эллинистический период -- период распада единой науки. Начинают появляться отдельные направления\\
    Этот процесс начался еще до Аристотеля
    \item Попытка создания новой единой науки о природе\\
    Нужны межпредметные связи, из-за чего возникает попытка объединения отраслей
\end{enumerate}
За эти 3 периода возник Квадриум -- 4 аспекта греческой образованности: арифметика (число само по себе), геометрия (число на плоскости), музыка (число в звуке), астрономия (число в космосе)\\
(В порядке убывания крутости)
\subsection{Эллин}
Эллинизм начался после смерти Александр Македонского (330 г до н.э. -- 1 век до н.э.)\\
Т.к. после его смерти не осталось наследников, его владения было поделены между его сподвижниками\\
Эллин -- грек\\
В период экспансии империи Македонского в страны была принесена культура Греции\\
Знаменитый мусейон был основан в этот период в Египте\\
В этом мусейоне работал Евклид\\
Его наиболее известная работа -- <<Начала>> -- один из первых письменных источников, дошедших до наших дней\\
Первую редакцию смог восстановить Йохан Людвиг Гейберг \\
Другой известный ученый этого периода -- Демокрит\\
Им была придумана теория атомизма\\
Изначально это было про атомы, но потом идея была применена на все остальное\\
Демокрит считал, что атомы двигаются в одном направлении, от которого не могут отклоняться(позже из этого вытек фатализм)\\
Его ученик -- Эпикур -- наоборот считал, что атомы могут отклоняться. Хаос рождает хаос, а значит человек имеет волю\\
Также он считал, цель жизни человека -- удовлетворять свои потребности (но также он считал, что удовлетворять их надо в меру)\\
Следующий крупный ученый -- Аристарх Самосский -- крупный математик и основатель гелеоцентрической системы мира\\
Он работал в мусейоне и выдвинул теорию, что сферическая Земля вращается вокруг Солнца, а также попытался посчитать ее размеры, а также расстояние\\
Другой крупный ученый -- Ктесибий\\
Он один из отцов гидравлики и пневматики\\
Создал первое пневматическое ружье\\
Еще один известный ученый -- Архимед\\
Он родился на юге Сицилии\\
Погиб Архимед в ходе захвата Сиракузы Римом\\
Конец периода -- захват Греции Римом
\subsection{Римский этап}
1 век до н.э. -- 5 век н.э.\\
В этот период Рим -- центр науки\\
В ходе перенятия Римом культуры Греков он стал двуязычным -- языком общения стала латынь, а языком науки -- греческий\\
Римская империя подарила миру системы управления\\
Каждый правитель в Риме проходил сложную систему обучения -- грамматика (начальная школа), язык и 7 свободных искусств(грамматика, диалектика, риторика, музыка, арифметика, геометрия, астрономия)\\
В этот момент жили и работали Гален (медик), Герон, Витрувий (архитектор, автор формулы Витрувия: польза, прочность, красота), Птолемей(географ, астроном, автор <<Альмагеста -- Великого построения>>)\\
\section{Лекция 02.11}
Периодизация средних веков:
\begin{enumerate}
    \item Темное время: 5-6 век -- 9 вен н.э.
    \item Средние средние века: 10 век -- 11 век н.э.
    \item Зрелое средневековье: 12 век -- 15 век
\end{enumerate}
Потом идет возрождение
\subsection{Темное время}
Кризис рабочей силы\\
Переход от рабства к феодализму под влиянием церкви\\
На смену рабам приходит "свободный человек"\\
Основное отношение -- нетранзитивное подчинение ("вассал моего вассала -- не мой вассал")\\
(прямое подчинение)\\
Царствуют войны, голод, болезни\\
Это приводит к упадку науки\\
В ходу "поросячья латынь" (упрощенный диалект латыни)\\
С приходом церкви происходит смена парадигмы\\
Вместо демонстрации красоты человеческого тела появляется табу на демонстрацию телесного\\
Практика вскрытия появилась в позднем средневековье. До этого считалось, что тело неприкасаемо\\
\subsection{Христианская ученость}
Зарождение в 2-3 веке\\
3 основы
\begin{enumerate}
    \item Владение словом (красноречие)\\
    Человек должен уметь читать и понимать Библию\\
    Очень трепетное отношение к слову\\
    Некоторые слова, связанные со священным, даже не писались полностью
    \item Построение картины мира (Теологии)
    \item Обоснование праведной жизни\\
    Формирование новой этики, политики, экономики\\
    Создание нового понимания правильного
\end{enumerate}
Античная ученость была направленная на поиск во нового, поиск вовне\\
С точки зрения христианской учености истина уже уже известна\\
Нужно лишь найти путь к ней, понять ее\\
Христианская ученость не отрицала всю античную ученость\\
В первую очередь, христианская ученость взяла у античной язык -- латынь\\
Полноценное возрождение науки происходит с окончанием темного времени\\
Этому способствовало завершение периода смуты и контакты с востоком(крестовые походы, торговля)\\
Через эти контакты в Европу стали проникать античные знания\\
Центрами науки стали центы торговли -- Италия, Франция, часть Испании\\
Открываются университеты\\
Все университеты независимы от государств, где они расположены, и подчинялись Папе Римскому\\
Все обучение велось на латыни\\
Появляется схоластика -- наука объяснения религии методами классической науки\\
Схоластика делилась на 
\begin{enumerate}
    \item Учение о движении (кинематика)
    \item Учение о свете (оптика)
    \item Учение о живом
    \item Алхимия
\end{enumerate}
Ученые считали, что Бог уже все создал. Надо лишь понять это\\
Цель -- не доказательство, а построение логической мысли\\
Раз они к этому пришли -- значит это воля Божья\\
Ученые того времени
\begin{enumerate}
    \item Фибоначчи (Леонардо Пизанский, 12-13 века) -- крупнейший математик своего времени и один из крупнейших в средних веках\\
    Был сыном потомственного торговца\\
    Со временем они переехали в Алжир\\
    По просьбе отца стал учиться у арабов математике для ведения бизнеса\\
    Но так увлекся ей, что бросил торговлю\\
    Написал "книгу абака"\\
    Ее заметил Фридрих II -- император Священной Римской империи\\
    В этой книге он описал главные научные знания о геометрии и алгебре того времени\\
    Благодаря ему в Европе распространилась десятичная система счисления\\
    Т.к. Фридрих II был большим фанатом учености, то при нем при дворе проводились матбои
    \item Альберт Великий (13 век)\\
    Родился в Священной Римской империи\\
    Принадлежал к ордену доминиканцев\\
    Был крупнейшим энциклопедистом своего времени\\
    Первым выделил мышьяк\\
    Переводил труды Аристотеля, спорил с ним\\
    Также переводил труды арабский ученых\\
    Все свои труды он систематизировал и публиковал в сборниках\\
    Был учителем Фомы Аквинского
    \item Фома Аквинский\\
    Стал причиной расцвета схоластики\\
    Был приглашен к Папской курии\\
    Изучал познание и выделял 3 зоны
    \begin{enumerate}
        \item Ум -- духовные силы
        \item Интеллект -- способность управлять умственными возможностями
        \item Разум -- способность к рассуждению
    \end{enumerate}
    Используя эти 3 зоны, человек совершает 3 умственно-познавательные операции
    \begin{enumerate}
        \item Созерцание (наблюдение, введение понятий)
        \item Суждение (попоставление понятий между собой)
        \item Умозаключение (связывание суждений друг с другом)
    \end{enumerate}
    Умея совершать все 3 операции, он приближает себя к Богу\\
    Ученый человек занимается благим делом -- двигается к Богу\\
    Поэтому его нельзя порицать\\
    Оффтоп: главные "развлечения", приписываемые средневековью: гонения ученых, казни, на самом деле свойствены Возрождению. В средневековье было спокойнее\\
    Чувства Фома Аквинский делил на 4 составляющих
    \begin{enumerate}
        \item Общие чувства
        \item Пассивная память 
        \item Активная память
        \item Интеллект (уникален у каждого человека. Чем образованнее человек, тем выше интеллект)
    \end{enumerate}
    Известна его работа о доказательстве существования Бога (сюрприз: от противного)\\
    Он приводит 5 доказательств:
    \begin{enumerate}
        \item Через причину (если все движется, значит есть причина)
        \item Через первопричину (причина причин -- Бог)
        \item Через необходимость
        \item Через стремление к совершенству (а совершенство есть Бог)
        \item Через целевую причину (если все стремится к финалу, то финал -- это Бог)
    \end{enumerate}
    Схоласты работали так: садились и думали
\end{enumerate}
\section{Лекция 23.11}
Десятинная церковь -- самая древняя (ныне не сохранившаяся) церковь на Руси\\
Десятинная -- потому что 1/10 часть всех доходов уходило церкви\\
Эта церковь показывает высокий уровень развития Руси в момент принятия христианства\\
Церковь Покрова на Нерли -- другая древняя церковь\\
Вместе с христианством на Русь пришла и ученость\\
...\\
Во второй половине XX века была выдвинута теория, что Слово о полку Игореве -- историческая реконструкция, написанная в недавнем времени\\
Но Лихачев в своем сочинении смог доказать, что Слово было написано в средневековье\\
Другой памятник истории -- <<Русская правда>> Ярослава Мудрого\\
Памятники оружия, доспехов домонгольской эпохи почти не сохранились\\
Скань -- художественная обработка драгоценных металлов\\
Скань -- техника закручивания из проволок\\
Нашествие Татаро-Монгольского ига негативно сыграло на культуре Руси\\
Многие ученые, ремесленники были убиты или уведены в рабство, многие культурные памятники были утрачены\\
Русь оправлялась от этого несколько веков\\
Однако по Белокаменному Кремлю, постоенному при Дмитрии Донском, мы видим, что к моменту его правления Русь частичо смогла восстановить знания и техники\\
Другим памятником, показывающим высокий уровень развития, является Колокольня Ивана Великого (Ивана III)\\
Большой толчок к развитию культуры Русь получает в момент правления Михаила Романова\\
Соборное уложение -- свод законов Руси\\\\
Многие думают, что Петр I пришел и все изменил, но на самом деле его изменения основаны на тех изменениях, что сделали его предшественники\\
По периоду правления Петра I можно сказать, что в тот период Русь отставала от Европы\\
Эпоха просвещения -- эпоха, когда научный прогресс замедлился\\
Сделанные открытия опережали развитие социального строя\\
В этот период появляется большое количество неродовитых деятелей, а многая старая элита погибает или разоряется\\
В этот период происходят два значимых события: Великая Французская революция и Война за независимость в США\\
Оба этих события подготовлены движением просвещения\\
Это люди, мыслившие иначе, чем люди старого времени\\
Они критиковали все: церковь, социальный строй\\
Их главной ценностью был рационализм и свобода мысли\\
В этот момент происходит национализация науки: латынь уходит, статьи начинают писаться на национальных языках\\
Правители того времени тесно общались с учеными, окружали себя ими\\
Они видели в науке источник своей силы\\
Так Петр I был членом Голландской Академии наук, был лично знаком с Ньютоном\\
Екатерина II вела переписку с Вольтером, а позже выкупила его библиотеку\\
Термин <<Военная революция>> -- полная модернизация экономики через научную модернизацию армии и флота\\
Все просветители того времени стремились создать единую энциклопедию, объясняющую все
\section{Лекция 30.11}
*про школы в Российской империи*\\
Петр I собой заменял целые толпы людей\\
После смерти Петра замены ему не нашлось\\
Поэтому появлются совещательные органы\\
После смерти Петра количество цифирных школ уменьшается на треть\\
Но это символ не регресса, а саморегуляции реформ\\
Школами управляло Адмиралтейство\\
Тогда строительство корабля было сопоставимо со страительством космического корабля сейчас\\
После смерти Петра в цифирных школах учатся только дьячьи и подьячи дети\\
Была введена коллегиальная система управления (как в Европе)\\
Широкое распространение после Петра получили епархиальные школы. Они предназначались для детей духовенства\\
По приказу Петра такие школы обязательны быть в каждой епархии\\
Священнослужитель должен быть грамотным, ведь в церквях хранились метрические книги(книги с записями о рождении, смерти, свадьбе)\\
Также священнослужитель должен быть грамотным, чтобы проводить хорошую проповедь (и пропагандировать в том числе ценности, выгодные Петру)\\
Также при Петре появились первые гарнизонные школы. Это было необходимо, т.к. Петр создал в России регулярную армию. В отличие от старой модели, здесь военный -- профессия, т.е. человек занимается этим как основным делом\\
Создавались и другие профессиональные школы: школа переводчиков при коллегии иностранных дел, концелярские школы при различных коллегиях, технические учебные заведения(Навигацкая школа) и т.п.\\
Появились школы для хорошо образованных людей (греческая, славяно-латинская академии). Часто там готовились высшие слои для духовества, т.к. духовенство было основным транслятором идей Петра\\
В Навигацкой школе людей обучали астрономии, геометрии, живописи навигации, рапирному делу (обращению с холодным оружием). Эта школа была элитным заведением.\\
Первая такая школа появилась в Москве, но позже они появились в Нарве, Таллине\\
В Петербурге появилась Морская академия для лучших учеников Навигацкой школы\\
Если в Навигацкой школе основными преподавателями были русские, то в академии уже преподавали иностранцы\\
При Петре были переведены и изданы сотни и тысячи иностранных книг\\
В 1701 году в Москве открывается Артиллерийская школа, в 1712 -- Инженерная школа, в 1707 -- Медицинское училище\\
Петр проводит реформу алфавита: в 1710 году он вводит новый русский алфавит (гражданский шрифт)\\
При Петре основывается первая типография и печатается первая газета -- Ведомости\\
Проводится много экспедиций: на Кавказ, на Камчатку\\
Одной из мечт Петра было перенаправить торговый путь из Индии в Европу через Россию\\
Ведутся геологические изыскания\\
Для улучшения логистики строятся каналы (в т.ч. Ладожский канал)\\
В Петербурге открывается обсерватория\\
Инициируются исторические экспедиции\\
Во время великого посольства Петр становится членом Французской академии наук\\
В России Петр основывает Российскую академию наук. В ней было 3 направления: Математическое, Физическое и Гуманитарное\\
Академики должны были заниматься наукой, выступать с докладами и следить за актуальностью библиотеки\\
Петр создал основу Российского образования\\
Это образование -- светское, основанное на принципах рационализма, практической применимости\\
Особенностью Петра было то, что он довал реализоваться каждому: его не заботило, из какого сословия человек\\
Примером является Калмыков. Он был из низжего сословия. Но своими знаниями он так впечатлил Петра, что тот сделал его дворянином, дал ему звание офицера\\
Основой карьерной лестницы был табель о рангах\\
Эта система позволяла любому человеку заработать дворянский титул\\
С другой стороны, всё дворянство обязали служить\\
В 1731 Анна Иоановна образовывает Шляхетский (Кадетский) корпус\\
Это учебное заведение для дворян\\
Он располагался в бывшем дворце Меншинова\\
В 1752 образуется Морской Шляхетский корпус\\
В 1758 -- Инженерная Шляхетская школа\\
В XVIII веке также в ход вошло домашнее образование. Учителями ставились иностранцы (в том числе многие французы, сбежавшие от Французской революции)\\
Это часто приводило к тому, что первым языком многих дворянинов становится французский\\
Это сделало Россию более открытой для мира\\
В 1755 году основывается Московский университет при поддержке графа Шувалова\\
Шувалов взял за основу идеи Ломоносова\\
Этот университет был исключительно светским: в нем были Философский, Юридический и Медицинский факультеты\\
Основными учениками стали разночинцы -- жители городов средних чинов (не низких, но еще и не дворянства)\\
Во время правления Екатерины II основными учениками становится дворянство (в целом при Екатерине II расцвело дворянство)\\
Политика Екатерины привели к расцвету культуры: дворяне были освобождены от службы и могли заниматься тем, чем они хотят\\
В начале XIX века происходит университетская реформа: во многих городах(Казань, Харьков и другие крупнейшие города) империи открываются Императорские университеты\\
Для подготовки учеников университетов создаются гимназии\\
Они делились на гимназии для дворян и для разночинцев\\
Учились они уже на русских учебниках, написанных Ломоносовым\\
Российские университеты отличались тем, что образование велось на национальном языке(а не на латыни, как в Европе)\\
Центр науки смещается с Академии наук на университеты\\
В конце XVIII века основывается Смольный институт благородных девиц -- первое учебное образование для женщин (как для знатных девушек, так и для разночинцев)\\
В конце XVIII века создается Царско-Сельский лицей
\section{Лекция 07.12}
Петр понимает, что необходимо исследовать территорию России\\
При нем было открыто 121 месторождение\\
На местах этих месторождений были построены заводы\\
\underline{Машина Якова-Батищева} -- машина для ковки стволовых досок, необходимых для производства ружий\\
Другой известный изобретатель -- механик Нартов\\
Он создал русские токарные станки\\
Помимо этого развивается гуманитарная мысль: биология, агрономия, экономика, история, философия\\
Ратищев -- автор первых книг по истории России\\
1758 -- открывается Академия художеств\\
\subsection*{XIX век}
1802г -- министерство народного просвещения\\
Оно реализует системы косплексного образования\\
1804 -- школьный устав. Он вводит систему в образование. Возникает несколько ступеней:
\begin{enumerate}
    \item Двухклассное уездное училище
    \item Гимназия (4 класса)\\
    Эти две ступени финансировались из бюджета страны. Программа этих двух этапов была совместная (т.е. одно -- продолжение другого)
\end{enumerate}
Параллельно существовала церковно-приходская школа, где давалось самое базовое образование (1 год)\\
Они не финансировались государством, а поддерживались на деньги с приходов\\
1804 -- университетский устав\\
Это один из самых либеральных и свободных уставов страны и Европы\\
Создавались императорские университеты\\
Они обладали полной автономностью: сами выбирали программу, управление\\
Все должности там были выборными, а не назначались государством\\
Был и университетский суд, который судил студентов \underline{до} гражданского суда\\
Позже университеты получили право осуществлять контроль над гимназиями: составлять планы, контроллировать издание учебников и т.п.\\
Также существовали лицеи -- средние учебные заведения. Они были направлены на подготовку кадров для правительства\\
Появляются институты: Технологический, Горный\\
Они были менее привилегированными, обладали меньшей автономией\\
В 1825 году к власти приходит Николай I\\
Он усиливает сословность в образование (т.е. увеличивает количество барьеров):\\
Уездные училища стали доступны только для детей купцов, ремесленников и городских жителей\\
Гимназии -- только для дворян, чиновников и богатых купцов\\
Идея была в том, чтобы в низших слоях не возникали идеи революции\\
Помимо образования школа еще и воспитывала людей\\
1835 -- отмена университетского устава и принятие нового\\
Ликвидируется часть автономности\\
Но увеличивается финансирование и дается больше прав в научной сфере\\
Со временем унивеситеты становятся центрами наук вместо Академии наук\\
Первое значение имело техническое и утилитарное образование\\
Появляются \underline{ведомственные школы} -- при ведомствах\\
(среднее образование)\\
Но общий уровень образования был крайне низким\\
Но элита была крайне образована\\
В середине XIX века за год издается более 2000 книг\\
Происходит бум газет и журналов\\
Они очень делились на консервативные (Москвитянин и Северная пчела) и демократические (Современник, Отечественные записки)\\
Появляются большие типографии\\
В 1814 году происходит открытие Императорской публичной библиотки (ныне Российская национальная библиотека)\\
Также в России востребованы иностранные издания\\
Университеты -- центры науки\\
При них работают крупные деятели и совершаются крупные открытия\\
1819-1821 -- Лазарев и Белинсгаузен открывают Антарктиду\\
1803-1806 -- Крузенштерн и Лисянский совершают кругосветное путешествие\\
1826 -- Лобачевский создает свою геометрию (работал в Казанском университете)\\
В Петербурге работал математик Чебышев\\
Остроградский и Буняковский занимались теорвером\\
Идет изучение электричества и магнитизма (Якоби и Шилинг -- изобрели и усовершенствовали один из первых ЭМ телеграфов)\\
В первой половине XIX века органической химией занимался Зимин и Бутлеров\\
Развивается и медицина (Пирогов -- один из основателей военно-полевой хирургии)\\
История: Карамзин -- <<История государства Российского>> (12 томов), Соловьев\\
Помимо университетов существовали научные объединения по интересам\\
1811 -- образование <<Общества любителей российской словесности>> при Московском университете\\
1845 -- Русское географическое общество\\
XIX век -- Золотой век русской литературы\\
Развивается и архитектура\\
Промышленная революция -- середина XIX века -- переход от ручного труда к фабрикам и мануфактурам\\
Мануфактуру от фабрики отличает постоянное использование машин\\
Начинают использоваться паровые машины\\
В XIX веке в США создается первый пароход (плавал на реке Гудзона)\\
Ричард Треветик -- сконструировал паровоз и предложил использовать его для перемещения угля из шахты\\
Далее эксперименты проводятся в США и Европе\\
Один из самых успешных экспериментов -- англичанин Джордж Стефенсон\\
1822-1825 -- строительство первой гражданской ЖД в центральной Великобритании\\
Ломомотив -- locomotion -- самодвижение
30 октября 1837 -- ЖД между Петербургом и Царским селом\\
Затем Москва -- Санкт-Петербург\\
Авангард промышленной революции -- текстильная промышленность (первые автоматические предельные машины)\\
Также появляются маленькие предельные машины (самая известная -- машина Зингера)\\
Появляются первые токарные станки\\
Происходит стандартизация резьб\\
Резко возрастает количество морских перевозок\\
Появляется гражданский винтовой пароход\\
Затем появляются первые броненосцы\\
Во Франции появляется <<Лаглуа>> -- первый броненосец на службе армии\\
1812-1820 -- появление в Лондоне первого уличного освещения (благодаря Уильяму Мердаку)\\
Из-за этого увеличился рабочий день(лол)\\

\end{document}
