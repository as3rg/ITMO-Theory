\documentclass[12pt]{article}
\usepackage{bbold}
\usepackage{amsfonts}
\usepackage{amsmath}
\usepackage{amssymb}
\usepackage{color}
\setlength{\columnseprule}{1pt}
\usepackage[utf8]{inputenc}
\usepackage[T2A]{fontenc}
\usepackage[english, russian]{babel}
\usepackage{graphicx}
\usepackage{hyperref}
\usepackage{mathdots}
\usepackage{xfrac}


\def\columnseprulecolor{\color{black}}

\graphicspath{ {./resources/} }


\usepackage{listings}
\usepackage{xcolor}
\definecolor{codegreen}{rgb}{0,0.6,0}
\definecolor{codegray}{rgb}{0.5,0.5,0.5}
\definecolor{codepurple}{rgb}{0.58,0,0.82}
\definecolor{backcolour}{rgb}{0.95,0.95,0.92}
\lstdefinestyle{mystyle}{
    backgroundcolor=\color{backcolour},   
    commentstyle=\color{codegreen},
    keywordstyle=\color{magenta},
    numberstyle=\tiny\color{codegray},
    stringstyle=\color{codepurple},
    basicstyle=\ttfamily\footnotesize,
    breakatwhitespace=false,         
    breaklines=true,                 
    captionpos=b,                    
    keepspaces=true,                 
    numbers=left,                    
    numbersep=5pt,                  
    showspaces=false,                
    showstringspaces=false,
    showtabs=false,                  
    tabsize=2
}

\lstset{extendedchars=\true}
\lstset{style=mystyle}

\newcommand\0{\mathbb{0}}
\newcommand{\eps}{\varepsilon}
\newcommand\overdot{\overset{\bullet}}
\DeclareMathOperator{\sign}{sign}
\DeclareMathOperator{\re}{Re}
\DeclareMathOperator{\im}{Im}
\DeclareMathOperator{\Arg}{Arg}
\DeclareMathOperator{\const}{const}
\DeclareMathOperator{\rg}{rg}
\DeclareMathOperator{\Span}{span}
\DeclareMathOperator{\alt}{alt}
\DeclareMathOperator{\Sim}{sim}
\DeclareMathOperator{\inv}{inv}
\DeclareMathOperator{\dist}{dist}
\newcommand\1{\mathbb{1}}
\newcommand\ul{\underline}
\renewcommand{\bf}{\textbf}
\renewcommand{\it}{\textit}
\newcommand\vect{\overrightarrow}
\newcommand{\nm}{\operatorname}
\DeclareMathOperator{\df}{d}
\DeclareMathOperator{\tr}{tr}
\newcommand{\bb}{\mathbb}
\newcommand{\lan}{\langle}
\newcommand{\ran}{\rangle}
\newcommand{\an}[2]{\lan #1, #2 \ran}
\newcommand{\fall}{\forall\,}
\newcommand{\ex}{\exists\,}
\newcommand{\lto}{\leftarrow}
\newcommand{\xlto}{\xleftarrow}
\newcommand{\rto}{\rightarrow}
\newcommand{\xrto}{\xrightarrow}
\newcommand{\uto}{\uparrow}
\newcommand{\dto}{\downarrow}
\newcommand{\lrto}{\leftrightarrow}
\newcommand{\llto}{\leftleftarrows}
\newcommand{\rrto}{\rightrightarrows}
\newcommand{\Lto}{\Leftarrow}
\newcommand{\Rto}{\Rightarrow}
\newcommand{\Uto}{\Uparrow}
\newcommand{\Dto}{\Downarrow}
\newcommand{\LRto}{\Leftrightarrow}
\newcommand{\Rset}{\bb{R}}
\newcommand{\Rex}{\overline{\bb{R}}}
\newcommand{\Cset}{\bb{C}}
\newcommand{\Nset}{\bb{N}}
\newcommand{\Qset}{\bb{Q}}
\newcommand{\Zset}{\bb{Z}}
\newcommand{\Bset}{\bb{B}}
\renewcommand{\ker}{\nm{Ker}}
\renewcommand{\span}{\nm{span}}
\newcommand{\Def}{\nm{def}}
\newcommand{\mc}{\mathcal}
\newcommand{\mcA}{\mc{A}}
\newcommand{\mcB}{\mc{B}}
\newcommand{\mcC}{\mc{C}}
\newcommand{\mcD}{\mc{D}}
\newcommand{\mcJ}{\mc{J}}
\newcommand{\mcT}{\mc{T}}
\newcommand{\us}{\underset}
\newcommand{\os}{\overset}
\newcommand{\ol}{\overline}
\newcommand{\ot}{\widetilde}
\newcommand{\vl}{\Biggr|}
\newcommand{\ub}[2]{\underbrace{#2}_{#1}}

\def\letus{%
    \mathord{\setbox0=\hbox{$\exists$}%
             \hbox{\kern 0.125\wd0%
                   \vbox to \ht0{%
                      \hrule width 0.75\wd0%
                      \vfill%
                      \hrule width 0.75\wd0}%
                   \vrule height \ht0%
                   \kern 0.125\wd0}%
           }%
}
\DeclareMathOperator*\dlim{\underline{lim}}
\DeclareMathOperator*\ulim{\overline{lim}}

\everymath{\displaystyle}

% Grath
\usepackage{tikz}
\usetikzlibrary{positioning}
\usetikzlibrary{decorations.pathmorphing}
\tikzset{snake/.style={decorate, decoration=snake}}
\tikzset{node/.style={circle, draw=black!60, fill=white!5, very thick, minimum size=7mm}}

\DeclareMathOperator{\Lin}{Lin}
\DeclareMathOperator{\Int}{Int}
\DeclareMathOperator{\grad}{grad}
\newcommand{\ppart}[2]{\frac{\partial #1}{\partial #2}}

\title{История науки и техники}
\author{Александр Сергеев}
\date{}
\begin{document}
\maketitle
\section{Лекция 05.10}
Технологии надо внедрять грамотно\\
Цели истории науки и техники:
\begin{itemize}
    \item Обеспечить постоянное повышение качества научно-технического потенциала человечества путем внедрения новых знаний
    \item Служить основой для интеграции естественно-научной, технической и гуманитарной форм знаний
    \item Постоянный ввод в оборот нового фактического и концептуального научно-технического знания\\
    (понять, почему знания появились в этот период)
    \item Создать фактологическую и концептуальную основу для моделирования будущего прогресса
\end{itemize}
Задачи:
\begin{itemize}
    \item Поиск, систематизация, анализ и обобщение историко-научных фактов
    \item Расширение базы источников для исследований
    \item Выявление и обоснование законов научно-технического развития
    \item Анализ роли и значения научно-технического развития в истории
    \item Совершенствование методологического инструментария
    \item Рассмотрение вопросов приоритета различных новшеств
\end{itemize}
Источники:
\begin{itemize}
    \item Письменные
    \item Визуальные
    \item Материальные
    \item Устные
    \item Этнографические
    \item Цифровые
    \item Археологические
\end{itemize}
\textbf{Определение}\\
Наука -- это непрерывно развивающаяся система знаний объективных законов природы, общества и мышления, получаемых и превращаемых в непосредственную производительную силу общества в результате социально-экономической деятельности\\
\textbf{Определение}\\
Техника -- это
\begin{enumerate}
    \item совокупность технических устройств (артефактов)
    \item совокупность различных видов технической деятельности по созданию технических артефактов
    \item совокупность технических знаний
\end{enumerate}
Можешь выбрать наиболее понравившееся или доказать эквивалентность\\
\textbf{Определение}\\
Технология -- совокупность процессов получения и обработки сырья и материалов\\
Существует несколько точек зрения на связь между наукой и технологией\\
\begin{itemize}
    \item Наука -- теоретическая часть, технология -- прикладная часть
    \item Развитие техники обгоняет развитие науки
    \item Развитие науки всегда обгоняет развитие техники
    \item Наука и техника -- автономные процессы, дополняющие друг друга
\end{itemize}
Наука декультурна и денациональна\\
Наукой может заниматься любой индивида\\
Философы науки:
\begin{itemize}
    \item Томас Кун 
    \item Жан Бодрийяр
    \item Маршалл Маклюэн
    \item Поль Вирильо
\end{itemize}
Все они имеют негативный взгляд на технологии\\
Люди становятся зависимыми от них\\
Они окружают себя <<протезами>>, усиливающими наши способности\\
Периоды историиЖ
\begin{itemize}
    \item Первобытное общество -- от возникновения человека до возникновения первых древних цивилизаций -- Древняя Индия, Древний Китай (6-5 век до нашей эры)
    \item Эпоха древности -- от возникновения первых цивилизаций до 476 г.н.э. (падение Западной Римской Империи)
    \item Средние века -- от 476 г. н.э. до эпохи Ренесансва (в каждой стране она случилась в разное время) (либо до 16 века, либо до Великой Французской революции в конце 18 века)
    \item Эпоха Ренесанса -- от момента, когда она началась (большинство считает, что в 16 веке) до Нового времени (модерна)
    \item Новое время -- от момента начала до 1914 (Первая Мировая Война)
    \item Новешая история -- от конца ПМВ до нашего времени
\end{itemize}
Этапы периодизации историии:
\begin{itemize}
    \item Пранаука (величайшее изобретение -- речь)
    \item Античная наука (Древняя Греция, Рим) -- период формирования первых научных теорий, трактатов
    \item Период средневековой магии (деградация, первые эксперименты)
    \item Классическая наука (с начала Ренесанса) -- Ньютон, Галилей, Коперник
    \item Постклассическая наука -- после мировых войн
\end{itemize}
Альтернативный взгляд:
\begin{itemize}
    \item Доклассическая наука 
    \item Классическая наука
    \item Постклассическая наука
\end{itemize}
\section{Лекция 19.10}
Никакая цивилизаций не могла быть построена без земледелия\\
Земледелие появляется в результате \textit{неолитической} революции\\
В результате этой революции народы смогли осесть, что стало началом цивилизации\\
Земледелие дает кратно больше продукта, чем собирательство и охота\\
Древняя Греция строилась по системе полисов (городов-государств), которые были независимы друг от друга\\
Экономика полисов была основана на рабстве, а свободные жители было освобождены от тяжелого труда\\
Античная наука делилась на следующие этапы
\begin{enumerate}
    \item Период единой науки о природе -- философия\\
    Самый известный деятель -- Аристотель (4 век до н.э.)
    \item Эллинистический период -- период распада единой науки. Начинают появляться отдельные направления\\
    Этот процесс начался еще до Аристотеля
    \item Попытка создания новой единой науки о природе\\
    Нужны межпредметные связи, из-за чего возникает попытка объединения отраслей
\end{enumerate}
За эти 3 периода возник Квадриум -- 4 аспекта греческой образованности: арифметика (число само по себе), геометрия (число на плоскости), музыка (число в звуке), астрономия (число в космосе)\\
(В порядке убывания крутости)
\subsection{Эллин}
Эллинизм начался после смерти Александр Македонского (330 г до н.э. -- 1 век до н.э.)\\
Т.к. после его смерти не осталось наследников, его владения было поделены между его сподвижниками\\
Эллин -- грек\\
В период экспансии империи Македонского в страны была принесена культура Греции\\
Знаменитый мусейон был основан в этот период в Египте\\
В этом мусейоне работал Евклид\\
Его наиболее известная работа -- <<Начала>> -- один из первых письменных источников, дошедших до наших дней\\
Первую редакцию смог восстановить Йохан Людвиг Гейберг \\
Другой известный ученый этого периода -- Демокрит\\
Им была придумана теория атомизма\\
Изначально это было про атомы, но потом идея была применена на все остальное\\
Демокрит считал, что атомы двигаются в одном направлении, от которого не могут отклоняться(позже из этого вытек фатализм)\\
Его ученик -- Эпикур -- наоборот считал, что атомы могут отклоняться. Хаос рождает хаос, а значит человек имеет волю\\
Также он считал, цель жизни человека -- удовлетворять свои потребности (но также он считал, что удовлетворять их надо в меру)\\
Следующий крупный ученый -- Аристарх Самосский -- крупный математик и основатель гелеоцентрической системы мира\\
Он работал в мусейоне и выдвинул теорию, что сферическая Земля вращается вокруг Солнца, а также попытался посчитать ее размеры, а также расстояние\\
Другой крупный ученый -- Ктесибий\\
Он один из отцов гидравлики и пневматики\\
Создал первое пневматическое ружье\\
Еще один известный ученый -- Архимед\\
Он родился на юге Сицилии\\
Погиб Архимед в ходе захвата Сиракузы Римом\\
Конец периода -- захват Греции Римом
\subsection{Римский этап}
1 век до н.э. -- 5 век н.э.\\
В этот период Рим -- центр науки\\
В ходе перенятия Римом культуры Греков он стал двуязычным -- языком общения стала латынь, а языком науки -- греческий\\
Римская империя подарила миру системы управления\\
Каждый правитель в Риме проходил сложную систему обучения -- грамматика (начальная школа), язык и 7 свободных искусств(грамматика, диалектика, риторика, музыка, арифметика, геометрия, астрономия)\\
В этот момент жили и работали Гален (медик), Герон, Витрувий (архитектор, автор формулы Витрувия: польза, прочность, красота), Птолемей(географ, астроном, автор <<Альмагеста -- Великого построения>>)\\
\end{document}
