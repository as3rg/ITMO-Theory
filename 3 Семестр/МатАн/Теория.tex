\documentclass[12pt]{article}
\usepackage{bbold}
\usepackage{amsfonts}
\usepackage{amsmath}
\usepackage{amssymb}
\usepackage{color}
\setlength{\columnseprule}{1pt}
\usepackage[utf8]{inputenc}
\usepackage[T2A]{fontenc}
\usepackage[english, russian]{babel}
\usepackage{graphicx}
\usepackage{hyperref}
\usepackage{mathdots}
\usepackage{xfrac}


\def\columnseprulecolor{\color{black}}

\graphicspath{ {./resources/} }


\usepackage{listings}
\usepackage{xcolor}
\definecolor{codegreen}{rgb}{0,0.6,0}
\definecolor{codegray}{rgb}{0.5,0.5,0.5}
\definecolor{codepurple}{rgb}{0.58,0,0.82}
\definecolor{backcolour}{rgb}{0.95,0.95,0.92}
\lstdefinestyle{mystyle}{
    backgroundcolor=\color{backcolour},   
    commentstyle=\color{codegreen},
    keywordstyle=\color{magenta},
    numberstyle=\tiny\color{codegray},
    stringstyle=\color{codepurple},
    basicstyle=\ttfamily\footnotesize,
    breakatwhitespace=false,         
    breaklines=true,                 
    captionpos=b,                    
    keepspaces=true,                 
    numbers=left,                    
    numbersep=5pt,                  
    showspaces=false,                
    showstringspaces=false,
    showtabs=false,                  
    tabsize=2
}

\lstset{extendedchars=\true}
\lstset{style=mystyle}

\newcommand\0{\mathbb{0}}
\newcommand{\eps}{\varepsilon}
\newcommand\overdot{\overset{\bullet}}
\DeclareMathOperator{\sign}{sign}
\DeclareMathOperator{\re}{Re}
\DeclareMathOperator{\im}{Im}
\DeclareMathOperator{\Arg}{Arg}
\DeclareMathOperator{\const}{const}
\DeclareMathOperator{\rg}{rg}
\DeclareMathOperator{\Span}{span}
\DeclareMathOperator{\alt}{alt}
\DeclareMathOperator{\Sim}{sim}
\DeclareMathOperator{\inv}{inv}
\DeclareMathOperator{\dist}{dist}
\newcommand\1{\mathbb{1}}
\newcommand\ul{\underline}
\renewcommand{\bf}{\textbf}
\renewcommand{\it}{\textit}
\newcommand\vect{\overrightarrow}
\newcommand{\nm}{\operatorname}
\DeclareMathOperator{\df}{d}
\DeclareMathOperator{\tr}{tr}
\newcommand{\bb}{\mathbb}
\newcommand{\lan}{\langle}
\newcommand{\ran}{\rangle}
\newcommand{\an}[2]{\lan #1, #2 \ran}
\newcommand{\fall}{\forall\,}
\newcommand{\ex}{\exists\,}
\newcommand{\lto}{\leftarrow}
\newcommand{\xlto}{\xleftarrow}
\newcommand{\rto}{\rightarrow}
\newcommand{\xrto}{\xrightarrow}
\newcommand{\uto}{\uparrow}
\newcommand{\dto}{\downarrow}
\newcommand{\lrto}{\leftrightarrow}
\newcommand{\llto}{\leftleftarrows}
\newcommand{\rrto}{\rightrightarrows}
\newcommand{\Lto}{\Leftarrow}
\newcommand{\Rto}{\Rightarrow}
\newcommand{\Uto}{\Uparrow}
\newcommand{\Dto}{\Downarrow}
\newcommand{\LRto}{\Leftrightarrow}
\newcommand{\Rset}{\bb{R}}
\newcommand{\Rex}{\overline{\bb{R}}}
\newcommand{\Cset}{\bb{C}}
\newcommand{\Nset}{\bb{N}}
\newcommand{\Qset}{\bb{Q}}
\newcommand{\Zset}{\bb{Z}}
\newcommand{\Bset}{\bb{B}}
\renewcommand{\ker}{\nm{Ker}}
\renewcommand{\span}{\nm{span}}
\newcommand{\Def}{\nm{def}}
\newcommand{\mc}{\mathcal}
\newcommand{\mcA}{\mc{A}}
\newcommand{\mcB}{\mc{B}}
\newcommand{\mcC}{\mc{C}}
\newcommand{\mcD}{\mc{D}}
\newcommand{\mcJ}{\mc{J}}
\newcommand{\mcT}{\mc{T}}
\newcommand{\us}{\underset}
\newcommand{\os}{\overset}
\newcommand{\ol}{\overline}
\newcommand{\ot}{\widetilde}
\newcommand{\vl}{\Biggr|}
\newcommand{\ub}[2]{\underbrace{#2}_{#1}}

\def\letus{%
    \mathord{\setbox0=\hbox{$\exists$}%
             \hbox{\kern 0.125\wd0%
                   \vbox to \ht0{%
                      \hrule width 0.75\wd0%
                      \vfill%
                      \hrule width 0.75\wd0}%
                   \vrule height \ht0%
                   \kern 0.125\wd0}%
           }%
}
\DeclareMathOperator*\dlim{\underline{lim}}
\DeclareMathOperator*\ulim{\overline{lim}}

\everymath{\displaystyle}

% Grath
\usepackage{tikz}
\usetikzlibrary{positioning}
\usetikzlibrary{decorations.pathmorphing}
\tikzset{snake/.style={decorate, decoration=snake}}
\tikzset{node/.style={circle, draw=black!60, fill=white!5, very thick, minimum size=7mm}}

\DeclareMathOperator{\Lin}{Lin}

\title{Математический анализ. Теория}
\author{Александр Сергеев}
\date{}
\begin{document}
\maketitle
\section{Friends and strangers graph}
Рассмотрим связный неориентированный граф $G$ из $n$ вершин\\
В $n-1$ вершину поставим нумерованные фишки. Одну вершину оставим свободной\\
Будем перемещать фишки по ребрам (перемещать фишку можно только в пустую вершину)\\
Для каких графов $G$ можно получить любое расположение фишек в графе из исходного?\\
\textbf{Теорема Уилсона}\\
Если граф $G$:
\begin{itemize}
    \item Нет точек сочленения
    \item Граф не двудольный
    \item Граф -- не цикл длины $n\geq 4$
    \item $G$ -- не следующий граф:\\
    \begin{tikzpicture}
        %Nodes
        \node[node] (a) at (0.5, 0) {};
        \node[node] (b) at (1.5, 0) {};
        \node[node] (c) at (0, 1) {};
        \node[node] (d) at (1, 1) {};
        \node[node] (e) at (2, 1) {};
        \node[node] (f) at (0.5, 2) {};
        \node[node] (g) at (1.5, 2) {};
        %Lines
        \draw[-] (a) -- (b);
        \draw[-] (a) -- (c);
        \draw[-] (c) -- (d);
        \draw[-] (d) -- (e);
        \draw[-] (e) -- (g);
        \draw[-] (b) -- (e);
        \draw[-] (c) -- (f);
        \draw[-] (f) -- (g);
    \end{tikzpicture}\\
\end{itemize}
то в таком графе возможно получить любое расположение фишек, иначе невозможно\\
\textbf{Доказательство необходимости}
\begin{itemize}
    \item Рассмотрим граф\\
    \begin{tikzpicture}
        %Nodes
        \node[node] (a) at (0, 0) {1};
        \node[node] (b) at (0, 2) {2};
        \node[node] (c) at (1, 1) {3};
        \node[node] (d) at (2, 0) {4};
        \node[node] (e) at (2, 2) {$\circ$};
        %Lines
        \draw[-] (a) -- (b);
        \draw[-] (a) -- (c);
        \draw[-] (c) -- (b);
        \draw[-] (d) -- (e);
        \draw[-] (c) -- (e);
        \draw[-] (c) -- (d);
    \end{tikzpicture}\\
    Заметим, что в таком графе 1 никогда не попадет в правую половину
    \item Рассмотрим двудольный граф\\
    \begin{tikzpicture}
        %Nodes
        \node[node] (a) at (0, 0) {1};
        \node[node] (b) at (1, 0) {2};
        \node[node] (c) at (2, 0) {3};
        \node[node] (d) at (0, 2) {4};
        \node[node] (e) at (1, 2) {5};
        \node[node] (f) at (2, 2) {$\circ$};
        %Lines
        \draw[-] (a) -- (d);
        \draw[-] (a) -- (e);
        \draw[-] (a) -- (f);
        \draw[-] (b) -- (d);
        \draw[-] (b) -- (e);
        \draw[-] (b) -- (f);
        \draw[-] (c) -- (d);
        \draw[-] (c) -- (e);
        \draw[-] (c) -- (f);
    \end{tikzpicture}\\
    Рассмотрим его как перестановку\\
    Поменяем 2 и $\circ$ местами\\
    \begin{tikzpicture}
        %Nodes
        \node[node] (a) at (0, 0) {1};
        \node[node, text=red] (b) at (1, 0) {$\circ$};
        \node[node] (c) at (2, 0) {3};
        \node[node] (d) at (0, 2) {4};
        \node[node] (e) at (1, 2) {5};
        \node[node, text=red] (f) at (2, 2) {2};
        %Lines
        \draw[-] (a) -- (d);
        \draw[-] (a) -- (e);
        \draw[-] (a) -- (f);
        \draw[-] (b) -- (d);
        \draw[-] (b) -- (e);
        \draw[-] (b) -- (f);
        \draw[-] (c) -- (d);
        \draw[-] (c) -- (e);
        \draw[-] (c) -- (f);
    \end{tikzpicture}\\
    Заметим, что величина (четность перестановки + номер доли, где находится $\circ$) не изменилась\\
    Отсюда из исходной перестановки невозможно получить перестановку\\
    \begin{tikzpicture}
        %Nodes
        \node[node, text=red] (a) at (0, 0) {2};
        \node[node, text=red] (b) at (1, 0) {1};
        \node[node] (c) at (2, 0) {3};
        \node[node] (d) at (0, 2) {4};
        \node[node] (e) at (1, 2) {5};
        \node[node] (f) at (2, 2) {$\circ$};
        %Lines
        \draw[-] (a) -- (d);
        \draw[-] (a) -- (e);
        \draw[-] (a) -- (f);
        \draw[-] (b) -- (d);
        \draw[-] (b) -- (e);
        \draw[-] (b) -- (f);
        \draw[-] (c) -- (d);
        \draw[-] (c) -- (e);
        \draw[-] (c) -- (f);
    \end{tikzpicture}
    \item Рассмотрим граф:\\
    \begin{tikzpicture}
        %Nodes
        \node[node] (a) at (0, 0) {1};
        \node[node] (b) at (1, 0) {2};
        \node[node] (c) at (1, 1) {3};
        \node[node] (d) at (0, 1) {$\circ$};
        %Lines
        \draw[-] (a) -- (b);
        \draw[-] (b) -- (c);
        \draw[-] (c) -- (d);
        \draw[-] (a) -- (d);
    \end{tikzpicture}\\
    В нем невозможно поменять порядок фишек\\
\end{itemize}
\textbf{Доказательство достаточности}\\
Отсутствует...\\
Рассмотрим обобщение предыдущей игры:\\
Пусть фишки могут меняться местами, если они соседи и "друзья"\\
Т.е. теперь у нас два графа: "географический" граф $X$ и граф дружбы $Y$\\
В прошлой задаче граф дружбы -- "звездочка" с $\circ$ в центре (т.е. $\circ$ дружит со всеми, а остальные не дружат между собой)\\
Построим граф $FS(X, Y)$ -- граф друзей и врагов\\
В нем будет $n!$ вершин\\
Каждая вершина графа -- биекция $\sigma: V(X) \rto V(Y)$\\
Получается, что $\sigma$ -- какой-то способ расположить фишки по вершинам\\
(т.к. $V(x)$ -- множество вершин, а $V(Y)$ -- множество фишек)\\
Вершины $\sigma$ и $\sigma'$ соединены ребром, если они отличаются одним "дружеским" обменом\\
Тогда из любой перестановки можно получить любую $\LRto$ граф связен\\
Из теоремы Уилсона: $FS(G, K_{1, n-1}), G$ -- из теоремы Уилсона, $K_{1,n-1}$ -- звездочка с вершиной $n$ в центре\\
Ответим на другой вопрос: Для каких $G$ граф $FS(G, C_n), C_n$ -- цикл длины $n$ -- связен\\
\textbf{Лемма}\\
Графы $FS(X, Y)$ и $FS(Y,X)$ -- изоморфны\\
Т.е. можно построить биекцию $u \os{\theta}{\lrto} u'$ такую, что ребра $uv$ и $\theta(u)\theta(v)$ существуют одновременно\\
\textbf{Доказательство}\\
Построим биекцию $\sigma \in FS(X,Y) \lrto \sigma^{-1} \in FS(Y, X)$\\
Теперь рассмотрим граф $FS(C_n, G)$\\
Теперь рассмотрим следующую ситуацию:\\
По кругу стоят $3n$ человек: $n$ семей вида папа-мама-ребенок\\
Ребенок не может меняться со своими родителями, все остальные могут меняться между собой\\
//todo т.к. оффтоп, мне лень дальше конспектировать\\
См. первую лекцию, начиная с 30 минуты\\
\section{Функции и отображения в $\Rset^m$}
\subsection{Линейные отображения}
\textbf{Определение}\\
$\nm{Lin}(X, Y)$ -- множество линейных отображений из $X$ в $Y$ (линейные пространства)\\
$\nm{Lin}(X, Y)$ -- линейное пространство\\
\textbf{Обозначение}\\
Пусть $A \in \Lin(\Rset^m, \Rset^n)$\\
$\|A\| := \sup_{|x| = 1} |A(x)|$\\
\textbf{Замечание 1}\\
$\|A\| \in \Rset$\\
\textbf{Доказательство}\\
Было доказано: $|Ax| \leq C_a |x|, C_A = \sqrt{\sum a_{ij}^2}$\\
\textbf{Замечание 2}\\
Из теоремы Вейерштрасса $\sup \lrto \max$ в конечномерном случае\\
\textbf{Замечание 3}\\
$\fall x \in \Rset^m\ |Ax| \leq \|A\||x|$\\
\textbf{Доказательство}\\
Для $x=0$ очевидно\\
$\ot{x} := \frac{x}{|x|}$\\
$|A\ot{x}| \leq \|A\|$\\
\textbf{Замечание 4}\\
Если $\ex C: \fall x\ |Ax| \leq C|x|$, то $\|A\| \leq C$\\
\textbf{Пример}
\begin{itemize}
    \item $m=n=1$\\
    $A$ -- линейное отображение: $x \mapsto ax$\\
    $\|A\| = |a|$
    \item $m=1, n$ -- любое\\
    $A:\Rset\rto \Rset^n$\\
    Тогда $\ex \ol v \in \Rset^n Ax = x\ol{v}$\\
    $\|A\| = |\ol v|$
    \item $n=1, m$ -- любое\\
    $A: \Rset^m \rto \Rset$\\
    Тогда $\ex l \in \Rset^m: Ax=\an xl$\\
    $\|A\| = |l|$
    \item $m,n$ -- любые\\
    $A=(a_{ij})$\\
    $x \mapsto Ax$\\
    $\|A\|$ так легко не считается((((
\end{itemize}
\textbf{Лемма}\\
Пусть $X, Y$ -- нормированные линейное пространство\\
$A \in \Lin(X, Y)$\\
Тогда следующие утверждения эквивалентны:\\
\begin{enumerate}
    \item $A$ -- ограничен, т.е. $\|A\| < +\infty$
    \item $A$ -- непрерывно в $\0 \in X$
    \item $A$ -- непрерывно на $X$
    \item $A$ -- равномерно непрерывное ($\fall \eps > 0\ \ex \delta > 0: \fall x_1, x_2: |x_1-x_2| < \delta\ |Ax_1 - Ax_2| < \eps$)
\end{enumerate}
\textbf{Доказательство}\\
$4 \Rto 3 \Rto 2$ -- очевидно\\
Докажем $2 \Rto 1$\\
Возьмем $\eps = 1$\\
$\ex \delta > 0:\ \fall x: |x| < \delta\ |Ax| < 1$\\
Возьмем $|x| = 1$\\
$Ax| = \frac1\delta|A(\delta x)| < \frac1\delta$\\
Тогда $\|A\| \leq \frac1\delta$\\
Докажем $1 \Rto 4$\\
$\fall \eps > 0\ \ex \delta = \frac\eps{\|a\|}:\ \fall x_1, x_2: |x_1-x_2| < \delta$\\
$|Ax_1-Ax_2| = |A(x_1-x_2)| \leq \|A\||x_1-x_2| < \eps$\\
\textbf{Теорема о пространстве линейных отображений}
\begin{itemize}
    \item $\|\cdot \|$ -- норма в $\Lin(X, Y), X, Y$ -- конечномерные нормированные пространства\\
    Т.е.\begin{enumerate}
        \item $\|A\| \geq 0, \|A\| = 0 \LRto A = 0$
        \item $\fall \alpha \in \Rset\ \|\alpha A\| = |\alpha| \|A\|$
        \item $\|A + B\| \leq \|A\|+\|B\|$
    \end{enumerate}
    \item $\|BA\|\leq \|B\|\cdot\|A\|$
\end{itemize}
\textbf{Доказательство}\\
$\|A\| \geq 0$ -- тривиально\\
$\|A\|=\sup_{|x|=1} \|Ax\| = 0 \Rto A = 0$\\
$\|\alpha A\| = |\alpha| \|A\|$\\
$|(A+B)x| \leq |Ax| + |Bx| \leq \ub{C}{(\|A\|+\|B\|)}|x| \Rto \|A+B\| \leq C = \|A\|+\|B\|$\\
$|BAx|\leq \|B\||Ax| \leq \|B\|\|A\||x|$\\
\textbf{Замечание}\\
В $\Lin(X, Y)$\\
$\|A\|=\sup_{|x|=1} |Ax| = \sup_{|x|\leq 1} |Ax| = \sup_{|x|<1}|Ax| = \sup_{x\neq 0} \frac{|Ax|}{|x|} = \inf\{C\in \Rset: \fall x \in X\ |Ax| \leq C|x|\}$\\
\textbf{Теорема Лагранжа (для отображений)}\\
$F: D \subset \Rset^m \rto \Rset^n$ -- дифференцируема в $D$ -- открытом\\
$a, b \in D, [a,b] \subset D$\\
Тогда $\ex c \in [a,b]$, т.е. $\ex \theta \in [0,1]: c = a+\theta(b-a):\ |F(b)-F(a)| \leq \|F'(c)\||b-a|$\\
\textbf{Доказательство}\\
$f(t) = F(a+t(b-a)), t \in [0,1]$\\
$f'(t) = F'(a+t(b-a))(b-a)$\\
$|F(b)-F(a)|=|f(1)-f(0)| \leq |f'(\theta)|(1-0) = |F'(a+\theta(b-a))(b-a)| \leq \|F'(a+\theta(b-a))\||b-a|$\\
\textbf{Лемма}\\
$B\in \Lin(\Rset^m, \Rset^m)$\\
$\ex c > 0: \fall x\ |Bx| \geq c|x|$\\
Тогда $B$ -- обратим и $\|B^{-1}\| \leq \frac1c$\\
\textbf{Доказательство}\\
Обратимость очевидна, т.к. $\ker B = \{\0\}$. Тогда $\ex B^{-1}$\\
$|\ub{x}{B^{-1}y}| \leq \frac1c |BB^{-1}y| = \frac1c|y|$\\
\textbf{Замечание}\\
$\Omega_m$ -- множество обратимых линейных операторов $\Rset^m \rto\Rset^m$\\
$|x| = |A^{-1}Ax| \leq \|A^{-1}\||Ax|$\\
Т.е. $|Ax| \geq \frac1{\|A^{-1}\|}|x|$\\
\textbf{Теорема об обратимости линейного операторого, близкого к обратимому}\\
$L \in \Omega_m$ -- обратимый линейный оператор $\Rset^m \rto \Rset^m$\\
$M \in \Lin(\Rset^m, \Rset^m): \|L-M\| \leq \frac1{\|L^{-1}\|}$ -- $M$ -- близкий к $L$\\
Тогда
\begin{itemize}
    \item $M \in \Omega_m$ -- т.е. $\Omega_m$ -- открытое
    \item $\|M^{-1}\| \leq \frac1{\|L^{-1}\|^{-1} - \|L-M\|}$
    \item $\|L^{-1}-M^{-1}\| \leq \frac{\|L^{-1}\|}{\|L^{-1}\|^{-1}-\|L-M\|}\|L-M\|$
\end{itemize}
\textbf{Доказательство 1 и 2}\\
$Mx = Lx + (M-L)x$\\
$|Mx| \geq |Lx| - |(M-L)x| \geq \frac1{\|L^{-1}\|}|x| -\|M-L\||x| = (\|L^{-1}\|^{-1}-\|M-L\|)|x|$\\ 
1 и 2 следуют из леммы\\
\textbf{Доказательство 3}\\
$\frac1l - \frac1m = \frac{m-l}{lm}$\\
Аналогично $L^{-1}-M^{-1} = M^{-1}(M-L)L^{-1}$\\
$\|L^{-1}-M^{-1}\| = \|M^{-1}(M-L)L^{-1}\| \leq \|M^{-1}\|\|M-L\|\|L^{-1}\| \leq $ из пункта 2\\
\textbf{Следствие}\\
Отображение $L \mapsto L^{-1}$ задано на $\Omega_m$ и непрерывно\\
//todo лекции 3 и 4
\end{document}
