\documentclass[12pt]{article}
\usepackage{bbold}
\usepackage{amsfonts}
\usepackage{amsmath}
\usepackage{amssymb}
\usepackage{color}
\setlength{\columnseprule}{1pt}
\usepackage[utf8]{inputenc}
\usepackage[T2A]{fontenc}
\usepackage[english, russian]{babel}
\usepackage{graphicx}
\usepackage{hyperref}
\usepackage{mathdots}
\usepackage{xfrac}


\def\columnseprulecolor{\color{black}}

\graphicspath{ {./resources/} }


\usepackage{listings}
\usepackage{xcolor}
\definecolor{codegreen}{rgb}{0,0.6,0}
\definecolor{codegray}{rgb}{0.5,0.5,0.5}
\definecolor{codepurple}{rgb}{0.58,0,0.82}
\definecolor{backcolour}{rgb}{0.95,0.95,0.92}
\lstdefinestyle{mystyle}{
    backgroundcolor=\color{backcolour},   
    commentstyle=\color{codegreen},
    keywordstyle=\color{magenta},
    numberstyle=\tiny\color{codegray},
    stringstyle=\color{codepurple},
    basicstyle=\ttfamily\footnotesize,
    breakatwhitespace=false,         
    breaklines=true,                 
    captionpos=b,                    
    keepspaces=true,                 
    numbers=left,                    
    numbersep=5pt,                  
    showspaces=false,                
    showstringspaces=false,
    showtabs=false,                  
    tabsize=2
}

\lstset{extendedchars=\true}
\lstset{style=mystyle}

\newcommand\0{\mathbb{0}}
\newcommand{\eps}{\varepsilon}
\newcommand\overdot{\overset{\bullet}}
\DeclareMathOperator{\sign}{sign}
\DeclareMathOperator{\re}{Re}
\DeclareMathOperator{\im}{Im}
\DeclareMathOperator{\Arg}{Arg}
\DeclareMathOperator{\const}{const}
\DeclareMathOperator{\rg}{rg}
\DeclareMathOperator{\Span}{span}
\DeclareMathOperator{\alt}{alt}
\DeclareMathOperator{\Sim}{sim}
\DeclareMathOperator{\inv}{inv}
\DeclareMathOperator{\dist}{dist}
\newcommand\1{\mathbb{1}}
\newcommand\ul{\underline}
\renewcommand{\bf}{\textbf}
\renewcommand{\it}{\textit}
\newcommand\vect{\overrightarrow}
\newcommand{\nm}{\operatorname}
\DeclareMathOperator{\df}{d}
\DeclareMathOperator{\tr}{tr}
\newcommand{\bb}{\mathbb}
\newcommand{\lan}{\langle}
\newcommand{\ran}{\rangle}
\newcommand{\an}[2]{\lan #1, #2 \ran}
\newcommand{\fall}{\forall\,}
\newcommand{\ex}{\exists\,}
\newcommand{\lto}{\leftarrow}
\newcommand{\xlto}{\xleftarrow}
\newcommand{\rto}{\rightarrow}
\newcommand{\xrto}{\xrightarrow}
\newcommand{\uto}{\uparrow}
\newcommand{\dto}{\downarrow}
\newcommand{\lrto}{\leftrightarrow}
\newcommand{\llto}{\leftleftarrows}
\newcommand{\rrto}{\rightrightarrows}
\newcommand{\Lto}{\Leftarrow}
\newcommand{\Rto}{\Rightarrow}
\newcommand{\Uto}{\Uparrow}
\newcommand{\Dto}{\Downarrow}
\newcommand{\LRto}{\Leftrightarrow}
\newcommand{\Rset}{\bb{R}}
\newcommand{\Rex}{\overline{\bb{R}}}
\newcommand{\Cset}{\bb{C}}
\newcommand{\Nset}{\bb{N}}
\newcommand{\Qset}{\bb{Q}}
\newcommand{\Zset}{\bb{Z}}
\newcommand{\Bset}{\bb{B}}
\renewcommand{\ker}{\nm{Ker}}
\renewcommand{\span}{\nm{span}}
\newcommand{\Def}{\nm{def}}
\newcommand{\mc}{\mathcal}
\newcommand{\mcA}{\mc{A}}
\newcommand{\mcB}{\mc{B}}
\newcommand{\mcC}{\mc{C}}
\newcommand{\mcD}{\mc{D}}
\newcommand{\mcJ}{\mc{J}}
\newcommand{\mcT}{\mc{T}}
\newcommand{\us}{\underset}
\newcommand{\os}{\overset}
\newcommand{\ol}{\overline}
\newcommand{\ot}{\widetilde}
\newcommand{\vl}{\Biggr|}
\newcommand{\ub}[2]{\underbrace{#2}_{#1}}

\def\letus{%
    \mathord{\setbox0=\hbox{$\exists$}%
             \hbox{\kern 0.125\wd0%
                   \vbox to \ht0{%
                      \hrule width 0.75\wd0%
                      \vfill%
                      \hrule width 0.75\wd0}%
                   \vrule height \ht0%
                   \kern 0.125\wd0}%
           }%
}
\DeclareMathOperator*\dlim{\underline{lim}}
\DeclareMathOperator*\ulim{\overline{lim}}

\everymath{\displaystyle}

% Grath
\usepackage{tikz}
\usetikzlibrary{positioning}
\usetikzlibrary{decorations.pathmorphing}
\tikzset{snake/.style={decorate, decoration=snake}}
\tikzset{node/.style={circle, draw=black!60, fill=white!5, very thick, minimum size=7mm}}

\DeclareMathOperator{\Lin}{Lin}
\DeclareMathOperator{\Int}{Int}
\DeclareMathOperator{\grad}{grad}
\newcommand{\ppart}[2]{\frac{\partial #1}{\partial #2}}

\title{Математический анализ. Теория}
\author{Александр Сергеев}
\date{}
\begin{document}
\maketitle
\section{Friends and strangers graph}
Рассмотрим связный неориентированный граф $G$ из $n$ вершин\\
В $n-1$ вершину поставим нумерованные фишки. Одну вершину оставим свободной\\
Будем перемещать фишки по ребрам (перемещать фишку можно только в пустую вершину)\\
Для каких графов $G$ можно получить любое расположение фишек в графе из исходного?\\
\textbf{Теорема Уилсона}\\
Если граф $G$:
\begin{itemize}
    \item Нет точек сочленения
    \item Граф не двудольный
    \item Граф -- не цикл длины $n\geq 4$
    \item $G$ -- не следующий граф:\\
    \begin{tikzpicture}
        %Nodes
        \node[node] (a) at (0.5, 0) {};
        \node[node] (b) at (1.5, 0) {};
        \node[node] (c) at (0, 1) {};
        \node[node] (d) at (1, 1) {};
        \node[node] (e) at (2, 1) {};
        \node[node] (f) at (0.5, 2) {};
        \node[node] (g) at (1.5, 2) {};
        %Lines
        \draw[-] (a) -- (b);
        \draw[-] (a) -- (c);
        \draw[-] (c) -- (d);
        \draw[-] (d) -- (e);
        \draw[-] (e) -- (g);
        \draw[-] (b) -- (e);
        \draw[-] (c) -- (f);
        \draw[-] (f) -- (g);
    \end{tikzpicture}\\
\end{itemize}
то в таком графе возможно получить любое расположение фишек, иначе невозможно\\
\textbf{Доказательство необходимости}
\begin{itemize}
    \item Рассмотрим граф\\
    \begin{tikzpicture}
        %Nodes
        \node[node] (a) at (0, 0) {1};
        \node[node] (b) at (0, 2) {2};
        \node[node] (c) at (1, 1) {3};
        \node[node] (d) at (2, 0) {4};
        \node[node] (e) at (2, 2) {$\circ$};
        %Lines
        \draw[-] (a) -- (b);
        \draw[-] (a) -- (c);
        \draw[-] (c) -- (b);
        \draw[-] (d) -- (e);
        \draw[-] (c) -- (e);
        \draw[-] (c) -- (d);
    \end{tikzpicture}\\
    Заметим, что в таком графе 1 никогда не попадет в правую половину
    \item Рассмотрим двудольный граф\\
    \begin{tikzpicture}
        %Nodes
        \node[node] (a) at (0, 0) {1};
        \node[node] (b) at (1, 0) {2};
        \node[node] (c) at (2, 0) {3};
        \node[node] (d) at (0, 2) {4};
        \node[node] (e) at (1, 2) {5};
        \node[node] (f) at (2, 2) {$\circ$};
        %Lines
        \draw[-] (a) -- (d);
        \draw[-] (a) -- (e);
        \draw[-] (a) -- (f);
        \draw[-] (b) -- (d);
        \draw[-] (b) -- (e);
        \draw[-] (b) -- (f);
        \draw[-] (c) -- (d);
        \draw[-] (c) -- (e);
        \draw[-] (c) -- (f);
    \end{tikzpicture}\\
    Рассмотрим его как перестановку\\
    Поменяем 2 и $\circ$ местами\\
    \begin{tikzpicture}
        %Nodes
        \node[node] (a) at (0, 0) {1};
        \node[node, text=red] (b) at (1, 0) {$\circ$};
        \node[node] (c) at (2, 0) {3};
        \node[node] (d) at (0, 2) {4};
        \node[node] (e) at (1, 2) {5};
        \node[node, text=red] (f) at (2, 2) {2};
        %Lines
        \draw[-] (a) -- (d);
        \draw[-] (a) -- (e);
        \draw[-] (a) -- (f);
        \draw[-] (b) -- (d);
        \draw[-] (b) -- (e);
        \draw[-] (b) -- (f);
        \draw[-] (c) -- (d);
        \draw[-] (c) -- (e);
        \draw[-] (c) -- (f);
    \end{tikzpicture}\\
    Заметим, что величина (четность перестановки + номер доли, где находится $\circ$) не изменилась\\
    Отсюда из исходной перестановки невозможно получить перестановку\\
    \begin{tikzpicture}
        %Nodes
        \node[node, text=red] (a) at (0, 0) {2};
        \node[node, text=red] (b) at (1, 0) {1};
        \node[node] (c) at (2, 0) {3};
        \node[node] (d) at (0, 2) {4};
        \node[node] (e) at (1, 2) {5};
        \node[node] (f) at (2, 2) {$\circ$};
        %Lines
        \draw[-] (a) -- (d);
        \draw[-] (a) -- (e);
        \draw[-] (a) -- (f);
        \draw[-] (b) -- (d);
        \draw[-] (b) -- (e);
        \draw[-] (b) -- (f);
        \draw[-] (c) -- (d);
        \draw[-] (c) -- (e);
        \draw[-] (c) -- (f);
    \end{tikzpicture}
    \item Рассмотрим граф:\\
    \begin{tikzpicture}
        %Nodes
        \node[node] (a) at (0, 0) {1};
        \node[node] (b) at (1, 0) {2};
        \node[node] (c) at (1, 1) {3};
        \node[node] (d) at (0, 1) {$\circ$};
        %Lines
        \draw[-] (a) -- (b);
        \draw[-] (b) -- (c);
        \draw[-] (c) -- (d);
        \draw[-] (a) -- (d);
    \end{tikzpicture}\\
    В нем невозможно поменять порядок фишек\\
\end{itemize}
\textbf{Доказательство достаточности}\\
Отсутствует...\\
Рассмотрим обобщение предыдущей игры:\\
Пусть фишки могут меняться местами, если они соседи и "друзья"\\
Т.е. теперь у нас два графа: "географический" граф $X$ и граф дружбы $Y$\\
В прошлой задаче граф дружбы -- "звездочка" с $\circ$ в центре (т.е. $\circ$ дружит со всеми, а остальные не дружат между собой)\\
Построим граф $FS(X, Y)$ -- граф друзей и врагов\\
В нем будет $n!$ вершин\\
Каждая вершина графа -- биекция $\sigma: V(X) \rto V(Y)$\\
Получается, что $\sigma$ -- какой-то способ расположить фишки по вершинам\\
(т.к. $V(x)$ -- множество вершин, а $V(Y)$ -- множество фишек)\\
Вершины $\sigma$ и $\sigma'$ соединены ребром, если они отличаются одним "дружеским" обменом\\
Тогда из любой перестановки можно получить любую $\LRto$ граф связен\\
Из теоремы Уилсона: $FS(G, K_{1, n-1}), G$ -- из теоремы Уилсона, $K_{1,n-1}$ -- звездочка с вершиной $n$ в центре\\
Ответим на другой вопрос: Для каких $G$ граф $FS(G, C_n), C_n$ -- цикл длины $n$ -- связен\\
\textbf{Лемма}\\
Графы $FS(X, Y)$ и $FS(Y,X)$ -- изоморфны\\
Т.е. можно построить биекцию $u \os{\theta}{\lrto} u'$ такую, что ребра $uv$ и $\theta(u)\theta(v)$ существуют одновременно\\
\textbf{Доказательство}\\
Построим биекцию $\sigma \in FS(X,Y) \lrto \sigma^{-1} \in FS(Y, X)$\\
Теперь рассмотрим граф $FS(C_n, G)$\\
Теперь рассмотрим следующую ситуацию:\\
По кругу стоят $3n$ человек: $n$ семей вида папа-мама-ребенок\\
Ребенок не может меняться со своими родителями, все остальные могут меняться между собой\\
//todo т.к. оффтоп, мне лень дальше конспектировать\\
См. первую лекцию, начиная с 30 минуты\\
\section{Функции и отображения в $\Rset^m$}
\subsection{Линейные отображения}
\textbf{Определение}\\
$\Lin(X, Y)$ -- множество линейных отображений из $X$ в $Y$ (линейные пространства)\\
$\Lin(X, Y)$ -- линейное пространство\\
\textbf{Обозначение}\\
Пусть $A \in \Lin(\Rset^m, \Rset^n)$\\
$\|A\| := \sup_{|x| = 1} |A(x)|$\\
\textbf{Замечание 1}\\
$\|A\| \in \Rset$\\
\textbf{Доказательство}\\
Было доказано: $|Ax| \leq C_a |x|, C_A = \sqrt{\sum a_{ij}^2}$\\
\textbf{Замечание 2}\\
Из теоремы Вейерштрасса $\sup \lrto \max$ в конечномерном случае\\
\textbf{Замечание 3}\\
$\fall x \in \Rset^m\ |Ax| \leq \|A\||x|$\\
\textbf{Доказательство}\\
Для $x=0$ очевидно\\
$\ot{x} := \frac{x}{|x|}$\\
$|A\ot{x}| \leq \|A\|$\\
\textbf{Замечание 4}\\
Если $\ex C: \fall x\ |Ax| \leq C|x|$, то $\|A\| \leq C$\\
\textbf{Пример}
\begin{itemize}
    \item $m=n=1$\\
    $A$ -- линейное отображение: $x \mapsto ax$\\
    $\|A\| = |a|$
    \item $m=1, n$ -- любое\\
    $A:\Rset\rto \Rset^n$\\
    Тогда $\ex \ol v \in \Rset^n Ax = x\ol{v}$\\
    $\|A\| = |\ol v|$
    \item $n=1, m$ -- любое\\
    $A: \Rset^m \rto \Rset$\\
    Тогда $\ex l \in \Rset^m: Ax=\an xl$\\
    $\|A\| = |l|$
    \item $m,n$ -- любые\\
    $A=(a_{ij})$\\
    $x \mapsto Ax$\\
    $\|A\|$ так легко не считается((((
\end{itemize}
\textbf{Лемма}\\
Пусть $X, Y$ -- нормированные линейное пространство\\
$A \in \Lin(X, Y)$\\
Тогда следующие утверждения эквивалентны:
\begin{enumerate}
    \item $A$ -- ограничен, т.е. $\|A\| < +\infty$
    \item $A$ -- непрерывно в $\0 \in X$
    \item $A$ -- непрерывно на $X$
    \item $A$ -- равномерно непрерывное ($\fall \eps > 0\ \ex \delta > 0: \fall x_1, x_2: |x_1-x_2| < \delta\ |Ax_1 - Ax_2| < \eps$)
\end{enumerate}
\textbf{Доказательство}\\
$4 \Rto 3 \Rto 2$ -- очевидно\\
Докажем $2 \Rto 1$\\
Возьмем $\eps = 1$\\
$\ex \delta > 0:\ \fall x: |x| < \delta\ |Ax| < 1$\\
Возьмем $|x| = 1$\\
$Ax| = \frac1\delta|A(\delta x)| < \frac1\delta$\\
Тогда $\|A\| \leq \frac1\delta$\\
Докажем $1 \Rto 4$\\
$\fall \eps > 0\ \ex \delta = \frac\eps{\|a\|}:\ \fall x_1, x_2: |x_1-x_2| < \delta$\\
$|Ax_1-Ax_2| = |A(x_1-x_2)| \leq \|A\||x_1-x_2| < \eps$\\
\textbf{Теорема о пространстве линейных отображений}
\begin{itemize}
    \item $\|\cdot \|$ -- норма в $\Lin(X, Y), X, Y$ -- конечномерные нормированные пространства\\
    Т.е.\begin{enumerate}
        \item $\|A\| \geq 0, \|A\| = 0 \LRto A = 0$
        \item $\fall \alpha \in \Rset\ \|\alpha A\| = |\alpha| \|A\|$
        \item $\|A + B\| \leq \|A\|+\|B\|$
    \end{enumerate}
    \item $\|BA\|\leq \|B\|\cdot\|A\|$
\end{itemize}
\textbf{Доказательство}\\
$\|A\| \geq 0$ -- тривиально\\
$\|A\|=\sup_{|x|=1} \|Ax\| = 0 \Rto A = 0$\\
$\|\alpha A\| = |\alpha| \|A\|$\\
$|(A+B)x| \leq |Ax| + |Bx| \leq \ub{C}{(\|A\|+\|B\|)}|x| \Rto \|A+B\| \leq C = \|A\|+\|B\|$\\
$|BAx|\leq \|B\||Ax| \leq \|B\|\|A\||x|$\\
\textbf{Замечание}\\
В $\Lin(X, Y)$\\
$\|A\|=\sup_{|x|=1} |Ax| = \sup_{|x|\leq 1} |Ax| = \sup_{|x|<1}|Ax| = \sup_{x\neq 0} \frac{|Ax|}{|x|} = \inf\{C\in \Rset: \fall x \in X\ |Ax| \leq C|x|\}$\\
\textbf{Теорема Лагранжа (для отображений)}\\
$F: D \subset \Rset^m \rto \Rset^n$ -- дифференцируема в $D$ -- открытом\\
$a, b \in D, [a,b] \subset D$\\
Тогда $\ex c \in [a,b]$, т.е. $\ex \theta \in [0,1]: c = a+\theta(b-a):\ |F(b)-F(a)| \leq \|F'(c)\||b-a|$\\
\textbf{Доказательство}\\
$f(t) = F(a+t(b-a)), t \in [0,1]$\\
$f'(t) = F'(a+t(b-a))(b-a)$\\
$|F(b)-F(a)|=|f(1)-f(0)| \leq |f'(\theta)|(1-0) = |F'(a+\theta(b-a))(b-a)| \leq \|F'(a+\theta(b-a))\||b-a|$\\
\textbf{Лемма}\\
$B\in \Lin(\Rset^m, \Rset^m)$\\
$\ex c > 0: \fall x\ |Bx| \geq c|x|$\\
Тогда $B$ -- обратим и $\|B^{-1}\| \leq \frac1c$\\
\textbf{Доказательство}\\
Обратимость очевидна, т.к. $\ker B = \{\0\}$. Тогда $\ex B^{-1}$\\
$|\ub{x}{B^{-1}y}| \leq \frac1c |BB^{-1}y| = \frac1c|y|$\\
\textbf{Замечание}\\
$\Omega_m$ -- множество обратимых линейных операторов $\Rset^m \rto\Rset^m$\\
$|x| = |A^{-1}Ax| \leq \|A^{-1}\||Ax|$\\
Т.е. $|Ax| \geq \frac1{\|A^{-1}\|}|x|$\\
\textbf{Теорема об обратимости линейного операторого, близкого к обратимому}\\
$L \in \Omega_m$ -- обратимый линейный оператор $\Rset^m \rto \Rset^m$\\
$M \in \Lin(\Rset^m, \Rset^m): \|L-M\| \leq \frac1{\|L^{-1}\|}$ -- $M$ -- близкий к $L$\\
Тогда
\begin{itemize}
    \item $M \in \Omega_m$ -- т.е. $\Omega_m$ -- открытое
    \item $\|M^{-1}\| \leq \frac1{\|L^{-1}\|^{-1} - \|L-M\|}$
    \item $\|L^{-1}-M^{-1}\| \leq \frac{\|L^{-1}\|}{\|L^{-1}\|^{-1}-\|L-M\|}\|L-M\|$
\end{itemize}
\textbf{Доказательство 1 и 2}\\
$Mx = Lx + (M-L)x$\\
$|Mx| \geq |Lx| - |(M-L)x| \geq \frac1{\|L^{-1}\|}|x| -\|M-L\||x| = (\|L^{-1}\|^{-1}-\|M-L\|)|x|$\\ 
1 и 2 следуют из леммы\\
\textbf{Доказательство 3}\\
$\frac1l - \frac1m = \frac{m-l}{lm}$\\
Аналогично $L^{-1}-M^{-1} = M^{-1}(M-L)L^{-1}$\\
$\|L^{-1}-M^{-1}\| = \|M^{-1}(M-L)L^{-1}\| \leq \|M^{-1}\|\|M-L\|\|L^{-1}\| \leq $ из пункта 2\\
\textbf{Следствие (непрерывность вычисления обратного оператора)}\\
Отображение $L \mapsto L^{-1}$, заданное на $\Omega_m$, непрерывно\\
\textbf{Доказательство}\\
Доказательство по Гейне\\
Рассмотрим последовательность операторов $B_k: B_k \rto L$\\
Проверим, что $B_k^{-1} \rto L^{-1}$\\
Н.С.Н.М. $\|B_k-L\|< \frac1{\|L^{-1}\|}$\\
$\|B_k^{-1}-L^{-1}\| \leq \ub{\text{огр}}{\frac{\|L^{-1}\|}{\frac1{\|L^{-1}\|} - \ub{\rto 0}{\|L-B_k\|}}}\ub{\rto 0}{\|L-B_k\|} \rto 0$\\
\textbf{Теорема (о непрерывно дифференцируемых отображениях)}\\
$F: \ub{\text{откр}}D\subset \Rset^m \rto \Rset^l$, дифф. на $D$\\
$F': D \rto \Lin(\Rset^m, \Rset^l)$\\
Тогда $1 \lrto 2$
\begin{enumerate}
    \item $F\in C^1(D)$ (все частные производные непрерывны на $D$)
    \item $F': D \rto \Lin(\Rset^m, \Rset^l)$ -- непрерывно на $D$\\
    $\fall x \in D\ \fall \eps > 0\ \ex \delta > 0: \fall \ot x:|x-\ot x| < \delta\ \|F'(x)-F'(\ot x)\| < \eps$\\
\end{enumerate}
\textbf{Доказательство $1 \rto 2$}\\
Пусть $F\in C^1(D)$\\
$\fall i,j\ \fall x \in D\ \fall \eps > 0\ \ex \delta > 0: \fall \ot x:|x-\ot x| < \delta\ |\frac{\partial f_i}{\partial x_j}(x)-\frac{\partial f_i}{\partial x_j}(\ot x)| < \frac\eps{\sqrt{mn}}$\\
Тогда $\|F'(x) - F'(\ot x)\| \leq \sqrt{\sum_{ij} (\frac{\partial f_i}{\partial x_j}(x)-\frac{\partial f_i}{\partial x_j}(\ot x))^2} \leq \eps$\\
\textbf{Доказательство $2 \rto 1$}\\
Пусть $\fall x \in D\ \fall \eps > 0\ \ex \delta > 0: \fall \ot x:|x-\ot x| < \delta\ \|F'(x)-F'(\ot x)\| < \eps$\\
$h = (0, \ldots, 0, \ub{k}1, 0, \ldots 0)^T$\\
$|\ub{\sum_{i=1}^l (\frac{\partial f_i}{\partial x_k}(x)-\frac{\partial f_i}{\partial x_k}(\ot x))}{(F'(x)-F'(\ot x)h)}| \leq \| F'(x)-F'(\ot x)\||h| \leq \eps$\\
Отсюда $\sqrt{\sum_{i=1}^l (\frac{\partial f_i}{\partial x_k}(x)-\frac{\partial f_i}{\partial x_k}(\ot x))^2} \leq \eps$\\
Тогда для $i=i_0$ -- $|\frac{\partial f_{i_0}}{\partial x_k}(x)-\frac{\partial f_{i_0}}{\partial x_k}(\ot x)| \leq \eps$
\section{Экстремумы}
\textbf{Определение}\\
$f: D\subset \Rset^m \rto \Rset$\\
$a \in D$ -- локальный максимум $f \LRto \ex U(a): \fall x \in U(a) \cap D\ f(x) \leq f(a)$ (нестрогий экстремум)\\
$a \in D$ -- локальный строгий максимум $f \LRto \ex U(a): \fall x \in U(a) \cap D\ f(x) < f(a)$ (строгий экстремум)\\
\textbf{Теорема Ферма}\\
$f: D \subset \Rset^m \rto \Rset$\\
$a \in \Int D, f$ -- дифференцируема\\
$a$ -- экстремум\\
Тогда $\fall$ направление $l\ \frac{\partial f}{\partial l} (a) = 0$\\
\textbf{Доказательство}\\
$g(t) = f(a+tl), t \in \Rset$ -- задана в окрестности 0\\
$g'(0) = 0$\\
$g'(t) = f'l = \begin{pmatrix}
    \ppart f{x_1} & \ldots & \ppart f{x_m}
\end{pmatrix}\cdot \begin{pmatrix}
    l_1\\\vdots\\l_m
\end{pmatrix} = \ppart{f}{l}$\\
\textbf{Следствие (необходимое условие экстремума)}\\
$a$ -- локальный экстремум\\
Тогда $\fall 1 \leq k \leq m\ \ppart{f}{x_k}(a) = 0$\\
\textbf{Следствие (т. Ролля)}\\
Пусть $K \subset \Rset^m$ -- компакт\\
$f$ -- дифференцируема в $\Int K$ ($f: K \rto \Rset$, непрерывна)\\
$f_{\partial K} = \const, \partial K$ -- граница компакта\\
Тогда $\ex x_0 \in \Int K: \grad f(x_0) = 0$\\
\textbf{Доказательство}\\
По теореме Вейерштрасса $f$ достигает $\max, \min$ на $K$\\
Если оба на $\partial K$, то $f \equiv \const$ на $K$\\
Иначе применим теорему Ферма\\
\textbf{Определение}\\
$Q(h): \Rset^m \rto \Rset$ -- квадратичная форма, если она представляет однородный многочлен 2 степени\\
т.е. $Q(h) = \sum_{1 \leq i \leq m, 1 \leq j \leq m} a_{ij}h_i h_j, a_{ij} = a_{ji}$\\\\
$Q$ -- положительно определенная $\LRto \fall h \neq 0\ Q(h) > 0$\\
$Q$ -- отрицательно определенная $\LRto \fall h \neq 0\ Q(h) < 0$\\
$Q$ -- незнакоопределенная $\LRto \ex h: Q(h) > 0, \ex h: Q(h) < 0$\\
$Q$ -- полуопределенная (положительно определенная вырожденная) $\LRto \fall h\ Q(h) \geq 0, \ex h \neq 0: Q(h) = 0$\\
\textbf{Лемма об оценке квадратичной формы и об эквивалентных нормах}
\begin{enumerate}
    \item $Q: \Rset^m \rto \Rset$ -- кв. форма, $Q > 0$\\
    Тогда $\ex \gamma_Q > 0: \fall x\ Q(x) > \gamma_Q|x|:2$
    \item $p: \Rset^m \rto \Rset$ -- норма\\
    Тогда $\ex C_1, C_2 > 0: \fall x \in \Rset^m\ C_1|x_1| \leq p(x) \leq C_2|x_2|$
\end{enumerate}
\textbf{Доказательство}
\begin{enumerate}
    \item $\gamma_Q:= \min_{|x|=1} Q(x) > 0$\\
    Тогда $Q(x) = |x|^2Q(\frac x{|x|}) \geq \gamma_Q |x|^2, x \neq 0$
    \item Проверим, что $p(x)$ непрерывна, чтобы доказать существование минимума и максимума:\\
    $|p(x) - p(y)| \leq p(x-y) = p(\sum (x_k-y_k) \ol{e_k}) \leq \sum |x_k - y_k| p(e_k) \leq M|x-y|, M = \sqrt{\sum p(e_k)^2}$ -- по КБШ\\
    $C_1 = \min_{|x|=1}p(x), C_2 = \max_{|x|=1}p(x)$\\
    $p(x) = |x|p(\frac x{|x|}) \leq |x|C_2, \geq |x|C_1$
\end{enumerate}
\textbf{Напоминание}\\
$f(x+h) = f(x) + \df f(x, h) + \frac1{2!}\df^2 f(x,h) + \ldots$\\
$\df^2 f(x,h) = f''_{x_1x_1}(x) h_1^2 + \ldots + f''_{x_nx_n}h_n^2 + 2f''_{x_1x_2}h_1h_2 + \ldots$\\
\textbf{Теорема (достаточное условие экстремума)}\\
$f: D \subset \Rset^m \rto \Rset, a \in \Int D, \grad f(a) = 0, f \in C^2(D)$\\
$Q(h):= \df^2 f(a,h)$\\
Тогда $Q > 0 \Rto a$ -- локальный минимум\\
$Q < 0 \Rto a$ -- локальный максимум\\
$Q \lessgtr 0$ -- не точка локального экстремума\\
$Q \geq 0$ -- информации недостаточно\\
\textbf{Доказательство}\\
$\fall h\ \ex t \in (0,1): f(a+h) = f(a) + \ub0{\df f(a,h)} + \frac1{2!}\df f(a+th, h)$ -- остаток в формуле Лагранжа\\
$f(a+h) - f(a) = \frac1{2!}Q(h) + \frac1{2!}(\ub{|\text{б.м.}\cdot h_i^2| = o(|h|^2)}{f''_{x_1x_1}(a+th)h_1^2 - f''_{x_1x_1}(a)h_1^2}+ \ldots + \ub{|\text{б.м.}\cdot h_ih_j| = o(|h|^2)}{2f''_{x_1x_2}h_1h_2 - 2 f''_{x_1x_2}(a)h_1h_2}) + \ldots)$\\
$f(a+h)-f(a) \geq \frac12 Q(h)-|\alpha(h)||h|^2  \geq \frac12 \gamma_Q|h|^2 - |\alpha(h)||h|^2 \geq \frac14 \gamma_Q|h|^2 > 0, \alpha(h)$ -- б.м., при достаточно малых $|h|$\\
Пункт 1 доказан\\
Пункт 2 доказывается заменой $f \rto -f$\\
Пункт 3: $h: Q(h) > 0, \ot h: Q(\ot h) < 0$\\
Аналогично п.1. $f(a+s\cdot h) - f(a) \geq \frac12 Q(sh) - |\alpha(s)|s^2 = \frac12Q(h)s^2 - |\alpha(s)|s^2 \geq \frac14 Q(h) \cdot s^2$\\
С другой стороны $f(a+s\cdot \ot h) < 0$ по аналогичным соображениям\\
Пункт 4: $f: \Rset^2 \rto \Rset, a = (0, 0)$\\
$f(x_1, x_2) = x_1^2 - x_2^4$\\
$Q(h) = 2h_1^2$ -- полуопределенный\\
Тут нет экстремума\\
$f(x_1,x_2) = x_1^2 + x_2^4$ -- в нуле экстремум
\section{Функциональные последовательности и ряды}
\subsection{Равномерная сходимость последовательностей и функций}
\textbf{Определение}\\
Последовательность функций -- отображение $N \leadsto$ множество функций\\
Пусть $f_1(x), f_2(x), \ldots: X \rto \Rset, X$ -- любое множество\\
Последовательность $(f_n)$ сходится поточечно на $E$ -- существует функция $f(x)$\\
$f_n \us E\rto f$\\
$\fall x \in E\ \fall \eps > 0\ \ex N: \fall n > N\ |f_n(x)-f(x)|<\eps$\\
Последовательность $(f_n)$ сходится равномерно на $E$ к функции $f$\\
$f_n \us E\rightrightarrows f$
$\fall \eps > 0\ \ex N: \fall n > N\ \ub{\sup_{x\in E} |f_n(x)-f(x)| = \rho(f_n, f)\leq \eps}{\fall x \in E\ |f_n(x)-f(x)| < \eps}$\\
\textbf{Замечание}\\
$f \rrto f$ на $E, E_0 \subset E$\\
Тогда $f_n \us {E_0}\rrto$\\
\textbf{Замечание}\\
$f_n \us E\rrto f$\\
Тогда $f_n \us E\rto f$\\
\textbf{Замечание}\\
$\mc F = \{f: X \rto \Rset, f\text{ -- огр.}\}$\\
Тогда $\rto(f, g) = \sup_{x\in X} |f(x) - g(x)|$ является метрикой на $\mc F$\\
\textbf{Доказательство}\\
Первые две аксиомы очевидны\\
Докажем неравенство треугольника\\
$\fall \eps > 0\ \ex x_0: \rho(f, g) - \eps < |f(x_0) - g(x_0)| \leq |f(x_0) - h(x_0)| + |h(x_0)-g(x_0)| \leq \rho(f,h)+\rho(h,g)$\\
Отсюда $\rho(f,g) \leq \rho(f,h) + \rho(h,g)$\\
\textbf{Замечание}\\
$f_n \rrto f$ на $E_1$ и на $E_2$\\
Тогда $f_n \rrto f$ на $E_1 \cup E_2$\\
\textbf{Теорема 1 (Стокса - Зайдля)}\\
$X$ -- метрическое пространство\\
$f_n, f: X \rto \Rset$\\
$c \in X, f_n$ -- непрерывная в $c$\\
$f_n \rrto f$ на $X$\\
Тогда $f$ -- непрерывна в $c$\\
\textbf{Доказательство}\\
Дано утверждение о равномерной сходимости\\
$\fall \eps > 0\ \ex N: \fall n > N\ \sup_{x\in X} |f_n(x) - f(x)|<\eps$\\
$|f(x)-f(c)| \leq |f(x)-f_n(x)| + |f_n(x)-f_n(c)| + |f_n(c)-f(c)|$\\
Начиная с некоторого большого $n:$\\;
$|f(x)-f(c)| \leq \ub{< \eps}{|f(x)-f_n(x)|} + |f_n(x)-f_n(c)| + \ub{< \eps}{|f_n(c)-f(c)|}$\\
Т.к. $f_n$ непрерывна, то $\ex U(c): \fall x \in U(c)\ |f_n(x)-f_n(c)| < \eps$\\
Отсюда $|f(x)-f(c)| < 3\eps$\\
Тогда $\fall \eps > 0\ \ex U(c): \fall x \in U(c)\ |f(x)-f(c)| < 3\eps$\\
\textbf{Замечание}\\
Вместо метрических пространств можно рассматривать топологические пространства(без изменения доказательства)\\
\textbf{Следствие}\\
$f_n \in C(X), f_n \rrto f$ на $X$. Тогда $f \in C(X)$\\
\textbf{Следствие 2}\\
$f_n \in C(X), \fall c\ \ex W(c): f_n \rrto f$ на $W(c)$. Тогда $f \in C(X)$\\
\textbf{Замечание}\\
$f_n \rrto f$ на $X \not \Rto \fall c\ \ex W(c): f_n \rrto f$\\
Пример: $f_n = x^n, x \in (0,1)$\\
$f \equiv 0$\\
Рассмотрим точку $x$ в $(\alpha, \beta), \beta \neq 1$\\
$\rho(f_n, f) = \beta^n \rto 0$\\
$f_n(x) \rrto f$ на $(\alpha, \beta)$\\
Но $\rho(f_n, f) = 1$ на $(0,1)$\\
$f_n \not\rrto f$ на $(0,1)$\\
%лекция 5-6 (прошлый год) 2:13:00
\textbf{Теорема}\\
$X$ -- компакт\\
$\rho(f,g) = \sup_{x\in X}|f(x)-g(x)$ в $C(X)$\\
Тогда $(C(X), \rho)$ -- полное метрическое пространство\\
(т.е. $\fall x_n\text{ -- фунд.}\ \fall \eps > 0\ \ex N: \fall m,n > N\ \rho(x_n, x_m) < \eps)$\\
\textbf{Доказательство}\\
Пусть $f_n$ -- фундаментальная последовательность в $C(X)$\\
$\fall \eps > 0\ \ex N : \fall n, m > N\ \rho(f_n, f_m) = \sum_{x \in X} |f_n(x) - f_m(x)| < \eps$\\
Тогда $\fall x_0 \in X$ последовательность $n \mapsto f_n(x_0)$ -- фунд. вещ. посл.\\
Тогда $\ex \lim_{n\rto +\infty} f_n(x_0) = f(x_0)$ -- конечная\\
Проверим, что $f_n \rrto f, f \in C(X)$\\
$\fall \eps > 0 \ex N: \fall n, m > N\ \fall x\ |f_n(x) - f_m(x)| < \eps$\\
$\fall \eps > 0 \ex N: \fall n > N\ \fall x\ |f_n(x) - f(x)| \leq \eps$ (предельный переход $m \rto \infty$)\\
Т.е. $f_n \rrto f$ на $X$\\
$f\in C(X)$ по теореме 1\\
\textbf{Замечание}\\
$\mc F(X) = $ пространство ограниченных функций на $X$\\
$(\mc F(X), \rho)$ -- полное м.п.\\
\textbf{Доказательство}\\
Аналогичное\\
%лекция 5-6 (прошлый год) 2:32:00\\
\textbf{Теорема (критерий Коши)}\\
$f_n \in C(X)$\\
$\ex f \in C(x): f_n \rrto f \LRto \fall \eps > 0\ \ex N: \fall n,m > N\ \fall x \in X\ |f_n(x)-f_m(x)| < \eps$\\
\subsection{Предельный переход под знаком интеграла}
Подумаем о следующем правиле: $f_n \rto f \Rto \int_a^b f_n \rto \int_a^b f$\\
\textbf{Анти-пример}\\
$f_n(x) = nx^{n-1}(1-x^n), x\in [0,1], f_n \rto f\equiv 0$\\
$\int_0^1 nx^{n-1}(1-x^n)\df x \us{t=x^n}= \int_0^1 (1-t)\df t = \frac12$\\
\textbf{Теорема 2}\\
$f_n \in C[a,b]$\\
$f_n \rrto f$ на $[a,b]$\\
Тогда $\int_a^b f_n \rto \int_a^b f$\\
(по т.1. $f$ -- непрерывна)\\
\textbf{Доказательство}\\
$|\int_a^b f_n - \int_a^b f| = |\int f_a^b f_n - f| \leq \int_a^b |f_n-f| \leq \sup |f_n-f|(b-a) \rto 0$\\
\textbf{Следствие (правило Лейбница дифференцирования интеграла по параметру)}\\
$f: \ub{x}{[a,b]} \times \ub{y}{[c,d]} \rto \Rset$\\
$\fall x,y \ex f'_y(x,y)$ и $f, f'_y$ -- непрерывные на $[a,b]\times[c,d]$\\
Тогда для $\Phi(y) = \int_a^b f(x,y)\df x$ верно, что $\Phi$ -- дифференцируема на $[c,d]$ и $\Phi'(y) = \int_a^b f'_y(x,y)\df x$\\
\textbf{Доказательство}\\
$\frac{\Phi(y+t_n)-\Phi(y)}{t_n} = \int_a^b \frac{f(x,y+t_n)-f(x,y)}{t_n}\df x \us{\text{по т.Лагранжа}}= \int_a^b f_y'(x, t+\Theta_x t_n)\df x \os{(*)}\rto \int_a^b f'_y(x,y)\df x$\\
Проверим $(*)$:\\
Вспомним теорему Кантора о равномерной непрерывности\\
$f \in C(K)$. Тогда $f$ -- равномерно непрерывная\\
Т.е. $\fall \eps > 0\ \ub{\ex N: \fall n > N\ |t_n|<\delta}{\ex \delta > 0}: \fall x, \ol x: \rho(x, \ot x) < \delta\ |f(x)-f(\ot x)| < \eps$\\
Тогда $\rho((x, y + \Theta_x t_n), (x,y)) < \delta$\\
И значит $|f(x, y + \Theta_x t_n) - f(x,y)| < \eps$\\
Т.е. $|\int_a^b f'_y(x,y+\Theta_x t_n)-\int_a^b f'_y(x,y)| \leq \eps(b-a)$\\
%лекция 5-6 (прошлый год) 3:00:00
\textbf{Теорема 3 (о предельном переходе по знаку производной)}\\
Пусть $f_n \in C^1\an ab$\\
$f_n \rto f_0$ поточечно на $\an ab$\\
$f'_n \rrto \phi$ на $\an ab$\\
Тогда $f_0 \in C^1\an ab, f'_0 = \phi$ на $\an ab$\\
\textbf{Пояснение}\\
$\lim_{n\rto +\infty} f'_n(x) = (\lim_{n\rto +\infty} f_n(x))'$\\
\textbf{Доказательство}\\
$[x_0, x_1] \subset \an ab$\\
$f'_n \rrto \phi$ на $[x_0, x_1]$ (отсюда $\phi$ непрерывна)\\
Тогда по т.2 $\int_{x_0}^{x_1} f'_n \rto \int_{x_0}^{x_1} \phi$\\
$\ub{\rto (f_0(x_1)-f_0(x_0))}{f_n(x_1)-f_n(x_0)} \rto \int_{x_0}^{x_1} \phi$\\
Т.о. $f_0$ -- первообразная $\phi$\\
$\phi$ -- непрерывна по т.1\\
Отсюда $f'_0 = \phi$\\
% лекция 7-8 (этот год) 1:40\\
\textbf{Определение}\\
$u_n(x) : E \rto \Rset$\\
$\sum u_n(x_0)$\\
$S_N(x)=\sum_{n=1}^N u_n(x), S = \sum_{n=1}^\infty u_n(x)$\\
$S_N \rrto S$ на $E \LRto$ ряд равномерно сходится в $E$\\
$\LRto M_N = \sup_{x\in E}|\sum_{n>N} u_n(x)| \us{N\rto +\infty}\rto 0$\\
\textbf{Теорема (признак Вейерштрасса)}\\
$\sum u_n(x), x\in E$\\
Пусть $\ex (c_n) \in R:\fall x \in E |u_n(x)|\leq c_n$\\
и $\sum c_n$ -- сходится\\
Тогда $u_n$ равномерно сходится на $E$\\
\textbf{Доказательство}\\
Рассмотрим $M_N = \sup_{x\in E}|\sum_{n>N} u_n(x)| \leq \sum_{n > N} c_n \us{N\rto \infty}\rto 0$\\
\textbf{Критерий Больцано-Коши равномерной сходимости}\\
$\fall \eps > 0\ \ex N: \fall n > N\ \fall k \in \Nset\ \fall x \in E\ |u_{n+1}(x)+\ldots + u_{n+k}(x)| < \eps$\\
эквивалентно равномерной сходимости\\
(если нет равномерной сходимости, то нет и критерия Больцано-Коши)\\
\textbf{Пример}\\
$\sum \frac{x}{1+n^4x^2}, x \in (0, +\infty)$\\
Есть ли равномерная сходимость?\\
$c_n = \max_x \frac{x}{1+n^4x^2}=\frac1{2n^2}$\\
$\sum c_n = \sum \frac1{2n^2}$ -- сходится (т.е. есть равномерная сходимость)\\
\textbf{Пример 2}\\
$\sum \frac{xn}{1+n^4x^2}, x \in (0,+\infty)$\\
$c_n = \frac1{2n}, \sum \frac1{2n}$ -- расходится\\
Применим критерий Больцано-Коши\\
$\ex \eps = \frac1{17}: \fall N\ \ex n > N, m=n, x = \frac1{n^2}: |u_{n+1}(x)+\ldots + u_{2n}(x)| \geq \frac{(n+1)\frac1{n^2}}{1+(n+1)^4\frac1{n^4}}+\ldots + \frac{2n\frac1{n^2}}{1+(2n)^4\frac1{n^4}} \geq n\frac{\ob{\text{нижняя оценка числителя}}{\frac1n}}{\ub{\text{верхняя оценка знаменателя}}{17}} = \frac1{17}$\\
\textbf{Теорема 1' (Стокса-Зайдля для рядов)}\\
$u_n: X \rto \Rset, X$ -- метрическое пространство\\
$u_n$ -- непрерывно в $x_0 \in X$\\
$\sum u_n$ -- равномерно сходится в $X$\\
Тогда $S(x)=\sum u_n$ -- непрерывно в $x_0$\\
\textbf{Доказательство}\\
$f_n \lrto S_n, f \lrto S$ + теорема Стокса-Зайдля для функций\\
\textbf{Пример}\\
$\sum \frac{x}{1+n^4x^2}$ -- непрерывно\\
\textbf{Пример 2}\\
$\sum \frac{nx}{1+n^4x^2}$\\
Возьмем $x_0 > 0$ и окрестность $(a,b): 0 < a < x < b$\\
$|\frac{nx}{1+n^4x^2}| \leq \frac{nb}{1+n^4a^2} =: c_n$\\
$\sum c_n$ -- сходится\\
Тогда наш ряд равномерно сходится в $(a,b)$\\
Тогда он непрерывен\\
\textbf{Теорема 2'}\\
$u_n \in C[a.b]$\\
$\sum u_n$ равномерно сходится на $[a,b]$\\
$S(x)=\sum u_n(x)$\\
Тогда $\int_a^b S(x) \df x = \sum (\int_a^b u_n(x)\df x)$\\
\textbf{Доказательство}\\
Применим теорему 2\\
$\int_a^b S(n) \us{n\rto\infty}\rto \int_a^b S$\\
$\int_a^b (\sum_{k=1}^n u_k) = \sum_{k=1}^n(\int_a^b u_k)$\\
\textbf{Пример}\\
$\sum_{n=0}^{+\infty} (-1)^nx^n$ -- равномерно сходится на $[-q,q]$, где $0<q<1$\\
$|(-1)^nx^n| \leq q^n, \sum q^n$ -- сходится (т. Вейерштрасса)\\
$\sum_{n=0}^{+\infty} (-1)^nx^n = 1 - x + x^2 - x^3 + \ldots = \frac1{1+x}$
$\int_0^q \frac1{1+x}\df x = \ln(1+q) = \sum_{n=0}^{+\infty} (-1)^n \frac{q^{n+1}}{n+1}$\\
$\ln (1+q) = q-\frac{q^2}2+\frac{q^3}3-\frac{q^4}4 +\ldots, q \in (-1, 1)$\\
Заметим, что формула верна и при $q=1$, но исходный ряд не будет сходиться, т.к. слагаемые не будут бесконечно малыми нснм\\
\textbf{Пример 2}\\
Ряд $q-\frac{q^2}2+\frac{q^3}3-\frac{q^4}4 +\ldots$ равномерно сходится на $[0,1]$\\
по секретному приложению к признаку Лейбница\\
$|\sum_{n>N}(-1)^{n-1}\frac{q^n}n| \leq \frac{q^{N+1}}{N+1}\leq \frac1{N+1} \rto 0$ (тогда равномерно сходится)\\
Тогда сумма в правой части непрерывна на $[0,1]$ по Т.1'\\
Тогда формула из примера 1 <<продолжается>> по непрерывности на точку 1\\
$\ln 2 = 1 - \frac12 + \frac13 - \frac14 + \ldots$\\
\textbf{Теорема 3' (о дифференцировании ряда по параметру)}\\
$u_n \in C^1\an ab$
\begin{enumerate}
    \item $\sum u_n (x) = S(x)$ -- поточечная сходимость на $\an ab$
    \item $\sum u'_n(x) = \phi(x)$ -- равномерная сходимость на $\an ab$
\end{enumerate}
Тогда $S(x)\in C^1$ и $S' = \phi$ на $\an ab$\\
Другими словами, $(\sum u_n(x))' = \sum u'_n(x)$, если исходный ряд \underline{равномерно сходится}\\
\textbf{Доказательство}\\
Применим Т.3:\\
$f_n = S_n, f_0 = S$\\
$f'_n=\sum_{k=1}^n u'_k(x), \phi=\phi$\\
\textbf{Пример}\\
$\frac1{\Gamma(x)} = xe^{\gamma x}\prod_{n=1}^{\infty} (1+\frac xn)e^{-\frac xn}$\\
$-\ln \Gamma(x) = \ln x + \gamma x + \sum_{n=1}^{+\infty} (\ln (1+\frac xn)-\frac xn)$\\
$(\ln (1+\frac xn)-\frac xn)' = \frac{\frac1n}{1+\frac xn}-\frac1n = \frac1{n+1}-\frac1n = -\frac{x}{n(n+x)}$\\
$\sum -\frac{x}{n(n+x)}$ -- сходится\\
Проверим, что ряд равномерно сходится при больших $x$\\
$m>0, x \in (0, m)$\\
$|-\frac{x}{n(n+x)}| \leq \frac{M}{n(n+M)}$\\
$\sum \frac{M}{n(n+M)} \leq +\infty$\\
По признаку Вейерштрасса $\sum -\frac{x}{n(n+x)}$ -- равномерно сходится на $(0, M)$\\
$\sum (\ln (1+\frac xn)-\frac xn)$ -- дифференцируемо при $x>0$\\
$\Gamma(x) = xe^{\gamma x}\exp(\sum (\ln (1+\frac xn)-\frac xn))^{-1}$ -- дифференцируемо при $x>0$ и ее производная непрерывна\\
На самом деле $\Gamma \in C^\infty$\\
% \textbf{Пример}\\
% $S(x)=\sum \frac{\sin nx}{n^3}, x \in \Rset$\\
% Рассмотрим сумму производных: $\sum \frac{\cos nx}{n^2}$\\
% $|\frac{\cos nx}{n^2}| \leq \frac1{n^2}$ -- отсюда равномерная сходимость\\
% Тогда $S'(x) = \sum \frac{\cos nx}{n^2}$
\textbf{Теорема 4' (о почленном предельном переходе в суммах)}\\
% лекция 10 (этот год) 0:21
$u_n: X \rto \Rset, X$ -- м.п.\\
$x_0 \in X, x_0$ -- предельная точка в $E$\\
Пусть \begin{enumerate}
    \item $\ex \lim_{x\rto x_0} u_n(x) = a_n \in \Rset$
    \item $\sum u_n(x)$ -- равномерно сходится на $E$
\end{enumerate}
Тогда \begin{enumerate}
    \item $\sum a_n$ -- сходится
    \item $\sum a_n = \lim_{x\rto x_0} \sum u_n(x)$
\end{enumerate}
\textbf{Доказательство п.1}\\
Докажем сходимость\\
$S_n(x) := \sum_{k=1}^n u_k(x)$\\
$S_n^a = \sum_{k=1}^n a_k$\\
Проверим фундаментальность $S_n^a$\\
Возьмем $\eps > 0$\\
$|S_{n+p}^a - S_n^a| \leq \us{\text{выберем $x$ близко к $x_0$, чтобы $\ldots < \frac\eps3$}}{|S_{n+p}^a - S_{n+p}(x)|} + |S_{n+p}(x)-S_n(x)| + \us{\text{выберем $x$ близко к $x_0$, чтобы $\ldots < \frac\eps3$}}{|S_n(x)-S_n^a|} < \eps$\\
Из равномерной сходимости $\sum u_n(x)$:\\
$\ex N: \fall n > N\ \fall p \in \Nset\ \fall x\ |S_{n+p}(x)-S_n(x)|\leq \frac\eps3$\\
Отсюда $\fall \eps > 0\ \ex N: \fall n > N\ \fall p \in \Nset\ |S_{n+p}^a-S_n^a| < \eps$\\
\textbf{Доказательство п.2}\\
Результат следует из теоремы Стокса-Зайдля\\
$\ot u_n(x) = \left[\begin{array}{cc}
    u_n(x), & x\in E\\
    a_n, & x = x_0
\end{array}\right.$\\
$\ot u_n(x)$ -- непрерывная в $x_0$\\
$\sup_{E \cup \{x_0\}} |\sum_{n\geq N} \ot U_n(x)| \leq \ub{\xrto[N\rto +\infty]{} 0}{\sup_E |\sum_{n\geq N}u_n(x)|} + \ub{\xrto[N\rto +\infty]{} 0}{|\sum_{n\geq N}a_n|} \xrto[N\rto +\infty]{} 0$\\
$\sum \ot u_n$ -- равномерно сходится на $E \cup \{x_0\}$\\
\textbf{Теорема 4 (перестановки в предельных переходах)}\\
$f_n: E \subset X \rto \Rset, x_0$ -- предельная точка $E$
\begin{enumerate}
    \item $f_n(x) \xrto[x\rto x_0]{} A_n$ -- конечный 
    \item $\ex S: E \rto \Rset:\ f_n \us{n\rto +\infty}\rrto S$ на $E$
\end{enumerate}
Тогда \begin{enumerate}
    \item $\ex \lim A_n = A$ -- конечный
    \item $S(x)\xrto[x\rto x_0]{} A$
\end{enumerate}
Т.е. $\lim_{x\rto x_0} \lim_{n\rto +\infty} f_n(x) =  \lim_{n\rto +\infty} \lim_{x\rto x_0} f_n(x)$\\
% лекция 10 (этот год) 0:52
\textbf{Доказательство}\\
Применим теорему 4'\\
$f_n(x) \lrto S_n(x)$\\
$u_n \lrto f_n-f_{n-1}$ (за исключением $u_1 = f_1$)\\
$a_k \lrto A_k - A_{n-1}$ (кроме $a_1 = A_1$)\\
$\sum u_n(x) \lrto S(x)$\\
\textbf{Замечание}\\
$f_n \rrto S$ на $E$\\
Тогда $\fall \eps > 0\ \ex N: \fall n > N\ \sup_{x\in E}|f_n(x)-S(x)|<\eps$\\
Определим равномерный предел при $t\rto t_0$\\
$f: E \times D\rto \Rset, E$ -- множество, $D\subset Y$ -- м.п., $t_0$ -- предельная тока $D$\\
$f(x, t) \us{t\rto t_0}\rrto h(x)$, где $h: E\rto\Rset$\\
$\fall \eps > 0\ \ex U(t_0): \fall t \in U(t_0)\ \sup_{x\in E}|f(x,t)-h(x)| < \eps$\\
\textbf{Теорема 4''}\\
$f: E\times D \rto \Rset, E \subset X$ -- м.п., $D\subset Y$ -- м.п., $x_0$ -- предельная точка $E$, $t_0$ -- предельная точка $D$
\begin{enumerate}
    \item $\fall t\ \ex \lim_{x\rto x_0} f(x,t) = A(t)$ -- конечный
    \item $f(x,t) \us{t\rto t_0}\rrto S(x)$, где $S: E\rto \Rset$
\end{enumerate}
Тогда \begin{enumerate}
    \item $\ex$ конечный $\lim_{t\rto t_0} A(t) = A$
    \item $\lim_{x\rto x_0} S(x) = A$
\end{enumerate}
\textbf{Теорема (признак Дирихле равномерной сходиости ряда)}\\
% лекция 10 (этот год) 1:08
Пусть $\sum_{n=1}^{\infty} a_n(x)b_n(x), x \in X$\\
Пусть \begin{enumerate}
    \item $\ex C_A: \fall N \ \fall x\ |\sum_{n=1}^N a_n(x)|\leq C_A$\\
    (частичные суммы ряда $\sum a_n(x)$ равномерно ограничены)
    \item $b_n(x) \us{n\rto \infty} \rrto 0$ на $X$ и $\fall x\ b_n\us{n\rto \infty}\rto 0$ монотонно при каждом фиксированном $X$  //todo проверить, правда ли это
\end{enumerate}
Тогда $\sum a_n(x)b_n(x)$ -- равномерно сходится на $X$\\
\textbf{Доказательство}\\
$\sum_{N \leq k \leq M} a_kb_k = A_Mb_M - A_{N-1}b_{N-1} + \sum_{N \leq k \leq M-1}(b_k-b_{k+1})A_k$\\
$A_k = a_1 + \ldots + a_k$\\
Из равномерной сходимости: $\fall \eps>0\ \ex T: \fall M,N > T\ \sup|b_M(x)|<\eps;\sup|b_{N-1}(x)|<\eps,\sup|b_{M-1}(x)|<\eps,\sup|b_N(x)|<\eps$\\
$|\sum_{N\leq k \leq M} a_k(x)b_k(x)| \leq |A_M||b_M|+|A_{N-1}||b_{N-1}| + \sum |b_k-b_{k+1}||A_k| \leq C_A(|b_M| + |b_{N-1}|+|b_{N-1}|+|b_N|) < 4C_A\eps$\\
% лекция 15 (этот год) 0:00
\textbf{Следствие (признак Абеля)}\\
$\sum a_n(x)b_n(x), x \in E$
\begin{enumerate}
    \item $\sum a_n(x)$ равномерно сходится на $E$
    \item $b_n$ монотонно по $n$ при каждом $x$  //todo проверить, правда ли это\\
    $b_n(x)$ -- равномерно ограничена: $\ex C_B: \fall x\ \fall n\ |b_n(x)|\leq C_B$
\end{enumerate}
Тогда $\sum a_n(x)b_n(x), x \in E$ равномерно сходится на $E$\\
% лекция 12 (этот год) 0:00
\textbf{Пример}\\
$f(x) = \sum \frac{\sin nx}{n^3}, x \in \Rset$\\
$f$ -- непрерывно по признаку Вейерштрасса: $|\frac{\sin nx}{n^3}| \leq \frac1{n^3}, \sum \frac1{n^3}$ -- сходится\\
$f$ -- дифференцируемо: $f' = \sum \frac{\cos nx}{n^2}$ -- это выполнено по теореме 3', т.к. ряд равномерно сходится\\
Но дважды не дифференцируемо, т.к. нет равномерной сходимости\\
$\ex \eps=\frac1{100}: \fall N\ \ex n > N: \ex m=n\ \ex x=\frac1n: |\frac{\sin (n+1)x}{n+1} + \ldots + \frac{\sin (n+m)x}{n+m}| > \eps$\\
$\frac{\sin \frac{n+1}n}{n+1} + \ldots + \frac{\sin \frac{2n}n}{2n} > \frac1{20}$\\
Тогда докажем локальную равномерную сходимость (в окрестности некоторой точки)\\
В окрестности не должно быть $2\pi k$, иначе аналогично доказательству\\
Рассмотим окрестность $(\alpha, \beta), 2\pi k \not\in (\alpha, \beta)$\\
$a_n = \sin nx, b_n(x) = \frac1n$ -- монотонно, $b_n \rrto 0$\\
$|\sin x + \sin 2x + \ldots + \sin nx| \leq |\im(e^{ix} + \ldots + e^{inx})| \leq |e^{ix}\frac|{|e^{inx}-1|}{|e^{ix}-1|} \leq \frac2{|e^{ix}-1|} \leq \frac2d, d = \min(|e^{i\alpha}-1|, |e^{i\beta}-1|)$\\
Т.о. $\fall x_0 \in (0, 2\pi)\ \ex U(x_0)$ на которой имеется равномерная сходимость\\
Таким образом $f''(x) = \sum - \frac{\sin nx}{n}\ \fall x \in (0, 2\pi)$\\
\textit{Спойлер}: $\sum \frac{\sin nx}{x}$ -- не непрерывна в 0 -- там имеется скачок\\
$\not \ex f''(0)$\\
\subsection{Степенные ряды}
\textbf{Определение}\\
$B(z_0, r) \in \Cset := \{z: |z-z_0| < r\}$\\
\textit{Степенной ряд}: $\sum a_n(z-z_0)^n, z_0 \in C, a_n$ -- комплексная последовательность\\
\textbf{Теорема (о круге сходимости степенного ряда)}\\
$\sum a_n(z-z_0)^n$\\
Тогда выполнено ровно одно из трех
\begin{enumerate}
    \item Ряд сходится при всех $z \in \Cset$
    \item Ряд сходится только при $z=z_0$
    \item $\ex R \in (0, +\infty): $ при $|z-z_0| > R$ -- расходися; $|z-z_0|<R$ -- ряд сходится абсолютно
\end{enumerate}
\textbf{Утверждение}\\
$\sum a_n$ -- сходится $\LRto \sum \re a_n, \sum \im a_n$ -- сходятся\\
\textbf{Доказательство}\\
Изучим ряд на абсолютную сходимость\\
Применим признак Коши: изучим $\ulim \sqrt[n]{|a_n(z-z_0)^n} = \ulim |z-z_0|\sqrt[n]{|a_n|}$:\\
При $\ldots<1$ -- абсолютно сходится\\
При $\ldots > 1$ -- расходится, т.к. слагаемые $\not\rto 0$\\
Рассмотрим $|z-z_0|\ulim \sqrt[n]{|a_n|}$
\begin{enumerate}
    \item $\ulim \sqrt[n]{|a_n|} = 0$ -- всегда сходится (случай 1)
    \item $\ulim \sqrt[n]{|a_n|} = +\infty$ -- тогда сходится при $z=z_0$, иначе расходится (случай 2)
    \item $|z-z_0| < \frac1{\ulim \sqrt[n]{|a_n|}} =: R$ -- сходится в $B(z_0, R)$ (случай 3)
\end{enumerate}
$R$ -- \textit{радиус сходимости}\\
$R = \frac1{\ulim \sqrt[n]{|a_n|}}$ -- формула Коши-Адамара\\
Если применим признак Даламбера, то $R = \lim |\frac{a_n}{a_{n+1}}|$\\
\textbf{Пример}
\begin{enumerate}
    \item $\sum_{n=0}^{+\infty} z^n$\\
    $S_n = 1 + z + \ldots + z^n = \frac{z^{n+1}-1}{z-1} \rto \frac1{1-z}$\\
    Сходится при $|z| < 1$\\
    При $|z| = 1$ ряд расходится\\
    $R=1$
    \item $\sum \frac{z^n}n, R = 1$\\
    Рассмотрим $|z| = 1$: $z=1$ -- расходится; $z=-1$ -- сходится\\
    $z=e^{i\phi}$: $\sum \frac{e^{in\phi}}{n} = \sum \frac{\cos n\phi}{m} + i\sum \frac{\sin n\phi}{n}$, -- сходится при $\phi \in(0, 2\pi)$\\
    Т.е. сходимость при $|z| \leq 1$, кроме $z=1$
    \item $\sum \frac{z^n}{n^2}, R = 1$\\
    Рассмотрим $|z|=1$\\
    $|\frac{z^n}{n^2}| \leq \frac1n^2$ -- сходится при $|z| \leq 1$
    \item $\sum n!z^n$ -- сходится при $z=0$
    \item $\sum \frac{z^n}{n!}, R = \frac1{\lim \sqrt[n]{\frac1{n!}}} = \frac1{\lim \sqrt[n]{\frac1{n^n e^{-n}\sqrt{2\pi n}\ldots}}} = \frac1{\lim \frac en} = +\infty$ -- сходится при всех $z \in \Rset$
\end{enumerate}
\textbf{Теорема (о равномерной сходимости и непрерывности степенного ряда)}\\
$\sum a_n (z-z_0)^n, 0 < R \leq +\infty$\\
Тогда \begin{enumerate}
    \item Для $r: 0 < r < R$ ряд равномерно сходится на $\ol B (z_0, r)$
    \item $f(z) = \sum a_n(z-z_0)^n$ -- непрерывна на $B(z_0, R)$
\end{enumerate}
\textbf{Доказательство}
\begin{enumerate}
    \item по признаку Вейерштрасса\\
    $|a_n(z-z_0)^n| \leq |a_n|r^n$\\
    $\sum |a_n|r^n$ -- сходится: подставим в ряд $z:=z_0+r$ -- должен абсолютно сходиться
    % лекция 10 (этот год) 1:15
    \item Проверим, что $a_n(z-z_0)^n$ -- непрерывная функция:\\
    Возьмем $z_1 \in B(z_0, R)$\\
    Возьмем $r: |z_1-z_0| < r < R$\\
    В круге $\ol B(z_0, r)$ есть равномерная сходимость $\Rto$ есть непрерывность
\end{enumerate}
\textbf{Определение}\\
Будем считать комплексную функцию дифференцируемой, если $\lim_{z\rto z_0} \frac{f(z)-f(z_0)}{z-z_0} = A = f'(z)$ (двойной предел)\\
Эквивалентно $f(z)=f(z_0) + A\cdot (z-z_0) + o(|z-z_0|)$
Т.к. мы ввели другое определение производной, не будем пользоваться теоремой 2\\
\textbf{Лемма}\\
Пусть $w, w_0 \in \Cset, r > 0$\\
$|w| < r, |w_0| < r$\\
$|w^n-w_0^n| = |(w-w_0)(w^{n-1} + w^{n-2}w_0 + \ldots + ww_0^{n-2}+w_0^{n-1})|\leq |w-w_0|nr^{n-1}$\\
% лекция 13 (этот год) 0:0
\textbf{Теорема (о почленном дифференцировании ряда)}\\
Пусть ряд A: $\sum_{n=0}^{\infty} a_n(z-z_0)^n, 0 < R \leq +\infty$\\
Ряд A': $\sum_{n=1}^{\infty} a_n n (z-z_0)^{n-1}$\\
Тогда\begin{enumerate}
    \item A' имеет тот же радиус сходимости
    \item Если $f(z) = \sum a_n (z-z_0)^n$, то $\fall z \in B(z_0, R)\ f'(z)=\sum_{n=1}^\infty a_n n (z-z_0)^{n-1}$
\end{enumerate}
\textbf{Доказательство}\\
Множество сходимости ряда A' такое же, как у ряда $\sum_{n=1}^{+\infty} a_nn(z-z_0)^n$\\
$R^{(A')} = \frac1{\ulim \sqrt[n]{|a_n|n}} = \frac1{\ulim \sqrt[n]{|a_n|}}$\\
Возьмем $a$ в окрестности сходимости\\
$\frac{f(z)-f(a)}{z-a} = \sum a_n\frac{(z-z_0)^n-(a-z_0)^n}{(z-z_0)-(a-z_0)} = \left[\begin{array}{cc}
    w=z-z_0\\
    w_0 = a-z_0
\end{array}\right] = \sum a_n \frac{w^n-w_0^n}{w-w_0} \xrto[w\rto w_0]{} \sum a_nnw_0^{n-1}$ -- при условии наличия равномерной сходимости в $U(w_0)$\\
Воспользуемся леммой, взяв $|a-z_0| < r < R$\\
Тогда при $|w| < r$ (и $|w_0| < r$): $|a_0 \frac{w^n-w_0^n}{w-w_0}| \leq n |a_n|r^{n-1}$\\
$\sum na_nr^{n-1}$ -- ряд A' в точке $z_0+r$ -- абсолютно сходится\\
Т.о. в $B(z_0, r)$ ряд $\sum a_n\frac{(z-z_0)^n-(a-z_0)^n}{(z-z_0)-(a-z_0)}$ равномерно сходится по признаку Вейерштрасса\\
\textbf{Следствие 1}\\
$f(z) = \sum a_n(z-z_0)^n, 0 < R \leq +\infty$\\
Тогда $f\in C^\infty(B(z_0, R))$\\
и все производные получаются почленным дифференцированием\\
\textbf{Следствие 2}\\
$a_n, x_0 \in \Rset$\\
$f(x) = \sum a_n(x-x_0)^n, x \in \Rset$ при $|x-x_0| < R$\\
Тогда при почленном интегрировании радиус сходимости сохраняется и $\int_{x_0}^x f(x)\df x = \sum_{n=0}^\infty a_n \frac{(x-x_0)^{n+1}}{n+1}$\\
% лекция 15 (этот год) 0:07
\textbf{Теорема (Метод Абеля суммирования рядов)}\\
Пусть $\sum c_n$ сходится\\
$f(x) := \sum c_n x^n, x \in (-1, 1)$\\
Тогда $\lim_{x\rto 1-0} f(x) = \sum c_n$\\
\textbf{Доказательство}\\
Признак Абеля: $a_n(x) \lrto c_n$\\
$b_n(x) = x^n$ -- здесь считаем, что $x \in [0,1)$\\
Отсюда $\sum c_nx^n$ равномерно сходится на $[0,1)$\\
Тогда $R \geq 1 \Rto$ равномерно сходится на $(-1, 1)$\\
По Т.4' о предельном переходе в сумме предел суммы ($\lim_{x\rto 1-0} f(x)$) равен сумме пределов $\sum c_n$\\
\textbf{Следствие}\\
$\sum a_n = A$\\
$\sum b_n = B$\\
$c_n = a_nb_0 + a_{n-1}b_1 + \ldots + a_0b_n$\\
Пусть ряд $\sum c_n$ сходится и $\sum c_n = C$\\
Тогда $AB = C$\\
\textbf{Доказательство}\\
$f(x) = \sum a_nx^n, g(x) = \sum b_nx^n, h(x) = \sum c_n x^n$\\
$x\in (0, 1)$ -- ряды сходятся абсолютно\\
Тогда $f(x)g(x) = h(x)$\\
Тогда из предельного перехода $x\rto 1-0\ AB=C$\\
\textbf{Пример}\\
$\sum \frac1{n^2}$\\
$f(x) := \sum_{n=1}^{\infty} \frac{x^n}{n^2}$\\
$f'(x) = \sum_{n=1}^\infty \frac{x^{n-1}}{n}$\\
$xf'(x) = \sum_{n=1}^\infty \frac{x^n}{n}$\\
$(xf')' = \sum_{n=1}^\infty x^{n-1} = \frac{1}{1-x}$\\
$xf' = -\ln(1-x)+c$\\
Из $f(0): c=0$\\
Тогда $f' = -\frac{\ln(1-x)}{x}$\\
$\sum \frac{x^n}{n^2} = -\int_0^x \frac{1-t}{t}\df t$\\
$\sum \frac1{n^2} = -\int_0^1 \frac{1-t}{t}\df t$\\
\section{Ряды Тейлора}
% лекция 15 (этот год) 0:36
\textbf{Определение}\\
$f$ раскладывается в степенной ряд в окрестности $x_0$,\\
если $\ex (a_n), U(x_0): \fall x\in U(x_0)\ f(x) = \sum a_n(x-x_0)^n$\\
\textbf{Замечание}\\
$f$ -- раскладывается $\Rto f\in C^\infty(U(x_0))$\\
\textbf{Теорема о единственности}\\
$f$ -- раскладывается $\Rto$ ряд определен однозначно\\
($\ex! a_n$)\\
\textbf{Доказательство}\\
$\sum a_n(x-x_0)^n = f(x)$\\
Тогда $a_n = \frac{f^{(n)}(x_0)}{n!}$\\
\textbf{Определение}\\
Пусть $f\in C^{\infty} ((x_0 - \eps, x_0 + \eps))$\\
Тогда ряд $\sum_{n=0}^\infty \frac{f^{(n)}(x_0)}{n!}(x-x_0)^n$ -- ряд Тейлора функции $f$ в точке(окрестности точки) $x_0$\\
\textbf{Замечание}
\begin{enumerate}
    \item Ряд Тейлора может сходиться <<не туда>> (не к исходной функции)\\
    К примеру, $f(x) = \left\{\begin{array}{cc}
        e^{-\frac1{x^2}}, & x\neq0\\
        0, & x = 0
    \end{array}\right.$\\
    $f\in C^\infty(\Rset)$\\
    $\us{x\neq 0}{f^{(k)}(x)} = \ub{\xrto[x\rto 0]{} 0}{P_k(\frac1x)e^{e^{-\frac1{x^2}}}}, P_k$ -- многочлен степени $\leq 3k$\\
    По следствию из т. Лагранжа функция $k$ раз дифференцируема и $f^{(k)}(0) = 0$\\
    Тогда у $f(x)$ ряд Тейлора $\equiv 0$\\
    Т.е. существуют $f \in C^\infty$, которые не раскладываются в ряд\\
    (функции, раскладывающиеся в ряд -- \textit{аналитические})
    % лекция 15 (этот год) 0:54
    \item Ряд Тейлора может расходиться при всех $x\neq x_0$\\
    $f(t) = \int_0^\infty \frac{e^{-x}}{1+t^2x}\df x$\\
    //todo продолжить 00:55:24
\end{enumerate}
\section{Диффеоморфизм}
%лекция 5-6 (прошлый год) 0:0
\textbf{Определение}\\
\textit{Область} в $\Rset^m$ -- открытое связное множество\\
$f: O \subset \Rset^m \rto \Rset^m, O$ -- область\\
$f$ -- \textit{диффеоморфизм}, если $f$ -- обратимо, $f, f^{-1}$ -- дифференцируема\\
\textbf{Замечание}\\
Если это так, то $f^{-1}\circ f = \nm{id}$\\
$(f^{-1})'(y) = (f'(x))^{-1}, y = f(x)$\\
\textbf{Лемма (о <<почти>> локальной инъективности)}\\
$F: O \subset \Rset^m \rto \Rset^m, O$ -- область, $x_o \in O, F$ -- дифференцируемо в $x_0$\\
$\det F'(x_0) \neq 0$\\
Тогда $\ex C>0, \delta > 0: \fall h: |h| < \delta\ |F(x_0+h)-F(x)| \geq c|h|$\\
\textbf{Доказательство}
\begin{enumerate}
    \item $f$ -- линейное\\
    Тогда $|h| = |f^{-1}\circ F\cdot h| \leq \|F^{-1}\||Fh|$\\
    $|F(x_0+h) - F(x_0)| = |Fh| \geq \frac1{\|F^{-1}\|}|h|$\\
    $\delta$ -- любое
    \item $|F(x_0+h)-F(x_0)| = |F'(x_0)h+\alpha(h)|h|| \geq \ub{\text{из п.1}}{C}|h|-|\alpha(h)||h|$\\
    Берем $\delta$, чтобы $|\alpha(h)| \leq \frac C2$
\end{enumerate}
\textbf{Замечание}\\
$\fall x\ \det F'(x) \neq 0$, то отсюда не следует инъективность\\
В далеких точках значения могут совпадать\\
\textbf{Пример}\\
$(x_1, x_2) \rto (x_1^2-x_2^2, 2x_1x_2)$\\
$\det F'(x) = 2(x_1^2 + x_2^2)$\\
Данное отображение склеивает точки\\
\textbf{Теорема о сохранении области}\\
$F: O \subset \Rset^m \rto \Rset^m, O$ -- \underline{открытое}, $\fall x\ F$ -- дифференцируемый в $x$ и $\det F'(x) \neq 0$\\
Тогда $F(O)$ -- открытое множество\\
\textbf{Доказательство}\\
Пусть $x_0 \in O, y_0 \in F(x_0)$\\
Проверим, что $y_0$ -- внутренняя точка $F(O)$\\
По лемме $\ex C, \delta: \fall h \in \ol{B(0, \delta)}$\\
$|F(x_0 + h) - F(x_0)| \geq C|h|$\\
$r:= \frac12 \nm{dist}(y_0, \ub{\text{компакт}}{F(\ub{\text{сфера}}{S(x_0, \delta)})})$\\
$r > 0$ -- потому что $\dist = \inf$ на компакте, а значит $\inf$ реализуется\\
Проверим, что $B(y_0, r) \subset F(O)$\\
Т.е. проверим, что $\fall y \in B(y_0, r)\ \ex x \in B(x_0, \delta): F(x) = y$\\
Рассмотрим $g(x) := |F(x)-y|^2, y \in B(y_0, r)$ -- функция на $\ol{B(x_0, \delta)}$\\
(надеемся, что она обращается в 0)\\
$g(x_0) = |F(x_0) - y| < r^2$\\
%лекция 5-6 (прошлый год) 41:58\\
$\fall x \in S(x_0, \delta)\ \gamma(x) \geq r^2$\\
Тогда $\min g$ достигается(т.к. функция на компакте) внутри $B(x_0, r)$\\
В этой точке все частные производные = 0\\
Пусть в точке $x$ достигается минимум\\
$g(x) = (F_1(x)-y_1)^2 + \ldots + (F_m(x)-y_m)^2$\\
$\left\{\begin{array}{ccc}
0 & = & \ppart g{x_1} = 2(F_1-y_1)\ppart{F_1}{x_1} + \ldots + 2(F_m-y_m)\ppart{F_m}{x_1}\\
& \vdots &\\
0 & = & \ppart g{x_m} = 2(F_1-y_1)\ppart{F_1}{x_m} + \ldots + 2(F_m-y_m)\ppart{F_m}{x_m}
\end{array}\right.$\\
$(F(x)-y)^T F'(x) = 0$\\
Т.е. $\det F'(x)\neq 0$, то $g(x) = F(x)-y = 0$\\
Отсюда $g(x)$ достигает $0$\\
\textbf{Замечание}\\
$F$ -- непрерывное $\LRto \fall \ub{\text{откр.}}W\ F^{-1}(W)$ -- открытое\\
(из определения)\\
\textbf{Замечание}\\
Если  $O$ -- связное, $F$ -- непрерывное\\
Тогда $F(O)$ -- связное\\
(Отсюда область переходит в область)\\
\textbf{Доказательство}\\
Пусть это не так\\
Тогда $F(O) = W_1 \cup W_2$\\
Тогда $O = F^{-1}(W_1)\cup F^{-1}(W_2), F^{-1}(W_1), F^{-1}(W_2)$ -- открытые и не пересекающиеся\\
Но это невозможно из связности\\
Тогда $F(O)$ -- связное\\
%лекция 5-6 (прошлый год) + лекция 7-8 (этот год) 0:00
\textbf{Следствие}\\
$F: \ub{\text{откр}}O \subset \Rset^m \rto \Rset^l, l < m$\\
$F \in C^1(O)$\\
$\fall x \in O\ \rg F'(x) = l$ ($\rg$ -- ранг матрицы)\\
Тогда $F(O)$ -- открытое\\
\textbf{Доказательство}\\
%% Доказательство прошлого года:
% Если что, смотри лекцию 7-8 (этот год) 0:20
Пусть $x_0 \in O$\\
Проверим, что $F(x_0)$ -- внутренняя точка в $F(O)$\\
$\rg F'(x_0) = l$\\
Н.у.о. пусть ранг реализуется на первых $l$ столбцах\\
Т.е. $\det (\ppart{F_i}{x_j})_{i,j \in 1\ldots l} \neq 0$\\
Тогда $\ex U(x_0): \fall x \in U(x_0)\ \det (\ppart{F_i}{x_j}(x))_{i,j \in 1\ldots l} \neq 0$\\
Тогда $F(x_0)$ -- внутренняя в $F(U(x_0))$:\\
Рассмотрим $U_l:=\{ (t_1, \ldots, t_l) : (t_1, \ldots, t_l, (x_0)_{l+1}, \ldots, (x_0)_{m}) \in U(x_0)\}$ -- $l$-мерная окрестность\\
$\ot F: U_l \rto \Rset^l$\\
$(t_1, \ldots, t_l) \mapsto F(t_1, \ldots, t_l, (x_0)_{l+1}, \ldots, (x_0)_m)$\\
$\ppart{\ot F_i}{t_j} = \begin{pmatrix}
    \ppart{F_i}{x_j}(t_1, \ldots, t_l, (x_0)_{l+1}, \ldots, (x_0)_m)
\end{pmatrix}$\\
%лекция 5-6 (прошлый год) 1:16:0 || лекция 7-8 (этот год) 0:30
\textbf{Теорема о дифференцировании обратного отображения}\\
$F: O\subset \Rset^m \rto \Rset^m, O$ -- область\\
$F \in C^r(O), r \in \Nset \cup \{\infty\}$\\
Пусть $F$ -- обратимо и невырождено ($\fall x\ \det F'(x) \neq 0$)\\
Тогда $F^{-1} \in C^r$ (отсюда $(F^{-1})'(y) = (F'(x))^{-1})$\\
\textbf{Доказательство}\\
Индукция по $r$\\
База: $r = 1$\\
Пусть $S = F^{-1}$\\
$S$ -- непрерывна по теореме о сохранении области (по топологическому определению)\\
Возьмем $x_0 \in O, y_0 = F(x_0) \in O_1, O_1 = F(O)$\\
По лемме $\ex C, \delta: \fall x \in B(x_0, \delta)\ |F(x)-F(x_0)| \geq C|x-x_0|$\\
$A = F'(x_0)$\\
$\ub{y}{F(x)}-\ub{y_0}{F(x_0)} = A(\ub{S(y)}x-\ub{S(y_0)}{x_0}) + \alpha(x)|x-x_0|$\\
$S(y)-S(y_0) = A^{-1}(y-y_0)- A^{-1}\alpha(S(y))|S(y)-S(y_0)|$\\
%лекция 5-6 (прошлый год) 1:54
Надо проверить: $\beta(y) = |S(y)-S(y_0)|A^{-1}\alpha(S(y)) = o(|y-y_0|)$\\
Пусть $|x-x_0| = |S(y)-S(y_0)|<\delta$ -- выполнено при $y$ близких к $y_0$\\
$|\beta(y)| = |S(y)-S(y_0)||A^{-1}\alpha(S(y))| \leq \frac1C |F(x)-F(x_0)|\|A^{-1}\||\alpha(S(y))| = \frac{\|A^{-1}\|}C |y-y_0||\alpha(S(y))| = o(|y-y_0|)$\\
Отсюда $S$ -- дифференцируемо\\
Проверим, что в $S \in C^1, S' = A^{-1}$\\
$y \ub{\text{непр}}\mapsto S(y) = x \ub{\text{непр}}\mapsto T'(x) = A \ub{\text{непр}}\mapsto A^{-1}$\\
Индукционный переход:\\
% лекция 5-6 (прошлый год) 2:11:0 || лекция 7-8 (этот год) 0:33
Проведем цепочку вычислений:\\
$y \mapsto F^{-1}(y)=x\mapsto F'(x)=A\mapsto A^{-1}=(F'(x))^{-1} = (F^{-1})'(y)$\\
Переходы -- композиции непрерывных отображений\\
Т.о.:\\
Пусть $F^{-1} \in C^{r-1}$\\
\textbf{Лемма}\\
$F(x_0+h)-F(x_0) - F'(x_0)h| \leq M|h|$\\
$M=\sup_{x_1\in [x_0, x_0+h]} \|F'(x_1)-F'(x_0)$\\
//todo доказать\\
% лекция 7-8 (этот год) 0:40
\textbf{Теорема о локально обратимости}\\
Пусть $F \in C^1(O, \Rset^m)$ (Т.е. $F: O \rto \Rset^m, F \in C^1)$\\
$x_0 \in O$\\
$\det F'(x_0) \neq 0$\\
Тогда $\ex U(x_0): F\vl_{U(x_0)}$ -- диффеоморфизм\\
\textbf{Доказательство}\\
$\ex V(x_0): \fall x \in V\ \det F(x) \neq 0$\\
(не представляю, откуда)\\
По предыдущей теореме достаточно построить окрестность $x_0$, где $F$ -- обратимо\\
$F'(x_0)$ -- невырожденный\\
Тогда $\ex c: \fall h\ |F'(x_0)-h| \geq c|h|$\\
Тогда $\ex U(x_0) = V(x_0) \cap B(x_0, r) \subset O$ такая, что $\fall x \in U(x_0)\ \|F'(x)-F'(x_0)\| < \frac c4$ и попрежнему $\det F'(x) \neq 0$\\
Проверяем, что $F$ -- обратимо на $U(x_0)$\\
$x \in U(x_0), y = x+h \in U(x_0)$\\
$F(y)-F(x) = (F(x+h)-F(x)-F'(x)h) + (F'(x)h-F'(x_0)h)+F'(x_0)h$\\
$|F(y)-F(x)|\geq |F'(x_0)h| - |F(x+h)-F(x)-F'(x)h| - |(F'(x)-F'(x_0))h|$ (неравенство треугольнка)\\
$F(x_0+h)-F(x_0) - F'(x_0)h| \leq M|h|$\\
$M=\sup_{x_1\in [x_0, x_0+h]} \ub{\leq \|F'(x_1)-F'(x_0)\| + \|F'(x)-F'(x_0)\|}{\|F'(x_1)-F'(x_0)} \leq \frac c2$\\
$|F(y)-F(x)| \geq c|h|-\frac c2|h| - \frac c4|h| = \frac c4|h|$ -- т.е. $F(y)\neq F(x)$, а значит точки не склеиваются\\\\
% лекция 9 (этот год) 0:0
Пусть $x = (x_1, \ldots, x_m) \in \Rset^m$\\
$y = (y_1, \ldots, y_n) \in \Rset^n$\\
$F:O\subset \Rset^{m+n} \rto \Rset^n$ -- дифференцируема, $O$ -- открытое\\
$\df F = \begin{pmatrix}
    \ppart {F_1}{x_1} & \ldots & \ppart {F_1}{x_m} & \ppart {F_1}{y_1} & \ldots & \ppart {F_1}{y_n}\\
    \vdots & \ddots & \vdots & \vdots & \ddots & \vdots\\
    \ppart {F_n}{x_1} & \ldots & \ppart {F_n}{x_m} & \ppart {F_n}{y_1} & \ldots & \ppart {F_n}{y_n}
\end{pmatrix} =: (F'_x, F'_y)$\\
\textbf{Теорема о неявном отображении}\\
$F: O \subset \Rset^{m+n} \rto \Rset^n, F \in C^r, r \in \Nset \cup \{\infty\}$\\
Пусть $(a, b): F(a,b) = 0$\\
$\det F'_y(a,b)\neq 0$\\
Тогда 
\begin{enumerate}
    \item $\ex P \subset \Rset^m, a \in P, \ex Q \subset \Rset^n, b \in Q$\\
    $\ex! \phi: P \rto Q \in C^r$ -- гладкое\\
    $\fall x \in P\ F(x, \phi(x)) = 0$
    \item $\phi'(x) = -(F'_y(x,\phi(x)))^{-1}F'_x(x,\phi(x))$
\end{enumerate}
\textbf{Доказательство}\\
Построим $\Phi: O \rto \Rset^{m+n}$\\
$(x,y) \rto (x, F(x, y))$\\
$\Phi(a,b) = (0,0)$\\
$\Phi' = \begin{pmatrix}
    E & 0\\
    F'_x & F'_y
\end{pmatrix}$\\
$\det \Phi'(a,b) \neq 0$\\
$\ex \ot U(a,b): \Phi\vl_{\ot U}$ -- диффеоморфизм\\
Можно считать, что $\ot U = P_1 \times Q$\\
% лекция 9 (этот год) 0:25
$\ot V = \Phi(\ot U)$ -- открытое\\
$\ex \Psi: \ot V \rto \ot U, \Psi = \Phi^{-1}\subset C^r$\\
$\Phi, \Psi$ не меняют первые $n$ координат\\
$\Psi(u,v) = (u, H(u,v)), H: \ot V \rto \Rset^m, H \in C^r$\\
Пусть $P = \ot V \cap (\Rset^m \times \{\0_n\})$ (подмножество $\ot V$, где последние $n$ координат -- нули)\\
$\phi(x):= H(x,0)$\\
Что $F(x, \phi(x)) = 0$ -- тривиально\\
$F(x, \phi(x))' = (F'_x, F'_y) \begin{pmatrix}
    E_m\\\phi'
\end{pmatrix} = 0$\\
$F'_x + F'y\phi' = 0$\\
Докажем единственность\\
$x \in P, y \in Q$\\
$F(x,y) = 0$\\
$\Phi(x,y) = (x,0)$\\
$(x,y) = \Psi\Phi(x,y) = \Psi(x,0) = (x, H(x,0)) = (x, \phi(x))$\\
\textbf{Замечание}\\
Пусть есть система\\
$\left\{\begin{array}{l}
    F_1(x_1, \ldots, x_{m+n}) = 0\\
    \vdots\\
    F_m(x_1, \ldots, x_{m+n}) = 0\\
\end{array}\right.$\\
$\rg F'(x_0) = m$\\
Н.у.о. пусть ранг реализуется на $n$ последних переменных\\
Обозначим последние $n$ переменных $x_i$ как $y_{j}$\\
Пусть $a = ((x_0)_1, \ldots, (x_0)_m)$\\
Тогда для $\ex U(a), V(b)$\\
$\fall x \in U(a)\ \ex y\in V(b)$ -- решение, которое гладко зависит от $x$\\
\textbf{Определение}\\
Пусть $k < m$\\
$M \subset \Rset^m$ -- \textit{простое $k$-мерное многообразие в $\Rset^m$}, если\\
$\ex O \subset \Rset^k$ -- область\\
$\ex \Phi: O \rto M$ -- биекция, гомеоморфизм ($\Phi, \Phi^{-1}: \Phi(O)\rto O$ -- непрерывно)\\
$\Phi$ -- \textit{параметризация}\\
\textbf{Определение}\\
Пусть $k < m$\\
$M \subset \Rset^m$ -- \textit{простое $k$-мерное $C^r$-гладкое многообразие в $\Rset^m$}, если\\
$\ex O \subset \Rset^k$ -- область\\
$\ex \Phi: O \rto M$\\
$\Phi \in C^r$ -- гомеоморфизм ($\Phi, \Phi^{-1}: \Phi(O)\rto O$ -- непрерывно)\\
$\fall t \in O\ \rg \Phi'(t) = k$\\
% лекция 10 (этот год) 0:00 (пример про тор)
\textbf{Пример}
\begin{itemize}
    \item $x^2+y^2+z^2 = 1, z > 0$\\
    $(x,y) \os \Phi \mapsto (x,y, \sqrt{1-x^2-y^2})$\\
    $\df \Phi = \begin{pmatrix}
        1 & 0\\
        0 & 1\\
        -\frac{x}{\sqrt{1-x^2-y^2}} & -\frac{y}{\sqrt{1-x^2-y^2}}
    \end{pmatrix}$
    \item Цилиндр\\
    $x=R\cos t, y = R\sin t, z = z$\\
    $F: \Rset^2 \rto \Rset^3$\\
    $(t, z) \os F\mapsto (R\cos t, R\sin t, z)$\\
    $\df F = \begin{pmatrix}
        -R\sin t & 0\\
        R\cos t & 0\\
        0 & 1
    \end{pmatrix}$
    \item Шар -- не является\\
    % лекция 10 (этот год) 0:19
    Возьмем какую-то точку $a$ в исходном множестве\\
    На шаре ей будет соответствовать точка $A$\\
    Удалим точку $a$ и $A$ из исходного множества и шара соответственно\\
    В исходном множестве возьмем петлю вокруг точки $a$\\
    Она не может быть стянута в одну точку\\
    С другой стороны, петля вокруг $a$ на шаре -- может
\end{itemize}
% лекция 11 (этот год) 0:00
\textbf{Теорема}\\
$M\subset \Rset^m, 1 \leq k < m$\\
$1 \leq r \leq +\infty$\\
Тогда $\fall p \in M$ эквивалентно:
\begin{enumerate}
    \item $\ex U = U(p) \subset \Rset^m$ (откр. множество)\\
    $M \cap U(p)$ -- простое $k$-мерное $C^r$-гладкое многообразие в $\Rset^m$
    \item $\ex \ot U(p)\subset \Rset^m$ и $\ex f_1,\ldots, f_{m-1}: \ot U\rto \Rset \in C^r$\\
    $x\in M\cap \ot U(p)\LRto \left\{\begin{array}{c}
        f_1(x_1,\ldots, x_m) = 0\\
        \vdots\\
        f_{m-k}(x_1,\ldots, x_m) = 0
    \end{array}\right.$\\
    и $\grad f_1(p), \ldots, \grad f_{m-k} (p)$ -- линейно независимые
\end{enumerate}
\textbf{Доказательство $1\Rto 2$}\\
$\Phi: O\subset \Rset^k \rto \Rset^m, \Phi\in C^r$ -- параметризация $M\cap U$\\
$\phi_1, \ldots, \phi_m$ -- координатные функции $\Phi$\\
$p = \Phi(t^0)$\\
Н.у.о. пусть $(\ppart{\phi_i}{t_k}(t^0))_{i,j=1\ldots k}$ -- невырожденная матрица\\
(по непрерывности можно считать, что выполнено на всем $O$)\\
$L: \Rset^m \rto \Rset^k$ -- проекция $x\mapsto (x_1,\ldots, x_k)$\\
Посмотрим на $L\circ \Phi$:\\
$(L\circ \Phi)' = (\ppart{\phi_i}{t_k}(t^0))_{i,j=1\ldots k}$\\
$\det (L\circ \Phi)' \neq 0$ на $O$\\
При необходимости сузим $O$ на окрестность точки $t^0$, чтобы $L\circ \Phi$ было диффеоморфизмом\\
$W \subset O$ -- область определения $L\circ \Phi$\\
$V = L\circ \Phi(W) \subset \Rset^k$\\
$\ex \Psi = (L\circ \Phi)^{-1}, \Psi\in C^r$\\ 
% лекция 11 (этот год) 0:30
Для $x \in V$ однозначно задано $\Phi(\Psi(x)) \in M\cap U$\\
Т.е. множество $\Phi(W)$ -- график некоторого $H: V\rto \Rset^{m-k}$\\
Для $x'\in V$ $\Phi(\Psi(x)) = (x', H(x')) \Rto H\in C^r$\\
$\Phi(W)$ -- множество, открытое в $M$, т.к. $\Phi$ -- гомеоморфизм\\
Значит $\ex \ot U$ -- открытое в $\Rset^m: \Phi(W) = M \cap \ot U$ (теорема об открытых множествах в пространствах и подпространствах)\\
Можно считать, что $\ot U \subset \ub{\text{открытое в $\Rset^m$}}{V\times \Rset^{m-k}}$\\
(пусть это не так. Тогда возьмем $U' := U \cap V\times \Rset^{m-k}$)\\
Определим $f_j: \ot U \rto R$\\
$f_j(x)=H_j(L(x))-x_{k+j}, f_j \in C^r$\\
Тогда $x\im M\cap \ot U \LRto \fall j\ f_j(x) = 0$\\
$\begin{pmatrix}
    \grad f_1\\
    \vdots\\
    \grad f_{m-k}
\end{pmatrix} = \begin{pmatrix}
    T_{(m-k)\times k} & \nm{diag}(-1, \ldots, -1)
\end{pmatrix}$ -- градиенты(строки) линейно независимые\\
\textbf{Доказательство $2 \Rto 1$}\\
$(\ppart{f_i}{x_j})_{i=1,\ldots m-k, j=1\ldots m}$ -- матрица, у которогой строки -- градиенты $f_i$\\
Можно считать, что $\det(\ppart{f_i}{x_j}(p))_{i=1,\ldots m-k, j={k+1}\ldots m} \neq 0$\\
% лекция 11 (этот год) 0:58
Тогда из теоремы о неявном отображении:\\
$\ex P((p_1,\ldots, p_k)) \subset \Rset^k\ \ex Q((p_{k+1}, \ldots, p_m))\subset \Rset^{m-k}$\\
$\ex H: P \rto Q:\ \Phi(u, H(u)) = 0$, где $u \in P, F=(f_1,\ldots, f_{m-k})$, равносильно $\fall x \in M\cap (P\times Q)\ F(x)=0$, т.е. $x=(u, H(u))$\\
Т.е. $\Phi: P\rto P\times Q \subset \Rset^m$\\
$u\mapsto (u, H(u)), H\in C^r$\\
$\Phi \in C^r$\\
$\Phi$ -- гомеоморфизм, т.к. $\Phi^{-1}$ -- это проекция(а она непрерывна)\\
$\rg \Phi' = k$\\
\textbf{Следствие (о двух параметризациях)}\\
Пусть $M \subset \Rset^m$ -- $k$-мерное простое $C^r$-гладкое многообразие\\
$p\in M, \ex U(p)$\\
$\Phi_1: O_1\subset \Rset^k \rto U(p)$ -- параметризация, $\Phi_1 \in C^r$\\
$\Phi_2: O_2\subset \Rset^k \rto U(p)$ -- параметризация, $\Phi_2 \in C^r$\\
(обе действуют инъективно)\\
Тогда $\ex$ диффеоморфизм $\Theta: O_1 \rto O_2, \Theta \in C^r$\\
$\Phi_1 = \Phi_2\circ \Omega$\\
\textbf{Доказательство}\\
$\ex \Theta = \Phi_2^{-1}\circ \Phi_1$ -- гомеоморфизм\\
$\Theta=\Psi_2\circ L_2\circ \Phi_1\in C^r$\\
$\Theta^{-1} = \Psi_1 \circ L_1 \circ \Phi_2 \in C^r$\\
% лекция 14 (этот год) 0:0
\textbf{Лемма}\\
$\Phi: O\subset \Rset^k \rto \Rset^m, O$ -- открытое множество\\
$\Phi$ -- это $C^1$-параметризация некоторого $M$ -- простого гладкого многообразия в $\Rset^m$\\
$t_0 \in O, \Phi(t_0) = p\in M$\\
Тогда $\Phi'(t_0): \Rset^k \rto \Rset^m$ -- не зависит от $\Phi$ и представляет собой $k$-мерное линейное подпространство в $\Rset^m$\\
\textbf{Доказательство}\\
$\fall \Phi$ -- гладкая параметризация $\rg \Phi' = k$ -- образ $k$-мерный\\
Пусть есть $\Phi_1$ и $\Phi_2$ -- две параметризации\\
$\ex \psi$ -- диффеоморфизм\\
$\psi: O_1 \rto O_2$\\
$\Phi_1 = \Phi_2 \circ \psi$\\
$\Phi'_1 = \Phi'_2 \psi'$\\
Заметим, что $E = (\psi^{-1}\psi)' = (\psi^{-1})' \psi'$\\
Тогда $\psi'$ и $(\psi^{-1})'$ -- обратимые\\
$\Phi'_1(\Rset^k) = \Phi'_2\psi(\Rset^k) = \Phi'_2(\Rset^k)$ (т.к. $\psi(\Rset^k) = \Rset^k$)\\
\textbf{Определение}\\
$M$ -- простое $k$-мерное гладкое многообразие в $\Rset^m$\\
$p \in M, \Phi$ -- параметризация, $\Phi(t_0) = p$\\
Рассмотрим в $\Rset^m$ подпространство $\Phi'(t_0)(\Rset^k)$\\
Оно называется \textit{касательным подпространством} к $M$ в точке $p$\\
Обозначается $T_pM$\\
Множество $p+T_pM$ будем называть \textit{афинным} пространством (касательное линейное многообразие)\\
\textbf{Замечание}
\begin{enumerate}
    \item Если есть $v \in T_pM$, то $\ex$ гладкий путь $\gamma_v: [-\eps, \eps] \rto M\subset \Rset^m, \gamma_v(0) = 0$ и $\gamma'_v(0) = v$\\
    \textbf{Доказательство}\\
    $\rg \Phi'(t_0) = k \Rto \ex u \in \Rset^j: \Phi'(t_0) u = v$\\
    $\ot\gamma_v(s) = t_0 + t_0 + us$\\
    $\gamma_v = \Phi\circ \ot \gamma_v$\\
    $\gamma'_v(0) = \Phi'\ot\gamma'_v(0)= \Phi'(u) = v$
    \item $\gamma: [-\eps, \eps] \rto M$ -- гладкий путь\\
    $\gamma(0) = p$\\
    Тогда $\gamma'(0) \in T_pM$\\
    % лекция 14 (этот год) 0:30
    \textbf{Доказательство}\\
    Что-то рукомахательное
    \item Рассмотрим касательное пространство к графику\\
    $f: O \subset \Rset^m \rto \Rset$\\
    $(x, y=f(x))$ -- поверхность в $\Rset^{m+1}$ -- это простое $k$-мерное гладкое многообразие в $\Rset^{m+1}$:\\
    $\Phi: O\subset \Rset^m \rto \Rset^{m+1}, \Phi(x) = (x, f(x))$ -- параметризация\\
    Тогда линейное касательное многообразие в $(x_0, y_0)$, где $y = f(x_0)$, задается уравнением $y-y_0 = \ppart{}{x_1}f(x_0)(x_1-x_1^0) + \ldots + \ppart{}{x_m}f(x_0)(x_m-x_m^0)$\\
    \textbf{Доказательство}\\
    $\Phi(x) = \begin{pmatrix}
        x_1\\\vdots\\x_m\\f(x)
    \end{pmatrix}$\\
    $\Phi' = \begin{pmatrix}
    &E_m&\\
    \ppart{}{x_1}f & \ldots & \ppart{}{x_m}f
    \end{pmatrix}$\\
    $a \in \Rset^m$\\
    $\Phi'(x)\cdot a = \begin{pmatrix}
        a_1\\\vdots\\a_m\\\ppart{f}{x_1} a + \ldots + \ppart{f}{x_m}a
    \end{pmatrix}$\\
    Линейное многообразие -- множество векторов, перпендикулярных вектору
    $\begin{pmatrix}
        \ppart{}{x_1}f(x_0), \ldots, \ppart{}{x_m}f(x_0), -1
    \end{pmatrix}$\\
    % лекция 14 (этот год) 0:50
    Тогда надо проверить, что $\begin{pmatrix}
        \ppart{}{x_1}f(x_0), \ldots, \ppart{}{x_m}f(x_0), -1
    \end{pmatrix} \cdot \Phi'(x)a = 0$
    \item $f: O \subset \Rset^m \rto \Rset, O$ -- открытое\\
    $p \in O$\\
    $U(p)$ -- $m-1$-мерное простое гладкое многообразие в $\Rset^m$б заданное уравнением $f(x) = 0$\\
    При это выполняется:\\
    $f(p) = 0$\\
    $\grad f(p) \neq 0$\\
    Тогда линейное касательное многообразие в точке $p$ есть $(*)$ $f'_{x_1}(p)(x_1-p_1) + \ldots + f'_{x_m}(p)(x_m-p_m) = 0$\\
    \textbf{Доказательство}\\
    Н.у.о. пусть $f'_{x_m}(p) \neq 0$\\
    $\ex \phi: U(p_1, \ldots, p_{m-1}) \subset \Rset^{m-1} \rto \Rset$ -- вычисляет последнюю координату\\
    $f(x_1, \ldots, x_{m-1}, \phi(x_1, \ldots, x_{m-1})) = 0 \LRto (x_1, \ldots, x_{m-1}) \in U(p)$, где $x_m = \phi(x_1, \ldots, x_{m-1})$\\
    Т.е. $U(p)$ -- есть график $\phi$ над $U(p_1, \ldots, p_{m-1})$\\
    Тогда уравнение касательного линейного многообразия $(**)$ $x_m - p_m = \phi'_{x_1}(x_1-p_1) + \ldots + \phi'_{x_{m-1}}(x_{m-1}-p_{m-1})$\\
    Мы знаем, что $f(x_1, \ldots, x_{m-1}, \phi(x_1, \ldots, x_{m-1})) = 0$\\
    $f'_{x_1} + f'_{x_m} \phi'_{x_1} = \ldots = f'_{x_{m-1}} + f'_{x_m} \phi'_{x_{m-1}} = 0$\\
    Домножим $(**)\cdot f'_{x_m} = (*)$, ч.т.д.
\end{enumerate}
% лекция 14 (этот год) 1:09
\subsection{Относительный (= условный) экстремум}
\textbf{Определение}\\
$f: E \subset \Rset^{m+n} \rto \Rset$\\
$\Phi: E \rto \Rset^n$\\
$M_\Phi = \{ x \in \Rset^{n+m}: \Phi(x) = 0\}$\\
($\Phi(x) = 0$ -- уравнения связи)\\
$x_0 \in E$\\
$\Phi(x_0) = 0$, т.е. $x_0 \in M_\Phi$\\
$x_0$ -- точка относительного локального максимума, если $x_0$ -- локальный максимум $f\vl_{M_\Phi}$\\
Т.е. $\ex U(x_0): \fall x\in U(x_0): \Phi(x) = 0\ f(x_0) \geq f(x)$\\
\section{Экспонента}
% лекция 13 (этот год) 0:51
\textbf{Определение}\\
$\exp(z) := \sum_{n=0}^{+\infty} \frac{z^n}{n!}, R = +\infty$\\
\textbf{Свойства}
\begin{enumerate}
    \item $\exp(0) = 1$
    \item $\exp(z)' = \exp(z)$
    \item $\ol{\exp(z)} = \ol{\sum_{n=0}^{+\infty} \frac{z^n}{n!}} = \sum_{n=0}^{+\infty} \ol{\frac{z^n}{n!}} = \sum_{n=0}^{+\infty} \frac{\ol{z}^n}{n!} = \exp(\ol z)$
    \item $\fall z, w \in \Cset\ \exp(z+w) = \exp(z)\exp(w)$\\
    \textbf{Доказательство}\\
    $\exp z \exp w = \sum_{n=0}^{+\infty} \frac{z^n}{n!} \sum_{n=0}^{+\infty} \frac{w^n}{n!} = \sum_{k=0}^{\infty} (\frac{z^k}{k!} + \frac{z^{k-1}w^1}{(k-1)!1!} + \ldots + \frac{w^k}{k!}) = \ldots$ -- по теореме Коши о произведении рядов\\
    $\ldots = \sum_{k=0}^{\infty} (\frac{k!z^k}{k!} + \frac{k!z^{k-1}w^1}{(k-1)!1!} + \ldots + \frac{k!w^k}{k!})\frac1{k!} = \sum_{k=0}^{\infty} (C_k^0 z^k + C_k^1 z^{k-1}w^1 + \ldots + C_k^k w^k)\frac1{k!} = \sum \frac{(z+w)^k}{k!}$
\end{enumerate}
Тогда из 1, 2, 4 $\exp$ -- показательная функция из теоремы о существовании показательной функции\\
\textbf{Следствие}\\
$\fall z\in \Cset\ \exp(z) \neq 0$\\
\textbf{Доказательство}\\
Пусть $\ex z: \exp(z) = 0$\\
Тогда $\fall w\ \exp(z+w) = \exp(z)\exp(w) = 0$\\
Пусть $w = u-z$\\
Тогда $\fall u\ \exp(z+u-z) = \exp(u) = \exp(z)\exp(u-z) = 0$\\
$\exp(u) \equiv 0$ -- что неверно\\\\
Пусть $x\in \Rset$\\
Обозначим $e^{ix} = \alpha(x) + i\beta(x)$\\
Тогда $e^{-ix} = \ol{e^{ix}} = \alpha(x) - i\beta(x)$\\
$\alpha(x) = \frac{e^{ix}+e^{-ix}}{2}$\\
$\beta(x) = \frac{e^{ix}-e^{-ix}}{2i}$\\
Тогда $\alpha(x) = \frac12 \sum\frac{(ix)^n+(-ix)^n}{n!} = 1 - \frac{x^2}{2} + \frac{x^4}{4!} + \ldots$ (похоже на $\cos x$)\\
$\beta(x) = x - \frac{x^3}{3!} + \frac{x^5}{5!}  + \ldots$ (похоже на $\sin x$)\\
(желающие могут думать, что $x\in \Cset$)\\
Заметим, что $\alpha(x+y) = \alpha(x)\alpha(y) - \beta(x)\beta(y)$\\
$\beta(x+y) = \alpha(x)\beta(y) + \alpha(y)\beta(x)$\\
$\alpha^2(x) + \beta^2(y) = 1$\\
$(e^{ix})' = ie^{ix}$\\
Отображение $ie^{ix}$ -- вектор скорости\\
Тогда $e^{ix}$ описывает движение с постоянной скоростью по окружности единичной длины\\
\end{document}
