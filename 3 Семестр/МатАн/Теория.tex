\documentclass[12pt]{article}
\usepackage{bbold}
\usepackage{amsfonts}
\usepackage{amsmath}
\usepackage{amssymb}
\usepackage{color}
\setlength{\columnseprule}{1pt}
\usepackage[utf8]{inputenc}
\usepackage[T2A]{fontenc}
\usepackage[english, russian]{babel}
\usepackage{graphicx}
\usepackage{hyperref}
\usepackage{mathdots}
\usepackage{xfrac}


\def\columnseprulecolor{\color{black}}

\graphicspath{ {./resources/} }


\usepackage{listings}
\usepackage{xcolor}
\definecolor{codegreen}{rgb}{0,0.6,0}
\definecolor{codegray}{rgb}{0.5,0.5,0.5}
\definecolor{codepurple}{rgb}{0.58,0,0.82}
\definecolor{backcolour}{rgb}{0.95,0.95,0.92}
\lstdefinestyle{mystyle}{
    backgroundcolor=\color{backcolour},   
    commentstyle=\color{codegreen},
    keywordstyle=\color{magenta},
    numberstyle=\tiny\color{codegray},
    stringstyle=\color{codepurple},
    basicstyle=\ttfamily\footnotesize,
    breakatwhitespace=false,         
    breaklines=true,                 
    captionpos=b,                    
    keepspaces=true,                 
    numbers=left,                    
    numbersep=5pt,                  
    showspaces=false,                
    showstringspaces=false,
    showtabs=false,                  
    tabsize=2
}

\lstset{extendedchars=\true}
\lstset{style=mystyle}

\newcommand\0{\mathbb{0}}
\newcommand{\eps}{\varepsilon}
\newcommand\overdot{\overset{\bullet}}
\DeclareMathOperator{\sign}{sign}
\DeclareMathOperator{\re}{Re}
\DeclareMathOperator{\im}{Im}
\DeclareMathOperator{\Arg}{Arg}
\DeclareMathOperator{\const}{const}
\DeclareMathOperator{\rg}{rg}
\DeclareMathOperator{\Span}{span}
\DeclareMathOperator{\alt}{alt}
\DeclareMathOperator{\Sim}{sim}
\DeclareMathOperator{\inv}{inv}
\DeclareMathOperator{\dist}{dist}
\newcommand\1{\mathbb{1}}
\newcommand\ul{\underline}
\renewcommand{\bf}{\textbf}
\renewcommand{\it}{\textit}
\newcommand\vect{\overrightarrow}
\newcommand{\nm}{\operatorname}
\DeclareMathOperator{\df}{d}
\DeclareMathOperator{\tr}{tr}
\newcommand{\bb}{\mathbb}
\newcommand{\lan}{\langle}
\newcommand{\ran}{\rangle}
\newcommand{\an}[2]{\lan #1, #2 \ran}
\newcommand{\fall}{\forall\,}
\newcommand{\ex}{\exists\,}
\newcommand{\lto}{\leftarrow}
\newcommand{\xlto}{\xleftarrow}
\newcommand{\rto}{\rightarrow}
\newcommand{\xrto}{\xrightarrow}
\newcommand{\uto}{\uparrow}
\newcommand{\dto}{\downarrow}
\newcommand{\lrto}{\leftrightarrow}
\newcommand{\llto}{\leftleftarrows}
\newcommand{\rrto}{\rightrightarrows}
\newcommand{\Lto}{\Leftarrow}
\newcommand{\Rto}{\Rightarrow}
\newcommand{\Uto}{\Uparrow}
\newcommand{\Dto}{\Downarrow}
\newcommand{\LRto}{\Leftrightarrow}
\newcommand{\Rset}{\bb{R}}
\newcommand{\Rex}{\overline{\bb{R}}}
\newcommand{\Cset}{\bb{C}}
\newcommand{\Nset}{\bb{N}}
\newcommand{\Qset}{\bb{Q}}
\newcommand{\Zset}{\bb{Z}}
\newcommand{\Bset}{\bb{B}}
\renewcommand{\ker}{\nm{Ker}}
\renewcommand{\span}{\nm{span}}
\newcommand{\Def}{\nm{def}}
\newcommand{\mc}{\mathcal}
\newcommand{\mcA}{\mc{A}}
\newcommand{\mcB}{\mc{B}}
\newcommand{\mcC}{\mc{C}}
\newcommand{\mcD}{\mc{D}}
\newcommand{\mcJ}{\mc{J}}
\newcommand{\mcT}{\mc{T}}
\newcommand{\us}{\underset}
\newcommand{\os}{\overset}
\newcommand{\ol}{\overline}
\newcommand{\ot}{\widetilde}
\newcommand{\vl}{\Biggr|}
\newcommand{\ub}[2]{\underbrace{#2}_{#1}}

\def\letus{%
    \mathord{\setbox0=\hbox{$\exists$}%
             \hbox{\kern 0.125\wd0%
                   \vbox to \ht0{%
                      \hrule width 0.75\wd0%
                      \vfill%
                      \hrule width 0.75\wd0}%
                   \vrule height \ht0%
                   \kern 0.125\wd0}%
           }%
}
\DeclareMathOperator*\dlim{\underline{lim}}
\DeclareMathOperator*\ulim{\overline{lim}}

\everymath{\displaystyle}

% Grath
\usepackage{tikz}
\usetikzlibrary{positioning}
\usetikzlibrary{decorations.pathmorphing}
\tikzset{snake/.style={decorate, decoration=snake}}
\tikzset{node/.style={circle, draw=black!60, fill=white!5, very thick, minimum size=7mm}}

\DeclareMathOperator{\Lin}{Lin}
\DeclareMathOperator{\Int}{Int}
\DeclareMathOperator{\grad}{grad}
\newcommand{\ppart}[2]{\frac{\partial #1}{\partial #2}}

\title{Математический анализ. Теория}
\author{Александр Сергеев}
\date{}
\begin{document}
\maketitle
\section{Friends and strangers graph}
Рассмотрим связный неориентированный граф $G$ из $n$ вершин\\
В $n-1$ вершину поставим нумерованные фишки. Одну вершину оставим свободной\\
Будем перемещать фишки по ребрам (перемещать фишку можно только в пустую вершину)\\
Для каких графов $G$ можно получить любое расположение фишек в графе из исходного?\\
\textbf{Теорема Уилсона}\\
Если граф $G$:
\begin{itemize}
    \item Нет точек сочленения
    \item Граф не двудольный
    \item Граф -- не цикл длины $n\geq 4$
    \item $G$ -- не следующий граф:\\
    \begin{tikzpicture}
        %Nodes
        \node[node] (a) at (0.5, 0) {};
        \node[node] (b) at (1.5, 0) {};
        \node[node] (c) at (0, 1) {};
        \node[node] (d) at (1, 1) {};
        \node[node] (e) at (2, 1) {};
        \node[node] (f) at (0.5, 2) {};
        \node[node] (g) at (1.5, 2) {};
        %Lines
        \draw[-] (a) -- (b);
        \draw[-] (a) -- (c);
        \draw[-] (c) -- (d);
        \draw[-] (d) -- (e);
        \draw[-] (e) -- (g);
        \draw[-] (b) -- (e);
        \draw[-] (c) -- (f);
        \draw[-] (f) -- (g);
    \end{tikzpicture}\\
\end{itemize}
то в таком графе возможно получить любое расположение фишек, иначе невозможно\\
\textbf{Доказательство необходимости}
\begin{itemize}
    \item Рассмотрим граф\\
    \begin{tikzpicture}
        %Nodes
        \node[node] (a) at (0, 0) {1};
        \node[node] (b) at (0, 2) {2};
        \node[node] (c) at (1, 1) {3};
        \node[node] (d) at (2, 0) {4};
        \node[node] (e) at (2, 2) {$\circ$};
        %Lines
        \draw[-] (a) -- (b);
        \draw[-] (a) -- (c);
        \draw[-] (c) -- (b);
        \draw[-] (d) -- (e);
        \draw[-] (c) -- (e);
        \draw[-] (c) -- (d);
    \end{tikzpicture}\\
    Заметим, что в таком графе 1 никогда не попадет в правую половину
    \item Рассмотрим двудольный граф\\
    \begin{tikzpicture}
        %Nodes
        \node[node] (a) at (0, 0) {1};
        \node[node] (b) at (1, 0) {2};
        \node[node] (c) at (2, 0) {3};
        \node[node] (d) at (0, 2) {4};
        \node[node] (e) at (1, 2) {5};
        \node[node] (f) at (2, 2) {$\circ$};
        %Lines
        \draw[-] (a) -- (d);
        \draw[-] (a) -- (e);
        \draw[-] (a) -- (f);
        \draw[-] (b) -- (d);
        \draw[-] (b) -- (e);
        \draw[-] (b) -- (f);
        \draw[-] (c) -- (d);
        \draw[-] (c) -- (e);
        \draw[-] (c) -- (f);
    \end{tikzpicture}\\
    Рассмотрим его как перестановку\\
    Поменяем 2 и $\circ$ местами\\
    \begin{tikzpicture}
        %Nodes
        \node[node] (a) at (0, 0) {1};
        \node[node, text=red] (b) at (1, 0) {$\circ$};
        \node[node] (c) at (2, 0) {3};
        \node[node] (d) at (0, 2) {4};
        \node[node] (e) at (1, 2) {5};
        \node[node, text=red] (f) at (2, 2) {2};
        %Lines
        \draw[-] (a) -- (d);
        \draw[-] (a) -- (e);
        \draw[-] (a) -- (f);
        \draw[-] (b) -- (d);
        \draw[-] (b) -- (e);
        \draw[-] (b) -- (f);
        \draw[-] (c) -- (d);
        \draw[-] (c) -- (e);
        \draw[-] (c) -- (f);
    \end{tikzpicture}\\
    Заметим, что величина (четность перестановки + номер доли, где находится $\circ$) не изменилась\\
    Отсюда из исходной перестановки невозможно получить перестановку\\
    \begin{tikzpicture}
        %Nodes
        \node[node, text=red] (a) at (0, 0) {2};
        \node[node, text=red] (b) at (1, 0) {1};
        \node[node] (c) at (2, 0) {3};
        \node[node] (d) at (0, 2) {4};
        \node[node] (e) at (1, 2) {5};
        \node[node] (f) at (2, 2) {$\circ$};
        %Lines
        \draw[-] (a) -- (d);
        \draw[-] (a) -- (e);
        \draw[-] (a) -- (f);
        \draw[-] (b) -- (d);
        \draw[-] (b) -- (e);
        \draw[-] (b) -- (f);
        \draw[-] (c) -- (d);
        \draw[-] (c) -- (e);
        \draw[-] (c) -- (f);
    \end{tikzpicture}
    \item Рассмотрим граф:\\
    \begin{tikzpicture}
        %Nodes
        \node[node] (a) at (0, 0) {1};
        \node[node] (b) at (1, 0) {2};
        \node[node] (c) at (1, 1) {3};
        \node[node] (d) at (0, 1) {$\circ$};
        %Lines
        \draw[-] (a) -- (b);
        \draw[-] (b) -- (c);
        \draw[-] (c) -- (d);
        \draw[-] (a) -- (d);
    \end{tikzpicture}\\
    В нем невозможно поменять порядок фишек\\
\end{itemize}
\textbf{Доказательство достаточности}\\
Отсутствует...\\
Рассмотрим обобщение предыдущей игры:\\
Пусть фишки могут меняться местами, если они соседи и "друзья"\\
Т.е. теперь у нас два графа: "географический" граф $X$ и граф дружбы $Y$\\
В прошлой задаче граф дружбы -- "звездочка" с $\circ$ в центре (т.е. $\circ$ дружит со всеми, а остальные не дружат между собой)\\
Построим граф $FS(X, Y)$ -- граф друзей и врагов\\
В нем будет $n!$ вершин\\
Каждая вершина графа -- биекция $\sigma: V(X) \rto V(Y)$\\
Получается, что $\sigma$ -- какой-то способ расположить фишки по вершинам\\
(т.к. $V(x)$ -- множество вершин, а $V(Y)$ -- множество фишек)\\
Вершины $\sigma$ и $\sigma'$ соединены ребром, если они отличаются одним "дружеским" обменом\\
Тогда из любой перестановки можно получить любую $\LRto$ граф связен\\
Из теоремы Уилсона: $FS(G, K_{1, n-1}), G$ -- из теоремы Уилсона, $K_{1,n-1}$ -- звездочка с вершиной $n$ в центре\\
Ответим на другой вопрос: Для каких $G$ граф $FS(G, C_n), C_n$ -- цикл длины $n$ -- связен\\
\textbf{Лемма}\\
Графы $FS(X, Y)$ и $FS(Y,X)$ -- изоморфны\\
Т.е. можно построить биекцию $u \os{\theta}{\lrto} u'$ такую, что ребра $uv$ и $\theta(u)\theta(v)$ существуют одновременно\\
\textbf{Доказательство}\\
Построим биекцию $\sigma \in FS(X,Y) \lrto \sigma^{-1} \in FS(Y, X)$\\
Теперь рассмотрим граф $FS(C_n, G)$\\
Теперь рассмотрим следующую ситуацию:\\
По кругу стоят $3n$ человек: $n$ семей вида папа-мама-ребенок\\
Ребенок не может меняться со своими родителями, все остальные могут меняться между собой\\
//todo т.к. оффтоп, мне лень дальше конспектировать\\
См. первую лекцию, начиная с 30 минуты\\
\section{Функции и отображения в $\Rset^m$}
\subsection{Линейные отображения}
\textbf{Определение}\\
$\Lin(X, Y)$ -- множество линейных отображений из $X$ в $Y$ (линейные пространства)\\
$\Lin(X, Y)$ -- линейное пространство\\
\textbf{Обозначение}\\
Пусть $A \in \Lin(\Rset^m, \Rset^n)$\\
$\|A\| := \sup_{|x| = 1} |A(x)|$\\
\textbf{Замечание 1}\\
$\|A\| \in \Rset$\\
\textbf{Доказательство}\\
Было доказано: $|Ax| \leq C_a |x|, C_A = \sqrt{\sum a_{ij}^2}$\\
\textbf{Замечание 2}\\
Из теоремы Вейерштрасса $\sup \lrto \max$ в конечномерном случае\\
\textbf{Замечание 3}\\
$\fall x \in \Rset^m\ |Ax| \leq \|A\||x|$\\
\textbf{Доказательство}\\
Для $x=0$ очевидно\\
$\ot{x} := \frac{x}{|x|}$\\
$|A\ot{x}| \leq \|A\|$\\
\textbf{Замечание 4}\\
Если $\ex C: \fall x\ |Ax| \leq C|x|$, то $\|A\| \leq C$\\
\textbf{Пример}
\begin{itemize}
    \item $m=n=1$\\
    $A$ -- линейное отображение: $x \mapsto ax$\\
    $\|A\| = |a|$
    \item $m=1, n$ -- любое\\
    $A:\Rset\rto \Rset^n$\\
    Тогда $\ex \ol v \in \Rset^n Ax = x\ol{v}$\\
    $\|A\| = |\ol v|$
    \item $n=1, m$ -- любое\\
    $A: \Rset^m \rto \Rset$\\
    Тогда $\ex l \in \Rset^m: Ax=\an xl$\\
    $\|A\| = |l|$
    \item $m,n$ -- любые\\
    $A=(a_{ij})$\\
    $x \mapsto Ax$\\
    $\|A\|$ так легко не считается((((
\end{itemize}
\textbf{Лемма}\\
Пусть $X, Y$ -- нормированные линейное пространство\\
$A \in \Lin(X, Y)$\\
Тогда следующие утверждения эквивалентны:\\
\begin{enumerate}
    \item $A$ -- ограничен, т.е. $\|A\| < +\infty$
    \item $A$ -- непрерывно в $\0 \in X$
    \item $A$ -- непрерывно на $X$
    \item $A$ -- равномерно непрерывное ($\fall \eps > 0\ \ex \delta > 0: \fall x_1, x_2: |x_1-x_2| < \delta\ |Ax_1 - Ax_2| < \eps$)
\end{enumerate}
\textbf{Доказательство}\\
$4 \Rto 3 \Rto 2$ -- очевидно\\
Докажем $2 \Rto 1$\\
Возьмем $\eps = 1$\\
$\ex \delta > 0:\ \fall x: |x| < \delta\ |Ax| < 1$\\
Возьмем $|x| = 1$\\
$Ax| = \frac1\delta|A(\delta x)| < \frac1\delta$\\
Тогда $\|A\| \leq \frac1\delta$\\
Докажем $1 \Rto 4$\\
$\fall \eps > 0\ \ex \delta = \frac\eps{\|a\|}:\ \fall x_1, x_2: |x_1-x_2| < \delta$\\
$|Ax_1-Ax_2| = |A(x_1-x_2)| \leq \|A\||x_1-x_2| < \eps$\\
\textbf{Теорема о пространстве линейных отображений}
\begin{itemize}
    \item $\|\cdot \|$ -- норма в $\Lin(X, Y), X, Y$ -- конечномерные нормированные пространства\\
    Т.е.\begin{enumerate}
        \item $\|A\| \geq 0, \|A\| = 0 \LRto A = 0$
        \item $\fall \alpha \in \Rset\ \|\alpha A\| = |\alpha| \|A\|$
        \item $\|A + B\| \leq \|A\|+\|B\|$
    \end{enumerate}
    \item $\|BA\|\leq \|B\|\cdot\|A\|$
\end{itemize}
\textbf{Доказательство}\\
$\|A\| \geq 0$ -- тривиально\\
$\|A\|=\sup_{|x|=1} \|Ax\| = 0 \Rto A = 0$\\
$\|\alpha A\| = |\alpha| \|A\|$\\
$|(A+B)x| \leq |Ax| + |Bx| \leq \ub{C}{(\|A\|+\|B\|)}|x| \Rto \|A+B\| \leq C = \|A\|+\|B\|$\\
$|BAx|\leq \|B\||Ax| \leq \|B\|\|A\||x|$\\
\textbf{Замечание}\\
В $\Lin(X, Y)$\\
$\|A\|=\sup_{|x|=1} |Ax| = \sup_{|x|\leq 1} |Ax| = \sup_{|x|<1}|Ax| = \sup_{x\neq 0} \frac{|Ax|}{|x|} = \inf\{C\in \Rset: \fall x \in X\ |Ax| \leq C|x|\}$\\
\textbf{Теорема Лагранжа (для отображений)}\\
$F: D \subset \Rset^m \rto \Rset^n$ -- дифференцируема в $D$ -- открытом\\
$a, b \in D, [a,b] \subset D$\\
Тогда $\ex c \in [a,b]$, т.е. $\ex \theta \in [0,1]: c = a+\theta(b-a):\ |F(b)-F(a)| \leq \|F'(c)\||b-a|$\\
\textbf{Доказательство}\\
$f(t) = F(a+t(b-a)), t \in [0,1]$\\
$f'(t) = F'(a+t(b-a))(b-a)$\\
$|F(b)-F(a)|=|f(1)-f(0)| \leq |f'(\theta)|(1-0) = |F'(a+\theta(b-a))(b-a)| \leq \|F'(a+\theta(b-a))\||b-a|$\\
\textbf{Лемма}\\
$B\in \Lin(\Rset^m, \Rset^m)$\\
$\ex c > 0: \fall x\ |Bx| \geq c|x|$\\
Тогда $B$ -- обратим и $\|B^{-1}\| \leq \frac1c$\\
\textbf{Доказательство}\\
Обратимость очевидна, т.к. $\ker B = \{\0\}$. Тогда $\ex B^{-1}$\\
$|\ub{x}{B^{-1}y}| \leq \frac1c |BB^{-1}y| = \frac1c|y|$\\
\textbf{Замечание}\\
$\Omega_m$ -- множество обратимых линейных операторов $\Rset^m \rto\Rset^m$\\
$|x| = |A^{-1}Ax| \leq \|A^{-1}\||Ax|$\\
Т.е. $|Ax| \geq \frac1{\|A^{-1}\|}|x|$\\
\textbf{Теорема об обратимости линейного операторого, близкого к обратимому}\\
$L \in \Omega_m$ -- обратимый линейный оператор $\Rset^m \rto \Rset^m$\\
$M \in \Lin(\Rset^m, \Rset^m): \|L-M\| \leq \frac1{\|L^{-1}\|}$ -- $M$ -- близкий к $L$\\
Тогда
\begin{itemize}
    \item $M \in \Omega_m$ -- т.е. $\Omega_m$ -- открытое
    \item $\|M^{-1}\| \leq \frac1{\|L^{-1}\|^{-1} - \|L-M\|}$
    \item $\|L^{-1}-M^{-1}\| \leq \frac{\|L^{-1}\|}{\|L^{-1}\|^{-1}-\|L-M\|}\|L-M\|$
\end{itemize}
\textbf{Доказательство 1 и 2}\\
$Mx = Lx + (M-L)x$\\
$|Mx| \geq |Lx| - |(M-L)x| \geq \frac1{\|L^{-1}\|}|x| -\|M-L\||x| = (\|L^{-1}\|^{-1}-\|M-L\|)|x|$\\ 
1 и 2 следуют из леммы\\
\textbf{Доказательство 3}\\
$\frac1l - \frac1m = \frac{m-l}{lm}$\\
Аналогично $L^{-1}-M^{-1} = M^{-1}(M-L)L^{-1}$\\
$\|L^{-1}-M^{-1}\| = \|M^{-1}(M-L)L^{-1}\| \leq \|M^{-1}\|\|M-L\|\|L^{-1}\| \leq $ из пункта 2\\
\textbf{Следствие}\\
Отображение $L \mapsto L^{-1}$, заданное на $\Omega_m$, непрерывно\\
\textbf{Доказательство}\\
Доказательство по Гейне\\
Рассмотрим последовательность операторов $B_k: B_k \rto L$\\
Проверим, что $B_k^{-1} \rto L^{-1}$\\
Н.С.Н.М. $\|B_k-L\|< \frac1{\|L^{-1}\|}$\\
$\|B_k^{-1}-L^{-1}\| \leq \ub{\text{огр}}{\frac{\|L^{-1}\|}{\frac1{\|L^{-1}\|} - \ub{\rto 0}{\|L-B_k\|}}}\ub{\rto 0}{\|L-B_k\|} \rto 0$\\
\textbf{Теорема (о непрерывно дифференцируемых отображениях)}\\
$F: \ub{\text{откр}}D\subset \Rset^m \rto \Rset^l$, дифф. на $D$\\
$F': D \rto \Lin(\Rset^m, \Rset^l)$\\
Тогда $1 \lrto 2$
\begin{enumerate}
    \item $F\in C^1(D)$ (все частные производные непрерывны на $D$)
    \item $F': D \rto \Lin(\Rset^m, \Rset^l)$ -- непрерывно на $D$\\
    $\fall x \in D\ \fall \eps > 0\ \ex \delta > 0: \fall \ot x:|x-\ot x| < \delta\ \|F'(x)-F'(\ot x)\| < \eps$\\
\end{enumerate}
\textbf{Доказательство $1 \rto 2$}\\
Пусть $F\in C^1(D)$\\
$\fall i,j\ \fall x \in D\ \fall \eps > 0\ \ex \delta > 0: \fall \ot x:|x-\ot x| < \delta\ |\frac{\partial f_i}{\partial x_j}(x)-\frac{\partial f_i}{\partial x_j}(\ot x)| < \frac\eps{\sqrt{mn}}$\\
Тогда $\|F'(x) - F'(\ot x)\| \leq \sqrt{\sum_{ij} (\frac{\partial f_i}{\partial x_j}(x)-\frac{\partial f_i}{\partial x_j}(\ot x))^2} \leq \eps$\\
\textbf{Доказательство $2 \rto 1$}\\
Пусть $\fall x \in D\ \fall \eps > 0\ \ex \delta > 0: \fall \ot x:|x-\ot x| < \delta\ \|F'(x)-F'(\ot x)\| < \eps$\\
$h = (0, \ldots, 0, \ub{k}1, 0, \ldots 0)^T$\\
$|\ub{\sum_{i=1}^l (\frac{\partial f_i}{\partial x_k}(x)-\frac{\partial f_i}{\partial x_k}(\ot x))}{(F'(x)-F'(\ot x)h)}| \leq \| F'(x)-F'(\ot x)\||h| \leq \eps$\\
Отсюда $\sqrt{\sum_{i=1}^l (\frac{\partial f_i}{\partial x_k}(x)-\frac{\partial f_i}{\partial x_k}(\ot x))^2} \leq \eps$\\
Тогда для $i=i_0$ -- $|\frac{\partial f_{i_0}}{\partial x_k}(x)-\frac{\partial f_{i_0}}{\partial x_k}(\ot x)| \leq \eps$
\section{Экстремумы}
\textbf{Определение}\\
$f: D\subset \Rset^m \rto \Rset$\\
$a \in D$ -- локальный максимум $f \LRto \ex U(a): \fall x \in U(a) \cap D\ f(x) \leq f(a)$ (нестрогий экстремум)\\
$a \in D$ -- локальный строгий максимум $f \LRto \ex U(a): \fall x \in U(a) \cap D\ f(x) < f(a)$ (строгий экстремум)\\
\textbf{Теорема Ферма}\\
$f: D \subset \Rset^m \rto \Rset$\\
$a \in \Int D, f$ -- дифференцируема\\
$a$ -- экстремум\\
Тогда $\fall$ направление $l\ \frac{\partial f}{\partial l} (a) = 0$\\
\textbf{Доказательство}\\
$g(t) = f(a+tl), t \in \Rset$ -- задана в окрестности 0\\
$g'(0) = 0$\\
$g'(t) = f'l = \begin{pmatrix}
    \ppart f{x_1} & \ldots & \ppart f{x_m}
\end{pmatrix}\cdot \begin{pmatrix}
    l_1\\\vdots\\l_m
\end{pmatrix} = \ppart{f}{l}$\\
\textbf{Следствие (необходимое условие экстремума)}\\
$a$ -- локальный экстремум\\
Тогда $\fall 1 \leq k \leq m\ \ppart{f}{x_k}(a) = 0$\\
\textbf{Следствие (т. Ролля)}\\
Пусть $K \subset \Rset^m$ -- компакт\\
$f$ -- дифференцируема в $\Int K$ ($f: K \rto \Rset$, непрерывна)\\
$f_{\partial K} = \const, \partial K$ -- граница компакта\\
Тогда $\ex x_0 \in \Int K: \grad f(x_0) = 0$\\
\textbf{Доказательство}\\
По теореме Вейерштрасса $f$ достигает $\max, \min$ на $K$\\
Если оба на $\partial K$, то $f \equiv \const$ на $K$\\
Иначе применим теорему Ферма\\
\textbf{Определение}\\
$Q(h): \Rset^m \rto \Rset$ -- квадратичная форма, если она представляет однородный многочлен 2 степени\\
т.е. $Q(h) = \sum_{1 \leq i \leq m, 1 \leq j \leq m} a_{ij}h_i h_j, a_{ij} = a_{ji}$\\\\
$Q$ -- положительно определенная $\LRto \fall h \neq 0\ Q(h) > 0$\\
$Q$ -- отрицательно определенная $\LRto \fall h \neq 0\ Q(h) < 0$\\
$Q$ -- незнакоопределенная $\LRto \ex h: Q(h) > 0, \ex h: Q(h) < 0$\\
$Q$ -- полуопределенная (положительно определенная вырожденная) $\LRto \fall h\ Q(h) \geq 0, \ex h \neq 0: Q(h) = 0$\\
\textbf{Лемма}
\begin{enumerate}
    \item $Q: \Rset^m \rto \Rset$ -- кв. форма, $Q > 0$\\
    Тогда $\ex \gamma_Q > 0: \fall x\ Q(x) > \gamma_Q|x^2|$
    \item $p: \Rset^m \rto \Rset$ -- норма
    Тогда $\ex C_1, C_2 > 0: \fall x \in \Rset^m\ C_1|x_1| \leq p(x) \leq C_2|x_2|$
\end{enumerate}
\textbf{Доказательство}
\begin{enumerate}
    \item $\gamma_Q:= \min_{|x|=1} Q(x) > 0$\\
    Тогда $Q(x) = |x|^2Q(\frac x{|x|}) \geq \gamma_Q |x|^2, x \neq 0$
    \item Проверим, что $p(x)$ непрерывна, чтобы доказать существование минимума и максимума:\\
    $|p(x) - p(y)| \leq p(x-y) = p(\sum (x_k-y_k) \ol{e_k}) \leq \sum |x_k - y_k| p(e_k) \leq M|x-y|, M = \sqrt{\sum p(e_k)^2}$ -- по КБШ\\
    $C_1 = \min_{|x|=1}p(x), C_2 = \max_{|x|=1}p(x)$\\
    $p(x) = |x|p(\frac x{|x|}) \leq |x|C_2, \geq |x|C_1$
\end{enumerate}
\textbf{Напоминание}\\
$f(x+h) = f(x) + \df f(x, h) + \frac1{2!}\df^2 f(x,h) + \ldots$\\
$\df^2 f(x,h) = f''_{x_1x_1}(x) h_1^2 + \ldots + f''_{x_nx_n}h_n^2 + 2f''_{x_1x_2}h_1h_2 + \ldots$\\
\textbf{Теорема (достаточное условие экстремума)}\\
$f: D \subset \Rset^m \rto \Rset, a \in \Int D, \grad f(a) = 0, f \in C^2(D)$\\
$Q(h):= \df^2 f(a,h)$\\
Тогда $Q > 0 \Rto a$ -- локальный минимум\\
$Q < 0 \Rto a$ -- локальный максимум\\
$Q \lessgtr 0$ -- не точка локального экстремума\\
$Q \geq 0$ -- информации недостаточно\\
\textbf{Доказательство}\\
$\fall h\ \ex t \in (0,1): f(a+h) = f(a) + \ub0{\df f(a,h)} + \frac1{2!}\df f(a+th, h)$ -- остаток в формуле Лагранжа\\
$f(a+h) - f(a) = \frac1{2!}Q(h) + \frac1{2!}(\ub{|\text{б.м.}\cdot h_i^2| = o(|h|^2)}{f''_{x_1x_1}(a+th)h_1^2 - f''_{x_1x_1}(a)h_1^2}+ \ldots + \ub{|\text{б.м.}\cdot h_ih_j| = o(|h|^2)}{2f''_{x_1x_2}h_1h_2 - 2 f''_{x_1x_2}(a)h_1h_2}) + \ldots)$\\
$f(a+h)-f(a) \geq \frac12 Q(h)-|\alpha(h)||h|^2  \geq \frac12 \gamma_Q|h|^2 - |\alpha(h)||h|^2 \geq \frac14 \gamma_Q|h|^2 > 0, \alpha(h)$ -- б.м., при достаточно малых $|h|$\\
Пункт 1 доказан\\
Пункт 2 доказывается заменой $f \rto -f$\\
Пункт 3: $h: Q(h) > 0, \ot h: Q(\ot h) < 0$\\
Аналогично п.1. $f(a+s\cdot h) - f(a) \geq \frac12 Q(sh) - |\alpha(s)|s^2 = \frac12Q(h)s^2 - |\alpha(s)|s^2 \geq \frac14 Q(h) \cdot s^2$\\
С другой стороны $f(a+s\cdot \ot h) < 0$ по аналогичным соображениям\\
Пункт 4: $f: \Rset^2 \rto \Rset, a = (0, 0)$\\
$f(x_1, x_2) = x_1^2 - x_2^4$\\
$Q(h) = 2h_1^2$ -- полуопределенный\\
Тут нет экстремума\\
$f(x_1,x_2) = x_1^2 + x_2^4$ -- в нуле экстремум
\section{Функциональные последовательности и ряды}
\subsection{Равномерная сходимость последовательностей и функций}
\textbf{Определение}\\
Последовательность функций -- отображение $N \leadsto$ множество функций\\
Пусть $f_1(x), f_2(x), \ldots: X \rto \Rset, X$ -- любое множество\\
Последовательность $(f_n)$ сходится поточечно на $E$ -- существует функция $f(x)$\\
$f_n \us E\rto f$\\
$\fall x \in E\ \fall \eps > 0\ \ex N: \fall n > N\ |f_n(x)-f(x)|<\eps$\\
Последовательность $(f_n)$ сходится равномерно на $E$ к функции $f$\\
$f_n \us E\rightrightarrows f$
$\fall \eps > 0\ \ex N: \fall n > N\ \ub{\sup_{x\in E} |f_n(x)-f(x)| = \rho(f_n, f)\leq \eps}{\fall x \in E\ |f_n(x)-f(x)| < \eps}$\\
\textbf{Замечание}\\
$f \rrto f$ на $E, E_0 \subset E$\\
Тогда $f_n \us {E_0}\rrto$\\
\textbf{Замечание}\\
$f_n \us E\rrto f$\\
Тогда $f_n \us E\rto f$\\
\textbf{Замечание}\\
$\mc F = \{f: X \rto \Rset, f\text{ -- огр.}\}$\\
Тогда $\rto(f, g) = \sup_{x\in X} |f(x) - g(x)|$ является метрикой на $\mc F$\\
\textbf{Доказательство}\\
Первые две аксиомы очевидны\\
Докажем неравенство треугольника\\
$\fall \eps > 0\ \ex x_0: \rho(f, g) - \eps < |f(x_0) - g(x_0)| \leq |f(x_0) - h(x_0)| + |h(x_0)-g(x_0)| \leq \rho(f,h)+\rho(h,g)$\\
Отсюда $\rho(f,g) \leq \rho(f,h) + \rho(h,g)$\\
\textbf{Замечание}\\
$f_n \rrto f$ на $E_1$ и на $E_2$\\
Тогда $f_n \rrto f$ на $E_1 \cup E_2$\\
\textbf{Теорема 1 (Стокса - Зайдля)}\\
$X$ -- метрическое пространство\\
$f_n, f: X \rto \Rset$\\
$c \in X, f_n$ -- непрерывная в $c$\\
$f_n \rrto f$ на $X$\\
Тогда $f$ -- непрерывна в $c$\\
\textbf{Доказательство}\\
Дано утверждение о равномерной сходимости\\
$\fall \eps > 0\ \ex N: \fall n > N\ \sup_{x\in X} |f_n(x) - f(x)|<\eps$\\
$|f(x)-f(c)| \leq |f(x)-f_n(x)| + |f_n(x)-f_n(c)| + |f_n(c)-f(c)|$\\
Начиная с некоторого большого $n:$\\;
$|f(x)-f(c)| \leq \ub{< \eps}{|f(x)-f_n(x)|} + |f_n(x)-f_n(c)| + \ub{< \eps}{|f_n(c)-f(c)|}$\\
Т.к. $f_n$ непрерывна, то $\ex U(c): \fall x \in U(c)\ |f_n(x)-f_n(c)| < \eps$\\
Отсюда $|f(x)-f(c)| < 3\eps$\\
Тогда $\fall \eps > 0\ \ex U(c): \fall x \in U(c)\ |f(x)-f(c)| < 3\eps$\\
\textbf{Замечание}\\
Вместо метрических пространств можно рассматривать топологические пространства(без изменения доказательства)\\
\textbf{Следствие}\\
$f_n \in C(X), f_n \rrto f$ на $X$. Тогда $f \in C(X)$\\
\textbf{Следствие 2}\\
$f_n \in C(X), \fall c\ \ex W(c): f_n \rrto f$ на $W(c)$. Тогда $f \in C(X)$\\
\textbf{Замечание}\\
$f_n \rrto f$ на $X \not \Rto \fall c\ \ex W(c): f_n \rrto f$\\
Пример: $f_n = x^n, x \in (0,1)$\\
$f \equiv 0$\\
Рассмотрим точку $x$ в $(\alpha, \beta), \beta \neq 1$\\
$\rho(f_n, f) = \beta^n \rto 0$\\
$f_n(x) \rrto f$ на $(\alpha, \beta)$\\
Но $\rho(f_n, f) = 1$ на $(0,1)$\\
$f_n \not\rrto f$ на $(0,1)$\\
%лекция 5-6 2:13:00
\textbf{Теорема}\\
$X$ -- компакт\\
$\rho(f,g) = \sup_{x\in X}|f(x)-g(x)$ в $C(X)$\\
Тогда $(C(X), \rho)$ -- полное метрическое пространство\\
(т.е. $\fall x_n\text{ -- фунд.}\ \fall \eps > 0\ \ex N: \fall m,n > N\ \rho(x_n, x_m) < \eps)$\\
\textbf{Доказательство}\\
Пусть $f_n$ -- фундаментальная последовательность в $C(X)$\\
$\fall \eps > 0\ \ex N : \fall n, m > N\ \rho(f_n, f_m) = \sum_{x \in X} |f_n(x) - f_m(x)| < \eps$\\
Тогда $\fall x_0 \in X$ последовательность $n \mapsto f_n(x_0)$ -- фунд. вещ. посл.\\
Тогда $\ex \lim_{n\rto +\infty} f_n(x_0) = f(x_0)$ -- конечная\\
Проверим, что $f_n \rrto f, f \in C(X)$\\
$\fall \eps > 0 \ex N: \fall n, m > N\ \fall x\ |f_n(x) - f_m(x)| < \eps$\\
$\fall \eps > 0 \ex N: \fall n > N\ \fall x\ |f_n(x) - f(x)| \leq \eps$ (предельный переход $m \rto \infty$)\\
Т.е. $f_n \rrto f$ на $X$\\
$f\in C(X)$ по теореме 1\\
\textbf{Замечание}\\
$\mc F(X) = $ пространство ограниченных функций на $X$\\
$(\mc F(X), \rho)$ -- полное м.п.\\
\textbf{Доказательство}\\
Аналогичное\\
%лекция 5-6 2:32:00\\
\textbf{Теорема (критерий Коши)}\\
$f_n \in C(X)$\\
$\ex f \in C(x): f_n \rrto f \LRto \fall \eps > 0\ \ex N: \fall n,m > N\ \fall x \in X\ |f_n(x)-f_m(x)| < \eps$\\
\subsection{Предельный переход под знаком интеграла}
Подумаем о следующем правиле: $f_n \rto f \Rto \int_a^b f_n \rto \int_a^b f$\\
\textbf{Анти-пример}\\
$f_n(x) = nx^{n-1}(1-x^n), x\in [0,1], f_n \rto f\equiv 0$\\
$\int_0^1 nx^{n-1}(1-x^n)\df x \us{t=x^n}= \int_0^1 (1-t)\df t = \frac12$\\
\textbf{Теорема 2}\\
$f_n \in C[a,b]$\\
$f_n \rrto f$ на $[a,b]$\\
Тогда $\int_a^b f_n \rto \int_a^b f$\\
(по т.1. $f$ -- непрерывна)
\textbf{Доказательство}\\
$|\int_a^b f_n - \int_a^b f| = |\int f_a^b f_n - f| \leq \int_a^b |f_n-f| \leq \sup |f_n-f|(b-a) \rto 0$\\
\textbf{Следствие (правило Лейбница дифференцирования интеграла по параметру)}\\
$f: \ub{x}{[a,b]} \times \ub{y}{[c,d]} \rto \Rset$\\
$\fall x,y \ex f'_y(x,y)$ и $f, f'_y$ -- непрерывные на $[a,b]\times[c,d]$\\
Тогда для $\Phi(y) = \int_a^b f(x,y)\df x$ верно, что $\Phi$ -- дифференцируема на $[c,d]$ и $\Phi'(y) = \int_a^b f'_y(x,y)\df x$\\
\textbf{Доказательство}\\
$\frac{\Phi(y+t_n)-\Phi(y)}{t_n} = \int_a^b \frac{f(x,y+t_n)-f(x,y)}{t_n}\df x \us{\text{по т.Лагранжа}}= \int_a^b f_y'(x, t+\Theta_x t_n)\df x \os{(*)}\rto \int_a^b f'_y(x,y)\df x$\\
Проверим $(*)$:\\
Вспомним теорему Кантора о равномерной непрерывности\\
$f \in C(K)$. Тогда $f$ -- равномерно непрерывная\\
Т.е. $\fall \eps > 0\ \ub{\ex N: \fall n > N\ |t_n|<\delta}{\ex \delta > 0}: \fall x, \ol x: \rho(x, \ot x) < \delta\ |f(x)-f(\ot x)| < \eps$\\
Тогда $\rho((x, y + \Theta_x t_n), (x,y)) < \delta$\\
И значит $|f(x, y + \Theta_x t_n) - f(x,y)| < \eps$\\
Т.е. $|\int_a^b f'_y(x,y+\Theta_x t_n)-\int_a^b f'_y(x,y)| \leq \eps(b-a)$\\
%лекция 5-6 3:00:00
\textbf{Теорема 3 (о предельном переходе по знаку производной)}\\
Пусть $f_n \in C^1\an ab$\\
$f_n \rto f_0$ поточечно на $\an ab$\\
$f'_n \rrto \phi$ на $\an ab$\\
Тогда $f_0 \in C^1\an ab, f'_0 = \phi$ на $\an ab$\\
\textbf{Пояснение}\\
$\lim_{n\rto +\infty} f'_n(x) = (\lim_{n\rto +\infty} f_n(x))'$\\
\textbf{Доказательство}\\
$[x_0, x_1] \subset \an ab$\\
$f'_n \rrto \phi$ на $[x_0, x_1]$ (отсюда $\phi$ непрерывна)\\
Тогда по т.2 $\int_{x_0}^{x_1} f'_n \rto \int_{x_0}^{x_1} \phi$\\
$\ub{\rto (f_0(x_1)-f_0(x_0))}{f_n(x_1)-f_n(x_0)} \rto \int_{x_0}^{x_1} \phi$\\
Т.о. $f_0$ -- первообразная $\phi$\\
$\phi$ -- непрерывна по т.1\\
Отсюда $f'_0 = \phi$
\section{Диффеоморфизм}
%лекция 5-6 0:0
\textbf{Определение}\\
\textit{Область} в $\Rset^m$ -- открытое связное множество\\
$f: O \subset \Rset^m \rto \Rset^m, O$ -- область\\
$f$ -- \textit{диффеоморфизм}, если $f$ -- обратимо, $f, f^{-1}$ -- дифференцируема\\
\textbf{Замечание}\\
Если это так, то $f^{-1}\circ f = \nm{id}$\\
$(f^{-1})'(y) = (f'(x))^{-1}, y = f(x)$\\
\textbf{Лемма (о <<почти>> локальной инъективности)}\\
$F: O \subset \Rset^m \rto \Rset^m, O$ -- область, $x_o \in O, F$ -- дифференцируемо в $x_0$\\
$\det F'(x_0) \neq 0$\\
Тогда $\ex C>0, \delta > 0: \fall h: |h| < \delta\ |F(x_0+h)-F(x)| \geq c|h|$\\
\textbf{Доказательство}
\begin{enumerate}
    \item $f$ -- линейное\\
    Тогда $|h| = |f^{-1}\circ F\cdot h| \leq \|F^{-1}\||Fh|$\\
    $|F(x_0+h) - F(x_0)| = |Fh| \geq \frac1{\|F^{-1}\|}|h|$\\
    $\delta$ -- любое
    \item $|F(x_0+h)-F(x_0)| = |F'(x_0)h+\alpha(h)|h|| \geq \ub{\text{из п.1}}{C}|h|-|\alpha(h)||h|$\\
    Берем $\delta$, чтобы $|\alpha(h)| \leq \frac C2$
\end{enumerate}
\textbf{Определение}\\
$\fall x\ \det F'(x) \neq 0$, то отсюда не следует инъективность\\
В далеких точках значения могут совпадать\\
\textbf{Пример}\\
$(x_1, x_2) \rto (x_1^2-x_2^2, 2x_1x_2)$\\
$\det F'(x) = 2(x_1^2 + x_2^2)$\\
Данное отображение склеивает точки\\
\textbf{Теорема о сохранении области}\\
$F: O \subset \Rset^m \rto \Rset^m, O$ -- \underline{открытое}, $\fall x\ F$ -- дифференцируемый в $x$ и $\det F'(x) \neq 0$\\
Тогда $F(O)$ -- открытое множество\\
\textbf{Доказательство}\\
Пусть $x_0 \in O, y_0 \in F(x_0)$\\
Проверим, что $y_0$ -- внутренняя точка $F(O)$\\
По лемме $\ex C, \delta: \fall h \in \ol{B(0, \delta)}$\\
$|F(x_0 + h) - F(x_0)| \geq C|h|$\\
$r:= \frac12 \nm{dist}(y_0, \ub{\text{компакт}}{F(\ub{\text{сфера}}{S(x_0, \delta)})})$\\
$r > 0$ -- потому что $\dist = \inf$ на компакте, а значит $\inf$ реализуется\\
Проверим, что $B(y_0, r) \subset F(O)$\\
Т.е. проверим, что $\fall y \in B(y_0, r)\ \ex x \in B(x_0, \delta): F(x) = y$\\
Рассмотрим $g(x) := |F(x)-y|^2, y \in B(y_0, r)$ -- функция на $\ol{B(x_0, \delta)}$\\
(надеемся, что она обращается в 0)\\
$g(x_0) = |F(x_0) - y| < r^2$\\
%лекция 5-6 41:58\\
$\fall x \in S(x_0, \delta)\ \gamma(x) \geq r^2$ 
Тогда $\min g$ достигается(т.к. функция на компакте) внутри $B(x_0, r)$\\
В этой точке все частные производные = 0\\
Пусть в точке $x$ достигается минимум\\
$g(x) = (F_1(x)-y_1)^2 + \ldots + (F_m(x)-y_m)^2$\\
$\left\{\begin{array}{ccc}
0 & = & \ppart g{x_1} = 2(F_1-y_1)\ppart{F_1}{x_1} + \ldots + 2(F_m-y_m)\ppart{F_m}{x_1}\\
& \vdots &\\
0 & = & \ppart g{x_m} = 2(F_1-y_1)\ppart{F_1}{x_m} + \ldots + 2(F_m-y_m)\ppart{F_m}{x_m}
\end{array}\right.$\\
$(F(x)-y)^T F'(x) = 0$\\
Т.е. $\det F'(x)\neq 0$, то $g(x) = F(x)-y = 0$\\
Отсюда $g(x)$ достигает $0$\\
\textbf{Замечание}\\
$F$ -- непрерывное $\LRto \fall \ub{\text{откр.}}W\ F^-1(W)$ -- открытое\\
(из определения)\\
\textbf{Замечание}\\
Если  $O$ -- связное, $F$ -- непрерывное\\
Тогда $F(O)$ -- связное\\
(Отсюда область переходит в область)\\
\textbf{Доказательство}\\
Пусть это не так\\
Тогда $F(O) = W_1 \cup W_2$\\
Тогда $O = F^{-1}(W_1)\cup F^{-1}(W_2), F^{-1}(W_1), F^{-1}(W_2)$ -- открытые и не пересекающиеся\\
Но это невозможно из связности\\
Тогда $F(O)$ -- связное\\
\textbf{Следствие}\\
$F: \ub{\text{откр}}O \subset \Rset^m \rto \Rset^l, l < m$\\
$F \in C^1(O)$\\
$\fall x \in O\ \rg F'(x) = l$ ($\rg$ -- ранг матрицы)\\
Тогда $F(O)$ -- открытое\\
\textbf{Доказательство}\\
Пусть $x_0 \in O$\\
Проверим, что $F(x_0)$ -- внутренняя точка в $F(O)$\\
$\rg F'(x_0) = l$\\
Н.у.о. пусть ранг реализуется на первых $n$ столбцах\\
Т.е. $\det (\ppart{F_i}{x_j})_{i,j \in 1\ldots l} \neq 0$\\
Тогда $\ex U(x_0): \fall x \in U(x_0)\ \det (\ppart{F_i}{x_j}(x))_{i,j \in 1\ldots l} \neq 0$\\
Тогда $F(x_0)$ -- внутренняя в $F(U(x_0))$:\\
Рассмотрим $U_l:=\{ (t_1, \ldots, t_l) : (t_1, \ldots, t_l, (x_0)_{l+1}, \ldots, (x_0)_{m}) \in U(x_0)\}$ -- $l$-мерная окрестность\\
$\ot F: U_l \rto \Rset^l$\\
$(t_1, \ldots, t_l) \mapsto F(t_1, \ldots, t_l, (x_0)_{l+1}, \ldots, (x_0)_m)$\\
$\ppart{\ot F_i}{t_j} = \begin{pmatrix}
    \ppart{F_i}{x_j}(t_1, \ldots, t_l, (x_0)_{l+1}, \ldots, (x_0)_m)
\end{pmatrix}$\\
%лекция 5-6 1:16:0\\
\textbf{Теорема о дифференцировании обратного отображения}\\
$F: O\subset \Rset^m \rto \Rset^m, O$ -- область\\
$F \in C^r(O), r \in \Rex$\\
Пусть $F$ -- обратимо и невырождено ($\fall x\ \det F'(x) \neq 0$)\\
Тогда $F^{-1} \in C^r$ (отсюда $(F^{-1})'(y) = (F'(x))^{-1})$\\
\textbf{Доказательство}\\
Индукция по $r$\\
База: $r = 1$\\
Пусть $S = F^{-1}$\\
$S$ -- непрерывна по теореме о сохранении области (по топологическому определению)\\
Возьмем $x_0 \in O, y_0 = F(x_0) \in O_1, O_1 = F(O)$\\
По лемме $\ex C, \delta: \fall x \in B(x_0, \delta)\ |F(x)-F(x_0)| \geq C|x-x_0|$\\
$A = F'(x_0)$\\
$|\ub{y}{F(x)}-\ub{y_0}{F(x_0)} = A(\ub{S(y)}x-\ub{S(y_0)}{x_0}) + \alpha(x)|x-x_0|$\\
$S(y)-S(y_0) = A^{-1}(y-y_0)- A^{-1}\alpha(S(y))|S(y)-S(y_0)|$\\
%лекция 5-6 1:54
Надо проверить: $\beta(y) = |S(y)-S(y_0)|A^{-1}\alpha(S(y)) = o(|y-y_0|)$\\
Пусть $|x-x_0| = |S(y)-S(y_0)|<\delta$ -- выполнено при $y$ близких к $y_0$\\
$|\beta(y)| = |S(y)-S(y_0)||A^{-1}\alpha(S(y))| \leq \frac1C |F(x)-F(x_0)|\|A^{-1}\||\alpha(S(y))| = \frac{\|A^{-1}\|}C |y-y_0||\alpha(S(y))| = o(|y-y_0|)$\\
Отсюда $S$ -- дифференцируемо\\
Проверим, что в $S \in C^1, S' = A^{-1}$\\
$y \ub{\text{непр}}\mapsto S(y) = x \ub{\text{непр}}\mapsto T'(x) = A \ub{\text{непр}}\mapsto A^{-1}$\\
Индукционный переход:\\
$(F^{-1})'(y) = (F'(x(y)))^{-1} \in C^r$\\
%лекция 5-6 2:11:0
Ни черта не понял, смотрите лекцию 5-6, время ~2:11:00\\

\end{document}
