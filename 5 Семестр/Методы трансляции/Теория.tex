\documentclass[12pt]{article}
\usepackage{bbold}
\usepackage{amsfonts}
\usepackage{amsmath}
\usepackage{amssymb}
\usepackage{color}
\setlength{\columnseprule}{1pt}
\usepackage[utf8]{inputenc}
\usepackage[T2A]{fontenc}
\usepackage[english, russian]{babel}
\usepackage{graphicx}
\usepackage{hyperref}
\usepackage{mathdots}
\usepackage{xfrac}


\def\columnseprulecolor{\color{black}}

\graphicspath{ {./resources/} }


\usepackage{listings}
\usepackage{xcolor}
\definecolor{codegreen}{rgb}{0,0.6,0}
\definecolor{codegray}{rgb}{0.5,0.5,0.5}
\definecolor{codepurple}{rgb}{0.58,0,0.82}
\definecolor{backcolour}{rgb}{0.95,0.95,0.92}
\lstdefinestyle{mystyle}{
    backgroundcolor=\color{backcolour},   
    commentstyle=\color{codegreen},
    keywordstyle=\color{magenta},
    numberstyle=\tiny\color{codegray},
    stringstyle=\color{codepurple},
    basicstyle=\ttfamily\footnotesize,
    breakatwhitespace=false,         
    breaklines=true,                 
    captionpos=b,                    
    keepspaces=true,                 
    numbers=left,                    
    numbersep=5pt,                  
    showspaces=false,                
    showstringspaces=false,
    showtabs=false,                  
    tabsize=2
}

\lstset{extendedchars=\true}
\lstset{style=mystyle}

\newcommand\0{\mathbb{0}}
\newcommand{\eps}{\varepsilon}
\newcommand\overdot{\overset{\bullet}}
\DeclareMathOperator{\sign}{sign}
\DeclareMathOperator{\re}{Re}
\DeclareMathOperator{\im}{Im}
\DeclareMathOperator{\Arg}{Arg}
\DeclareMathOperator{\const}{const}
\DeclareMathOperator{\rg}{rg}
\DeclareMathOperator{\Span}{span}
\DeclareMathOperator{\alt}{alt}
\DeclareMathOperator{\Sim}{sim}
\DeclareMathOperator{\inv}{inv}
\DeclareMathOperator{\dist}{dist}
\newcommand\1{\mathbb{1}}
\newcommand\ul{\underline}
\renewcommand{\bf}{\textbf}
\renewcommand{\it}{\textit}
\newcommand\vect{\overrightarrow}
\newcommand{\nm}{\operatorname}
\DeclareMathOperator{\df}{d}
\DeclareMathOperator{\tr}{tr}
\newcommand{\bb}{\mathbb}
\newcommand{\lan}{\langle}
\newcommand{\ran}{\rangle}
\newcommand{\an}[2]{\lan #1, #2 \ran}
\newcommand{\fall}{\forall\,}
\newcommand{\ex}{\exists\,}
\newcommand{\lto}{\leftarrow}
\newcommand{\xlto}{\xleftarrow}
\newcommand{\rto}{\rightarrow}
\newcommand{\xrto}{\xrightarrow}
\newcommand{\uto}{\uparrow}
\newcommand{\dto}{\downarrow}
\newcommand{\lrto}{\leftrightarrow}
\newcommand{\llto}{\leftleftarrows}
\newcommand{\rrto}{\rightrightarrows}
\newcommand{\Lto}{\Leftarrow}
\newcommand{\Rto}{\Rightarrow}
\newcommand{\Uto}{\Uparrow}
\newcommand{\Dto}{\Downarrow}
\newcommand{\LRto}{\Leftrightarrow}
\newcommand{\Rset}{\bb{R}}
\newcommand{\Rex}{\overline{\bb{R}}}
\newcommand{\Cset}{\bb{C}}
\newcommand{\Nset}{\bb{N}}
\newcommand{\Qset}{\bb{Q}}
\newcommand{\Zset}{\bb{Z}}
\newcommand{\Bset}{\bb{B}}
\renewcommand{\ker}{\nm{Ker}}
\renewcommand{\span}{\nm{span}}
\newcommand{\Def}{\nm{def}}
\newcommand{\mc}{\mathcal}
\newcommand{\mcA}{\mc{A}}
\newcommand{\mcB}{\mc{B}}
\newcommand{\mcC}{\mc{C}}
\newcommand{\mcD}{\mc{D}}
\newcommand{\mcJ}{\mc{J}}
\newcommand{\mcT}{\mc{T}}
\newcommand{\us}{\underset}
\newcommand{\os}{\overset}
\newcommand{\ol}{\overline}
\newcommand{\ot}{\widetilde}
\newcommand{\vl}{\Biggr|}
\newcommand{\ub}[2]{\underbrace{#2}_{#1}}

\def\letus{%
    \mathord{\setbox0=\hbox{$\exists$}%
             \hbox{\kern 0.125\wd0%
                   \vbox to \ht0{%
                      \hrule width 0.75\wd0%
                      \vfill%
                      \hrule width 0.75\wd0}%
                   \vrule height \ht0%
                   \kern 0.125\wd0}%
           }%
}
\DeclareMathOperator*\dlim{\underline{lim}}
\DeclareMathOperator*\ulim{\overline{lim}}

\everymath{\displaystyle}

% Grath
\usepackage{tikz}
\usetikzlibrary{positioning}
\usetikzlibrary{decorations.pathmorphing}
\tikzset{snake/.style={decorate, decoration=snake}}
\tikzset{node/.style={circle, draw=black!60, fill=white!5, very thick, minimum size=7mm}}

\title{Математическая статистика. Теория}
\author{Александр Сергеев}
\date{}
\begin{document}
\maketitle
\section{Введение}
\textbf{Процесс компиляции}\\
$\text{вход} \xrto{\text{лексический анализ}} \text{токены} \xrto{\text{парсинг}} \text{дерево разбора} \xrto{\text{вычисление/компиляция}} \text{результат}$\\
\textbf{Определение}\\
Токен -- неделимая единица парсинга\\
Синтаксически управляемая трансляция -- технология написания парсеров, когда одновременно задаются и зависят друг от друга парсинг и вычисление (правила вычисления применяются прямо во время разбора)
\textbf{Напоминание}\\
Контекстно-свободная грамматика:\\
Алфавит $\Sigma$\\
Нетерминалы $N$\\
Стартовый нетерминал $S \in N$\\
Правила $P \subset N \times (N \cup \Sigma)^*: \an A\alpha \in P$ или $A \rto \alpha$\\
$\alpha \Rto \beta$ -- из $\alpha$ выводится за 1 шаг $\beta$, если $\alpha = \alpha_1A\alpha_2, \beta = \alpha_1 \xi \alpha_2$ и есть правило $A \rto \xi \in P$\\
Язык грамматики $L(\Gamma) = \{x | S \Rto^* x, x \in \Sigma^*\}$\\
\textbf{Определение}\\
Грамматика $\Gamma \in LL(1)$, если из\\
$S \Rto^* xA\alpha \Rto x\xi\alpha \Rto^* xcy$\\
$S \Rto^* xA\beta \Rto x\eta\beta \Rto^* xcz$\\
$c \in \Sigma$ или $c=\eps, y=\eps, z = \eps$\\
следует $\xi = \eta$\\
\textbf{Замечание}\\
Буквы из конца латинского алфавита -- строки из терминалов\\
Буквы из греческого алфавита -- любые строки (возможно, содержащие нетерминалы)\\
\textbf{Замечание}\\
Другими словами, если мы хотим, чтобы из нетерминала $A$ получилась строка, начинающаяся на $c$, то у нас есть только одно правило для достижения этого\\
\textbf{Определение}\\
$LL(k)$ -- вместо символа $c$ у нас $k$ символов\\
Из\\
$S \Rto^* xA\alpha \Rto x\xi\alpha \Rto^* xcy$\\
$S \Rto^* xA\beta \Rto x\eta\beta \Rto^* xcz$\\
$c \in \Sigma^k$ или $c=\Sigma^{\leq k}, y=\eps, z = \eps$\\
следует $\xi = \eta$\\
\textbf{Замечание}\\
$LL(0)$-грамматики задают линейные программы (обобщение архиваторов)\\
\textbf{Утверждение}\\
$LL(1)$-грамматики -- это грамматики, для которых можно написать рекурсивный спуск\\
\textbf{Определение}\\
$FIRST: (N \cup \Sigma)^* \rto 2^{\Sigma \cup \{\eps\}}$\\
$FOLLOW: N \rto 2^{\Sigma \cup \{\$\}}$\\\\
Пока будем считать, что \textit{бесполезных} символов нет -- из любого нетерминала можно вывести терминал\\
$FIRST(\alpha) = \{c | \alpha \Rto^* c\beta\} \cup \{\eps | \alpha \Rto^* \eps\}$\\
$FOLLOW(A) = \{c| S\Rto^* \alpha Ac\beta\} \cup \{\$ | S \Rto^* \alpha A\}$\\
\textbf{Теорема}\\
Грамматика $\Gamma \in LL(1) \LRto \fall A \rto \alpha, A \rto \beta$ выполнено
\begin{enumerate}
	\item $FIRST(\alpha) \cap FIRST(\beta) = \varnothing$
	\item $\eps \in FIRST(\alpha) \Rto FIRST(\beta) \cap FOLLOW(A) = \varnothing$
\end{enumerate}
\textbf{Лемма}\\
$\alpha = c\beta \Rto FIRST(\alpha) = \{c\}$\\
$\alpha = \eps \Rto FIRST(\alpha) = \{\eps\}$\\
$\alpha = A\beta \Rto FIRST(\alpha) = FIRST(A) \setminus \eps \cup (FIRST(\beta)\ if\ \eps \in FIRST(A))$\\

\end{document}