\documentclass[12pt]{article}
\usepackage{bbold}
\usepackage{amsfonts}
\usepackage{amsmath}
\usepackage{amssymb}
\usepackage{color}
\setlength{\columnseprule}{1pt}
\usepackage[utf8]{inputenc}
\usepackage[T2A]{fontenc}
\usepackage[english, russian]{babel}
\usepackage{graphicx}
\usepackage{hyperref}
\usepackage{mathdots}
\usepackage{xfrac}


\def\columnseprulecolor{\color{black}}

\graphicspath{ {./resources/} }


\usepackage{listings}
\usepackage{xcolor}
\definecolor{codegreen}{rgb}{0,0.6,0}
\definecolor{codegray}{rgb}{0.5,0.5,0.5}
\definecolor{codepurple}{rgb}{0.58,0,0.82}
\definecolor{backcolour}{rgb}{0.95,0.95,0.92}
\lstdefinestyle{mystyle}{
    backgroundcolor=\color{backcolour},   
    commentstyle=\color{codegreen},
    keywordstyle=\color{magenta},
    numberstyle=\tiny\color{codegray},
    stringstyle=\color{codepurple},
    basicstyle=\ttfamily\footnotesize,
    breakatwhitespace=false,         
    breaklines=true,                 
    captionpos=b,                    
    keepspaces=true,                 
    numbers=left,                    
    numbersep=5pt,                  
    showspaces=false,                
    showstringspaces=false,
    showtabs=false,                  
    tabsize=2
}

\lstset{extendedchars=\true}
\lstset{style=mystyle}

\newcommand\0{\mathbb{0}}
\newcommand{\eps}{\varepsilon}
\newcommand\overdot{\overset{\bullet}}
\DeclareMathOperator{\sign}{sign}
\DeclareMathOperator{\re}{Re}
\DeclareMathOperator{\im}{Im}
\DeclareMathOperator{\Arg}{Arg}
\DeclareMathOperator{\const}{const}
\DeclareMathOperator{\rg}{rg}
\DeclareMathOperator{\Span}{span}
\DeclareMathOperator{\alt}{alt}
\DeclareMathOperator{\Sim}{sim}
\DeclareMathOperator{\inv}{inv}
\DeclareMathOperator{\dist}{dist}
\newcommand\1{\mathbb{1}}
\newcommand\ul{\underline}
\renewcommand{\bf}{\textbf}
\renewcommand{\it}{\textit}
\newcommand\vect{\overrightarrow}
\newcommand{\nm}{\operatorname}
\DeclareMathOperator{\df}{d}
\DeclareMathOperator{\tr}{tr}
\newcommand{\bb}{\mathbb}
\newcommand{\lan}{\langle}
\newcommand{\ran}{\rangle}
\newcommand{\an}[2]{\lan #1, #2 \ran}
\newcommand{\fall}{\forall\,}
\newcommand{\ex}{\exists\,}
\newcommand{\lto}{\leftarrow}
\newcommand{\xlto}{\xleftarrow}
\newcommand{\rto}{\rightarrow}
\newcommand{\xrto}{\xrightarrow}
\newcommand{\uto}{\uparrow}
\newcommand{\dto}{\downarrow}
\newcommand{\lrto}{\leftrightarrow}
\newcommand{\llto}{\leftleftarrows}
\newcommand{\rrto}{\rightrightarrows}
\newcommand{\Lto}{\Leftarrow}
\newcommand{\Rto}{\Rightarrow}
\newcommand{\Uto}{\Uparrow}
\newcommand{\Dto}{\Downarrow}
\newcommand{\LRto}{\Leftrightarrow}
\newcommand{\Rset}{\bb{R}}
\newcommand{\Rex}{\overline{\bb{R}}}
\newcommand{\Cset}{\bb{C}}
\newcommand{\Nset}{\bb{N}}
\newcommand{\Qset}{\bb{Q}}
\newcommand{\Zset}{\bb{Z}}
\newcommand{\Bset}{\bb{B}}
\renewcommand{\ker}{\nm{Ker}}
\renewcommand{\span}{\nm{span}}
\newcommand{\Def}{\nm{def}}
\newcommand{\mc}{\mathcal}
\newcommand{\mcA}{\mc{A}}
\newcommand{\mcB}{\mc{B}}
\newcommand{\mcC}{\mc{C}}
\newcommand{\mcD}{\mc{D}}
\newcommand{\mcJ}{\mc{J}}
\newcommand{\mcT}{\mc{T}}
\newcommand{\us}{\underset}
\newcommand{\os}{\overset}
\newcommand{\ol}{\overline}
\newcommand{\ot}{\widetilde}
\newcommand{\vl}{\Biggr|}
\newcommand{\ub}[2]{\underbrace{#2}_{#1}}

\def\letus{%
    \mathord{\setbox0=\hbox{$\exists$}%
             \hbox{\kern 0.125\wd0%
                   \vbox to \ht0{%
                      \hrule width 0.75\wd0%
                      \vfill%
                      \hrule width 0.75\wd0}%
                   \vrule height \ht0%
                   \kern 0.125\wd0}%
           }%
}
\DeclareMathOperator*\dlim{\underline{lim}}
\DeclareMathOperator*\ulim{\overline{lim}}

\everymath{\displaystyle}

% Grath
\usepackage{tikz}
\usetikzlibrary{positioning}
\usetikzlibrary{decorations.pathmorphing}
\tikzset{snake/.style={decorate, decoration=snake}}
\tikzset{node/.style={circle, draw=black!60, fill=white!5, very thick, minimum size=7mm}}

\title{Математическая статистика. Теория}
\author{Александр Сергеев}
\date{}
\begin{document}
\maketitle
\section{Введение}
Есть генеральная совокупность. Надо выбрать часть генеральной совокупности -- выборку. По выборке хотим сделать вывод о всей совокупности\\
В исследовании есть следующие этапы
\begin{enumerate}
	\item Сбор данных
	\item Препроцессинг (чистка)
	\item Построение модели и анализ
	\item Интерпретация
\end{enumerate}
\textbf{Определение}\\
Репрезентативность -- свойство выборки, означающее, что по выборке можно судить по всей совокупности\\
\section{Простейшая модель выборки. Эмпирическая функция распределения}
\textbf{Простейшая модель выборки}\\
$X_1, \ldots, X_n$ -- i.i.d.\\
$F$ -- функция распределения (теоретическая, мы ее не знаем)\\
$x_1, \ldots, x_n$ -- реализация выборки\\\\
Глобальная цель -- оценить из реализации $x_1, \ldots, x_n$ теоретическую функцию $F$\\\\
\textbf{Определение (эмпирическая функция выборки)}\\
$\mu_n(t) = \sum_{i=1}^n \1(X_i \leq t)$\\
$F_n(t) = \frac{\mu_n(t)}{n}$ -- эмпирическая функция распределения\\
\textbf{Замечание}\\
Заметим, что $\1(X_i \leq t) \sim Bin(F(t))$\\
Тогда $\mu_n(t) \sim Bin(n, F(t))$\\
$P(F_n(x) = \frac kn) = P(\mu_n(x) = k) = \binom{n}{k} F_n^k(x) (1-F_n(x))^{n-k}$\\ 
Отсюда $E(\mu_n(x)) = n F_n(x)$\\
$E(F_n(t)) = F(t)$ -- несмещенность\\
$\Var F_n(t) = \frac{F(t)(1-F(t))}{n}$\\
По ЦПТ $\frac{\mu_n(t) - n F(t)}{\sqrt{F(t)(1-F(t))n}} = \sqrt{n} \frac{F_n(t) - F(t)}{\sqrt{F(t)(1-F(t))}} \xrto{d} N(0,1)$ -- асимптотическая нормальность\\
\textbf{Теорема Гливенко -- Кантелли}\\
$\sup_{t \in \Rset} |F_n(t) - F(t)| \xrto[n \rto \infty]{a.s.} 0$\\
\textbf{Теорема Колмагорова}\\
$D_n = \sup_x |F_n(x) - F(x)|, F \in C(\Rset), t \geq 0 \Rto P(\sqrt{n} D_n \leq t) \rto K(t) = \sum_{j = -\infty}^\infty e^{-2j^2t^2}$ -- функция распределения Колмагорова\\
\textbf{Теорема Смирнова}\\
$X_1, \ldots, X_n, Y_1, \ldots, Y_n$ -- независимы\\
Обе распределены на $F \in C(\Rset)$\\
$D_{m, n} = \sum_x |F_n(x) - F_m(x)|$\\
Тогда $P(\sqrt{\frac{nm}{n+m}} D_{m,n} \leq t) \xrto[]{m,n \rto \infty} K(t)$\\
\section{Выборочные моменты}
$\alpha_k = E X_1^k$ -- $k$-ый теоретический момент\\
$\beta_k = E(X_1 - EX_1)^k$ -- $k$-ый теоретический момент\\
$\ol{g(X)} := \frac1n \sum_{k=1}^n g(X_k), g(\bullet): \Rset \rto \Rset$\\
$\widehat \alpha_k = \ol{X^k} = \frac1n \sum_{j=1}^n X^k_j$ -- $k$-ый выборочный момент\\
$E\widehat \alpha_k = \alpha_k$ -- несмещенность\\
$\Var \widehat \alpha_k = \frac1n \Var(X_1^k) = \frac1n ((EX_1^{2k}) - (EX_1^k)^2)$\\
По ЦПТ $\sqrt{n} \frac{\widehat \alpha_k - \alpha_k}{\sqrt{\alpha_{2k}-\alpha_k^2}} \approx N(0,1)$\\
$\sqrt{n} \frac{\widehat \alpha_k - \alpha_k}{\sqrt{\widehat \alpha_{2k}-\widehat\alpha_k^2}} = \sqrt{n} \frac{\widehat \alpha_k - \alpha_k}{\sqrt{\alpha_{2k}-\alpha_k^2}} \ub{\rto 1}{\sqrt{\frac{\alpha_{2k}-\alpha_k^2}{\widehat \alpha_{2k}-\widehat \alpha_k^2}}} \approx N(0,1)$\\
\textbf{Пояснение}\\
$\widehat \alpha_k \xrto{P} \alpha_k$ (ЗБЧ)\\\\
$\widehat \beta_k = \ol{(X-\ol X)^k} = \frac1n \sum_{j=1}^n (X_j - X)^k$ -- $k$-ый центральный выборочный момент\\
$\widehat \beta_2 = S_*^2$ -- выборочная дисперсия\\
$S_*$ -- выборочное отклонение\\
\textbf{Замечание}\\
Выборочные моменты -- моменты, посчитанные относительно эмпирического распределения\\
Тогда для них действуют утверждения, свойственные обычным моментам\\
$S_*^2 = \ol X^2 - (\ol X)^2$\\
$\widehat\beta_k = Poly(\widehat\alpha_k, \ldots, \widehat \alpha_1)$\\
Т.к. $\widehat \alpha_k \xrto{P} \alpha_k$, то $\widehat\beta_k \xrto{P} \beta_k$\\\\
\textbf{Отступление}\\
% видимо, это и есть дельта-метод
Пусть $\xi_n$ -- последовательность случайных векторов и $\sqrt{n}(\xi_n - \mu) \xrto{d} N(0, \Sigma)$\\
$\mu$ -- какой-то вектор (необязательно матожидание)
\begin{enumerate}
	\item $\xi_n \xrto{P} \mu$\\
	Т.к. $(\xi_n - \mu)\frac{\sqrt{n}}{\sqrt{n}} \xrto{d} 0$
	\item Пусть $\phi: \Rset^m \rto \Rset, \phi \in C^1(\Rset)$\\
	$\phi(\xi_n) \approx \phi(\mu) + \nabla \phi(\mu)(\xi_n.- \mu)$\\
	$\phi(\xi_n) - \phi(\mu) \approx \nabla\phi(\mu)(\xi_n - \mu)$\\
	$\Var(\phi(\xi_n) - \phi(\mu)) \approx \Var(\nabla \phi(\mu)(\xi_n - \mu)) = \nabla\phi(\mu)\Var(\xi_n) (\nabla\phi(\mu))^T$\\
	Тогда $\sqrt{n} (\phi(\xi_n)-\phi(\mu)) \approx \sqrt{n}\nabla \phi(\mu)(\xi_n - \mu) \rto N(0, \nabla\phi(\mu)\Sigma(\nabla\phi(\mu))^T)$
\end{enumerate}
\textbf{Теорема}\\
Пусть $\xi_n = (\ol X, \ldots, \ol X^k)$\\
(Многоперная ЦПТ $\Rto \sqrt(\xi_n - \alpha) \xrto{d} N(0, \sigma), \alpha = (\alpha_1, \ldots, \alpha_k), \Sigma = \Var(X_1, \ldots, X_1^k)$)\\
$\phi: \Rset^k \rto R, \phi \in C^1(\Rset)$\\
$\sigma^2 = \nabla \phi(\alpha)\Sigma(\nabla\phi(\alpha))^T > 0$\\
Тогда $\sqrt{n} \frac{\phi(\xi_n)-\phi(\alpha)}{\sigma} \xrto{d} N(0, 1)$\\
Кроме того, $\sigma = \sigma(\alpha) \in C^1(\Rset) \Rto \sqrt{n}\frac{\phi(\xi_n)-\phi(\alpha)}{\sigma(\xi_n)} \xrto{d} N(0,1)$\\
\textbf{Упражнение}\\
$\sqrt{n}\frac{S_*^2 - \sigma^2}{\sqrt{\widehat \beta_4 - S_*^4}} \approx N(0,1)$\\\\
$ES_*^2 = \frac{n-1}{n}\sigma^2$\\
$S^2 := \frac{n}{n-1}S_*^2 = \frac{1}{n-1}\sum_{i=1}^n (X - \ol X)^2$ -- исправленная (несмещенная) дисперсия\\
\textbf{Коэффициент асимметрии}\\
$\frac{E(X-EX)^3}{\sigma^3}$\\
Тогда $\frac{\widehat \beta_3}{S_*^3}$ -- выборочный коэффициент асимметрии\\
\textbf{Коэффициент эксцесса}\\
$\frac{E(X-EX)^4}{\sigma^4} - 3 \mapsto \frac{\widehat \beta_4}{\sigma^4_*} -3$\\
\textbf{Ковариация}\\
$\Cov(X, Y) = EXY- EXEY \mapsto S_{*XY} = \frac1n \sum_j (X_j - \ol X)(Y_j - \ot Y) = \frac1n \sum_j X_iY_i - \ol X \ol Y$\\
\textbf{Корреляция}\\
$\rho = \frac{\Cov(X,Y)}{\sqrt{\Var X \Var Y}}\mapsto \rho_n = \frac{S_{*XY}}{S_{*X}S_{*Y}}$
\section{Порядковые статистики}
\textbf{Определение}\\
Вариационный ряд -- выборка $X_{(1)} \leq \ldots \leq X_{(m)}$\\
\textbf{Определение}\\
$X_{(k)}$ -- $k$-ая порядковая статистика\\
\textbf{Напоминание}\\
Квантиль порядка $\alpha$ -- $q_\alpha: P(X \geq q_\alpha) \geq 1 - \alpha, P(X \leq q_\alpha) \geq \alpha$\\
Есди $F$ -- строго монотонная, то $F(q_\alpha) = \alpha \LRto q_\alpha = F^{-1}(\alpha)$\\
\textbf{Определение}\\
Выборочный квантиль порядка $0$ = $\min X_i$\\
Выборочный квантиль порядка $1 = \max X_i$\\
Выборочный квантиль порядка $\alpha \in (0, 1)$:\\
$\ex 0 \leq k \leq n-1: \frac kn \leq \alpha < \frac{k+1}n$\\
Тогда $X_{(k)}$ -- искомый квантиль\\
$\alpha = \frac14$ -- первый (нижний) квартиль\\
$\alpha = \frac12$ -- второй квартиль, выборочная медиана\\
$\alpha = \frac34$ -- третий (верхний) квартиль\\
$\alpha = 1$ -- четвертный квартиль / максимум\\\\
$n = 2m \Rto med(X) = \frac{X_{(m)} + X_{(m + 1)}}2$\\
$n = 2m + 1 \Rto \med(X) = X_{(m+1)}$\\\\
$P(X_{(k)}\leq t) = P(\mu_n(t) \geq k) = B(F(x), k, n - k + 1)$\\
Пусть $p(t)$ -- теоретическая плотность т.е. $p = F'$\\
$P(X_{(k)} \leq t)'_t = \frac{\Gamma(n+1)}{\Gamma(k)\Gamma(n-k+1)} F^{k-1}(t)(1-F(t))^{n-k}p(t)$ -- плотность $k$-ой порядковой статистики\\
$g(x_1, x_2) = \frac{n!}{(k-1)!(r-k-1)!(n-r)!}F^{k-1}(x_1)(F(x_2)-F(x_1))^{r-k-1}(1-F(x_2))^{n-r}p(x_1)p(x)2$ -- совместная плотность вектора $(x_{(k)}, x_{(r)}), k < r$\\
$g(x_1, \ldots, x_n) = n!p(x_1)\ldots p(x_n)$ -- плотность для вектора всех статистик $(X_{(1)}, \ldots, X_{(n)})$\\
\textbf{Определение}\\
Средний член вариационного ряда -- $X_{(k(n))}, \frac{k(n)}n \rto \const \in (0, 1)$\\
Крайний член варианционного ряда -- $X_{(r)}, r$ -- ограничено по $n$ или $X_{n + 1 - s}, s$ -- ограничено\\
\textbf{Теорема (об асимптотике среднего члена вариационного ряда)}\\
Пусть $0 < \alpha < 1, p$ -- теоретическая плотность, $q_\alpha$ -- теоретическая квантиль порядка $\alpha$\\
$p \in C^1(U(q_\alpha))$\\
Тогда $\sqrt{n} p(q_\alpha)\frac{X_{(\lfloor n\alpha \rfloor)} - q_\alpha}{\sqrt{\alpha (1-\alpha)}} \xrto d N(0,1)$\\
\textbf{Доказательство}\\
Комментарии к доказательству в лекции 3, 0:55
\begin{enumerate}
	\item $k := \lfloor n\alpha \rfloor$. Выпишем плотность $X_{(k)}$
	\item Напишем плотность преобразования над $X_{(k)}$
\end{enumerate}
\textbf{Теорема (об асимптотике крайнего члена вариационного ряда)}\\
Пусть $r,s$ -- фиксированные, $p$ -- плотность\\
Тогда $nF(X_{(r)}) \xrto d \Gamma(r, 1), nF(X_{(n+1-s)}) \xrto d \Gamma(s, 1)$ -- независимые
\section{Точечное оценивание параметров}
\subsection{Постановка задачи точечного оценивания параметрова}
Пусть $X_1, \ldots, X_n \sim F_\theta ,\theta \in \encirc H \subset \Rset^d$ -- параметр\\
В классической постановке $\theta$ -- фиксированный неизвестный вектор\\
Цель: оценить $\theta$ в виде функции $\hat \theta = \hat \theta(X_1, \ldots, X_n)$ от выборки\\
\textbf{Замечание}
\begin{enumerate}
	\item Функции от выборки принято называть статистиками
	\item Байесовская постановка: $\theta$ -- случайная величина из известного априорного распределения
\end{enumerate}
\textbf{Определение (Состоятельность)}\\
$\hat \theta$ -- состоятельная оценка $\theta \LRto \hat \theta \xrto p \theta \LRto P(\| \hat \theta - \theta\| > \eps) \xrto[n \rto \infty]{} 0$\\
\textbf{Определение (несмещенность)}\\
$bias(\hat \theta) = E\hat \theta - \theta$ -- смещение\\
$bias(\hat \theta) = 0 \LRto \hat \theta\ \fall n$ -- несмещенная\\
$bias(\hat \theta) = 0 \LRto \hat \theta$ при $n \rto \infty$ -- асимптотическая несмещенная\\
\textbf{Определение (асимптотическая нормальность)}\\
$\sqrt{n}(\hat \theta - \theta) \xrto{n} N(0, \Sigma_\theta)$\\
\textbf{Определение (эффективность/оптимальность)}\\
$\hat \theta_1$ -- эффективнее $\hat \theta_2 \LRto MSE(\hat \theta_1) < MSE(\hat \theta_2)$, где $MSE(\hat \theta) = E\|\hat\theta - \theta\|^2 = E(\hat\theta - \theta)^T(\hat\theta - \theta)$\\
$MSE(\hat \theta) = \tr(\Var \hat \theta) + \|bias(\hat \theta)\|^2$\\
\textbf{Доказательство}\\
$E(\hat\theta - \theta)^T(\hat\theta - \theta) = E(\hat\theta - E\hat\theta + E\hat\theta - \theta)^T(\hat\theta - E\hat\theta + E\hat\theta - \theta) = E\ub{\Var \hat \theta}{(\hat \theta - E\hat\theta)^T(\hat \theta - E\hat\theta)}  + \|bias(\hat \theta)\|^2$\\
\textbf{Метод моментов}\\
Рассмотрим $g_1, \ldots, g_d: \ex Eg_i(X_1) = m_j(\theta_1, \ldots, \theta_d) \xrto{\text{выборочные аналоги}} \ol{g_i(X_1)} = m_i(\hat \theta_1, \ldots, \hat \theta_d)$ -- получили уравнения от $\theta$\\
Решая уравнения, получаем оценки\\
Часто берут $g_i(x) = x^i$ -- отсюда метод моментов (но можно брать и другие функции)\\
\begin{enumerate}
	\item Асимптотическая нормальность $\Rto$ состоятельность\\
	\textbf{Доказательство}\\
	$\hat \theta - \theta = \frac1{\sqrt n} \sqrt{n} (\hat \theta - \theta) \xrto p 0$
	\item Асимптотическая нормальность $\Rto bias(\hat \theta) \rto 0$\\
	\textbf{Доказательство}\\
	Пусть $d=2$\\
	$P(|\theta-\hat \theta| > \eps) = P(\frac{\sqrt{n}|\theta-E\hat \theta|}{\sigma} > \frac{\sqrt n \eps}{\sigma}) = 1-P(\frac{\sqrt{n}|\theta-E\hat \theta|}{\sigma} < \frac{\sqrt n \eps}{\sigma}) \approx 2(1 - \Phi(\frac{\sqrt n \eps}{\sigma})) = 2(1-\Phi(\frac{\sqrt n \eps}{\sigma})) \rto 0$
	\item Состоятельность $\Rto bias(\hat \theta) \rto 0$\\
	\textbf{Доказательство}\\
	Следует из УЗБЧ\\
	$\ol X \xrto{a.s.} \mu \Rto E\ol X \rto \mu$
	\item Пусть $d = 1, bias \hat \theta \rto 0, \Var \ot \theta \rto 0 \Rto \hat\theta$ -- состоятельная
\end{enumerate}
\textbf{Замечание}
\begin{enumerate}
	\item Если $(\ol{g_1(X)}, \ldots, \ol{g_d(X)})$ -- состоятельная оценка для $(\ldots, \ol{Eg_i(X)}{m_i}, \ldots)$, $\alpha_1, \ldots, \alpha_d$ -- непрерывные от $\ol{g_1(X)}, \ldots, \ol{g_d(X)}$, то они состоятельные
	\item Если $(\ol{g_1(X)}, \ldots, \ol{g_d(X)})$ -- асимптотически нормальные и $g_1, \ldots, g_d$ -- гладкие, то каждая оценка асимптотически нормальная
\end{enumerate}
\section{Метод максимального правдоподобия}
$pmf: p(x, \theta) = p(x|\theta)$\\
$pdf: p(x, \theta) = p(x|\theta)$\\
Все это будем называть плотностью\\\\
$X_1, \ldots, X_n \sim p(X|\theta)$\\
$L(X|\theta) = \prod_i p(X_i|\theta)$ -- функция правдоподобия\\
$\hat \theta_* = \argmax L(X|\theta_i)$\\\\
Предположим, что $\theta \in B$ -- откр., $\theta_1 \neq \theta_2 \Rto L(X, \theta_1) \neq L(X, \theta_2)$
Алгоритм
\begin{enumerate}
	\item Рассмотрим $\ln L(X, \theta)$
	\item Приравняем производную к нулю
	\item Найдем максимум
\end{enumerate}
\textbf{Пример}\\
$Poly(1, p), p = (p_1, \ldots, p_m)$\\
$\nu_1, \ldots, \nu_m$ -- количество наблюдений типа $1, \ldots, m$\\
$L(X, p) = p_1^{\nu_1} \cdot \ldots \cdot p_m^{\nu_m}$\\
$\ln L(X, p) = \sum_{j=1}^{m-1} \nu_j \ln p_j + \nu_m \ln (1 - p_1 -\ldots - p_{m-1})$\\
$\ppart{\ln L}{\ln p_j} = \frac{\nu_j}{p_j} - \frac{\nu_m}{1-p_1 - \ldots - p_{m-1}}$\\
Суммируем уравнения: $\nu_j (1 - \hat p_1 -\ldots - \hat p_{n-1}) = \hat p_j \nu_m$\\
$\hat p_m (n - \nu_m) = \nu_m (1-\hat p_m)$\\
$\hat p_m = \frac{\nu_m}{n}$\\
Аналогично $\hat p_j = \frac{\nu_j}{n}$\\
\textbf{Определение (информация Фишера)}\\
Для $d=1$\\
$L(X, \theta) = \prod p(X_j, \theta)$\\
$\ln L(X, \theta) = \sum \ln p(X_j, \theta)$\\
$V(X,\theta) = \ppart{\ln L}{\theta} = \sum \ppart{\ln p}{\theta}$ -- вклад выборки\\
Пусть $\theta \in \encirc H$ -- открыто\\
$\theta_1 \neq \theta_2 \Rto p(X, \theta_1) = p(X, \theta_2)$\\
Регулярность:
\begin{enumerate}
	\item $\ppart{}{\theta} \int_X T(X) L(X, \theta) \int_X \ppart{}{\theta} L(X, \theta) T(X) \df x$\\
	\textbf{Необходимое условие}\\
	$\supp P_x$ нне зависит от $\theta$\\
	$U[0, \theta]: \int_0^\theta \frac1\theta \df t = 1$\\
	$(\int_0^\theta \frac1\theta \df t)' = ( \frac1\theta\int_0^\theta \df t)' = -\frac{1}{\theta^2} \int_0^\theta\df t + \frac1\theta \neq \int_0^\theta (\frac1\theta)' \df t$
	\item $EV^2(X,\theta) < \infty$\\
	$\int_X L(X, \theta) \df X = 1$\\
	$\int_X \ppart{L}{\theta} \df X = \int_X \frac{\frac{L}{\theta}}{L}L\df X = \int_X V(X, \theta) L(X,\theta) \df X = EV(X, \theta) = 0$
\end{enumerate}
$I(\theta) = Var V(X, \theta) = EV^2(X, \theta)$ -- информация Фишера всей выборки\\
$V(X, \theta) = \sum_j \ppart{\ln \hat p(X_j, \theta)}{\theta} = \Var V(X, \theta) = n \ub{i(\theta)\text{ -- информация Фишера набора}}{\Var \ppart{\ln p(X_j, \theta)}{\theta}}$\\
$i(\theta) = E(\ppart{\ln p(X_j, \theta)}{\theta})^2 = -E\ppart{\partial \ln p}{\theta^2}$\\
$\ppart{}{\theta}\int_\Rset \ppart{\ln p(x, \theta)}{\theta}p(x, \theta)\df x = \int_\Rset \ppart{\partial \ln p(x, \theta)}{\theta^2} p(x, \theta) + \int_\Rset \ppart{L}{\theta}\ppart{p}{\theta}\df x = E\ppart{\partial \ln p}{\theta^2} + \ub{i(\theta)}{E(\ppart{\ln p}{\theta})^2} = 0$\\
Для произвольного $d$\\
$i(\theta) = -(E\ppart{\partial \ln p(X_1, \theta)}{\theta_i\partial \theta_j})_{1\leq i,j \leq d}$ -- информационная матрица для 1 набора\\
$I(\theta) = n\cdot i(\theta)$\\\\
Рассмотрим $N(\theta_1, \theta_2)$\\
$p(x, \theta_1, \theta_2) = \frac{1}{\sqrt{2\pi \theta_2}} \exp(-\frac{(x-\theta_1)^2}{2\theta_2})$\\
$\ln p(x, \theta_1, \theta_2) = -\frac12 \ln 2\pi - \frac12\ln \theta_2 - \frac{(x-\theta_1)^2}{2\theta_2}$\\
$\ppart{\ln p}{\theta_1} = \frac{(x-\theta_1)}{\theta_2}$\\
$\ppart{\ln p}{\theta_2} = -\frac1{2\theta_2} + \frac{(x-\theta_1)^2}{2\theta_2^2}$\\
$\frac{\partial^2 f}{\partial \theta_1^2} = -\frac{1}{2\theta_2}$\\
$\frac{\partial^2 f}{\partial \theta_2^2} = \frac1{2\theta_2^2} - \frac{(x-\theta_1)^2}{\theta_2^3}$\\
$\frac{\partial^2 f}{\partial \theta_1\theta_2} = -\frac{(x-\theta_1)}{\theta_2^2}$\\
$i(\theta) = \begin{pmatrix}
	\frac{1}{\theta_2} & 0\\
	0 & \frac{1}{2\theta_2^2}
\end{pmatrix}$\\
По неравенству Рао-Крамера $\Var \hat \theta_1 \geq \frac{\theta_2}{n}$\\
\textbf{Теорема (неравенство Рао-Крамера)}\\
Пусть модель регулярна, $d=1$\\
$\tau(\theta)$ -- оцениваемая функция, $\tau \in C^1, \tau(\theta) \equiv \theta$\\
$\hat \tau(\theta)$ -- оценка несмещенная, т.е. $E(\hat \tau(\theta)) = \tau(\theta)$\\
Тогда $\Var \hat \tau(\theta) \geq \frac{(\tau'(\theta))^2}{n\cdot i(\theta)}$\\
\textbf{Доказательство}\\
$\tau'(\theta) = \int \hat \tau(\theta) \ppart{L}{\theta} \df x = \frac{\int \tau(\theta) V(X, \theta)L(X, \theta) \df x}{E\hat \tau(\theta) V(x, \theta)} - EV(X, \theta)E\hat \tau(\theta) = \Cov (V(X, \theta), \hat \tau(\theta))$\\
$\Cov^2 (V(X, \theta), \hat \tau(\theta)) \leq \Var V(X, \theta) \Var \hat \tau(\theta)$\\
\textbf{Замечание}
\begin{enumerate}
	\item $E\hat \tau(\theta) - \tau(\theta) = bias(\theta) \neq 0$\\
	$E\hat \tau(\theta) = \tau(\theta) + bias(\theta)$\\
	$\Var \hat \tau(\theta) \geq \frac{(\tau'(\theta) + bias'(\theta))^2}{ni(\theta)}$\\
	$MSE(\tau(\theta)) \geq \frac{(\tau'(\theta) + bias'(\theta))^2}{ni(\theta)} + bias^2(\theta)$
	\item $\Cov^2 (V(X, \theta), \hat \tau(\theta)) = \Var V(X, \theta) \Var \hat \tau(\theta)$, если $\hat\tau(\theta) = \alpha(\theta) V(X, \theta) + \tau(\theta)$
\end{enumerate}
\textbf{Теорема (неравенство Рао-Крамера)}\\
Пусть модель регулярна, $d>1$\\
$\tau(\theta): \Rset^d \rto \Rset, \tau \in C^1, \tau(\theta) \equiv \theta$\\
$\hat \tau(\theta)$ -- оценка несмещенная, т.е. $E(\hat \tau(\theta)) = \tau(\theta)$\\
Тогда $\Var \hat \tau(\theta) \geq \frac{\nabla \tau(\theta) i^{-1}(\theta) \nabla^T \tau(\theta)}{n}$\\
\textbf{Свойства оценки максимального правдоподобия}
Если существует несмещенная оптимальная оценка в регулярном случае, то она совпадает с оценкой максимального правдоподобия\\
\subsection{Состоятельность оценки максимального правдоподобия}
Пусть $\theta_0$ -- реальный параметр, $\theta \neq \theta_0$\\
Тогда $P_{\theta_0} (L(X, \theta_0) > L(X, \theta)) \rto 1$\\
\textbf{Доказательство}\\
$\frac{L(X, \theta)}{L(X, \theta_0)} < 1$\\
$\frac1n\sum \ln \frac{p(X_j, \theta)}{p(X_j, \theta_0)} < 0$\\
По ЗБЧ $E_{\theta_0} \ln\frac{p(X_j, \theta)}{p(X_j, \theta_0)} \leq E_{\theta_0} (\frac{p(X_j, \theta)}{p(X_j, \theta_0)} - 1) = \int_X p(X, \theta) \df X - \int p(X, \theta_0) \df X = 0$\\\\
Пусть $S_n = \{X: \ln L(X, \theta_0) > \ln L(X, \theta_0 - \alpha)\} \cap \{X: \ln L(X, \theta_0) > \ln L(X, \theta_0 + \alpha)\}$\\
$P_{\theta_0}(S_n) \rto 1$\\
$A_n = \{X: |\hat \theta - \theta_0| < \alpha\}$\\
$B_n = \{X: \ppart{\ln L(X, \theta)}{\theta} \vl_{\theta=\hat \theta} = 0\}$\\
$S_n \subset A_nB_n \subset A_n$\\
Отсюда $P(A_n) \rto 1$ --  т.о. оценка состоятельная
\subsection{Принцип инвариантности правдоподия}
% Лекция 6 0:00
$\theta \in \encirc H \us{\phi}{\os{\text{биекция}}\lrto} \gamma \in \Gamma$\\
$\gamma = \Phi(\theta)$\\
$\theta = \phi^{-1}$\\
$\sup_\theta L(X, \phi(\theta)) = \sup_\gamma L(X, \theta)$\\
\textbf{Пример}\\
Дано $Exp(\lambda)$\\
Тогда $\lambda e^{-\lambda x} \mapsto \frac1{\ol X}$\\
$\frac1\lambda e^{-\frac x\lambda} \mapsto \ol X$\\
\textbf{Теорема (асимптотическая нормальность оценки максимального правдоподобия)}\\
Пусь модель регулярна\\
$|\frac{\partial^3 \ln F(X, \theta)}{\partial \theta_i \partial \theta_j \partial \theta_k}| \leq M(X), EM(X) < \infty$\\
$\theta_*$ -- оценка максимального правдоподия\\
$\nabla \ln F(X, \theta) = 0$ -- имеет единственное решение\\
Тогда 
\begin{enumerate}
	\item $\sqrt{n}(\theta_* -\theta) \rto N(0, i^{-1}(\theta))$
	\item Если $\tau(\theta)$ -- оцениваемая и $\in C^1$, то\\
		  $\sqrt{n}(\tau(\theta_*) - \tau(\theta)) \rto N(0, \theta^2)$\\
		  $\sigma^2 = \nabla \tau(\theta)i^{-1}(\theta)\nabla \tau(\theta)$
	\item Если $\sigma^2$ непрерывная от $\sigma$, то $\sqrt{n} \frac{\tau(\sigma_*) - \tau(\theta)}{\sigma(\theta_*)} \rto N(0, 1)$ 
\end{enumerate}
\textbf{Доказательство 1}\\
$V(X, \theta) = \ppart{\ln L(X, \theta)}{\theta}$\\
$\theta_0$ -- реальный параметр\\
$V(X, \theta) = V(X, \theta_0) + V_\theta'(X, \theta_0)(\theta - \theta_0) + V_\theta''(X, \ot \theta)\frac{(\theta - \theta_0)^2}{2}, \ot \theta \in (\theta, \theta_0)$\\
Выполним подстановку $\theta = \theta_*$\\
$0 = V(X, \theta_0) + V'_\theta(X, \theta_0)(\theta_* - \theta_0) + V''_\theta(X, \ot \theta)\frac{(\theta_*-\theta_0)^2}{2}$\\
$V'_\theta(X, \theta_0)(\theta_* - \theta_0) = -V(X, \theta_0)-V''_\theta(X, \ot \theta)\frac{(\theta_*-\theta_0)^2}{2}$\\
$\sqrt{n}V'_\theta(X, \theta_0)(\theta_* - \theta_0) = -\sqrt{n}V(X, \theta_0)-\sqrt{n}V''_\theta(X, \ot \theta)\frac{(\theta_*-\theta_0)^2}{2}$\\
Заметим, что $V(X, \theta_0)$ на самом деле представляет сумму независимых одинаково распределенных величин с матожиданием 0 и дисперсией, равной информации Фишера\\
Тогда по ЦПТ $-\sqrt{n}V(X, \theta_0) = A_n = N(0, i(\theta))$\\
$\sqrt{n}V''_\theta(X, \ot \theta)\frac{(\theta_*-\theta_0)^2}{2} = n^{\frac32}\ub{\text{ограниченно по ЗБЧ}}{\frac{V''_\theta(X, \ot \theta)}{n}}\frac{(\theta_*-\theta_0)^2}{2}$\\
% Лекция 6 0:28
$V'_\theta(X, \theta_0) = n\frac{V'_\theta(X, \theta_0)}n \rto -ni(\theta)$ по ЗБЧ\\
$\sqrt{n}V'_\theta(X, \theta_0)(\theta_* - \theta_0) \rto N(0, i(\theta))$\\
TO BE CONTINUED\\
Какая-то лажа, смотри \url{https://t.me/c/2069367863/1/726}\\
\textbf{Определение}\\
*Вспомним неравенство Рао-Крамера*\\
Показатель эффективности -- $\frac{(\tau'(\theta))^2}{n\cdot i(\theta) \Var \widehat\tau (\theta)} \in [0, 1]$ для регулярных моделей\\
Если ПЭ = 1, то оценка эффективная\\
\textbf{Определение}\\
Пусть $\sqrt{n} (\widehat \theta - \theta_0) \rto N(0, \tub{какое-то число}{\frac{\sigma^2}{n}})$\\
Показатель асимптотической эффективности -- $\frac1{i(\theta)\sigma^2}$
\subsection{Экспоненциальное семейство распределений}
Пусть модель регулярна\\
Если $p(x, \theta) = \exp(A(\theta)B(x) + C(\theta) + D(x))$, то распределение относится к регулярному распределению\\
Примеры: $N, \Gamma, Pois, Bin, NB$\\
\textbf{Свойства}\\
$\ln p(x, \theta) = A(\theta)B(x) + C(\theta) + D(x)$\\
$\ppart{\ln p(x, \theta)}{\theta} = A'(\theta) B(x) + C'(x)$\\
$V(X, \theta) = A'(\theta)\sum B(X_i) + nC'(\theta)$\\
$V(X, \theta) = n(A'(\theta) \ol{B(X)} + C'(\theta))$\\
$\frac{V(X, \theta)}{n} - C'(\theta) = A'(\theta) \ol{B(X)}$\\
$\ol{B(X)} = \frac{V(X, \theta)}{n A'(\theta)} - \frac{C'(\theta)}{A'(\theta)}$\\
Тогда $\ol B(X)$ -- оптимальная оценка для $- \frac{C'(\theta)}{A'(\theta)}$ по критерию оптимальности
\subsection{Байесовская постановка}
$X_1, \ldots, X_n \in F_\theta$\\
$\theta$ -- неизвестный параметр \\
$\theta \sim \Pi(\theta)$ -- априорное распределение\\
$l(\widehat \theta, \theta)$ -- функция потерь\\
$R(\widehat \theta, \theta) = E_{F_\theta}l(\widehat \theta, \theta)$ -- риск\\
$r(\widehat \theta) = E_{\pi(\theta)} R(\widehat \theta, \theta)$ -- байесовский риск\\
$\widehat \theta_B = \argmin_{\widehat \theta} r(\widehat \theta)$\\
$r(\widehat \theta) = E_{(\pi(\theta), F_\theta)}l(\widehat \theta, \theta)$\\
\textbf{Теорема Байеса для плотностей}\\
$p(\theta|X) = \frac{L(X|\theta)\pi(\theta)}{\int L(X|\theta)\pi(\theta) \df \theta}$\\
\textbf{Утверждение}\\
$\widehat \theta_B = \argmin_{\widehat \theta} E[l(\widehat{\theta}, \theta)| X]$\\
\textbf{Доказательство}\\
Пусть $\theta_* = \argmin ...$\\
$r(\theta_*) = EE[l(\theta_*, \theta)|X] \leq EE[l(\widehat \theta, \theta)|X] = r(\widehat \theta)$\\
\textbf{Замечание}\\
$l(\widehat \theta, \theta) = (\theta - \widehat \theta)^2 \Rto \widehat{\theta_B} = E[\theta|X]$\\
% лекция 6 1:05
% пример для какого-то распределения и доказательство замечания
\subsection{Минимаксная оценка}
% лекция 7 0:00
$m(\hat \theta) = \sup_\theta R(\hat \theta, \theta)$\\
$\hat \theta_{wc} = \argmin m(\hat \theta)$ -- минимаксная оценка\\
\textbf{Утверждение}\\
$r(\hat \theta) \leq \mu(\hat \theta)$ -- по определению\\
\textbf{Утверждение}\\
Если $\ex \pi(\theta)$ -- априорное распределение $: R(\hat \theta_B, \theta) \equiv \const$\\
Тогда $\hat \theta_{wc} = \hat\theta_B$\\
\textbf{Доказательство}\\
Пусть $\ex \hat \theta: m(\hat \theta) < m(\theta)$\\
Тогда $r(\hat \theta) \leq m(\hat \theta) < m(\hat \theta_B) = r(\hat \theta_B)$ -- противоречие с определением $\hat \theta_B$\\
% лекция 7 0:15
% пример для Bern(p)
\subsection{Интервальное оценивание}
\textbf{Определение (доверительный интервал)}\\
$X_1, \ldots, X_n \sim F_\theta, \theta \in \encirc H \subset \Rset$\\
$1 - \alpha = \gamma \in (0, 1)$ -- уровень доверия\\
Рассмотрим $(T_l(X), T_r(X))$ -- доверительный интервал уровня $\gamma = 1-\alpha$, если $P(\theta \in (T_l(X), T_r(X))) \geq \gamma$\\
\textbf{Классическая схема построения доверительных интервалов}\\
Пусть $T(X, \theta) \sim G$ -- не зависит от $\theta$\\
Рассмотрим $P(q_1 < T(X, \theta) < q_2) = 1-\alpha$ -- доверительный интервал\\
Потребуем, чтобы $P(T(X, \theta) \leq q_1) = P(T(X, \theta) \geq q_2) = \frac\theta2$\\
Тогда $q_1 = q_{\frac\alpha2}, q_2 = q_{1-\frac\alpha2}, q_\bullet$ -- квантили\\
% \textbf{<<Универсальный>> рецепт}
% Пусть $X_i \sim F_\theta$
% \begin{enumerate}
% 	\item $F_\theta(X_i) \sim U(0, 1)$
% 	\item $-\ln F_\theta(X_k) \sim Exp(1)$
% 	\item $\sum -\ln F_\theta(X_k) \sim \Gamma(n, 1)$
% \end{enumerate}
% лекция 7 0:38
\subsection{Доверительные интервалы параметров нормального закона}
\textbf{Лемма о независимости линейной и квадратической статистик}\\
$X_1, \ldots, X_n \sim N(\theta, \sigma^2) -- i.i.d.$\\
$T=AX; X = (X_1, \ldots, X_n)^T, A \in M_{m\times n}(\Rset)$ -- линейная статистика\\
$Q = X^T B X, B \in M_n(\Rset), B = B^T$ -- квадратичная статистика\\
$AB = 0$\\
Тогда $T, Q$ -- независимые\\
\textbf{Доказательство}\\
$\Lambda = \nm{diag}(\lambda_1, \ldots, \lambda_m, 0, \ldots, 0), \lambda_i \neq 0$ -- потенциально нули в конце\\
$\Lambda - U^T BU$ -- в силу симметричности\\
$U = (u_1, \ldots, u_n)$ -- собственные вектора, образуют ортонормированный базис\\
$B = U\Lambda U^T = \sum_{j=1}^m \lambda_j u_j  u^T_j$\\
$Q = \sum_{j=1}^m \lambda_j (X^T u_j)  (u^T_j X) = \sum_j \lambda_j (u^T_j X)^2$\\
$A(\sum_{j=1}^m \lambda_j u_j  u^T_j) = 0$\\
$\sum_{j=1}^m \lambda_j A u_j  u^T_j = 0$\\
Возьмем $k \in [1, m]$\\
Домножим справа на $u_k$\\
$\sum_{j=1}^m \lambda_j A u_j  \ub{\1(j = k)}{u^T_j u_k} = 0$\\
$\lambda_k A u_k = 0$\\
Отсюда $A u_k = 0$\\
Рассмотрим вектор $\begin{pmatrix}
	u^T\\
	A
\end{pmatrix}X$ -- гауссовский вектор\\
Проверим, что $A_iX$ и $u^T_k X$ -- независимые $\fall i, k$\\
$\Cov (A_i X, u^T_k X) = A_i \ub{\neq 0}{\Cov (X, X^T)} u_k = A_i \ub{\sigma^2 \cdot E}{\Var X} u_k = \sigma^2 A_i u_k = 0$\\
Т.е. статистики независимые\\
\textbf{Лемма о независимости двух независимых статистик}\\
$Q_1 = X^TB_1X$\\
$Q_2 = X^T B_2 X$\\
$B_1B_2 = B_2B_1 = 0$\\
Тогда $Q_1, Q_2$ -- независимые\\
\textbf{Доказательство}\\
Аналогично\\
\textbf{Определение}\\
$X_1, \ldots, X_n \sim N(0, 1)$\\
Тогда $\sum_{k=1}^n X_k^2 \sim \chi^2(n)$ -- распределение хи-квадрат с $n$ степенями свободы\\
\textbf{Лемма о распределении квадратичной статистики}\\
$X_1, \ldots, X_n \sim N(0, 1)$\\
$Q = X^T B X, B = B^2$\\
$r := \rg (B)$\\
Тогда $Q \sim \chi^2(r), r = \tr(B)$\\
\textbf{Доказательство}\\
Заметим, что собтсвенные числа либо 0, либо 1:
$\lambda \tub{с.в.}u = Bu=B^2u = B\lambda u = \lambda^2 u$\\
$B=\sum_{k=1}^r u_k u_k^T$\\
$Q = \sum_{k=1}^n (u^T_k X)^2$\\
$u^T_k X \sim N(\ub0{u_k^T EX}, \ub1{u_k^T E u_k})$\\
$\Cov (u_k^T X, u_j^T X) = 0$\\
Тогда $Q \sim \chi(r)$\\
$B = U\Lambda U^T$\\
$\rg B = \rg \Lambda = \tr \Lambda$\\
$B_{jj} = \lambda_j u_j u^T_j = \lambda_j$\\
Тогда $\tr B = \tr \Lambda = \rg B$\\
\textbf{Теорема Фишера}\\
Пусть $X_1, \ldots, X_n \sim N(\mu, \sigma^2)$\\
Тогда
\begin{enumerate}
	\item $\ol X \sim N(\mu, \frac{\sigma^2}{n})$\\
	\textbf{Доказательство} очевидно
	\item $\frac{nS^2_*}{\sigma^2} = \frac{(n-1)S^2}{\sigma^2} \sim \chi^2 (n-1)$\\
	\textbf{Доказательство}\\
	$Y_j = \frac{X_j - \mu}{\sigma}$\\
	$\ol Y = \frac1\sigma (\ol X - \mu)$\\
	$S_*^2(Y) = \frac1n \sum_{j=1}^n (Y_j - \ol Y)^2 = \frac{S_*^2 (X)}{\sigma^2}$\\
	$\ol Y = \frac{\sum Y_j}{n} = \ub b{\begin{pmatrix}
		\frac{1}{n} & \ldots & \frac{1}{n}
	\end{pmatrix}} \begin{pmatrix}
		Y_1 \\
		\vdots\\
		Y_n
	\end{pmatrix} = bY$\\
	$n S_*^2 (Y) = (Y-bY)^T (Y-bY) = Y^T (E-B)^T(E-B)Y, B = \begin{pmatrix}
		b \\ \vdots \\ b
	\end{pmatrix}$\\
	$(I-B)^T(I-B) = I - B$\\
	По предыдущей лемме $Y^T (E-B)^T(E-B)Y \sim \chi^2(\tr (I-B))$\\
	$\tr (I-B) = \sum_{k=1}^n (1-\frac1n) = n - 1$
	\item $S^2, \ol X $ -- независимые
	$S^2_*, \ol X$ -- независимые\\
	\textbf{Доказательство}\\
	$b(I - B) = b - b = 0$\\
	Тогда по лемме 1
\end{enumerate}
% Лекция 8 0:00
% пример для поиска доверительных интервалов всяких величин
% Лекция 8 0:30
\textbf{Определение}\\
$X_0, \ldots, X_n$ -- i.i.d $N(0,1)$\\
Тогда $\frac{X_0}{\sqrt{\frac1n \sum_{k=1}^n X_k^2}} \sim T(n)$ -- распределение Стьюдента\\
\textbf{Определение}\\
$\chi_n^2 \sim \chi^2(n)$\\
$\chi_m^2 \sim \chi^2(m)$ -- независимые\\
$\frac{\chi_n^2}{\chi_m^2} \sim F(n,m)$ -- распределение Фишера\\
\subsection{Асимптотические доверительные интервалы}
$\lim_{n\rto \infty} P(\theta \in (l_n, r_n)) \geq 1 -\alpha$\\
$T(X, \theta) \xrto[d]{} G, G$ не зависит от $\theta$
\section{Проверка статистических гипотез}
Основное предположение (по умолчанию)\\
Альтернативное предположение (хотим доказать)\\
\textbf{Определение}\\
Пусть есть выборка в широком смысле $X_1, \ldots, X_n$\\
Будем считать, что $(X_1, \ldots, X_n) \sim F$\\
$H_0 := (F \in \mc F_0)$ -- нулевая гипотеза (основная гипотеза)\\
$H_1 := (F \in \mc F_1)$ -- альтернатива\\
$\alpha \in (0,1)$ -- уровень значимости\\
Проводим стат.тест/критерий $\delta(X, \alpha)$:\\
% Лекция 9 0:00$
$\delta(X, \alpha) = \left\{\begin{array}{l}
	\tub{данные не противоречат нулевой гипотезе}{\text{accept } H_0}\\
	\tub{данные противоречат нулевой и свидетельствуют альтернативной}{\text{reject } H_0\text{ with respect to } H_1}
\end{array}\right.$\\
Тест не подтверждает и не опровергает гипотезу, но позволяет делать некоторые выводы о гипотезе\\\\
Рассмотрим $T(X)$ -- статистику критерия\\
$T(X) \sim G$ или $T(X) \xrto[n\rto \infty]{d} G$ при условии $H_0$\\
$P(T(X) \in T_0(\alpha) | H_0) = 1 -\alpha$\\
$P(T(X) \in T_1(\alpha) | H_0) = \alpha$ (при сходимости $\approx$)\\\
Если $T(x) \in T_1(\alpha)$ -- reject $H_0$\\
Иначе -- accept $H_0$\\
$\supp G = \tub{область принятия}{T_0(\alpha)} \sqcup \tub{область опровержения}{T_1(\alpha)}$\\
Виды тестов в зависимости от критической области:
\begin{enumerate}
	\item Left-sided\\
	$T_0(\alpha) = [q_\alpha, +\infty), T_1(\alpha) = (-\infty, q_\alpha)$\\
	$p_l = P(\Gamma \leq T(x) | H_0)$
	\item Right-sided\\
	$T_0(\alpha) = (-\infty,q_{1-\alpha}], T_1(\alpha) = (q_{1-\alpha}, +\infty)$
	$p_r = P(\Gamma > T(x) | H_0)$
	\item Two-sided\\
	$T_0(\alpha) = [q_{\frac\alpha2}, q_{1-\frac\alpha2}], T_1 = \ol{T_0}$
	$p=2\min(p_l, p_r)$
\end{enumerate}
Тогда мы получили еще одно условие:\\
Если $p < \alpha$ -- reject $H_0$\\
Иначе accept $H_0$\\\\
$p$ (p-value) -- это максимальный уровень, при котором мы принимаем $H_0$\\
Виды ошибок:
\begin{enumerate}
	\item first type error / false positive\\
	$P(T(X) \in T_1(\alpha) | H_0) = \alpha$
	\item second type error / false negative\\
	$P(T(X) \in T_0(\alpha) | H_1) = \beta$\\
	$1 - \beta$ -- мощность критерия
\end{enumerate}
\textbf{Пример (spam classifier)}\\
$H_0$ -- не спам\\
$H_1$ -- спам\\
Письма не фильтруются, $\alpha = 0 \Rto \beta$ -- большое, т.к. спама много\\
Все письма -- спам, $\beta = 0 \Rto \alpha$ -- большое
\subsection{Статистические критерии и доверительные интервалы}
Вспомним задачу построения доверительного интервала\\
$X_1, \ldots, X_n \sim F_\theta, T(X, \theta) \rto G, P(q_{\frac{\alpha}{2}} \leq T(X, \theta) \leq q_{1-\frac{\alpha}{2}}) = 1 - \alpha$\\
Рассмотрим следующую стат. гипотезу\\
$H_0: \theta = \theta_0$\\
$P(T(X, \theta_0) \in T_{\theta}|\theta = \theta_0) = 1-\alpha$\\
$H_1: \theta \neq \theta_0$ или $\theta > \theta_0$ или $\theta < \theta_0$\\\\
%lection 9 0:48:00 пример
%lection 9 1:09:00 F-test, T-test
\subsection{Критерий Колмагорова}
$X_1, \ldots, X_n \sim F$\\
$H_0: F=F_0, F_0$ -- непрерывная\\
$H_1: F\neq F_0$\\
\textbf{Теорема Колмагорова (напоминание)}\\
$P(\sqrt{n} \sup_{x\in \Rset} |F_n(x) - F(x)| \leq t) \rto K(t)$\\
Статистика критерия: $D_n = \sqrt{n}\sup_{x\in \Rset} |F_n(x) - F(x)|, F_n$ -- э.ф.р.\\
Если $D_n > q_{1-\alpha}$ -- reject $H_0$\\
Иначе accept $H_0$\\
\subsection{Критерий Смирнова}
$X_1, \dots, X_n, Y_1, \ldots, Y_m$ -- независимые\\
$H_0: F_X = F_Y (= F_0), F_0$ -- непрерывная\\
$H_1: \lnot H_0$\\
$D_{m,n} = \sqrt{\frac{nm}{n+m}} \sup_x |F_n(x) - F_m(x)|$\\
$T_1(\alpha) = (q_{1-\alpha}, +\infty)$\\
\subsection{Критерии типа хи-квадрат}
%lection 10 0:00
\textbf{Критерий согласия Пирсона}\\
Пусть есть $N$-элементное множество\\
$p=(p_1, \ldots, p_N)$ -- настоящий вектор вероятностей (не знаем)\\
$p_0 = (p_{01}, \ldots, p_{0N})$ -- ожидаемый, фиксированный вектор вероятностей\\
$\nu_k$ -- кол-во элементов типа $k$ в выборке, $n=\sum_k \nu_k$\\
$H_0: p = p_0$\\
$H_1: p \neq p_1$\\
Тогда возьмем статистику $\chi_N^2 = \sum_{k=1}^N \frac{(\nu_k - n p_{0k})^2}{np_{0k}}$\\
\textbf{Теорема}\\
$\chi_N^2 \xrto[n\rto \infty]{} \chi^2 (N-1)$ -- при условии $H_0$\\
\textbf{Доказательство для $N=2$}\\
$\frac{(\nu_1 - np_{01})^2}{np_{01}} + \frac{(\nu_2 - np_{02})^2}{np_{02}} = \frac{(\nu_1 - np_{01})^2}{n}(\frac1{p_{01} + \frac1{1-p_{01}}}) = \frac{(\nu_1 - np_{01})^2}{np_{01}(1-p_{01})} \rto (N(0,1))^2$\\
\textbf{Замечание}\\
Критерий состоятельный (мощность стремится к 1)\\
Критическая область правосторонняя (если $H_0$ верна, то $\chi$ будет мало)\\\\
Усложним задачу\\
$H_0: p = p_0(\theta), \theta \in \circ H \subset \Rset^d, d < N - 1$\\
Тогда $\chi_N^2 = \sum_{k=1}^N \frac{(\nu_k - np_{0k}(\theta))^2}{np_{0k}(\theta)}$\\
Вместо $\theta$ подставим оценку максимального правдоподия\\
\textbf{Теорема}\\
$p_0(\theta) > 0 \fall \theta$\\
$\ppart{p_0}{\theta}, \ppart{}{\theta}\ppart{}{\theta} p_0$ -- непрерывные\\
$\rg (\ppart{p_{0k}}{\theta_j})_{1\leq k \leq N, 1 \leq j \leq d} = d$\\
Тогда $\chi_N^2 \rto \chi^2(N-1-\tub{количество неизвестных}{d})$\\
%lection 10 0:37
\textbf{Критерий однородности}\\
Есть $K$ независимых выборок\\
Все они из $\{1\ldots,N\}$\\
Пусть $p^{(j)}$ -- истинный вектор вероятностей для $j$-ой выборки\\
$H_0: p^{(1)} = \ldots = p^{(k)}$ -- мы их не знаем\\
$H_1: \lnot H_0$\\
$\nu_{ij}$ -- количество элементов $j$ в выборке $i$\\
$n_i = \sum \nu_{i*}$\\
$n = n_1 + \ldots + n_k$\\
Пусть $p^{(*)}$ известны\\
$\chi_{n_1}^2 = \sum_j \frac{(\nu_{ij} - n_i p_j^{(i)})^2}{n_i p_j^{(i)}}, df = N - 1$\\
$\chi_n^2 = \sum \chi_n^2, df = k(N-1)$\\
Теперь $p^{(*)}$ -- неизвестные. Тогда суммарно $(N-1)$ неизвестных\\
Тогда из прошлой теоремы $df = k(N-1) - (N-1) = (N-1)(k-1)$\\
Подставим вместо $p^{(j)}$ оценку максимального правдоподия $\widehat{p_j} = \frac{\nu_{1j}+\ldots + \nu_{kj}}{n}$\\
\textbf{Критерий независимости}\\
$x_1, \ldots, x_n \in \{1,\ldots,N\}$\\
$y_1, \ldots, y_n \in \{1,\ldots,M\}$\\
$\nu_{ij}$ -- количество пар, в которых первая компонента $i$, вторая -- $j$\\
$p_{ij} = P(X=i, Y=j)$\\
$p_{X_i} = p(X=i)$\\
$p_{Y_j} = p(Y=j)$\\
$H_0: p_{ij} = p_{X_i}p_{Y_j}$\\
$H_1: \lnot H_0$\\
$\chi^2 = \sum_{ij} \frac{(\nu_{ij} - p_{ij}n)^2}{np_{ij}}, df = NM-1$\\
Учитывая, что $p_{ij} = p_{X_i}p_{Y_j}, df = NM-1-N-M$\\

\end{document}