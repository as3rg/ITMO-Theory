\documentclass[12pt]{article}
\usepackage{bbold}
\usepackage{amsfonts}
\usepackage{amsmath}
\usepackage{amssymb}
\usepackage{color}
\setlength{\columnseprule}{1pt}
\usepackage[utf8]{inputenc}
\usepackage[T2A]{fontenc}
\usepackage[english, russian]{babel}
\usepackage{graphicx}
\usepackage{hyperref}
\usepackage{mathdots}
\usepackage{xfrac}


\def\columnseprulecolor{\color{black}}

\graphicspath{ {./resources/} }


\usepackage{listings}
\usepackage{xcolor}
\definecolor{codegreen}{rgb}{0,0.6,0}
\definecolor{codegray}{rgb}{0.5,0.5,0.5}
\definecolor{codepurple}{rgb}{0.58,0,0.82}
\definecolor{backcolour}{rgb}{0.95,0.95,0.92}
\lstdefinestyle{mystyle}{
    backgroundcolor=\color{backcolour},   
    commentstyle=\color{codegreen},
    keywordstyle=\color{magenta},
    numberstyle=\tiny\color{codegray},
    stringstyle=\color{codepurple},
    basicstyle=\ttfamily\footnotesize,
    breakatwhitespace=false,         
    breaklines=true,                 
    captionpos=b,                    
    keepspaces=true,                 
    numbers=left,                    
    numbersep=5pt,                  
    showspaces=false,                
    showstringspaces=false,
    showtabs=false,                  
    tabsize=2
}

\lstset{extendedchars=\true}
\lstset{style=mystyle}

\newcommand\0{\mathbb{0}}
\newcommand{\eps}{\varepsilon}
\newcommand\overdot{\overset{\bullet}}
\DeclareMathOperator{\sign}{sign}
\DeclareMathOperator{\re}{Re}
\DeclareMathOperator{\im}{Im}
\DeclareMathOperator{\Arg}{Arg}
\DeclareMathOperator{\const}{const}
\DeclareMathOperator{\rg}{rg}
\DeclareMathOperator{\Span}{span}
\DeclareMathOperator{\alt}{alt}
\DeclareMathOperator{\Sim}{sim}
\DeclareMathOperator{\inv}{inv}
\DeclareMathOperator{\dist}{dist}
\newcommand\1{\mathbb{1}}
\newcommand\ul{\underline}
\renewcommand{\bf}{\textbf}
\renewcommand{\it}{\textit}
\newcommand\vect{\overrightarrow}
\newcommand{\nm}{\operatorname}
\DeclareMathOperator{\df}{d}
\DeclareMathOperator{\tr}{tr}
\newcommand{\bb}{\mathbb}
\newcommand{\lan}{\langle}
\newcommand{\ran}{\rangle}
\newcommand{\an}[2]{\lan #1, #2 \ran}
\newcommand{\fall}{\forall\,}
\newcommand{\ex}{\exists\,}
\newcommand{\lto}{\leftarrow}
\newcommand{\xlto}{\xleftarrow}
\newcommand{\rto}{\rightarrow}
\newcommand{\xrto}{\xrightarrow}
\newcommand{\uto}{\uparrow}
\newcommand{\dto}{\downarrow}
\newcommand{\lrto}{\leftrightarrow}
\newcommand{\llto}{\leftleftarrows}
\newcommand{\rrto}{\rightrightarrows}
\newcommand{\Lto}{\Leftarrow}
\newcommand{\Rto}{\Rightarrow}
\newcommand{\Uto}{\Uparrow}
\newcommand{\Dto}{\Downarrow}
\newcommand{\LRto}{\Leftrightarrow}
\newcommand{\Rset}{\bb{R}}
\newcommand{\Rex}{\overline{\bb{R}}}
\newcommand{\Cset}{\bb{C}}
\newcommand{\Nset}{\bb{N}}
\newcommand{\Qset}{\bb{Q}}
\newcommand{\Zset}{\bb{Z}}
\newcommand{\Bset}{\bb{B}}
\renewcommand{\ker}{\nm{Ker}}
\renewcommand{\span}{\nm{span}}
\newcommand{\Def}{\nm{def}}
\newcommand{\mc}{\mathcal}
\newcommand{\mcA}{\mc{A}}
\newcommand{\mcB}{\mc{B}}
\newcommand{\mcC}{\mc{C}}
\newcommand{\mcD}{\mc{D}}
\newcommand{\mcJ}{\mc{J}}
\newcommand{\mcT}{\mc{T}}
\newcommand{\us}{\underset}
\newcommand{\os}{\overset}
\newcommand{\ol}{\overline}
\newcommand{\ot}{\widetilde}
\newcommand{\vl}{\Biggr|}
\newcommand{\ub}[2]{\underbrace{#2}_{#1}}

\def\letus{%
    \mathord{\setbox0=\hbox{$\exists$}%
             \hbox{\kern 0.125\wd0%
                   \vbox to \ht0{%
                      \hrule width 0.75\wd0%
                      \vfill%
                      \hrule width 0.75\wd0}%
                   \vrule height \ht0%
                   \kern 0.125\wd0}%
           }%
}
\DeclareMathOperator*\dlim{\underline{lim}}
\DeclareMathOperator*\ulim{\overline{lim}}

\everymath{\displaystyle}

% Grath
\usepackage{tikz}
\usetikzlibrary{positioning}
\usetikzlibrary{decorations.pathmorphing}
\tikzset{snake/.style={decorate, decoration=snake}}
\tikzset{node/.style={circle, draw=black!60, fill=white!5, very thick, minimum size=7mm}}

\title{Линейная алгебра. Практика}
\author{Александр Сергеев}
\date{}

\begin{document}
\maketitle
\textbf{Пример}\\\\
$\left\{
\begin{array}{c}
     a_{11}x_1+a_{12}x_2=b_1 \\
     a_{21}x_1+a_{22}x_2=b_1
\end{array}
\right.$ - Система линейных неоднородных уравнений.\\\\
$\left(\begin{array}{cc}
    a_{11} & a_{12} \\
    a_{21} & a_{22}
\end{array}\right)
\left(\begin{array}{cc}
    x_1  \\
    x_2
\end{array}\right) = 
\left(\begin{array}{cc}
    b_1  \\
    b_2
\end{array}\right)\\\\
\left\{
\begin{array}{cc}
     & (a_{11}a_{22}-a_{12}a_{21})x_1=b_1a_{22}-a_{12}b_2 \\
     & (a_{12}a_{21}-a_{11}a_{22})x_2=b_1a_{21}-a_{11}b_2
\end{array}
\right.$\\\\
$\left\{
\begin{array}{cc}
     & x_1=\frac{b_1a_{22}-a_{12}b_2}{a_{11}a_{22}-a_{12}a{21}} \\
     & x_2=\frac{b_1a_{21}-a_{11}b_2}{a_{12}a_{21}-a_{11}a_{22}}
\end{array}
\right.$\\\\
$\Delta_i$ - определитель матрицы, где $i$-ый столбец заменен на столбец свободных членов.\\\\
$\left\{
\begin{array}{cc}
     & x_1=\frac{\Delta_1}{\Delta} \\
     & x_2=\frac{\Delta_2}{\Delta}
\end{array}
\right.$\\\\
\textbf{Определение}\\
$\det A = \sum\limits(-1)^{inv(\sigma)}\cdot a_{1i_1}\cdot a_{2i_2}\cdot\ldots\cdot a_{ni_n}$, где $\sigma = (i_1, i_2, \ldots, i_n)$, где $inv(S)$ - количество пар $(a,b): a,b\in S, a > b$\\\\
\textbf{Теорема Крамера}\\
$\det A \neq 0 \Leftrightarrow \exists !\ \text{решение}\  x_i=\frac{\Delta_i}{\Delta},\ \text{где} \\
\Delta_i$ - определитель матрицы, где $i$-ый столбец заменен на столбец свободных членов.\\\\
Свойства:
\begin{enumerate}
    \item $|A|=|A^T|$
    \item Все свойства строк выполняются для столбцов
    \item $|\ldots, A_i+B_i, \ldots| = |\ldots, A_i, \ldots| + |\ldots, B_i, \ldots|$, где $A_i, B_i$ - столбцы - свойство аддитивности по столбцам
    \item $|\ldots, k\cdot A_i, \ldots| = k\cdot |\ldots, A_i, \ldots|$
    \item Если в матрице есть нулевой столбец, то определитель = 0
    \item $|\ldots, A_i,\ldots, A_j, \ldots| = -|\ldots, A_j, \ldots, A_i, \ldots|$
    \item Отслюда $|\ldots, S,\ldots, S, \ldots| = 0$
    \item $|\ldots, A_i,\ldots, A_j, \ldots| = |\ldots, A_i+\lambda A_j, \ldots, A_j, \ldots|$ - линейное преобразование 
    \item 
$\left|\begin{array}{ccc}
        \ldots&\ldots&\ldots \\
        a_{i1} & \ldots& a_{in}\\
        \ldots&\ldots&\ldots
\end{array}\right| = \sum\limits_{j=1}^n a_{ij}\cdot A_{ij}$, \\где $A_{ij}=(-1)^{i+j}\cdot M_{ij}$ - дополнение, \\$M_{ij}$ - минор(матрица, полученная вычеркиванием i строки и j столбца).
\end{enumerate}
\textbf{Теорема}\\
\textit{Пучок плоскостей}, проходящих через $\alpha_1 \cap \alpha_2$ задается уравнением $a(A_1x+B_1y+C_1z+D_1)+b(A_2x+B_2y+C_2z+D_2)=0$\\\\
\textbf{Пересечение линейных пространств}\\
Пусть $L_1, L_2 \subset V$ - линейные подпространства линейного пространства\\
$L_1 = \Span (a_1, \ldots, a_k), a_1, \ldots, a_k$ - базис
$L_2 = \Span (b_1, \ldots, b_m), b_1, \ldots, b_k$ - базис\\\\
Рассмотрим $u \in L_1 \cap L_2$\\
$u = \sum_{i=1}^k x_i a_i = \sum_{i=1}^m y_i b_i$\\
$\sum_{i=1}^k x_i a_i - \sum_{i=1}^m y_i b_i = 0$ - СЛОУ\\
$\begin{pmatrix}
    a_1 & a_2 & \ldots & a_k & -b_1 & \ldots & -b_m
\end{pmatrix}\begin{pmatrix}
    x_1\\x_2\\\vdots\\x_k\\y_1\\\vdots\\y_m
\end{pmatrix} = \0$\\
Решая систему в общем виде, можем найти все вектора пересечения\\
Отсюда можно найти базис\\

\end{document}