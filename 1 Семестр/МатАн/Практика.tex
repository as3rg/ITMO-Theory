\documentclass[12pt]{article}
\usepackage{bbold}
\usepackage{amsfonts}
\usepackage{amsmath}
\usepackage{amssymb}
\usepackage{color}
\setlength{\columnseprule}{1pt}
\usepackage[utf8]{inputenc}
\usepackage[T2A]{fontenc}
\usepackage[english, russian]{babel}
\usepackage{graphicx}
\usepackage{hyperref}
\usepackage{mathdots}
\usepackage{xfrac}


\def\columnseprulecolor{\color{black}}

\graphicspath{ {./resources/} }


\usepackage{listings}
\usepackage{xcolor}
\definecolor{codegreen}{rgb}{0,0.6,0}
\definecolor{codegray}{rgb}{0.5,0.5,0.5}
\definecolor{codepurple}{rgb}{0.58,0,0.82}
\definecolor{backcolour}{rgb}{0.95,0.95,0.92}
\lstdefinestyle{mystyle}{
    backgroundcolor=\color{backcolour},   
    commentstyle=\color{codegreen},
    keywordstyle=\color{magenta},
    numberstyle=\tiny\color{codegray},
    stringstyle=\color{codepurple},
    basicstyle=\ttfamily\footnotesize,
    breakatwhitespace=false,         
    breaklines=true,                 
    captionpos=b,                    
    keepspaces=true,                 
    numbers=left,                    
    numbersep=5pt,                  
    showspaces=false,                
    showstringspaces=false,
    showtabs=false,                  
    tabsize=2
}

\lstset{extendedchars=\true}
\lstset{style=mystyle}

\newcommand\0{\mathbb{0}}
\newcommand{\eps}{\varepsilon}
\newcommand\overdot{\overset{\bullet}}
\DeclareMathOperator{\sign}{sign}
\DeclareMathOperator{\re}{Re}
\DeclareMathOperator{\im}{Im}
\DeclareMathOperator{\Arg}{Arg}
\DeclareMathOperator{\const}{const}
\DeclareMathOperator{\rg}{rg}
\DeclareMathOperator{\Span}{span}
\DeclareMathOperator{\alt}{alt}
\DeclareMathOperator{\Sim}{sim}
\DeclareMathOperator{\inv}{inv}
\DeclareMathOperator{\dist}{dist}
\newcommand\1{\mathbb{1}}
\newcommand\ul{\underline}
\renewcommand{\bf}{\textbf}
\renewcommand{\it}{\textit}
\newcommand\vect{\overrightarrow}
\newcommand{\nm}{\operatorname}
\DeclareMathOperator{\df}{d}
\DeclareMathOperator{\tr}{tr}
\newcommand{\bb}{\mathbb}
\newcommand{\lan}{\langle}
\newcommand{\ran}{\rangle}
\newcommand{\an}[2]{\lan #1, #2 \ran}
\newcommand{\fall}{\forall\,}
\newcommand{\ex}{\exists\,}
\newcommand{\lto}{\leftarrow}
\newcommand{\xlto}{\xleftarrow}
\newcommand{\rto}{\rightarrow}
\newcommand{\xrto}{\xrightarrow}
\newcommand{\uto}{\uparrow}
\newcommand{\dto}{\downarrow}
\newcommand{\lrto}{\leftrightarrow}
\newcommand{\llto}{\leftleftarrows}
\newcommand{\rrto}{\rightrightarrows}
\newcommand{\Lto}{\Leftarrow}
\newcommand{\Rto}{\Rightarrow}
\newcommand{\Uto}{\Uparrow}
\newcommand{\Dto}{\Downarrow}
\newcommand{\LRto}{\Leftrightarrow}
\newcommand{\Rset}{\bb{R}}
\newcommand{\Rex}{\overline{\bb{R}}}
\newcommand{\Cset}{\bb{C}}
\newcommand{\Nset}{\bb{N}}
\newcommand{\Qset}{\bb{Q}}
\newcommand{\Zset}{\bb{Z}}
\newcommand{\Bset}{\bb{B}}
\renewcommand{\ker}{\nm{Ker}}
\renewcommand{\span}{\nm{span}}
\newcommand{\Def}{\nm{def}}
\newcommand{\mc}{\mathcal}
\newcommand{\mcA}{\mc{A}}
\newcommand{\mcB}{\mc{B}}
\newcommand{\mcC}{\mc{C}}
\newcommand{\mcD}{\mc{D}}
\newcommand{\mcJ}{\mc{J}}
\newcommand{\mcT}{\mc{T}}
\newcommand{\us}{\underset}
\newcommand{\os}{\overset}
\newcommand{\ol}{\overline}
\newcommand{\ot}{\widetilde}
\newcommand{\vl}{\Biggr|}
\newcommand{\ub}[2]{\underbrace{#2}_{#1}}

\def\letus{%
    \mathord{\setbox0=\hbox{$\exists$}%
             \hbox{\kern 0.125\wd0%
                   \vbox to \ht0{%
                      \hrule width 0.75\wd0%
                      \vfill%
                      \hrule width 0.75\wd0}%
                   \vrule height \ht0%
                   \kern 0.125\wd0}%
           }%
}
\DeclareMathOperator*\dlim{\underline{lim}}
\DeclareMathOperator*\ulim{\overline{lim}}

\everymath{\displaystyle}

% Grath
\usepackage{tikz}
\usetikzlibrary{positioning}
\usetikzlibrary{decorations.pathmorphing}
\tikzset{snake/.style={decorate, decoration=snake}}
\tikzset{node/.style={circle, draw=black!60, fill=white!5, very thick, minimum size=7mm}}

\title{Математический анализ. Практика}
\author{Александр Сергеев}
\date{}

\begin{document}
\maketitle
\section{Урок 05.09.2022}
Пусть есть функция $f(x)$\\\\
\textbf{Определение}\\
$f(x)$ - Инъекция, если $\forall x_0\ \overline{\exists x: f(x)=f(x_0)}$\\\\
\textbf{Определение}\\
$f(x)$ - Сюръекция, если $\forall y_0\ \exists x: f(x)=y_0$\\\\
\textbf{Определение}\\
$f(x)$ - Биекция, если $\forall y_0\ \exists! x: f(x)=y_0$\\\textit{(Сюръекция+Инъекция)}\\\\
Биекция из $A$ в $B$ $\Leftrightarrow |A| = |B|$\\\\
Сравним мощности групп чисел:\\
1) $|\mathbb{Z}| = |\mathbb{N}|$\\
\textbf{Доказательство}\\
Получим биекцию, сопоставив значениям $2k$ из $\mathbb{N}$ значения $k$ из $\mathbb{Z}$, а значениям $2k+1$ из $\mathbb{N}$ значения $-(1+k)$ из $\mathbb{Z}$. Ч.т.д.\\\\
2) $|\mathbb{Q}| = |\mathbb{N}|$\\
\textbf{Доказательство}\\
Представим числа из $\mathbb{Q}$ как $\frac{p}{q}$, где $p$ - целое, $q$ - положительное.
Составим таблицу, отложив по горизонтали p, а по вертикали q:\\
\begin{center}
\begin{tabular}{ |c|c c c c c c| } 
 \hline
   & 0 & 1 & -1 & 2 & -2 & 3\\
   \hline
 1 & 1 & 2 & 3 & 5 & 8 & 11 \\ 
 2 & . & 4 & 6 & . & . & \\ 
 3 & . & 7 & 9 & & & \\ 
 4 & . & 10 & & & & \\ 
 5 & . &  & & & & \\ 
 \hline
\end{tabular}
\end{center}
Мы получили бесконечную таблицу всех значений $\mathbb{Q}$. Пронумеруем ее по диагонали в направлении сверху вниз справа налево, пропуская повторяющиеся значения. Мы получили биекцию из $\mathbb{N}$ в $\mathbb{Q}$. Отсюда два множества равномощны, ч.т.д.\\\\
3)$|\mathbb{R}| = |\mathbb{N}|$\\
\textbf{Доказательство}\\
Выберем $X=[0;1)$, где все числа $x \in X$ содержат в своей записи только 0 и 1.\\
Пусть $|X| = |\mathbb{N}|$.\\
Сопоставим каждому числу из $\mathbb{N}$ число из $X$.\\
Мы получили биекцию:
\begin{center}
    \begin{tabular}{c|c}
        1 & $0,r_{11}\ r_{12}\ r_{13}\ r_{14}\ ...$ \\
        2 & $0,r_{21}\ r_{22}\ r_{23}\ r_{24}\ ...$\\
        3 & $0,r_{31}\ r_{32}\ r_{33}\ r_{34}\ ...$\\
        4 & $0,r_{41}\ r_{42}\ r_{43}\ r_{44}\ ...$\\
    \end{tabular}
\end{center}
Возьмем число $0,\overline{r_{11}}\ \overline{r_{22}}\ \overline{r_{33}}\ \overline{r_{44}}\ ...$, где $\overline{x} = 1-x$. Это число отличается как минимум одной цифрой от каждого числа в таблице. При этом в нем нет $(9)$, что гарантирует, что каждое число в $X$ может быть представлено единственным способом. Тогда мы получили число, не содержащееся в таблице. Отсюда $|X| > |\mathbb{N}| \Rightarrow |\mathbb{R}| > |\mathbb{N}|$, ч.т.д.\\\\
Свойства мощности:
\begin{enumerate}
    \item $|A| = \infty \Rightarrow |A^2| = |A|$
    \item $|2^A| > |A|$, где $2^A$ - \textit{булеан} - множество всех подмножеств A.
    \item {
            $\left.
          \begin{array}{l}
            \text{Существует инъекция из }A\text{ в }B\\
            \text{Существует инъекция из }B\text{ в }A \\
          \end{array}
        \right\} \Rightarrow |A| = |B|$
        }
\end{enumerate}
\section{Урок 12.09.2022\\Последовательности}
Методы нахождения пределов:
\begin{enumerate}
    \item Теорема о двух городовых
    \item Линейность: \\$x_n\rightarrow A, y_n\rightarrow B, \alpha, \beta \in \mathbb{R} \Rightarrow \alpha x_n+\beta y_n \rightarrow \alpha A+\beta B$\\
    $\$x_n\rightarrow A, y_n\rightarrow B \Rightarrow x_ny_n\rightarrow AB$\\
    $\$x_n\rightarrow A, y_n\rightarrow B \Rightarrow \frac{x_n}{y_n}\rightarrow \frac{A}{B}$, если $B \neq 0 \land \exists\,N:\ \forall\,n>N\ y_n \neq 0$
    \item Доказательство существование предела: \\
    Пусть $x_n$ монотонно возрастающая(убывающая) последовательность, ограниченная сверху(снизу). Тогда $x_n$  сходится.
\end{enumerate}
$\lim\limits_{n\rightarrow +\infty}{\frac{\ln{n}}{n}} = 0$
\section{Урок 19.09.2022\\Частичный предел}
\textbf{Определение}\\
$x_n \sim y_n \Leftrightarrow \lim\limits_{n\rightarrow\infty} \frac{x_n}{y_n} = 1$\\
$x_n, y_n$ - эквивалентные последовательности\\\\
$\lim\limits_{x\rightarrow 0} \frac{\sin x}{x} = 1$ - первый замечательный предел\\\\
Правило Лопиталя:\\
$f(x),g(x) \rightarrow \infty \Rightarrow \lim \frac{f(x)}{g(x)} = \lim \frac{f`(x)}{g`(x)}$\\\\
\textbf{Определение}\\
\textit{Подпоследовательность} - упорядоченное подмножество с индуцированным порядком(т.е. если в последовательности элемента стояли в определенном порядке, то те элементы, которые попали в подпоследовательность, стоят в ней в том же порядке)\\\\
\textbf{Определение}\\
Пусть $x_{n_k}$ - подпоследовательность $x_n$. Тогда $x_{n_k} \xrightarrow[n_k\rightarrow\infty]{} a$ - \textit{частичный предел $x_n$}\\\\
\textbf{Определение}\\
$\sup A$ - наименьшая верхняя граница $A$\\
$\inf A$ - наибольшая нижняя граница $A$\\
$\sup A,\inf A \in \overline{\mathbb{R}} = \mathbb{R}\cup\{\pm\infty\}$\\
$\sup A,\inf A$ всегда существуют.\\\\
$\sup \{\text{ частичные пределы } x_n\}$ - верхний предел\\
$\inf \{\text{ частичные пределы } x_n\}$ - нижние предел\\
\section{Урок 17.10.2022\\Предел функии}
$\forall\,\varepsilon > 0\ \exists\,\delta\ \forall\,x:\ 0<\rho(x,a)<\delta\quad \rho(f(x),A) < 0$\\
$f(x)$ - непрерывна в $a$, если $\lim\limits_{x\rightarrow a} f(x) = f(a)$\\
Для непрерывной $f(x)$ $\forall\,\varepsilon > 0\ \exists\,\delta\ \forall\,x:\ \rho(x,a)<\delta\quad \rho(f(x),A) < 0$ (в том числе при $x=a$)\\\\
Если $F$ непрерывна, то $\lim\limits_{x\rightarrow a} F(f(x)) = F(\lim\limits_{x\rightarrow a} f(x))$\\
$e^x = 1+\frac x{1!} + \frac{x^2}{2!}+\ldots$\\
\textbf{Определение}\\
Пусть $a: a$ - предельная в $D_f$ и не содержится там\\
Тогда $a$ - \textit{особая точка}\\
$a$ - \textit{устранимая}, если существует предел в точке $a$.
Тогда определим $f(x)=\left\{\begin{array}{ll}
     f(x), & x\neq a\\
     \lim\limits_{x\rightarrow a} f(x), & x = a 
\end{array}\right.$ - устранение особенностей\\
$f'(x) = \lim\limits_{\partial x\rightarrow 0} \frac{f(x+\partial x)-f(x)}{\partial x}$\\
$f(x+h)=f(x)+hf'(x)+g(h)$, где $g(h) = f(x+h)-f(x)-hf'(x)$\\
$\lim\limits_{h \rightarrow 0} \frac{g(h)}{h} = 0 \Leftrightarrow g(h) = o(h)$(на самом деле $g(h) \in o(h)$, но так не пишут)\\
Тогда $\exists\, f'(x)\rightarrow f(x+h)=f(x)+hf'(x)+o(h)$\\
Отсюда $e^{x+h} = e^x+he^x+o(h)$\\
$e^h = 1 + h + o(h)$\\\\
$\sin h = \sin (0+h) = \sin 0 + h\sin' x+o(x) = x+o(x)$
$\lim\limits_{x\rightarrow 0} \frac{\sin x}{x} = \lim\limits_{x\rightarrow 0} \frac{x+o(x)}{x} = 1$
\section{Урок 24.10.2022}
\begin{enumerate}
    \item $o(x)+o(x)=o(x)$
    \item $o(o(x)) = o(x)$
    \item $xo(x) = o(x^2)$
    \item $o(1) = \{f: f \rightarrow 0\}$
    \item $o(x^3) = o(x)$, но не наоборот
    \item $\frac{1+o(1)}{1+o(1)} = 1+o(1)$
    \item $o(x)+o(x^2) = o(x)$
\end{enumerate}
$f'(x) = \lim\limits_{x\rightarrow a} \frac{f(x)-f(a)}{x-a}$\\
Если функция дифференцируема:\\
$f(x) = f(a) + f'(a)(x-a) + o(x-a)$, где $o(x-a) = f(x)-f(a)-f'(a)(x-a)$
\section{Урок 14.11.2022}
\textbf{Определение}\\
Функция $f$ называется \textit{дифференцируемой} в точке $x$, если $f(x+\delta x) - f(x) = A\delta x +o(\delta x)$, где $A$ - константа\\
Если $f$ дифференцируема, то $A = f'(x)$\\
$A\delta x$ - \textit{дифференциал} в точке $x$\\
\textbf{Свойства дифференциала $\df x$}
\begin{enumerate}
    \item $\df (\alpha f + \beta g) = \alpha \df f + \beta \df g$
    \item $\df (fg) = f\df g + g \df f$ - Формула Лейбница
    \item $\df 1 = \df 1\cdot1 = 1\df 1 + 1\df 1$\\
    $\df 1 = 0$
    \item $\df (\df x) = 0$
\end{enumerate}
$\df (\df f) = \df^2 f = \df (f'(x)\df x) = (\df x)\df f'(x) + f'(x) \df (\df x) = f''(x)(\df x)^2$\\
\textbf{Определение}\\
$f$ - \textit{гладкая}, если она дифференцируема до бесконечности (на практике - дифференцируема столько, сколько нужно)
\section{Урок 5.12.2022}
Функция \textit{возрастает} в $x_*$, если в некоторой выколотой окрестности $x_*$ $\frac{f(x)-f(x_*)}{x-x_*} \geq0$\\
Тогда $f'(x_*) \geq 0$\\
Если $f'(x_*) > 0$, то функция возрастает\\\\

\end{document}

