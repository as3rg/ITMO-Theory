\documentclass[12pt]{article}
\usepackage{bbold}
\usepackage{amsfonts}
\usepackage{amsmath}
\usepackage{amssymb}
\usepackage{color}
\setlength{\columnseprule}{1pt}
\usepackage[utf8]{inputenc}
\usepackage[T2A]{fontenc}
\usepackage[english, russian]{babel}
\usepackage{graphicx}
\usepackage{hyperref}
\usepackage{mathdots}
\usepackage{xfrac}


\def\columnseprulecolor{\color{black}}

\graphicspath{ {./resources/} }


\usepackage{listings}
\usepackage{xcolor}
\definecolor{codegreen}{rgb}{0,0.6,0}
\definecolor{codegray}{rgb}{0.5,0.5,0.5}
\definecolor{codepurple}{rgb}{0.58,0,0.82}
\definecolor{backcolour}{rgb}{0.95,0.95,0.92}
\lstdefinestyle{mystyle}{
    backgroundcolor=\color{backcolour},   
    commentstyle=\color{codegreen},
    keywordstyle=\color{magenta},
    numberstyle=\tiny\color{codegray},
    stringstyle=\color{codepurple},
    basicstyle=\ttfamily\footnotesize,
    breakatwhitespace=false,         
    breaklines=true,                 
    captionpos=b,                    
    keepspaces=true,                 
    numbers=left,                    
    numbersep=5pt,                  
    showspaces=false,                
    showstringspaces=false,
    showtabs=false,                  
    tabsize=2
}

\lstset{extendedchars=\true}
\lstset{style=mystyle}

\newcommand\0{\mathbb{0}}
\newcommand{\eps}{\varepsilon}
\newcommand\overdot{\overset{\bullet}}
\DeclareMathOperator{\sign}{sign}
\DeclareMathOperator{\re}{Re}
\DeclareMathOperator{\im}{Im}
\DeclareMathOperator{\Arg}{Arg}
\DeclareMathOperator{\const}{const}
\DeclareMathOperator{\rg}{rg}
\DeclareMathOperator{\Span}{span}
\DeclareMathOperator{\alt}{alt}
\DeclareMathOperator{\Sim}{sim}
\DeclareMathOperator{\inv}{inv}
\DeclareMathOperator{\dist}{dist}
\newcommand\1{\mathbb{1}}
\newcommand\ul{\underline}
\renewcommand{\bf}{\textbf}
\renewcommand{\it}{\textit}
\newcommand\vect{\overrightarrow}
\newcommand{\nm}{\operatorname}
\DeclareMathOperator{\df}{d}
\DeclareMathOperator{\tr}{tr}
\newcommand{\bb}{\mathbb}
\newcommand{\lan}{\langle}
\newcommand{\ran}{\rangle}
\newcommand{\an}[2]{\lan #1, #2 \ran}
\newcommand{\fall}{\forall\,}
\newcommand{\ex}{\exists\,}
\newcommand{\lto}{\leftarrow}
\newcommand{\xlto}{\xleftarrow}
\newcommand{\rto}{\rightarrow}
\newcommand{\xrto}{\xrightarrow}
\newcommand{\uto}{\uparrow}
\newcommand{\dto}{\downarrow}
\newcommand{\lrto}{\leftrightarrow}
\newcommand{\llto}{\leftleftarrows}
\newcommand{\rrto}{\rightrightarrows}
\newcommand{\Lto}{\Leftarrow}
\newcommand{\Rto}{\Rightarrow}
\newcommand{\Uto}{\Uparrow}
\newcommand{\Dto}{\Downarrow}
\newcommand{\LRto}{\Leftrightarrow}
\newcommand{\Rset}{\bb{R}}
\newcommand{\Rex}{\overline{\bb{R}}}
\newcommand{\Cset}{\bb{C}}
\newcommand{\Nset}{\bb{N}}
\newcommand{\Qset}{\bb{Q}}
\newcommand{\Zset}{\bb{Z}}
\newcommand{\Bset}{\bb{B}}
\renewcommand{\ker}{\nm{Ker}}
\renewcommand{\span}{\nm{span}}
\newcommand{\Def}{\nm{def}}
\newcommand{\mc}{\mathcal}
\newcommand{\mcA}{\mc{A}}
\newcommand{\mcB}{\mc{B}}
\newcommand{\mcC}{\mc{C}}
\newcommand{\mcD}{\mc{D}}
\newcommand{\mcJ}{\mc{J}}
\newcommand{\mcT}{\mc{T}}
\newcommand{\us}{\underset}
\newcommand{\os}{\overset}
\newcommand{\ol}{\overline}
\newcommand{\ot}{\widetilde}
\newcommand{\vl}{\Biggr|}
\newcommand{\ub}[2]{\underbrace{#2}_{#1}}

\def\letus{%
    \mathord{\setbox0=\hbox{$\exists$}%
             \hbox{\kern 0.125\wd0%
                   \vbox to \ht0{%
                      \hrule width 0.75\wd0%
                      \vfill%
                      \hrule width 0.75\wd0}%
                   \vrule height \ht0%
                   \kern 0.125\wd0}%
           }%
}
\DeclareMathOperator*\dlim{\underline{lim}}
\DeclareMathOperator*\ulim{\overline{lim}}

\everymath{\displaystyle}

% Grath
\usepackage{tikz}
\usetikzlibrary{positioning}
\usetikzlibrary{decorations.pathmorphing}
\tikzset{snake/.style={decorate, decoration=snake}}
\tikzset{node/.style={circle, draw=black!60, fill=white!5, very thick, minimum size=7mm}}

\title{Математический анализ. Теория}
\author{Александр Сергеев}
\date{}

\begin{document}
\maketitle
\section{Введение}
\subsection{Множества}
\textit{Множество} - совокупность уникальных элементов. \\\textit{(Не является определением)}\\\\
Способы задания множества:
\begin{enumerate}
    \item $A = \{1,2, ...\}$ - \textit{перечисление}
    \item $A = \{ x \in B: \phi (x) \}$ - \textit{через другое множество}
\end{enumerate}
Отношения множеств:
\begin{enumerate}
    \item $A \subset B \Leftrightarrow \forall\,x \in A \quad x \in B$
    \item $A = B \Leftrightarrow A \subset B \land B \subset A$
\end{enumerate}
Операции над множествами:
\begin{enumerate}
    \item $X \times Y = \{ (x,y): x \in X, y \in Y\}$ - Декартово произведение 
    \item $\bigcup_{\alpha \in A} X_\alpha = \{ x: \exists\ \alpha\quad x \in X_\alpha\}$ - Объединение
    \item $\bigcap_{\alpha \in A} X_\alpha = \{ x: \forall\ \alpha\quad x \in X_\alpha\}$ - Пересечение
    \item $A^c = \{ x \in U: x \notin A \}$ - Дополнение
    \item $A \setminus B = A \cap B^c$ - Разность
    \item $A \triangle B = (A \cup B) \setminus (A \cap B)$ - Исключающее объединение(симметричная разность)
\end{enumerate}
Свойства объединения и пересечения:
\begin{enumerate}
    \item Коммутативность: $X \cap Y = Y \cap X;\ X \cup Y = Y \cup X$
    \item Нейтральный элемент: $X \cap U = X;\ X \cup \varnothing = X$
    \item Ассоциативность: $X\cap(Y \cap Z) = (X \cap Y)\cap Z;\ X \cup (Y \cup Z) = (X \cup Y) \cup Z$
    \item Дистрибутивность(законы де Моргана):\\ $X\cap(Y\cup Z) = (X \cap Y) \cup (X \cap Z);\ X\cup(Y\cap Z) = (X \cup Y) \cap (X \cup Z)$
\end{enumerate}
Законы де Моргана для разности:
\begin{enumerate}
    \item $X\setminus(Y \cup Z) = (X \setminus Y) \cap (X \setminus Z)$
    \item $X\setminus(Y \cap Z) = (X \setminus Y) \cup (X \setminus Z)$
\end{enumerate}
\subsection{Логические операции}
Правила отрицания:
\begin{enumerate}
    \item $\overline{\exists\,x: \phi(x)} \Leftrightarrow \forall\,x: \overline{\phi(x)}$
    \item $\overline{\forall\,x: \phi(x)} \Leftrightarrow \exists\,x: \overline{\phi(x)}$
\end{enumerate}
Операции над логическими выражениями:
\begin{enumerate}
    \item Имприкация\\$ P \Rightarrow Q \quad \Leftrightarrow \quad Q \lor \overline{P}$
    \item Эквивалентность\\$P \Leftrightarrow Q \quad \Leftrightarrow \quad Q \land P \lor \overline{Q} \land \overline{P}$
\end{enumerate}
\subsection{Семейства}
\textit{Семейство} - совокупность неупорядоченных элементов. \\\textit{(Не является определением)}\\\\
\textbf{Определение}\\
\textit{Семейство элементов X} - отображением множества индексов A в множество X.
Обозначения: 
\begin{enumerate}
    \item $(x_{\alpha})\ _{\alpha \in A},\ x_{\alpha} \in X$
    \item $A \rightarrow X$
    \item $\alpha \mapsto x_{\alpha}$
\end{enumerate}
Частные случаи семейств:
\begin{enumerate}
    \item Упорядоченный набор из n чисел\\$ \{1...n\} \mapsto \mathbb{R}$
    \item Упорядоченная пара\\$\{1,2\} \mapsto \mathbb{R}$
    \item Последовательность
\end{enumerate}
\subsection{Счетные и несчетные множества}
\textbf{Определение}\\
Назовем два множества эквивалентными, если существует биекция между ними\\
Классы эквивалентности по этому отношению называются \textit{мощностью множества}\\
Если множество конечно, то его мощность - число его элементов\\
\textbf{Определение}\\
Множество \textit{счетно}, если существует биекция между этим множеством и $\bb{N}$\\
\textbf{Теорема}\\
Если множество бесконечно, то оно содержит счетное подмножество\\
\textbf{Доказательство}\\
Будем по одному выкидывать элементы из множества, нумеруя их\\
Т.к. множество бесконечно, то для каждого номера такой элемент найдется\\
\textbf{Теорема}\\
Бесконечное подмножество в счетном множестве тоже счетно\\
\textbf{Доказательство}\\
Пусть $A$ - счетное множество\\
$B$ - бесконечное подмножество $A$\\
Пусть у каждого элемента $A$ был номер\\
Перенумеруем элементы $B$ в порядке возрастания номеров\\
\textbf{Определение}\\
Множество \textit{не более чем счетное} - множество, являющееся конечным или счетным\\
\textbf{Теорема}\\
Не более чем счетное объединение не более чем счетных множеств не более чем счетно\\
\textbf{Следствие}\\
$\bb{N} \times \bb{N}$ счетно\\
\textbf{Теорема}\\
$\bb{Q}$ счетно\\
\textbf{Теорема}\\
$[0, 1]$ несчетно\\
\textbf{Определение}\\
Если множество равномощно $[0, 1]$, то его мощность - \textit{континуум}\\
\textbf{Теорема}\\
Пусть $A$ - имеет мощность континуума, $B$ не более чем счетно\\
Тогда $A \cup B$ имеет мощность континуума\\
\textbf{Теорема}\\
Множество всех бесконечных бинарных последовательностей имеет мощность континуума\\
\textbf{Доказательство}\\
Сопоставим каждой последовательности $(\epsilon_1, \epsilon_2, \ldots)$ двоичную дробь $0,\epsilon_1\epsilon_2\ldots$\\
Заметим, что такое сопоставление не будет биективным из-за двойственности представления двоичных дробей\\
Пусть $A$ - множество конечных двоичных дробей (целая часть 0)\\
Множество $A$ счетно (можно сопоставить каждой дроби двоичное число из $\bb{N}\cup\{0\}$, полученное отражением числа относительно запятой)\\
Теперь мы можем построить биекцию между последовательностями и $[0, 1] \cup A$, считая, что элементы $A$ - это "другие" дроби, не содержащиеся в $[0,1]$\\
Тогда из предыдущей теоремы множество последовательностей равномощно $[0,1]$\\
\textbf{Континуум-гипотеза}\\
Пусть $A\subset [0,1]$ и не континуально\\
Утверждение "Тогда $A$ счетно" невозможно ни доказать, ни опровергунть\\
\textbf{Утверждение}\\
$\bb{R}^m, \bb{R}^\infty$ - континуум\\
$\{f: f:[a,b] \rightarrow \bb{R}\}$ - больше, чем континуум\\
Если $X$ - множество, то $2^X$ - множество всех подмножество - имеет б\textbf{о}льшую мощность\\
\section{Последовательности в метрическом пространстве}
\subsection{Предел вещественной последовательности}
\textbf{Определение}\\
Пусть $(x_{n})$ - вещественная последовательность\\
$(x_{n}) \rightarrow \alpha \quad \Leftrightarrow \quad \forall\,\varepsilon > 0\ \exists\,N \in \mathbb{R}\ \forall\,n > N\ |x_{n} - \alpha| < \varepsilon$\\\\
Замечания:
\begin{enumerate}
    \item $N = N(\varepsilon)$
    \item Необязательно брать самый оптимальный N
    \item $N(\varepsilon_0)$ подходит для $\varepsilon \geq \varepsilon_0$
    \item "$< \varepsilon$" можно заменить на "$< y\varepsilon$" или "$< \varepsilon^{y}$" ($y \in (0; +\infty$)
\end{enumerate}
\textbf{Определение}\\
$\varepsilon$-окружность $\alpha \quad U_\varepsilon(\alpha) = [\alpha-\varepsilon; \alpha+\varepsilon]$\\\\
\textbf{Определение}\\
$(x_{n}) \rightarrow \alpha \quad \Leftrightarrow \quad \forall\,\varepsilon > 0\ \exists\,N \in \mathbb{R}\ \forall\,n > N\ x_{n} \in U_\varepsilon(\alpha)$\\\\
\textbf{Определение}\\
\textit{Метрика} на $X$ - это отображение $\rho: X\times X \rightarrow \mathbb{R}$, удовлетворяющее свойствам(аксиомам метрики):
\begin{enumerate}
    \item $\forall\,x,y\in X\ \rho(x,y) \geq 0$, причем $\rho(x,y) = 0 \Leftrightarrow x=y$
    \item $\rho(x,y) = \rho(y,x)$
    \item $\rho(x,z) \leq \rho(x,y) + \rho(y,z)$ - неравенство треугольника
\end{enumerate}
\textbf{Определение}\\
Пара $(X, \rho)$ - \textit{метрическое пространство}\\\\
Примеры:
\begin{enumerate}
    \item \textit{Симплициальная метрика} - $\rho(x,y) = \begin{bmatrix}
         & 1,\ x\neq y \\
         & 0,\ x = y
    \end{bmatrix}.$
    \item \textit{Метрика Хемминга}\\
    $X = \text{множество байтов} = \{(\varepsilon_1, ..., \varepsilon_8):\forall\,i\ \varepsilon_i\in \{0,1\}\}$\\
    $\rho(x,y) = \text{число несовпадающих разрядов}$
    \item \textit{Метрика городских кварталов}\\
    $(\mathbb{R}^m, \rho): \rho(x,y) = |x_1-y_1|+|x_2-y_2|+...+|x_m-y_m|$
    \item \textit{Евклидова метрика}\\
    $(\mathbb{R}^m, \rho): \rho(x,y) = \sqrt{|x_1-y_1|^2+|x_2-y_2|^2+...+|x_m-y_m|^2}$
    \item $(\mathbb{R}^m, \rho): \rho(x,y) = \max{ |x_1-y_1|,|x_2-y_2|,...,|x_m-y_m|}$
\end{enumerate}
\textbf{Определение}\\
$(X,\rho)$ - метрическое пространство\\
$A \subset X$\\
$\rho_A: A \times A \rightarrow \mathbb{R} : \forall\,a,b\in A\  \rho_A(a,b)=\rho(a,b)$\\
$(A,\rho_A)$ - \textit{Подпространство метрического пространства}\\\\
\textbf{Определение}\\
$(X,\rho)$ - метрическое пространство\\
$a \in X, r > 0$\\
\textit{Открытый шар} $B(a,r) = \{x \in X: \rho(a,x) < r\}$\\
\textit{Закрытый шар} $\overline{B}(a,r) = \{x \in X: \rho(a,x) \leq r\}$\\
\textit{Сфера} $S(a,r) = \{x \in X: \rho(a,x) = r\}$\\\\
\textbf{Определение}\\
\textit{$\varepsilon$-окрестность} точки $a =B(a,\varepsilon)$\\
\textit{Проколотая $\varepsilon$-окрестность} точки $a = \overdot{B}(a,\varepsilon) = B(a,\varepsilon)\setminus \{a\}$\\\\
\textbf{Определение}\\
$A \subset X$ - \textit{Ограниченное} $\\\Leftrightarrow A$ содержится в каком-нибудь шаре(в том числе в шаре с фиксированным центром) $\\\Leftrightarrow \exists\,a\in X, r>0: A \subset B(a,r) \\\Leftrightarrow \exists\,r > 0: A\subset B(b,r)$ для фиксированного $b$\\\\
\textbf{Определение}\\
$(x_n)$ - последовательность в $(X,\rho)$\\
$x_n\rightarrow L \\\Leftrightarrow \lim_{n\rightarrow +\infty}{x_N} = L
\\\Leftrightarrow \forall\,\varepsilon>0\,\exists\,N>0\,\forall\,n>N\ \rho(x_n,L)<\varepsilon
\\\Leftrightarrow \forall\,\varepsilon>0\,\exists\,N>0\,\forall\,n>N\ x_n \in U_\varepsilon(L)
\\\Leftrightarrow (x_n\rightarrow L \Leftrightarrow \rho(x_n,L)\rightarrow 0)$\\\\
\textbf{Теорема}\\
Пусть $(x_n)$ - последовательность в $(X,\rho)$\\
$x_n\rightarrow L$, $x_n\rightarrow M$\\
Тогда $L=M$.\\
\textbf{Доказательство}\\
Для любой окружности верно, что вне нее содержится конечное количество $x_n$.\\
Пусть $L\neq M$.\\
Возьмем $U_\varepsilon(L)$ и $U_\varepsilon(M)$ с $\varepsilon = \frac{\rho(L,M)}{2}$. Тогда из свойства для $U_\varepsilon(L)$ следует, что в $U_\varepsilon(M)$ конечное количество членов, что неверно. Отсюда $L=M$, ч.т.д.\\
Уточнение: $U_\varepsilon(M) \cap U_\varepsilon(L) = \varnothing$, т.к. окрестности - открытые окружности(доказательство очевидно).\\\\
\textbf{Теорема\\(об ограниченности сходящейся последовательности)}\\
Пусть $(x_n)$ - последовательность в $(X,\rho)$\\
$x_n\rightarrow L$\\
Тогда множество значений $x_n$ ограничено.\\
($\exists\,B(a,r): \forall\,n\ x_n\in B(a,r))$\\
\textbf{Доказательство}\\
$\forall\,\varepsilon>0\,\exists\,N_\epsilon>0\,\forall\,n>N_\epsilon\ x_n \in U_\varepsilon(L)$\\
Возьмем $\varepsilon$. $R = \epsilon+\max_{1 \leq i \leq N_\epsilon}{\rho(x_i,L)}$. Тогда $\forall\,n\ x_n\in B(L,R)$\\\\
\subsection{Порядковые свойства пределов последовательностей в $\mathbb{R}$}
\textbf{Теорема\\(о предельном переходе в неравенствах)}\\
$(x_n),(y_n)$ - вещественные последовательности\\
$x_n\leq y_n$ для бесконечного количества $n$.\\
Пусть: $x_n\rightarrow a, y_n \rightarrow b$, где $a,b \in \mathbb{R}$\\
Тогда $a\leq b$\\
\textbf{Доказательство}\\
Пусть $a > b$. Возьмем $\varepsilon = \frac{\rho(a,b)}{2}$. Для некоторого $N \forall\,n>N\ x_n \in U_\varepsilon(a), y_n \in U_\varepsilon(b)$. Отсюда $\forall\,n>N\ y_n \leq b+\varepsilon < a-\varepsilon \leq x_n \Leftrightarrow \forall\,n > N\ y_n < x_n$, что неверно. Тогда $a \leq b$, ч.т.д.\\\\
\textbf{Теорема о двух городовых}\\
$(x_n),(y_n),(z_n)$ - вещественные последовательности.\\
$\forall\,x_n\leq y_n \leq z_n$\\
Пусть $x_n \rightarrow a, z_n \rightarrow a$\\
Тогда $y_n \rightarrow a$\\
\textbf{Доказательство}\\
$\forall\,\varepsilon >0\ \exists\,N\ \forall\,n>N\ a-\varepsilon<x_n$\\
Для того же $\varepsilon\ \exists\,K\ \forall\,n>K\ a+\varepsilon>z_n$\\
При $n>\max(N,K)\ a-\varepsilon<x_n\leq y_n \leq z_n < a+\varepsilon$.\\
Отсюда $\forall\,\varepsilon>0\ \exists\,S>0\ \forall\,n>S a-\varepsilon<y_n<a+\varepsilon \Leftrightarrow y_n\rightarrow a$\\
\textbf{Следствие}\\
$(y_n),(z_n)$ - вещественные последовательности.\\
$\forall\,n\ |y_n|\leq z_n$\\
$z_n \rightarrow 0$\\
Тогда $y_n$ сходится и $\lim_{n\rightarrow +\infty}{y_n}=0$\\\\
\textbf{Определение}\\
$(x_n)$ - бесконечно малая последовательность $\Leftrightarrow x_n \rightarrow 0$\\\\
\textbf{Теорема}\\
$(x_n)$ - бесконечно малая\\
$(y_n)$ - ограниченная последовательность\\
Тогда $(x_n\cdot y_n)$ - бесконечно малая\\
\textbf{Доказательство}\\
$x_n \rightarrow 0 \Leftrightarrow |x_n| \rightarrow 0$\\
$y_n$ - ограниченная $\Leftrightarrow \exists\,R\ \forall\,n\ |y_n|\leq R$\\
Отсюда $|x_n\cdot y_n| \leq R\cdot|x_n| \rightarrow 0$\\
Тогда по следствию из теоремы о двух городовых $x_n\cdot y_n \rightarrow 0$, ч.т.д.

\subsection{Отображение}
\textit{Отображение} - тройка объектов (f,X,Y), где X - область определения, Y - область значений.\\\\
Обозначения:
\begin{enumerate}
    \item $\forall\,x \in X\quad f(x) \in Y$
    \item $f: X \rightarrow Y$
    \item $x \mapsto y$
\end{enumerate}
Функция - отображение $X \rightarrow \mathbb{R}$\\
Векторнозначная функция - отображение $X \rightarrow \mathbb{R}^m$\\
Вещественная последовательность - отображение $\mathbb{N} \rightarrow \mathbb{R}$\\
Семейство - отображение\\\\
\textbf{Определение}\\
Пусть $f:X \rightarrow Y$. Образ $A \subset X\quad f(A) = \{ y \in Y: \exists\,x \in A\quad y=f(x)\}$\\\\
\textbf{Определение}\\
Пусть $f:X \rightarrow Y$. Прообраз $B \subset Y\quad f^{-1}(B) = \{ x \in X: f(x) \in B\}$\\
\textit{(Не является обратным отображением)}\\\\
Инъекция(взаимно однозначное отображение): $x \neq y\ \Rightarrow f(x) \neq f(y)$\\
Сюръекция(отображение "на"): $\forall\,y\ \exists\,x: f(x)=y$\\
Биекция(взаимно однозначное соответствие) = Инъекция $\land$ Сюръекция\\\\
\textbf{Определение}\\
\textit{График отображения} $f:X\rightarrow Y\\ \Gamma_f=\{(x,y) \in X \times Y:f(x) = y\}$\\\\
\textbf{Определение}\\
Пусть $f:X\rightarrow Y$ - инъективное.\\
\textit{Обратное отображение} $f^{-1}:f(X) \subset Y\rightarrow X$\\
$\forall\,y \in f(X)\ \exists\,x\in X: f(x)=y$.\\
В силу инъективности $f^{-1}(y)=x$\\\\\\
\textbf{Определение}\\
$f:X\rightarrow Y\\
A \subset X$\\
\textit{Сужение} $f$ на $A$ - это отображение $f|_A: A \rightarrow Y: \forall\,x\in A\ f|_A(x)=f(x)$\\\\
\textbf{Определение}\\
$f:X\rightarrow Y\\
X \subset B$\\
\textit{Продолжение} $f$ на $B$ - это отображение $F: B \rightarrow Y: \forall\,x\in X\ F(x)=f(x)$\\\\
\textbf{Определение}\\
\textit{Тождественное отображение} $id:X\rightarrow X$ - функция $id(x)=x$\\\\
\textbf{Определение}\\
$f:X\rightarrow Y\\
g:Y\rightarrow Z$\\
\textit{Композиция отображений} - отображение $g\circ f: X \rightarrow Z: (g\circ f)(x)=g(f(x))$\\

\subsection{Вещественные числа}
\textbf{Определение}\\
$\mathbb{R}$ - любое множество, которое удовлетворяет аксиомам 1-4
\begin{enumerate}
    \item Аксиомы поля
    \begin{enumerate}
        \item $"+": \mathbb{R}\times\mathbb{R} \rightarrow \mathbb{R}$:
        \begin{enumerate}
            \item Коммутативность $a+b=b+a$
            \item Ассоциативность $(a+b)+c=a+(b+c)$
            \item Нейтральный элемент $\exists\,\0: a+\0=a$
            \item Обратный элемент $\exists\,b: a+b=\0$
        \end{enumerate}
        \item $"\cdot": \mathbb{R}\times\mathbb{R} \rightarrow \mathbb{R}$:
        \begin{enumerate}
            \item Коммутативность $a\cdot b=b\cdot a$
            \item Ассоциативность $(a\cdot b)\cdot c=a\cdot(b\cdot c)$
            \item Нейтральный элемент $\exists\,\1\neq \0: a\cdot\1=a$
            \item Обратный элемент $\forall\,a \neq \0\ \exists\,b: a\cdot b=\1$
        \end{enumerate}
        \item[(v)] Дистрибутивность $a\cdot(b+c)=a\cdot b + a\cdot c$
    \end{enumerate}
    \item Аксиомы порядка
    \begin{enumerate}
        \item $"\leq"\,=\,\mathbb{R}\times\mathbb{R}$:
        \begin{enumerate}
            \item $\forall\,x,y\ x\leq y \lor y\leq x$ - полнота
            \item $x\leq y, y\leq z \Rightarrow x \leq z$ - транзитивность
            \item $x \leq y, y\leq x \Leftrightarrow x = y$ - антисимметричность
            \item $x\leq y \Rightarrow \forall\,z\ x+z \leq y+z$
            \item $\0\leq x, y \Rightarrow \0\leq xy$
        \end{enumerate}
    \end{enumerate}
    \item Аксиома Архимеда\\
    $\forall\,x,y>0\ \exists\,n\in \mathbb{N}\ nx>y$\\
    \textit{Пояснение}: не существует бесконечно больших чисел
    \item Аксиома Кантора\\
    $[a_1,b_1] \supset [a_2,b_2] \supset \ldots$ - бесконечное семейство вложенных отрезков в $\mathbb{R}$\\
    Тогда $\bigcap_{k\in\mathbb{N}}[a_k,b_k] \neq \varnothing$\\
    \textit{Замечание}: отрезки не могут быть заменены на (полу)интервалы\\

\end{enumerate}
Множество, удовлетворяющее аксиомам
\begin{itemize}
    \item поля - \textit{поле}
    \item поля и порядка - \textit{упорядоченное поле}
\end{itemize}
$[a,b] = {x: a\leq x\leq b}$ - отрезок\\
$[a,b) = {x: a\leq x< b}$ - полуинтервал\\
$(a,b] = {x: a < x \leq b}$ - полуинтервал\\
$(a,b) = {x: a < x < b}$ - интервал\\
$\langle a,b\rangle$ - любой из 4 промежутков\\\\
\textbf{Определение}\\
$\overline{\mathbb{R}} = \mathbb{R} \cup \{-\infty,+\infty\}\\
\forall\,a\in \mathbb{R}\ -\infty<a<+\infty$\\\\
\textbf{Теорема}\\
$\frac12\sum_{(i,k) \in A\times B} (a_ib_k-a_kb_i)^2 = (\sum_{i=1}^n a_i^2)(\sum_{i=1}^n b_i^2)-(\sum_{i=1}^n a_ib_i)^2 \geq 0$ - \textit{Тождество Лагранжа}\\
$(\sum_{i=1}^n a_i^2)(\sum_{i=1}^n b_i^2)\geq(\sum_{i=1}^n a_ib_i)^2$ - \textit{Неравенство Коши - Буняковского(КБШ)}

\subsection{Нормированное пространство}
\textbf{Определение}\\
$K$ - поле(поле скаляров)
$X$ - линейное(векторное) пространство над полем $K$, если заданы \\$"+": X\times X \rightarrow X$ и $"\cdot": K\times X \rightarrow X$, удволетворяющее аксиомам
\begin{enumerate}
    \item $a+b=b+a$
    \item $(a+b)+c=a+(b+c)$
    \item $\exists\,\0\in X:\ \forall\,a\ a+\0=a$
    \item $\forall\,a\ \exists\,-a\in X:\ a + (-a) = \0$
    \item $\forall\,\lambda,\mu \in K, a\in X\ (\lambda+\mu)\cdot a = \lambda\cdot a + \mu\cdot a$
    \item $\forall\,\lambda \in K, a,b\in X\ \lambda\cdot(a+b) = \lambda\cdot a + \lambda\cdot b$
    \item $\forall\,\lambda,\mu \in K, a\in X\ (\lambda\mu)\cdot a = \lambda(\mu\cdot a)$
    \item $1\cdot a = a$
\end{enumerate}
\textbf{Определение}\\
\textit{Нормированное пространство} - это линейное пространство, в котором задана норма.\\
\textit{Норма в линейном пространстве $X$ над полем $K$} - это отображение $\|\cdot\|: X \rightarrow \mathbb{R}$, удовлетворяющее аксиомам нормы:
\begin{enumerate}
    \item Положительная неопределенность:\\
    $\|x\| \geq 0$, причем $\|x\|=0 \Leftrightarrow x = \0$
    \item Положительная однородность\\
    $\|\lambda x\| = |\lambda|\cdot\|x\|$
    \item Неравенство треугольника\\
    $\|x+y\| \leq \|x\| +\|y\|$
\end{enumerate}
Свойства нормы:
\begin{enumerate}
    \item $p(\sum\lambda_kx_k) \leq \sum\lambda_k p(x_k)\\
    (x_k\in X, \lambda_k \in K)$
    \item $p(\0) = 0$
    \item $p(-x) = p(x)$
    \item $|p(x)-p(y)| \leq p(x-y)$
\end{enumerate}
\textbf{Определение}\\
\textit{Полунорма} - неотрицательная функция, удовлетворяющая 2 и 3 аксиомам нормы.\\\\
\textit{Замечание}: в нормированном пространстве $\|x-y\|$ является метрикой, но не всякая метрика может быть порождена нормой.
\subsection{Арифметические свойства пределов}
\textbf{Теорема(арифметические свойства предела в нормированном пространстве)}\\
$(X,\|\cdot\|)$ - нормированное пространство\\
$(x_n),(y_n)$ - последовательности в $X$\\
$(\lambda_n)$ - постедовательность скаляров.\\
Пусть $x_n\rightarrow a, y_n\rightarrow b, \lambda_n \rightarrow \mu$\\
Тогда:
\begin{enumerate}
    \item $x_n\pm y_n \rightarrow a\pm b$\\
    \textbf{Доказательство}\\
    $0 \leq \|x_n\pm y_n - (a\pm b)\| \leq \|x_n-a\|+\|y_n-b\| \rightarrow 0 \Rightarrow \|x_n\pm y_n - (a\pm b)\| \rightarrow 0$
    \item $x_ny_n \rightarrow ab$
    \item $\lambda_n x_n \rightarrow \mu a$\\
    \textbf{Доказательство}\\
    $\|\lambda_n x_n - \mu a\| = \|(\lambda_nx_n-\mu x_n) + (\mu x_n - \mu a)\| \leq \|(\lambda_n-\mu)x_n\| + \|\mu(x_n-a)\| = |\lambda_n-\mu|\cdot\|x_n\|+|\mu|\cdot\|x_n-a\| = |\text{б.м.}|\cdot\|\text{огр.}\|+|\text{огр.}|\cdot\|\text{б.м.}\| = \text{б.м.}$, ч.т.д.
    \item $\|x_n\| \rightarrow \|a\|$
    \item $y_n,b\neq 0$ начиная с некоторого места $\frac{x_n}{y_n} \rightarrow \frac{a}{b}$\\
    \textbf{Доказательство}\\
    Достаточно доказать, что $\frac 1{y_n} \rightarrow \frac 1b$\\
    Докажем ограниченность $\frac 1{y_n}$:\\
    Из предела для $\varepsilon=\frac{|b|}2\, \exists\,N\ \forall\,n>N\ |y_n|>\frac{|b|}2$. Тогда начиная с некоторого $N$ $\frac1{|y_n|} < \frac{2}{b}$\\
    $|\frac 1{y_n} - \frac 1b | = \frac 1{y_n}\cdot\frac1b\cdot|y_n-b| = \text{ограниченная c некоторого места}\cdot\text{ограниченная}\cdot\text{бесконечно малая} = 0$
\end{enumerate}
\subsection{Сходимость к $\infty$}
В $\mathbb{R}$:\\
\textbf{Определение}\\
$(x_n)$ - вещественная последовательность, $x_n \rightarrow +\infty$, если\\
$\forall\,E > 0\ \exists\, N\ \forall\,n>N\ x_n>E$\\
или на языке окресностей\\
$\forall\,U(+\infty)\ \exists\, N\ \forall\,n>N\ x_n\in U(+\infty)$, где $U(+\infty) = (a,+\infty]$ - окрестность $+\infty$\\\\
Аналогично $x_n\rightarrow -\infty$\\\\
$x_n\rightarrow \infty \Leftrightarrow |x_n| \rightarrow +\infty$\\
$x_n$ - бесконечно большая последовательность. При этом $\frac 1{x_n}$ - бесконечно малая последовательность\\\\
Если $x_n\rightarrow+\infty$, то $x_n \nrightarrow -\infty, x_n \nrightarrow a\in \mathbb{R}$ - единственность предела.\\
Другими словами если $x_n \rightarrow a \in \overline{\mathbb{R}}$, то он единственный.\\\\
Все утверждения о пределах актуальны для $\mathbb{R}^m$\\\\
\textbf{Теоремы}\\
Если $x_n\rightarrow +\infty$, то $x_n$ не ограничена сверху, но ограничена снизу и имеет минимум\\
Если $x_n\rightarrow -\infty$, то $x_n$ не ограничена снизу, но ограничена сверху и имеет максимум\\\\
Если $x_n \leq y_n, x_n \rightarrow a, y_n \rightarrow b, a,b\in \overline{\mathbb{R}}$, тогда $a\leq b$\\
\textbf{Доказательство теоремы 1}\\
Пусть для некоторого $E$ $\exists\,k\ \forall\,n>k\ x_n>E$. Тогда минимум $x_n$ - это $\min_{1\leq n \leq k} x_n$\\
\textbf{Доказательство теоремы 3}
\begin{enumerate}
    \item $a,b \in \mathbb{R}$ - доказано
    \item $b=+\infty$(включая $a=\pm\infty)$. Тогда $a\leq b$
    \item $a=+\infty$. Тогда возможно только $b = +\infty$
\end{enumerate}
\textbf{Теорема об арифметических свойствах предела в $\overline{\mathbb{R}}$}\\
$(x_n), (y_n)$ - последовательности в $\mathbb{R}$\\
$x_n\rightarrow a, y_n \rightarrow b$, где $a,b \in \overline{\mathbb{R}}$\\
Тогда
\begin{enumerate}
    \item $x_n+y_n \rightarrow a+b$
    \item $x_ny_n \rightarrow ab$
    \item Если $y_n \neq 0$ с некоторого места, $b\neq 0$, то $\frac{x_n}{y_n} \rightarrow \frac ab$
\end{enumerate}
при условии, что правые части утверждений имеют смысл(т.е. нет операций вида $(-\infty)+(+\infty)$, $0\cdot(\pm\infty), \frac{\pm\infty}{\pm\infty}, \frac 00)$\\
\textit{Замечание}\\
$\frac{\pm X}0$ может быть интерпретирован как $\pm\infty$
\subsection{Точные границы числовых множеств}
\textbf{Определение}\\
Пусть непустое $E \subset \mathbb{R}$ и ограничено сверху.\\
\textit{Супремум} $E$ - наименьшая верхняя граница $E$\\
или $\sup E = \min \{M: \forall\,x\in E\ x\leq M\}$\\
или $\sup E = S \Leftrightarrow \left\{\begin{array}{l}
     \forall\,x\in E\ x\leq S\\
     \forall\,\varepsilon > 0\ \exists\,x\in E\ S-\varepsilon < x 
\end{array}\right.$\\\\
\textit{Инфимум} $E$ - наибольшая нижняя граница $E$\\
или $\inf E = \max \{m: \forall\,x\in E\ m\leq x\}$\\
или $\inf E = I \Leftrightarrow \left\{\begin{array}{l}
     \forall\,x\in E\ I \leq x\\
     \forall\,\varepsilon > 0\ \exists\,x\in E\ x < I+\varepsilon
\end{array}\right.$
\subsection{Точки и множества в метрическом пространстве}
Далее считаем, что $X$ - метрическое пространство, $D \subset X, a \in X$\\\\
\textbf{Определение}
\begin{enumerate}
    \item $a$ - \textit{внутренняя точка} множества $D \Leftrightarrow \exists\,U(a) \subset D$
    \item $D$ - \textit{открытое множество}, если все его точки внутренние
    \item $\operatorname{Int}(D)$ - множество внуренних точек $D$\\
    $\nm{Int}(D) = \underset{F\text{ - открытое}}{\bigcup_{F \subset D}} F$
\end{enumerate}
\textbf{Замечания}
\begin{enumerate}
    \item $\varnothing$ и $X$ - открытые множества
    \item Открытый шар - открытое множество\\
    \textbf{Доказательство}\\
    Рассмотрим шар радиуса $r$ с центом в точке $a$, а также точку $x$\\
    $R = r-\rho(a,x)$\\
    $U(x) = B(x,R)$\\
    Докажем, что $U(x) \subset B(a,r)$:\\
    Рассмотрим $y \in U(x)$:\\
    $\rho(y,a) \leq \rho(y,x)+\rho(x,a) < R + (r-R) = r$
\end{enumerate}
\textbf{Теорема о свойствах открытых множеств}
\begin{enumerate}
    \item Объединение любого семейства открытых множеств открыто\\
    $(G_\alpha)_{\alpha \in A}$ - семейство открытых множеств\\
    $\bigcup_{\alpha \in A} G_\alpha$ - открытое множество\\
    \textbf{Доказательство}\\
    Пусть $x \in \bigcup_{\alpha \in A} G_\alpha$\\
    Тогда $\exists\,\alpha_0:\ x\in G_{\alpha_0}$\\
    Тогда $\exists\,U(x) \subset G_{\alpha_0} \subset \bigcup_{\alpha \in A} G_\alpha$, ч.т.д.
    \item Пересечение конечного семейства открытых множеств открыто\\
    $(G_\alpha)_{\alpha \in A}$ - семейство открытых множеств\\
    $\bigcap_{\alpha \in A} G_\alpha$ - открытое множество\\
    \textbf{Доказательство}\\
    Пусть $x \in \bigcap_{\alpha \in A} G_\alpha$\\
    Тогда $\forall\,\alpha_0\ x\in G_{\alpha_0}$\\
    Тогда $\exists\,U(x) \subset G_{\alpha_0} \subset \bigcap_{\alpha \in A} G_\alpha$, ч.т.д.\\
    \textbf{Контр-пример для бесконечного семейства}\\
    $\bigcap_{k=1}^\infty (-\frac 1k, \frac 1k) = \{0\}$ - не открытое множество в $\mathbb{R}$
\end{enumerate} 
%\textbf{Общее замечание}\\
%$X$ - множество\\
%$G$ - совокупность подмножеств $X$
%\begin{enumerate}
%    \item $\varnothing \in G, X \in G$
%    \item $G_1, G_2, \ldots \in G \Rightarrow G_1\cup G_2\cup \ldots \in G$ - не обязательно конечное количество
%    \item $G_1, \ldots, G_n \in G \Rightarrow G_1 \cap \ldots \cap G_n \in G$ - конечное количество
%\end{enumerate}
%Назовем члены $G$ открытыми множествами, а окрестностью $x$ любое открытое множество содержащее $x$\\
%Тогда пределом последовательности $x_n$ можно назвать $a: \forall\,U(a)\ \exists\,N\ \forall\,n>N\ x_n\in U(a)$, но теорема о единственности предела выполняться не будет(к примеру, при $G=\{\varnothing, X\}$ предел любой последовательности - любое число в $X$. Доказательство теоремы о единственности не работает в силу несуществования двух непересекающихся множеств, содержащих последовательность.)\\
%\textbf{Замечание}
%\begin{enumerate}
%    \item $\operatorname{Int} D = \underset{G-\text{открытое}}{\bigcup_{G:G\subset D}G}$
%    \item $D$ - открытое множество $\Leftrightarrow D=\operatorname{Int} D$
%\end{enumerate}
\textbf{Определение}
\begin{enumerate}
    \item Проколотая окрестность $\overdot{U}(a)=B(a,r)\setminus \{a\}$
    \item $D\subset X$, $a$ - \textit{предельная точка} $D \Leftrightarrow \forall\,\overdot{U}(a)\ \overdot{U}(a)\cap D \neq \varnothing$\\
    \textit{($a \in D$ или $a\notin D$)}
\end{enumerate}
\textbf{Замечание}
\begin{enumerate}
    \item $a$ - предельная точка $D \Leftrightarrow \forall\,U(a)\ |\overdot{U}(a)\cap D| = \infty$
    \item $a$ - предельная точка $D \Leftrightarrow \exists\,(x_n)\neq a \subset D:\ x_n\rightarrow a$\\
    \textbf{Доказательство $\Rightarrow$}\\
    Рассмотрим $U_{r_1}(a)$. Возьмем там $d_1$\\
    Положим $d_2 = \min \rho(a,d_1), \frac {r_1}2$\\
    Повторим для $d_2\ldots\infty$\\
    Тогда $(d_n)$ - искомая последовательность\\
    \textbf{Доказательство $\Leftarrow$}\\
    В каждой окрестности есть какой-то $x_n$, а значит любая окрестность непустая
\end{enumerate}
\textbf{Определение}\\
$a\in D$ - \textit{изолированная точка}, если $\exists\,U(a):\ U(a)\cap D = {a}$\\\\
\textbf{Определение}\\
$D$ - \textit{замкнутое множество} в $X$, если $D$ содержит все свои предельные точки\\\\
\textbf{Теорема}\\
$D\subset X$ - замкнутое $\Leftrightarrow D^c=X\setminus D$ - открытое\\
\textbf{Доказательство $\Rightarrow$}\\
Пусть правое утверждение ложно. Тогда $\exists\,x\in D^c:\ \forall\,U(x)\ U(x)\not\subset D^c$. Тогда для такого $x$\\
$\forall\,U(x)\ U(x)\cap D \neq\varnothing \Leftrightarrow \overdot{U}(x)\cap D \neq \varnothing \Leftrightarrow x$ - предельная точка $\Rightarrow x\in D$, т.к. D замкнутое. Отсюда противоречие\\
Тогда $\exists\,U(x) \subset D^c$, ч.т.д.\\
\textbf{Доказательство $\Leftarrow$}\\
$D^c$ - открыто\\
Если $D$ не замкнуто, то $\exists\,x$ - предельная точка $D, x\notin D$. Тогда $x\in D^c \Rightarrow \exists\, U(x)\subset D^c \Rightarrow x$, т.е. $U(x) \cap D = \varnothing$ - не предельная точка $D$. Тогда $D$ - замкнуто, ч.т.д.\\\\
\textbf{Теорема о свойствах замкнутых множеств}\\
В произвольном метрическом пространстве $X$:
\begin{enumerate}
    \item $(F_\alpha)_{\alpha \in A}$ - произвольное семейство замкнутых в $X$ множеств. Тогда $\bigcap_{\alpha \in A} F_\alpha$ - замкнутое
    \item $(F_\alpha)_{\alpha \in A}$ - произвольное конечное семейство замкнутых в $X$ множеств. Тогда $\bigcap_{\alpha \in A} F_\alpha$ - замкнутое\\
    \textbf{Контр-пример для бесконечного семейства}\\
    В $\mathbb{R}$ $\{x\}$ - замкнутое в $\mathbb{R}$. Рассмотрим $\bigcup_{k=1}^\infty \{\frac 1k\}$. Тогда $0$ - предельная точка, не содержащаяся в множестве. Тогда множество не замкнутое
\end{enumerate}
\textbf{Доказательство}\\
Из свойств открытых множеств и теоремы о связи открытых и замкнутых множеств\\\\
\textbf{Определение}\\
$D\subset X$ - произвольное множество. Тогда \textit{замыкание $\overline{D}$ множества $D$} - это $D\cup\text{(все его предельные точки)}$\\
\textbf{Замечание}\\
\textit{Обратим внимание, что $\overline{D}$ содержит все предельные точки $D$, а не свои. Но все же $\overline{D}$ - замкнуто}\\
\textbf{Замечание}
\begin{enumerate}
    \item $\overline{D} = \{a \in X: \exists\,(x_n):\ x_n\rightarrow a, x_n \in D\}$
    \item $\overline{D} = \underset{F-\text{замкнуто}}{\bigcap_{F:D\subset F\subset X}} F$,\\
    т.е. $\overline{D}$ - наименьшее по включению замкнутое множество, содержащее $D$
    \item $D$ - замкнуто $\Leftrightarrow \overline{D} = D$
\end{enumerate}
\textbf{Определение}\\
$D\subset X$ - произвольное множество\\
$a$ - \textit{граничная точка} $D$, если $\forall\,\overdot{U}(a)\ \begin{array}{l}
     \overdot{U}(a) \cap D \neq \varnothing \\
     \overdot{U}(a) \cap D^c  \neq \varnothing
\end{array}$\\
\textbf{Замечание}
\begin{enumerate}
    \item Граничная точка - невнутренняя предельная точка
    \item Граничная точка - предельная точка $D$ и $D^c$
    \item Множество граничных точек замкнуто
    \item Множество предельных точек замкнуто
\end{enumerate}
\subsection{Компактность и полнота}
\textbf{Лемма Гейне-Бореля}\\
Рассмотрим $\mathbb{R}$\\
Пусть $[a,b] \subset \bigcup_{k=1}^\infty (a_k,b_k)$\\
Тогда найдется конечное число отрезков $k_1\ldots k_n$ таких, что $[a,b] \subset \bigcup_{i=1}^n (a_{k_i}, b_{k_i})$\\\\
\textbf{Теоремы об открытых и замкнутых множествах в пространстве и подпространстве}\\
Пусть $D \subset Y \subset X$, $X,Y$ - метрические пространства с общей метрикой\\
Тогда
\begin{enumerate}
    \item $D$ - открытое в $Y \Leftrightarrow \exists\,G$ - открытое в $X: D=G\cap Y$\\
    \textbf{Доказательство $\Leftarrow$}\\
    $G$ - открыто в $X$, $D = G\cap Y$. Доказать, что $D$ открыто в $Y$\\
    Берем $a\in D$\\
    $a \in D \Rightarrow a \in G$, а $G$ - открыто. Тогда $\exists\,r:\ B^x(a,r) \subset G \Rightarrow B^x(a,r)\cap Y = B^y(a,r) \subset G\cap Y = D$. Отсюда $a$ - внутренняя точка\\
    \textbf{Доказательство $\Rightarrow$}\\
    $D$ - открытое в $Y$\\
    $D=\bigcup_{x\in D} B^y(x,r_x)$, где $r_x$ подбираем так, чтобы $B^y(x,r_x) \in D$\\
    Возьмем $D=\bigcup_{x\in D} B^x(x,r_x)$. $G$ - открытое множество. Тогда $D = G\cap Y$
    \item $D$ - замкнутое в $Y \Leftrightarrow \exists\,F$ - замкнутое в $X: D=F\cap Y$\\
    \textbf{Доказательство}\\
    $D$ - замкнутое в $Y \Leftrightarrow (Y\setminus D)$ - открытое множество $\Leftrightarrow \exists\,G$ - открытое в $X: (Y\setminus D) = G\cap Y$\\
    $Y\setminus(Y\setminus D) = Y\setminus(G\cap Y)$\\
    $D = (Y\setminus G)\cap Y$\\
    $D = (X\setminus G) \cap Y$. Отсюда $X \setminus G$ - замкнутое
\end{enumerate}
\textbf{Определение}\\
$X$ - метрическое пространство\\
$K\subset X$\\
Если $K \subset \bigcup_{\alpha \in A} G_\alpha$, то множества $G_\alpha$ образует \textit{покрытие} $K$\\
Если все $G_\alpha$ - открытые, то \textit{открытое покрытие}\\
$K \subset \bigcup_{\alpha \in A' \subset A} G_\alpha $ - \textit{подпокрытие}\\\\
Множество называется \textit{компактным}, если \\
$\forall\,(G_\alpha)$ - открытые $: K \subset \bigcup_{\alpha \in A} G_\alpha\ \exists\,G_{\alpha_1}\ldots G_{\alpha_n}: K \subset \bigcup_{i=1}^n G_{\alpha_i}$\\\\
\textbf{Теорема}\\
Пусть $K\subset Y \subset X$\\
Тогда $K$ - компактно в $Y \Leftrightarrow K$ - компактно в $X$\\
\textbf{Доказательство $\Rightarrow$}\\
$K$ - компактно в $Y$\\
Пусть $K \subset \bigcup_{\alpha \in A} G_\alpha$ - открытые в $X$\\
Тогда $K \subset (\bigcup_{\alpha \in A} G_\alpha) \cap Y = \bigcup_{\alpha \in A} (G_\alpha \cap Y)$. $G_\alpha \cap Y$ - открыто в $Y$.\\
В силу компактности $K$ в $Y \exists\,\alpha_1\ldots\alpha_n$\\
$K \subset \bigcup_{i=1}^n (G_{\alpha_i} \cap Y) = \bigcup_{i=1}^n G_{\alpha_i}$\\
\textbf{Доказательство $\Leftarrow$}\\
Пусть $K$ компактно в $X$\\
$K \subset \bigcup_{\alpha\in A} O_\alpha$, где $O_\alpha$ открыты $Y$\\
$\forall\,O_\alpha =G_\alpha \cap Y$, где $G_\alpha$ открыты в $X$\\
$K \subset \bigcup_{\alpha \in A} G_\alpha$\\
Тогда $\exists\,\alpha_1\ldots\alpha_n:\ K \subset \bigcup_{i=1}^n G_{\alpha_i}$\\
Отсюда $K \subset \bigcup_{i=1}^n (G_{\alpha_i} \cap Y) = \bigcup_{i=1}^n O_{\alpha_i}$, ч.т.д.\\
\textbf{Теорема о простейших свойствах компактных множеств}\\
$(X, \rho)$ - метрическое пространство\\
$K \subset X$
\begin{enumerate}
    \item $X$ - компактно $\Rightarrow X$ замкнуто и ограничено\\
    \textbf{Доказательство}\\
    Докажем, что $K^c$ - открытое\\
    Пусть $x \in K^c$\\
    $K \subset \bigcup_{a\in K} B(a, \frac{\rho(a,x)}2)$\\
    Т.к. $K$ - компактно, $K \subset \bigcup_{i=1}^n B(a_i,r_i)$, где $r_i=\frac{\rho(a_i,x)}2$\\
    $B(a_i,r_i) \cap B(x,r_i) = \varnothing$\\
    Отсюда $B(x, \min(r_1,\ldots,r_n))$ не пересекает ни одно $B(a_i,r_i)$. Тогда $B(x,\min(r_1,\ldots,r_n)) \cap K = \varnothing \Leftrightarrow B(x,\min(r_1,\ldots,r_n)) \subset K^c$\\
    Т.о. $x \in K^c \Rightarrow B(x) \subset K^c$, а значит $K^c$ открыто. Тогда $K$ замкнуто, ч.т.д.
    Выберем $x_0 \in X$. $K \subset X \subset \bigcup_{n\in \mathbb{N}}B(x_0,n)$\\
    Из компактности $\exists\,n_1,\ldots,n_m:\ K \subset \bigcup_{i=1}^m B(x_0, n_i)$\\
    Тогда $K \subset B(x_0,\max(n_1,\ldots,n_m))$
    \item $X$ - компактно, а $K$ - замкнутно. Тогда $K$ - компактно\\
    \textbf{Доказательство}\\
    $K \subset \bigcup_{a\in K} G_a$. Тогда $X = \bigcup_{a\in K} G_a \cup K^c$, где $K^c$ - открыто.\\
    Тогда $\exists\,a_1,\ldots,a_n:\ X = \bigcup_{i=1}^n (G_{a_i} \cup K^c)$. Отсюда $K \subset \bigcup_{a\in K} G_a$, ч.т.д.
\end{enumerate}
\textbf{Теорема о компактности в пространстве и в подпространстве}\\
$K \subset X \subset Y$ - компактно в $X \Leftrightarrow K$ компактно в $Y$\\
\textbf{Определение}\\
$a,b\in \mathbb{R}^m$\\
\textit{Параллелепипед} $[a,b]=\{x\in\mathbb{R}^m: \forall\,i\in\{1\ldots m\}\ a_i\leq x_i\leq b_i\}$\\\\
\textbf{Лемма о вложенных параллелепипедах}\\
$[a^{(1)},b^{(1)}] \supset [a^{(2)},b^{(2)}] \supset \ldots$\\
$[a^{(1)},b^{(1)}] \cap [a^{(2)},b^{(2)}] \cap \ldots \neq \varnothing$\\
\textbf{Доказательство}\\
Покоординатно следует из теоремы Кантора\\
\textbf{Лемма}\\
Замкнутый параллелепипед компактен\\
\textbf{Доказательство}\\
$[A^{(1)},B^{(1)}] \subset \bigcup_{a\in K} G_a$ - открытые.
Воспользуемся \textit{половинным делением}\\
Допустим, что нет конечного подпокрытия\\
По каждой координате разделим параллелепипед на две части. Тогда он будет разделен на $2^m$ частей\\
Если бы все части можно было накрыть конечным числом покрытий, то и весь параллелепипед можно\\
Тогда существует такой "кусочек", который не покрывается конечным числом подпокрытий.\\
Назовем этот параллелепипед $[A^{(2)}, B^{(2)}]$. Применим к нему такую же логику\\
Тогда мы получаем бесконечную последовательность вложенных параллелепипедов, каждый из которых не покрывается конечным количеством подпокрытий. $\exists\,x \in [A^{(1)}, B^{(1)}] \cap [A^{(2)}, B^{(2)}] \cap \ldots$. Тогда $\exists\,G_i:\ x \in G_i$. Вместе с $x$ в $G_i$ содержится некая окрестность $B(x, R)$.\\
Линейные размеры параллелепипедов стремятся к 0, а значит с некоторого момента его размеры по всем координатам будут такими, что $\rho(A,B) < 2R$. Отсюда весь этот параллелепипед поместится в $B(x, R)$. Тогда с некоторого места все параллелепипеды содержатся в некотором покрытии. Противоречие.\\
Отсюда параллелепипед компактен, ч.т.д.\\\\
\textbf{Теорема о характеристике компактов в $\mathbb{R}^m$}\\
Данные утверждения эквивалентны
\begin{enumerate}
    \item $K \subset \mathbb{R}^m$ замкнуто и ограничено
    \item $K \subset \mathbb{R}^m$ компактно
    \item $K \subset \mathbb{R}^m$ секвенциально компактно
\end{enumerate}
\textbf{Доказательство}
\begin{enumerate}
    \item[$1\Rightarrow 2$:] $K$ - ограничено $\Rightarrow$ содержится в шаре $\Rightarrow$ содержится в параллелепипеде $\Rightarrow$ содержится в компактном множестве и замкнуто $\Rightarrow$ компактно
    \item[$2\Rightarrow 3$:]
    \begin{enumerate}
        \item Если некая последовательность $(x_n)$ имеет конечное число значений, то какое-то значение повторяется бесконечное количество раз. Тогда оно является частным пределом
        \item Иначе: пусть $D$ - множество значений $(x_n)$, $|D| = \infty$
        \begin{enumerate}
            \item если D не имеет предельных точек\\
            Пусть $K\subset D$
            Тогда $\forall\,x\in K\ \exists\,\overdot{B}(x,r'):\ \overdot{B}(x,r')\cap D = \varnothing$\\
            $K \subset \bigcup_{a\in K} B(x,r')$\\
            Тогда каждая такая окрестность покрывает конечное множество точек, а значит $K$ - не компактное - противоречие.
            \item существует $x_0$ - предельная точка $D$\\
            Тогда из закрытости $D$ $x_0\in D$ и из определения предельной точки в $D$ существует сходящаяся последовательность к $x_0$. Выкинем из нее элементы, индексы которых меньше, чем у предыдущих и получим подпоследовательность $(x_n)$, сходящуюся к $x_0$, ч.т.д.
        \end{enumerate}
    \end{enumerate}
    \item [$3\Rightarrow 1$:] Пусть $a$ - предельная точка $K$\\
    Проверим, что $a \in K$\\
    $\exists\,(x_n): \begin{array}{l}
         x_n\in K \\
         x_n \neq a \\
         x_n\rightarrow a
    \end{array}$\\
    Выберем подпоследоватьность $(x_{n_k})$. Из секвенциальной компактности $\exists\,n_k: x_{n_k} \rightarrow x_0 \in K$.\\
    Из $x_n\rightarrow a$ $(x_{n_k}) \rightarrow a$. Отсюда $x_0 = a$ и $a \in K$. Тогда $K$ замкнуто\\\\
    Пусть $K$ не ограничено, то существуют сколь угодно большие числа. Выберем $(x_n) \rightarrow \infty$. Тогда $x_{n_k}\rightarrow \infty$, что противоречит секвенциальной компактности. Тогда $K$ ограничено, ч.т.д.
\end{enumerate}
\textbf{Определение}\\
$K \subset X$ - \textit{секвенциально компактно}, если $\forall\,(x_n) \subset K\ \exists\,(n_k)\in\mathbb{N}, x_0\in K:\ (n_k)\uparrow,\ x_{n_k}\rightarrow x_0$\\
\textbf{Замечания}
\begin{enumerate}
    \item В произвольном метрическом пространстве замкнутое + ограниченное $\nRightarrow$ компактное
    \item $2 \Leftrightarrow 3$ в любом метрическом пространстве
    \item $2 \nLeftrightarrow 3$ в произвольном топологическом пространстве
\end{enumerate}
\textbf{Следствие (принцип выбора Больцано-Вейерштрасса)}\\
$(x_n)$ - ограниченная последовательность в $\mathbb{R}^m$\\
Тогда существует сходящаяся подпоследовательность\\
\textbf{Доказательство}\\
$x_n$ - ограниченная последовательность\\
Тогда существует замкнутый параллелепипед $K: \forall\,n\ x_n\in K$\\
Параллелепипед компактный.
Тогда $K$ -  секвенциально компактный\\
Тогда из свойств секвенциальной компактности, ч.т.д.\\
\textbf{Замечание}\\
$x_n$ - не ограничено $\Rightarrow\ \exists\,(x_n):x_{n_k} \rightarrow \infty$\\
\textbf{Определение}\\
$X$ - метрическое пространство\\
$(x_n)$ - \textit{фундаментальная последовательность} (последовательность Коши, сходящаяся в себе)\\
$\forall\,\varepsilon>0\ \exists\,N\ \forall\,n,m>N\ \rho(x_n,x_m) < \varepsilon$\\
\textbf{Лемма}
\begin{enumerate}
    \item $(x_n)$ - фундаментальная последовательность $\Rightarrow (x_n)$ - ограничена\\
    \textbf{Доказательство}\\
    Пусть $\varepsilon = 1$\\
    $\forall\,n_0,m > N(1)\ \rho(x_m,x_{n_0}) < 1$\\
    Тогда $x_m \in B(x_{n_0},1)$ начиная с $m > N(1)$\\
    Тогда не в $B(x_{n_0},1)$ конечное число точек, а значит вся последовательность ограничена
    \item $(x_n)$ - фундаментальная последовательность, $x_{n_k} \rightarrow A$\\
    Тогда $(x_n) \rightarrow A$\\
    \textbf{Доказательство}\\
    $\forall\,\varepsilon>0\ \exists\,N\ \forall\,n,m>N\ \rho(x_m,x_n) < \frac\varepsilon2$\\
    $x_{n_k} \rightarrow a \Leftrightarrow \forall\,\varepsilon>0\ \exists\,K\ \forall\,k>K\ \rho(x_{n_k}, a) <\frac\varepsilon2$\\
    $\forall\,\varepsilon>0\ \exists\,M = \max(N(\varepsilon), K(\varepsilon))\ \forall\,m>M\ \rho(x_m,a) \leq \rho(x_m,x_{n_k})+\rho(x_{n_k}, a) < \frac\varepsilon2+\frac\varepsilon2 < \varepsilon$
\end{enumerate}
\textbf{Теорема}
\begin{enumerate}
    \item $X$ - метрическое пространство\\
     $(x_n)$ - сходящаяся $\Rightarrow (x_n)$ - фундаментальная\\
     \textbf{Доказательство}\\
     $\forall\,\varepsilon>0\ \exists\,N\ \forall\,n>N\ \rho(x_n,a) < \frac\varepsilon2$\\
     Тогда $\forall\,\varepsilon>0\ \exists\,N\ \forall\,n,m>N\ \rho(x_n,x_m)<\rho(x_n,a)+\rho(x_m,a) < \frac\varepsilon2+\frac\varepsilon2 < \varepsilon$
     \item в $\mathbb{R}^m: (x_n)$ - фундаментальная $\Rightarrow (x_n)$ - сходится\\
     \textbf{Доказательство}\\
     $(x_n)$ - фундаментальная $\Rightarrow\ (x_n)$ - ограниченная$ \Rightarrow \exists\ x_{n_k} \rightarrow a \Rightarrow x_n \rightarrow a$
\end{enumerate}
\textbf{Определение}\\
Метрическое пространство \textit{полно}, если в нем любая фундаментальная последовательность является сходящейся\\
\textbf{Утверждение (критерий Больцано-Коши)}\\
$(x_n)$ - сходится в $\mathbb{R}^m \Leftrightarrow \forall\,\varepsilon>0\ \exists\,N\ \forall\,n,m>N\ \|x_n-x_m\| < \varepsilon$
\section{Пределы и непрерывность отображений}
\subsection{Всякие прикольные теоремы}
\textbf{Теорема Кантора}\\
Пусть $[a_1, b_1] \supset [a_2, b_2] \supset \ldots$, $|[a_n,b_n]| = b_n-a_n \rightarrow 0$. Тогда $\exists!\, c: \bigcap_{k=1}^\infty [a_k,b_k] = \{c\}$\\
\textbf{Доказательство}\\
Т.к. пересечение не пусто, берем любую точку $c \in \bigcap_{k=1}^\infty [a_k,b_k]$.\\
Тогда $\forall\,k\ a_k\leq c \leq b_k$\\
$|a_k-c|\leq b_k-a_k \rightarrow 0$, т.е. $a_k\rightarrow c$\\
$|b_k-c|\leq b_k-a_k \rightarrow 0$, т.е. $b_k\rightarrow c$\\
Из единственности предела $c$ единственный\\
\textbf{Следствие}\\
$a_k, b_k \rightarrow c$\\\\
Алгоритм перевода в двоичную дробь: делим наш промежуток пополам. Если число попало в левую половинку, дописываем 0 и переходим в влево, иначе дописываем 1 и переходим вправо.\\
\textbf{Теорема}
\begin{enumerate}
    \item Пусть $\varepsilon_i$ - бесконечная последовательность из 0 и 1.\\
    Тогда $0,\varepsilon_1\varepsilon_2\varepsilon_3\ldots$ определяет некоторое число из $[0,1]$
    \item $\forall\,x\in [0,1]$ существует не более двух последовательностей $\varepsilon_i$, задающих $x$
    \begin{enumerate}
        \item Если $x$ - двоичное рациональное число кроме 0, т.е. $x = \frac a{2^b} \neq 1$ - две записи
        \item Иначе - одна запись\\
        \textbf{Доказательство}\\
    $x = \bigcap_{k=1}^\infty\ [0,\varepsilon_1\ldots\varepsilon_k; 0,\varepsilon_1\ldots\varepsilon_k+\frac{1}{2^k}]$
    \end{enumerate}
    \item Баг:\\
    $x$ может оказаться между половинками очередного отрезка, тогда он принадлежит обеим половинкам.\\
    $x$ окажется на стыке тогда и только тогда, когда он - двоичное рациональное число(на $b$-ом шаге)\\
    В этом случае $x$ имеет две записи
    \item Отдельно:
    $1,00\ldots = 0,11\ldots$
\end{enumerate}
В любом упорядоченном поле $\mathbb{R}$, $\mathbb{N} \subset \mathbb{R}$\\\\
Рассмотрим $A\subset\mathbb{R}$\\
$A$ - индуктивное, если\begin{enumerate}
    \item $\1 \in A$
    \item $\forall\,x\in A\ x+\1 \in A$
\end{enumerate}
Самое маленькое индуктивное множество:\\
$\mathbb{N} = \underset{A-\text{индуктивное}}{\bigcap_{A\subset R}} A$\\\\
\textbf{Неравенство Бернулли}\\
$\forall\,x\geq -1, n\in \mathbb{N}\quad(1+x)^n \geq 1+nx$\\
\textbf{Доказательство}
\begin{enumerate}
    \item База($n=1$): $1+x \geq 1+x$
    \item Шаг индукции:\\
    Пусть $(1+x)^n \geq 1+nx$ - верно\\
    Докажем $(1+x)^{n+1} \geq 1+(n+1)x$:\\
    $(1+x)^{n+1}=(1+x)(1+x)^n \geq (1+x)(1+nx) = 1 + nx + x + nx^2 \geq 1+(n+1)x$, ч.т.д.
\end{enumerate}
\textbf{Определение}\\
Множество $A \subset \mathbb{R}$ \textit{ограничено сверху}:\\
$\exists\,M\in\mathbb{R}:\ \forall\,a\in A\ a\leq M$\\
Множество $A \subset \mathbb{R}$ \textit{ограничено снизу}:\\
$\exists\,m\in\mathbb{R}:\ \forall\,a\in A\ a\geq m$\\
Множество $A \subset \mathbb{R}$ \textit{ограничено}, если оно ограничено сверху и снизу\\
$x \in A$ - \textit{максимум}, если $\forall\,a\in A\ a\leq x$\\
$x \in A$ - \textit{минимум}, если $\forall\,a\in A\ a\geq x$\\\\
\textbf{Теорема}\\
В любом конечном множестве существует максимальный(минимальный) элемент\\
\textbf{Доказательство}
\begin{enumerate}
    \item База: для $n=1$ $A=\{x\}$, $x$ - максимум
    \item Переход: рассмотрим множество из $n+1$ элементов $A$. Выберем элемент $x$. Множество $A\setminus\{x\}$ имеет максимум $y$. Тогда максимум множества $A$ - это $\max x,y$.
\end{enumerate}
\textbf{Определение}\\
Множество $\mathbb{Q}$ \textit{плотно} в $\mathbb{R}$, если $\forall\,a,b \subset \mathbb{R}\ \exists\,x\in \mathbb{Q}:\ x\in[a,b]$\\
\textbf{Доказательство для рациональных чисел}\\
Пусть $n \in \mathbb{N} > \frac 1{b-a}$(существует по теореме Архимеда)\\
$\frac 1n < b-a$\\
Возьмем $x=\frac {[na]+1}n \in \mathbb{Q}$\\
$a=\frac{na-1+1}n < \frac{[na]+1}n \leq \frac{na+1}n = a+\frac 1n < a + (b-a) = b$\\
Отсюда $a<x< b$, ч.т.д.\\\\
\textbf{Теорема о существовании супремума}\\
$E\neq \varnothing \subset \mathbb{R}$, ограниченное сверху\\
Тогда $\exists\,s \in \mathbb{R}: s = \sup E$\\\\
\textbf{Доказательство}\\
Пусть $b_1$ - верхняя граница $E$, $a_1 \in E$\\
$c_1=\frac{a_1+b_1}2$:
\begin{enumerate}
    \item если $c_1$ - верхняя граница, то рассмотрим промежуток $[a_2, b_2]: a_2 = a_1; b_2 = c_1$\\
    $b_2$ - верхняя граница
    \item если $c_1$ - не верхняя граница, то рассмотрим промежуток $[a_2, b_2]: a_2 = c_1; b_2 = b_1$\\
    $b_2$ - верхняя граница
\end{enumerate}
$|[a_n,b_n]| = b_n-a_n=\frac{b_1-a_1}{2^n}$\\
Повторяем аналогичные действия. Тогда по следствию из теоремы Кантора существует единственный $s = \bigcap_{k=1}^{\infty} [a_n,b_n]$\\
Проверим, что $s = \sup E$:
\begin{enumerate}
    \item $\forall\,x\in E, n\ x\leq b_n$\\
    $b_n\rightarrow s$\\
    Отсюда $\forall\,x\in E\ x\leq s$
    \item $\forall\,\varepsilon\ \exists\,n\ b_n-a_n<\varepsilon$\\
    $a_n \in E$
    \item $\forall\,\varepsilon\ \exists\,n\ s < b_n<\varepsilon+a_n$
\end{enumerate}
\textbf{Дополнительная часть опеределения}
\begin{enumerate}
    \item $E$ не ограничено сверху: $\sup E = +\infty$
    \item $E$ не ограничено снизу: $\inf E = -\infty$
    \item $E = \varnothing$: $\sup E = -\infty, \inf E = +\infty$
\end{enumerate}
\textbf{Лемма о свойствах супремума}
\begin{enumerate}
    \item $D\neq\varnothing \subset E \subset \mathbb{R}$\\
    Тогда $\sup D \leq \sup E$\\
    \textbf{Доказательство}\\
    $\sup E$ - верхняя точка $D$
    \item $X\subset\mathbb{R}, \lambda\in\mathbb{R}$\\
    $\lambda X = {\lambda\cdot x: x \in X}$
    Тогда $\forall\,\lambda>0\ \sup \lambda X = \lambda \sup X$
    \item $\sup -X = -\inf X$\\
    $\sup kX = k\sup X, k > 0$\\
    $\inf kX = k\inf X, k > 0$\\
    $\sup X+Y = \sup X + \sup Y$\\
    $\inf X+Y = \inf X + \inf Y$
\end{enumerate}
\textbf{Определение}
\begin{enumerate}
    \item $f: X \rightarrow \mathbb{R}$, $D \subset X$\\
    $f$ - ограничена(сверху/снизу) на $D \Leftrightarrow f(D) \subset R$ - ограниченное множество(сверху/снизу)
    \item $f: \langle a,b \rangle \rightarrow \mathbb{R}$\\
    $f$ - монотонна $\Leftrightarrow \forall\,x_1,x_2 \in \langle a,b\rangle: x_1 < x_2\quad f(x_1)\leq f(x_2)$
\end{enumerate}
\textbf{Теорема о пределе монотонной последовательности}
\begin{enumerate}
    \item $(x_n)$ - ограниченная сверху возрастающая вещественная последовательность\\
    Тогда эта последовательность сходится к $s = \sup x_n$
    \item $(x_n)$ - ограниченная снизу убывающая вещественная последовательность\\
    Тогда эта последовательность сходится к $i = \inf x_n$
    \item $(x_n)$ - ограниченная монотонная последовательность\\
    Тогда эта последовательность сходится
\end{enumerate}
\textbf{Доказательство}\\
$\forall\,\varepsilon>0\ \exists\,N:\ s-\varepsilon<x_n$\\
Из возрастания $\forall\,\varepsilon>0\ \exists\,N\ \forall\,n>N\ s-\varepsilon<x_n$\\
или $\forall\,\varepsilon>0\ \exists\,N\ \forall\,n>N\ 0 \leq s-x_n<\varepsilon$\\
Т.о. $s = \lim_{n\rightarrow +\infty} x_n$, ч.т.д.\\\\
\textbf{Замечание}\\
$x_n$ - возрастающая. Тогда $\exists\,\lim x_n = \sup x_n \in \overline{\mathbb{R}}$\\\\
\textbf{Лемма (о сходимости к нулю быстро убывающей последовательности)}\\
Пусть $x_n>0; \lim_{n\rightarrow\infty}\frac{x_{n+1}}{x_n} < 1$\\
Тогда $x_n \rightarrow 0$\\
\textbf{Доказательство}\\
Начиная с некоторого места, $x_n$ убывает. $x_n > 0$. Тогда существует $L: x_n \rightarrow L$.\\
$L \geq 0$
\begin{enumerate}
    \item $L = 0$ - ч.т.д.
    \item $L > 0$:\\
    Пусть $\lim_{n\rightarrow\infty}\frac{x_{n+1}}{x_n} = l < 1$\\
Тогда для $\varepsilon = \frac{1-l}{2}$: $\exists\, N:\ \forall\,n>N\ \frac{x_{n+1}}{x_n} < \frac{l+1}2 < 1$\\
    \textit{(Из определения предела)}\\
    В то же время для $\varepsilon = L\frac2{l+1}-L$: $\exists\, N:\ \forall\,n>N\ x_n-L < \varepsilon$\\
    Рассмотрим $x_n$ для $n>N$:\\
    $x_n < L\frac2{l+1}$\\
    % Для $n > N$ рассмотрим $y \in [L; L\frac2{l+1}], y < x_n$\\
    % $x_n < L\frac2{l+1}$\\
    % $\frac y{x_n} \geq \frac{L}{L\frac 2{l+1}} = \frac {l+1} 2$\\
    По лемме о пределе монотонной последовательности $\inf x_n = L$\\
    Тогда $x_{n+1} \geq L$\\
    Тогда $\frac {x_{n+1}}{x_n} \geq \frac{l+1}2$. Противоречие\\
\end{enumerate}
Отсюда $L = 0$\\\\
\textbf{Следствие}
\begin{enumerate}
    \item $a>1, k \in \mathbb{N}$. Тогда $\lim_{n\rightarrow \infty} \frac{n^k}{a^n} = 0$
    \item $a>0$. Тогда $\lim_{n\rightarrow \infty} \frac{a^n}{n!} = 0$
    \item $\lim_{n\rightarrow \infty} \frac{n!}{n^n} = 0$
\end{enumerate}
\subsection{Предел отображений}
\textbf{Определение}\\
$X, Y$ - метрическое пространство\\
$D\subset X, f:D\rightarrow Y$\\
$a$ - предельная точка $D$\\
Определим:\\
$\lim_{x\rightarrow a} f(x) = A \Leftrightarrow f(x)\rightarrow A$, если:
\begin{enumerate}
    \item Определение по Коши; на языке $\varepsilon-\delta$:\\
    $\forall\,\varepsilon > 0\ \exists\,\delta>0\ \forall\,x\in D: 0\neq \rho^x(x,a)<\delta\quad\rho^y(f(x),A) < \varepsilon$
    \item На языке окрестностей:\\
    $\forall\,U(A)\ \exists\,V(a)\ \forall\,x\in D \cap \overdot{V}(a)\quad f(x)\in U(A)$\\
    \textit{($U, V$ - окрестности)}
    \item По Гейне:\\
    $\forall\,(x_n): \left\{\begin{array}{l}
        x_n \in D  \\
        x_n \neq a\\
        x_n \rightarrow a
    \end{array}\right.\ f(x_n)\rightarrow A$
\end{enumerate}
\textbf{Теорема}\\
Определения по Коши и по Гейне эквивалентны\\
\textbf{Доказательство}\\
$x,y$ - метрические пространства\\
$f:D\subset X \rightarrow Y$\\
$a$ - предельная точчка $D$\\
$A \in Y$\\
$\lim_{x\rightarrow a} f(x) = A$\\
Тогда по Коши:\\
$\forall\,\varepsilon > 0\ \exists\,\delta>0\ \forall\,x\in D: 0\neq \rho^x(x,a)<\delta\quad\rho^y(f(x),A) < \varepsilon$\\
По Гейне:\\
$\forall\,(x_n): \left\{\begin{array}{l}
        x_n \in D  \\
        x_n \neq a\\
        x_n \rightarrow a
\end{array}\right.\ f(x_n)\rightarrow A$
\begin{enumerate}
    \item Докажем, что из определения Коши следует определение Гейне\\
        Возьмем $x_n \in D, x_n \neq a, x_n \rightarrow a$\\
        Из определения Коши для $\varepsilon$:\\
        $\exists\,\delta>0\,\forall\,x\in D: 0<\rho(x,a)<\delta\quad\rho(f(x),A) < \varepsilon$\\
        Из $x_n \rightarrow a$:\\
        $\exists\,N\ \forall\,n>N\ \rho(x_n,a)<\delta$\\\
        Тогда $\rho(f(x_n),A) < \varepsilon$, ч.т.д.
    \item Доказем, что из определения Гейне следует определение Коши\\
        Пусть определение Коши неверное\\
        $\exists\,\varepsilon > 0:\ \forall\,\delta>0\ \exists\,x\in D:\ 0<\rho(x,a)<\delta\quad\rho(f(x),A) \geq A$\\
        Возьмем $\delta = 1$\\
        $\exists\, x_1\in D\ 0<\rho(x_1,a)<1\quad \rho(f(x_1),A) \geq \varepsilon$\\
        $\vdots$\\
        Возьмем $\delta=\frac 1n$:
        $\exists\, x_n\in D\ 0<\rho(x_n,a)<\frac 1n\quad \rho(f(x_n),A) \geq \varepsilon$\\
        Отсюда $\rho(x_n,a) \rightarrow 0 \Leftrightarrow x_n \rightarrow a$, а $\rho(f(x_n),A) > \varepsilon$. Тогда $f(x_n)\nrightarrow A$ - противоречие
    \end{enumerate}
\textbf{Замечание}
\begin{enumerate}
    \item $a$ - предельная точка $D\Rightarrow$ последовательности из определения 3 существуют
    \item Если $a \in D$, то предел не зависит от $f(a)$
    \item $f\equiv g$ на некоторой $\overdot{W}(a)$(выколотой окрестности $a$) и $\exists\,\lim_{x\rightarrow a} f(x)=A$, то $\exists\,\lim_{x\rightarrow a} g(x)$ и $\lim_{x\rightarrow a} g(x) = A$
    \item Определение 2 можно обобщить на случай $X=\overline{\mathbb{R}}, Y = \overline{\mathbb{R}}, D \subset \mathbb{R}, a, A \in \overline{\mathbb{R}}$
    \item $X,Y$ - метрические пространства\\
    Определение 2 равносильно \\
    $\forall\,U\subset Y: A \in U, U - \text{открытое}\ \exists\,V\subset X: a \in V, V -\text{открытое}\ \forall\,x\in D\cap V \setminus \{a\}\ f(x)\in U$\\
    \textit{(Назовем его топологическим определением предела)}\\
    \textbf{Доказательство $\Rightarrow$}\\
    Выберем множество $U$. Тогда существует $U(A) \subset U$\\
    Выберем множество $V$. Тогда существует $V(a) \subset V$\\
    Пусть дано Определение 2. Для каждого $U$ будем рассматривать только $U(A)$, а вместо $V$ будем брать только $V(a)$\\
    \textbf{Доказательство $\Leftarrow$}\\
    Для каждого $V$ существует $V(a) \subset V$. Сузим $V$ до $V(a)$. От этого утверждение не пострадает\\
    Если для всех $U$ утверждение верно, то и для всех $U=U(A)$ работает, т.к. это частный случай
    \item Попробуем обобщить Определение 1 для предела $\infty$. Для этого можно ввести метрику $\rho(a,b)=|\arctan a - \arctan b|$ , считая, что $\arctan \pm\infty = \pm\frac \pi2$\\
    Тогда $x_n\rightarrow \pm\infty \Leftrightarrow \rho(x_n, \pm\infty) \rightarrow 0$
\end{enumerate}
\textbf{Свойства пределов отображений}\\
$f: D\subset X \rightarrow Y$
\begin{enumerate}
    \item $\lim_{x\rightarrow a} f(x) = A, \lim_{x\rightarrow a} f(x) = B \Rightarrow A = B$\\
    \textbf{Доказательство}\\
    Предел последовательности $f(x_n)$ из определения Гейне единственный
    \item (Локальная ограниченность отображения, имеющего предел)\\
    $\lim_{x\rightarrow a} f(x) = A$\\
    Тогда $\exists\,U(a): f|_{U(a)\cap D}$ - ограничено\\
    \textbf{Доказательство}\\
    Если $a \notin D$: Для $B(A)$ $\exists\,\overdot{U}(a)\ \forall\,x\in \overdot{U}(a)\cap D\ f(x)\in B(A)$\\
    Если $a \in D$: Для $B(A) = B(A,R+\rho(f(a),A))$ $\exists\,U(a)\ \forall\,x\in U(a)\cap D\ f(x)\in B(A)$
    \item (Теорема о стабилизации знака)\\
    $\lim_{x\rightarrow a} f(x) = A$\\
    $A \neq B$\\
    $\exists\,U(a):\ \forall\,x\in \overdot{U}(a)\cap D\ f(x) \neq B$\\
    \textbf{Доказательство}\\
    Для $B(A, \rho(A,B))$ $\exists\,\overdot{U}(a)\ \forall\,x\in \overdot{U}(a)\cap D\ f(x)\in B(A,\rho(A,B))$, а значит $f(x) \neq B$\\
    \textbf{Следствие}\\
    $B = 0$\\
    Тогда $\sign A = \sign f(x)$ в некоторой окрестности
    \item $g,f:D\subset X \rightarrow Y$, $X$ - метрическое пространство, $Y$ - нормированное пространство\\
    $\lambda: D \rightarrow \mathbb{R}$\\
    $f(x) \xrightarrow[x\rightarrow a]{} A \in Y, g(x) \xrightarrow[x\rightarrow a]{} B\in Y, \lambda(x) \xrightarrow[x\rightarrow a]{} L\in \mathbb{R}$\\
    Тогда
    \begin{enumerate}
        \item $f(x)\pm g(x)\xrightarrow[x\rightarrow a]{} A\pm B$
        \item $\lambda(x)f(x)\xrightarrow[x\rightarrow a]{} LA$
        \item $\|f(x)\|\xrightarrow[x\rightarrow a]{} \|A\|$
    \end{enumerate}
    \textbf{Доказательство}\\
    Из определения Гейне\\
    \textbf{Дополнение}\\
    При $B \neq 0$:\\
    $\frac {f(x)}{g(x)} \rightarrow \frac AB$
\end{enumerate}
\textbf{Определение}\\
%$G \subset \mathbb{R}, G$ - открытое множество. Тогда $G=\bigcup_{\alpha\in A} (a_\alpha, b_\alpha)$,где $\{(a_\alpha, b_\alpha)\}_{\alpha\in A}$\\\\\\\\
%Пусть $V(x)$ - окрестность, если $\left\{\begin{array}{l}
%     \forall\,V_1(x),V_2(x)\ \Rightarrow \exists\,V_3(x)\subset V_1(x)\cap V_2(x)\\
%     \forall\,x,y\ \exists\,V(x),V(y):\ V(x)\cap V(y) \neq \varnothing 
%\end{array}\right.$\\
%Отсюда можно определить открытые множества\\
%Топология \textit{метризуемая}, если $\exists$ метрика на $X$ *
% *TODO*\\
Рассмотрим в $\overline{\mathbb{R}}$ метрику $\rho(x,y) = |\arctan x - \arctan y|$
\begin{enumerate}
    \item $x_n\rightarrow a \in \mathbb{R} \Leftrightarrow \rho(x_n,a) \rightarrow 0$\\
    $x_n\rightarrow +\infty \Leftrightarrow \rho(x_n,+\infty) \rightarrow 0$
    \item Тогда из определения Гейне можно получить предел функции в $\overline{\mathbb{R}}$
    \item Теоремы об арифметических свойствах предела последовательности в $\overline{\mathbb{R}}$ также выполняются при условии, что все операции имеют смысл (нет выражений вида $+\infty-\infty$ и т.д)
    \item Также выполняются теоремы об арифметических свойствах пределов отображений\\
\end{enumerate}
\textbf{Теорема о предельном переходе в неравенствах}\\
$f,g:D\subset X \rightarrow \mathbb{R}$, $a$ - предельная точка $D$\\
$\forall\,x\in D\setminus\{a\}\ f(x) \leq g(x)$\\
$\lim_{x\rightarrow a} f(x) = A, \lim_{x\rightarrow a} g(x) = B$, где $A,B \in \overline{\mathbb{R}}$\\
Тогда $A\leq B$ из определения Гейне\\
\textbf{Следствие}\\
$f,g,h: D\subset X \rightarrow \mathbb{R}$, $a$ - предельная точка $D$\\
$f(x) \leq g(x) \leq h(x)$ при $x \in D \setminus\{a\}$\\
$\lim_{x\rightarrow a} f(x) = A, \lim_{x\rightarrow a} h(x) = A$. Тогда $\exists\,\lim_{x\rightarrow a} g(x) = A$ из Гейне\\
\textbf{Определение}\\
$f: D\subset X \rightarrow Y$, $a$ - предельная точка $D$\\
$D'\in D$, $a$ - предельная точка $D'$\\
Предел $f(x)$ при $x\rightarrow a$ по множеству $D':$ - это $\lim_{x\rightarrow a} f|_{D'}(x)$\\
\textbf{Определение}\\
$f:D\subset\mathbb{R} \rightarrow \mathbb{R}$, $a$ - предельная точка $D$\\
\textit{Левосторонний предел} при $x\rightarrow a, D' = (-\infty, a) \cap D$ - это\\
$\lim_{x\rightarrow a} f|_{D'} = \lim_{x\rightarrow a-0} f(x)$\\
\textit{Правосторонний предел} при $x\rightarrow a, D' = (a, +\infty) \cap D$ - это\\
$\lim_{x\rightarrow a} f|_{D'} = \lim_{x\rightarrow a+0} f(x)$\\
\textbf{Теорема о пределе монотонной функции}\\
$f:D\subset \mathbb{R} \rightarrow \mathbb{R}$ и монотонна, $a\in \overline{\mathbb{R}}$\\
$D'=(-\infty,a)\cap D$, $a$ - предельная точка $D'$\\
Тогда
\begin{enumerate}
    \item $f\uparrow$ и ограничена сверху $\Rightarrow\exists\,\lim_{x\rightarrow a-0} f(x)$ - конечный\\
    \textbf{Доказательство}\\
    Дополнение: $\lim_{x\rightarrow a-0}f(x) = \sup_{D'} f(x)$\\
    Пусть $\sup_{D'} f(x) = A$\\
    Докажем $\lim_{x\rightarrow a-0}f(x) = A$\\
    $\forall\, \varepsilon > 0\ \exists\,x\in D'\ A-\varepsilon<f(x)\leq A$\\
    Пусть $\delta = |x-a|$\\
    Тогда при $x':a-\delta =x < x' < a\quad A-\varepsilon < f(x) \leq f(x') \leq A$\\
    Т.е. $\forall\,\varepsilon > 0\ \exists\,\delta\ \forall\,x:a-\delta < x < a\ A-\varepsilon<f(x)\leq A$\\
    Т.е. $f(x) \xrightarrow[x\rightarrow a-0]{} A$\\
    Аналогично для неограниченной функции $f(x) \xrightarrow[x\rightarrow a-0]{} +\infty$
    \item $f\downarrow$ и ограничена снизу $\Rightarrow\exists\,\lim_{x\rightarrow a-0} f(x)$ - конечный
    \item Аналогично для правого предела возрастающей ограниченной снизу последовательности
    \item Аналогично для правого предела убывающей ограниченной сверху последовательности
\end{enumerate}
\textbf{Критерий Больцано-Коши для отображений}\\
Пусть $f:D\subset X \rightarrow Y$ - полное\\
$a$ - предельная точка $D$\\
Тогда данные выражения эквивалентны:
\begin{enumerate}
    \item $\exists\, \lim_{x\rightarrow a}f(x) \in Y$
    \item $\forall\,\varepsilon>0\ \exists\,V(a)\ \forall\,x,x' \in D \cap \overdot{V}(a)\ \rho(f(x),f(x')) < \varepsilon$
\end{enumerate}
\textbf{Доказательство}
\begin{enumerate}
    \item[$1 \Rightarrow 2$] Из существования предела:\\
    $\forall\,\varepsilon>0\ \exists\,V(a)\ \forall\,x \in D \cap \overdot{V}(a)\ \rho(f(x),A) < \frac\varepsilon2$\\
    $\forall\,\varepsilon>0\ \exists\,V(a)\ \forall\,x' \in D \cap \overdot{V}(a)\ \rho(f(x),A) < \frac\varepsilon2$\\
    Отсюда $\forall\,\varepsilon>0\ \exists\,V(a)\ \forall\,x,x' \in D \cap \overdot{V}(a)\ \rho(f(x),f(x')) \leq \rho(f(x),A)+\rho(f(x'),A) < \varepsilon$
    \item[$2 \Rightarrow 1$] по Гейне\\
    Возьмем $(x_n): \left\{\begin{array}{l}
         x_n\in D\\
         x_n \neq a\\
         x_n \rightarrow a
    \end{array}\right.$\\
    $\forall\,\varepsilon>0\ \exists\,V(a)\ \forall\,x,x' \in D \cap \overdot{V}(a)\ \rho(f(x),f(x')) < \varepsilon$\\
    И $\forall\,\varepsilon>0\ \exists\,V(a)\ \exists\,N\ \forall\,n>N\ x_n\in V(a)$\\
    Тогда можем взять $x=x_n, n > N; x' = x_m, m > N$\\
    Отсюда $(f(x_n))$ - фундаментальная, а значит в $Y$ существует конечный предел $f(x_n)$, ч.т.д.
\end{enumerate}
\textbf{Следствие}\\
$f:D\in\mathbb{R}\rightarrow \mathbb{R}$, $a$ - предельная точка $D$\\
Тогда $\exists\,\lim_{x\rightarrow a}f(x) \in \mathbb{R} \Leftrightarrow \forall\,\varepsilon>0\ \exists\,\delta>0\ \forall\,x,x'\in D: \begin{array}{l}
     0 < |x-a|<\delta\\
     0 < |x'-a| < \delta
\end{array}\ |f(x)-f(x')| < \varepsilon$\\
Для $a=+\infty$ аналогично
\subsection{Непрерывное отображение}
\textit{Определение}\\
Пусть $f:D\subset X \rightarrow Y$, $X,Y$ - метрические пространства\\
$x_0 \in D$\\
Говорят, что $f$ непрерывна в $x_0$, если верно одно из утверждений\\
\textit{(на самом деле тогда верны все)}
\begin{enumerate}
    \item $\lim_{x \rightarrow x_0} f(x) = f(x_0)$ либо $x_0$ - изолированная точка
    \item (по Коши) $\forall\,\varepsilon > 0\ \exists\,\delta > 0\ \forall\,x\in D\ \rho(x,x_0)<\delta\ \rho(f(x), f(x_0)) < \varepsilon$
    \item (на языке окрестностей) $\forall\,U(f(x_0))\ \exists\,V(x_0)\ \forall\, x\in D \cap V(x_0)\ f(x) \in U(f(x_0))$\\
    \textit{(Эквивалентна топологическому определению: $V(x_0)$ - открытое множество, содержащее $x_0$, $U(f(x_0))$ - открытое множество, содержащее $f(x_0)$)}
    \item (по Гейне) $\left.\begin{array}{c}
         x_n \in D \\
         x_n \rightarrow x_0
    \end{array}\right\}\Rightarrow f(x_0) \rightarrow f(x_0)$
\end{enumerate}
Доказательство эквивалентности аналогично доказательству эквивалентности определений пределов отображений\\
\textbf{Определение}\\
$f:D\subset\mathbb{R} \rightarrow Y$, $x_0\in D$\\
Если $f|_{D\cap (-\infty, x_0]}$ - непрерывна в $x_0$, то $f$ - \textit{непрерывна в $x_0$ слева}\\
Если $f|_{D\cap [x_0, +\infty)}$ - непрерывна в $x_0$, то $f$ - \textit{непрерывна в $x_0$ справа}\\
Если $f$ непрерывна слева и справа в точке $x_0$, то она непрерывна в точке $x_0$\\\\
\textbf{Обозначения}\\
Для непрерывных функций\\
$\lim_{x\rightarrow x_0+0} f(x) = f(x_0+0)$\\
$\lim_{x\rightarrow x_0-0} f(x) = f(x_0-0)$\\\\
\textbf{Определение}\\
Если $f(x_0+0), f(x_0-0) \in \bb{R}$ определены и $f(x_0+0) \neq f(x_0-0)$ или $x_0 \not\in D$ или $f(x_0\pm0) \neq f(x_0)$, то в точке $x_0$ $f(x)$ \textit{имеет скачок(разрыв I рода)}\\
В данном случае $f$ не является непрерывной, т.е. \textit{имеет разрыв в $x_0$}\\
Также бывает \textit{разрыв II рода} - $\not\exists\,f(x_0+0) \in \bb{R}$ или $\not\exists\,f(x_0-0) \in \bb{R}$\\
Если $x_0 \notin D$ и $f(x_0+0) = f(x_0-0)$, то разрыв будем считать \textit{устранимым}\\\\
\textbf{Определение}\\
$f:D\subset X \rightarrow Y$ \textit{непрерывна на $D$}, если непрерывна в каждой точке $D$\\
\textbf{Арифметрические свойства}
\begin{enumerate}
    \item $f,g: D\subset X \rightarrow Y$, $Y$ - нормированное пространство\\
    $\lambda:D \rightarrow \mathbb{R}$\\
    $x_0\in D, f,g,\lambda$ - непрерывны в $x_0$\\
    Тогда $f+g,\lambda f, \|f\|$ - непрерывны в $x_0$
    \item $f,g: D\subset X \rightarrow \mathbb{R}$, $x_0\in D$, $f,g$ - непрерывные в $x_0$\\
    Тогда $f+g, fg, |f|$ - непрерывны\\
    $\frac{f}{g}, g(x_0) \neq 0$ - непрерывна\\ 
\end{enumerate}
\textbf{Замечание}\\
Для непрерывности на множестве $D$ теоремы аналогичные\\\\
\textbf{Теорема о стабилизации знака для непрерывных функций}\\
$f:D\subset X \rightarrow \mathbb{R}$, $x_0 \in D$, $f$ - непрерывна в $x_0$\\
Тогда $f(x_0) > 0 \Rightarrow \exists\,U(x_0): f|_{U(x_0)} > 0$\\\\
\textbf{Теорема о непрерывности композиции}\\
$f:D\subset X \rightarrow Y$\\
$g:E\subset Y \rightarrow Z$\\
$f(D) \subset E, x_0 \in D, f$ - непрерывна в $x_0, f(x_0) \in E, g$ - непрерывна в $f(x_0)$\\
Тогда $g\circ f$ непрерывна в $x_0$\\
\textbf{Доказательство}\\
$\forall\,x_n:\ \left\{\begin{array}{l}
     x_n \in D\\
     x_n \rightarrow x_0
\end{array}\right.\ \left\{\begin{array}{l}
     f(x_n) \in E\\
     f(x_n) \rightarrow f(x_0)\\
\end{array}\right.$\\
Тогда $g(f(x_n)) \rightarrow g(f(x_0))$\\\\
\textbf{Теорема о пределе композиции}\\
$f: D \subset X \rightarrow Y$\\
$g: E\subset Y \rightarrow Z$\\
$f(D) \subset E, x_0$ - предельная точка $D, \lim_{x\rightarrow x_0} f(x) = A$\\
$A$ - предельная точка $E, \lim_{y\rightarrow A} g(y) = B$\\
Пусть $\exists\,U(x_0)\ \forall\, x\in U(x_0) \cap D\ f(x) \neq A$\\
Тогда $\exists\,\lim_{x\rightarrow x_0} g(f(x)) = B$\\
Также предел будет существовать и равен $B$, если $A\in E, g$ - непрерывна в $A$\\
\textbf{Доказательство}\\
По Гейне\\
$\left.\begin{array}{l}
     x_n \in D\\
     x_n \rightarrow x_0\\
     x_n \neq x_0\\
     f(x_n) \rightarrow A\\
     f(x_n) \in E\\
     f(x_n) \neq A \text{ начиная с некоторого места}
\end{array}\right\} \Rightarrow g(f(x_n)) \rightarrow B$\\\\
\textbf{Определение}\\
Функции $\const, x^\alpha (\alpha \in \mathbb{R}), \sin x, \cos x, e^x, \ln x, \arcsin x, \arctan x$ и полученные из них конечным числом арифметрических операций и композиций называются \textit{элементарными функциями}\\
\textbf{Теорема}\\
Все элементраные функции непрерывны на своих областях определения\\
\textbf{Теорема (о топологическом определении непрерывности)}\\
$f: X \rightarrow Y$, $X, Y$ - метрические пространства\\
Тогда $f$ - непрерывна на $X \Leftrightarrow \forall\,G\subset Y$ - открытое в $Y\ f^{-1}(G)$ - открыто в $X$\\
\textbf{Доказательство $\Leftarrow$}\\
Рассмотрим $a \in X$\\
Пусть $G\subset Y$ - открытое, $f(a) \in G$\\
Тогда $f^{-1}(G)$ - открыто в $X, a \in f^{-1}(G)$\\
Тогда $\exists\ U(a):\ U(a) \subset f^{-1}(G)$, ч.т.д.\\
\textbf{Доказательство $\Rightarrow$}\\
Пусть $G \subset Y$ - открытое\\
Выберем $a \subset f^{-1}(G)$\\
Тогда $f(a) \in G$\\
Тогда по определению существует окрестность $U(a) \subset f^{-1}(G)$, ч.т.д.\\
\textbf{Теорема Вейерштрасса о непрерывном образе компакта}\\
Пусть $f:X\rightarrow Y$ - непрерывно на $X$, $X,Y$ - метрические пространства, $X$ - компактно\\
Тогда $f(X)$ - компактно\\
\textbf{Доказательство}\\
Пусть $f(X) \subset \bigcup_{\alpha \in A} G_\alpha$, где $G_\alpha$ - открытые в $Y$\\
Тогда $X \subset \bigcup_{\alpha \in A} f^{-1}(G_\alpha)$. Из предыдущей теоремы $f^{-1}(G_\alpha)$ - открыты\\
Тогда существует конечное подпокрытие $f^{-1}(G_{\alpha_i}): X \subset \bigcup_{i=1}^n f^{-1}(G_{\alpha_i})$\\
Тогда $f(X) \subset \bigcup_{i=1}^n G_{\alpha_i}$\\
\textbf{Следствие 1}\\
В условиях теоремы $f(X)$ - замкнутое и ограниченное в $Y$\\
\textbf{Следствие 2(первая теорема Вейерштрасса)}\\
Пусть $f: [a,b] \rightarrow \mathbb{R}$ - непрерывно на $[a,b]$\\
Тогда $f([a,b])$ - ограниченное\\
\textbf{Следствие 3}\\
$f:X\neq \varnothing\rightarrow \mathbb{R}$ - непрерывна на $X$, $X$ - компактно\\
Тогда $\exists\,\min f(X), \max f(X)$\\
\textbf{Доказательство}\\
$f(X)$ - замкнуто и ограничено, а значит $\exists\,\sup f(X)$ и $\sup f(X) \in f(X)$, ч.т.д.\\
\textbf{Следствие 4(вторая теорема Вейерштрасса)}\\
Пусть $f:[a,b] \rightarrow\mathbb{R}$, непрерывна на $[a,b]$\\
Тогда $\exists\,\max f,\min f$\\
% \textbf{Исторический пример}\\
% Найдем макс. площадь вписанного $n$-угольника\\
% \textbf{Решение}\\
% Определим, существует ли максимум\\
% Заметим, что если $n$-угольник не содержит центр окружности, то его площадь может быть увеличена сдвигом вершины\\
% Тогда рассмотрим $n$-угольник, содержащий центр окружностиъ\\
% Пусть его центральные углы $\alpha_1, \ldots, \alpha_n, 0\leq \alpha \leq \pi, \sum_{i=1}^n \alpha_i = 2\pi$\\
% Площадь $n$-угольника равна $S = \frac12 R^2(\sin\alpha_1 + \ldots + \sin\alpha_n)$\\
% Рассмотрим множество наборов углов $X = \{\alpha \in \mathbb{R}^n: 0\leq \alpha_i \leq \pi, \sum \alpha_i = 2\pi\}$\\
% Тогда $S:X\rightarrow \mathbb{R}$\\
% $X$ - замкнутый шар в $\mathbb{R}^n$\\        //лажа, это не замкнутый шар, а его часть(не забывай об ограничении на каждую из координат
% Отсюда $X$ ограниченное и замкнутое, а значит компактное\\
% Отсюда $S(X)$ - компактно. Тогда оно замкнуто и ограничено. Отсюда у него существует максимум\\
\textbf{Определение}\\
Пусть $A$ - метрическое пространство\\
$A$ - \textit{связно}, если невозможно представить $A$ в виде объединения двух открытых непересекающихся множеств\\
\textbf{Лемма (о связности отрезка)}\\
$[a,b]$ в $\mathbb{R}$ невозможно представить в виде объединения двух непересекающихся непустых открытых множеств\\
$\nexists\,G_1, G_2$ - открытые в $\mathbb{R}: [a,b] \subset G_1 \cup G_2, [a,b] \cap G_1 \neq \varnothing, [a,b] \cap G_2 \neq \varnothing, G_1 \cap G_2 = \varnothing$\\
\textbf{Доказательство}\\
Пусть $G_1, G_2$ существуют\\
Пусть $a \in G_1$\\
Пусть $t = \sup \{x: [a,x] \subset G_1\}$\\
Пусть $b_2 \in G_2$\\
Тогда $t \leq b_2$\\
$t$ - корректно определенная точка на $[a,b]$\\
Если бы $t$ лежал в $G_1$, то она лежала бы там с некой окрестностью, а значит $t$ не был бы $\sup$. Тогда $t \notin G_1$.\\
Если бы $t$ лежал в $G_2$, то она лежала бы там с некой окрестностью, а значит $t$ не был бы $\sup$. Тогда $t \notin G_2$.\\
Отсюда $t \in [a,b], t \notin G_1 \cup G_2$, что невозможно.\\\\
\textbf{Следствие}\\
Утверждение верно не только для $[a,b]$, но и для $\langle a, b\rangle$\\
\textbf{Обозначение}\\
$C(\langle a,b\rangle)$ - множество функций $f:\langle a,b\rangle \rightarrow \mathbb{R}$, непрерывных на $\langle a,b\rangle$\\
\textbf{Теорема Больцано - Коши о промежуточном значении}\\
Пусть $f \in C[a,b]$. Тогда $\forall\,\min f(a),f(b) \leq t \leq \max f(a),f(b)\ \exists\,x\in[a,b]: f(x)=t$\\
\textbf{Доказательство}\\
Пусть существует $t_0$, не удовлетворяющее этому условию\\
Тогда $[a,b] = f^{-1}((-\infty, t_0)) \cup f^{-1}((t_0, -\infty))$, что противоречит теореме\\
\textbf{Теорема о бутерброде}\\
Пусть $A,B \subset \mathbb{R}^2, A\cap B = \varnothing$ - выпуклые многоугольники\\
Тогда существует прямая $l$, рассекающая оба многоугольника на равные многоугольники\\
\textbf{Доказательство}\\
Для начала решим задачу разреза одного многоугольника прямой, параллельной вектору $v\in \mathbb{R}^2$\\
Будем двигать прямую по прямоугольнику и считать $\delta$ - разность площадей частей прямоугольника, расположенных по разные части от прямой\\
$\delta$ принимает значения от $[-S_A, S_A]$\\
Заметим, что $\delta$ непрерывна(доказывается через две приближающиеся друг к другу прямые)\\
Тогда $\delta$ принимает все значения $[-S_A, S_A]$, а значит возможно добиться $\delta = 0$, т.е. разрезать прямоугольник на 2 равные по площади части\\\\
Будем задавать наш вектор $v$ через угол $\phi$. Для вектора построим прямую, разделяющую $A$ на две равные по площади части\\
Рассмотрим $\sigma(\phi)$ - разность двух половин, на которые данная прямая рассекает $B$. $\sigma$ будет принимать значения $[-S_B, S_B]$\\
Заметим, что $\sigma(\phi)$ и $\sigma(\phi + \pi)$ разных знаков\\
Заметим, что $\sigma(\phi)$(доказывается черед два вектора с близкими друк другу углами. Не забываем, что иногда образуется не треугольник, а четырехугольник. Также уточняем, что точка пересечения прямых лежит в $A$)\\
Тогда $\sigma$ пересекает 0, ч.т.д.\\\\
\textbf{Теорема о сохранении промежутка}\\
$f\in C\langle a,b\rangle, m = \inf f, M = \sup\limits_{} f, m,M \in \overline{\mathbb{R}}$\\
Тогда $f(\langle a,b\rangle) = \langle m, M\rangle$ (выбор скобок не согласован)\\
\textbf{Доказательство}\\
Достаточно проверить, что $\forall\,t\in (m,M)\ \exists\,c:\ f(c) = t$\\
Если это не так, рассмотрим $t_0$, для которого это не выполняется. Тогда $\langle a, b \rangle = f^{-1}((-\infty, t_0))\cup f^{-1}((t_0, +\infty))$\\
Заметим, что $(-\infty, t_0)$ и $(t_0, +\infty)$ не пусто\\
\textbf{Замечание}\\
Тип промежутка не сохраняется\\
\textbf{Доказательство}\\
$sin((0, 2\pi)) = [-1, 1]$\\
\textbf{Но}\\
По теореме Вейерштрасса образ отрезка - отрезок\\
\textbf{Определение}\\
Пусть $\gamma:[a,b] \rightarrow Y$, $Y$ - метрическое пространство, функция непрерывна\\
Тогда $\gamma$ - \textit{путь}\\
\textbf{Определение}\\
$E \in Y$, $Y$ - метрическое пространство\\
$E$ - \textit{линейно связное} множество, если $\forall\,A,B \in E\ \exists\,$ непрерывная $\gamma:[a,b] \rightarrow Y: \gamma(a) = A,\gamma(b) = B$\\
\textbf{Пример}\\
$E = (\Gamma_{y=\sin \frac1x}) \cup ([(0,-1), (0,1)]$ - отрезок $)$\\
$E$ - связное, но не линейно связное\\
\textbf{Лемма}\\
$E\subset \mathbb{R}$ - линейно связное $\Leftrightarrow E$ - промежуток\\
\textbf{Доказательство $\Leftarrow$}\\
Пусть $A, B \in \langle a, b \rangle$\\
Тогда $\gamma(t \in [0, 1]) = A + t(B-A)$ - искомая функция\\
\textbf{Доказательство $\Rightarrow$}\\
Если $E = \varnothing$ - очевидно\\
Иначе:\\
$m = \inf E, M = \sup E$\\
Пусть $\exists\,t \in (m, M), t \notin E$\\
Из линейной связности для $A < t < B$ существует непрерывный путь из $A$ в $B$. А значит этот путь принимает все значения, включая $t$. Тогда $t \in E$.\\
Отсюда $(m, M) \in E$, а значит $E = \langle m, M \rangle$\\
\textbf{Теорема о сохранении линейной связности}\\
$f: X \rightarrow Y$ - непрерывно\\
$X$ - линейно связное\\
Тогда $f(X)$ - линейно связное\\
\textbf{Доказательство}\\
Пусть $A, B \in f(X), U, V \in X, f(U) = A, f(V) = B$\\
Построим путь между $U, V$ - $c: [a,b] \rightarrow X, c(a) = U, c(b) = V, c$ - непрерывно\\
Тогда $f\circ c$ - путь в $f(x)$\\
\textbf{Теорема (о непрерывности монотонной функции)}\\
Пусть $f: \langle a, b \rangle \rightarrow \mathbb{R}$ - монотонная функция\\
Тогда
\begin{enumerate}
    \item У такой функции не может быть разрывов второго рода
    \item Непрерывность $f \Leftrightarrow f(\langle a, b \rangle)$ - промежуток
\end{enumerate}
\textbf{Доказательство п.1}\\
Не умоляя общности пусть $f$ - возрастающая\\
Возьмем $x_1 \leq x \leq x_2$\\
Тогда $f(x_1) \leq f(x) \leq f(x_2)$\\
Заметим, что у такой функции есть предел (из теоремы о пределе монотонной функции)\\
Пусть $x \rightarrow x_2 - 0$\\
Тогда $f(x_1) \leq f(x_1+0) \leq f(x_2)$\\
Тогда у функции существует конечный односторонний предел справа (аналогично слева)\\
Тогда в любой точке у данной функции существует односторонний предел, а значит не может быть разрывов второго рода, ч.т.д.\\
\textbf{Доказательство п.2}\\
$\Rightarrow$: из теоремы о сохранении промежутка\\
$\Leftarrow$:\\
Рассмотрим $x_0 \in (a, b)$\\
Пусть $f(x_0 + 0) \neq f(x_0)$\\
Из монотоности $f(x_0) < f(x_0 + 0)$\\
Тогда $(f(x_0), f(x_0 + 0))$ не лежит в множестве значений(для $x < x_0\ f(x) \leq f(x_0)$, для $x > x_0\ f(x) \geq f(x_0+0)$), но тогда множество значений - не промежуток - противоречие\\
Аналогично для левостороннего предела\\
\textbf{Следствие}\\
Пусть $f:\langle a,b\rangle \rightarrow \mathbb{R}$, $f$ - монотонна\\
Тогда множество точек разрыва не более чем счетно\\
\textbf{Доказательство}\\
Не умоляя общности, пусть $f$ возрастает\\
Пусть $X$ - множество точек разрыва $f$\\
Построим инъекцию $\phi: X \rightarrow \mathbb{Q}$:\\
Пусть $x_0 \in X$ - точка разрыва. Тогда $f$ имеет скачок в этой точке\\
Тогда $f(x_0-0) < f(x_0+0)$ (неравенство из разрывности)\\
Тогда $\forall\,x_1<x_0<x_2\ f(x_1) \leq f(x_0-0) < f(x_0+0) \leq f(x_2)$(см. доказательство п.1)\\
Отсюда пусть $\phi(x_0) = $ любое $a \in \mathbb{Q} \cap (f(x_0-0), f(x_0+0))$\\
Заметим, что $\phi$ является инъекцией, что и требовалось\\
Отсюда множество $X$ не более чем счетно\\
\textbf{Пример}\\
Пусть $\{r_n, r\in \mathbb{N}\} = \mathbb{Q}$\\
$f(x) := \sum_{n=1}^\infty \frac{\sign(x-r_n)}{2^n}$\\
\textbf{Теорема (о существовании и непрерывности обратной функции)}\\
Пусть $f \in C(\lan a, b \ran), f$ - строго монотонна, $m = \inf f, M = \sup f$\\
Тогда
\begin{enumerate}
    \item $f$ - обратима, $f^{-1}: \lan m, M\ran \rightarrow \lan a, b \ran$, функция биективна 
    \item $f^{-1}$ строго монотонна и имеет ту же монотонность
    \item $f^{-1}$ непрерывна
\end{enumerate}
\textbf{Определение}\\
Определим функцию $x^\alpha, \alpha \in \mathbb{Q} \leftrightarrow f_\alpha(x)$
\begin{enumerate}
    \item $\alpha = 1: f_1 = \nm{id}$\\
    $f_1(x)$ непрерывна
    \item $f_n(x) = x\cdot \ldots\cdot x, n \in \mathbb{N}, n \geq 2$ - непрерывна как произведение\\
    При нечетном $n$ - непрерывна на $\mathbb{R}$\\
    При четном $n$ - непрерывна на $(-\infty, 0]$ и $[0, +\infty)$
    \item $f_{-n}(x) = \frac1{f_n(x)}, n \in \mathbb{N}, x \neq 0$\\
    Непрерывна на $\mathbb{R}\setminus\{0\}$ и монотонна на $(-\infty, 0)$ и $(0, +\infty)$
    \item $f_0 = 1$ на $\mathbb{R}$
    \item $f_\frac1n (x), n \in \bb{R}, n$ - нечетная\\
    Рассмотрим $f_n$:\\
    $f_n:\bb{R} \rightarrow \bb{R}$, строго возрастает, непрерывна\\
    Тогда $\exists\,(f_n)^{-1}: \bb{R} \rightarrow \bb{R}$, она непрерывна и возрастает\\
    $f_\frac1n := (f_n)^{-1}$
    \item $f_\frac1n (x), n \in \bb{R}, n$ - нечетная\\
    Рассмотрим сужение $f_n$ на $\bb{R_+}$:\\
    $f_n:\bb{R}_+ \rightarrow \bb{R}_+$, строго возрастает, непрерывна\\
    Тогда $\exists\,(f_n)^{-1}: \bb{R}_+ \rightarrow \bb{R}_+$, она непрерывна и возрастает\\
    $f_\frac1n := (f_n)^{-1}$
    \item $f_\frac pq := f_\frac1q\circ f_p, \frac pq$ - несократимая дробь\\
    Если $p$ - четная или $q$ нечетная, то $f_\frac pq: \bb{R} \rightarrow \bb{R}$\\
    Иначе $f_\frac pq: \bb{R}_+ \rightarrow \bb{R}_+$
\end{enumerate}
\textbf{Свойства}\\
Пусть $x>0$
\begin{enumerate}
    \item $x^{r+s} = x^r\cdot x^s$
    \item $x^{rs}= (x^r)^s$
    \item $(xy)^s=x^sy^s$
\end{enumerate}
\subsection{Показательная функция}
\textbf{Определение}\\
Функция $f:\bb{R} \rightarrow \bb{R}$ называется \textit{показательной}, если она
\begin{enumerate}
    \item Непрерывна
    \item Не является $f \equiv 0$ или $f \equiv 1$
    \item Удовлетворяет свойству $f(x+y) = f(x)f(y)$
\end{enumerate}
\textbf{Свойства показательных функций}\\
Пусть $f$ - показательная функция. Тогда 
\begin{enumerate}
    \item $f(x) > 0, f(0) = 1$\\
    \textbf{Доказательство}\\
    Т.к. $f \not\equiv 0$, $\exists\, x_0:\ f(x_0) \neq 0$\\
    Тогда $f(0+x_0) = f(0)f(x_0)$\\
    Отсюда $f(0) = 1$\\
    $\forall\,x\ f(x) \neq 0$, т.к. если $f(x_1) = 0$, то $\forall\,t\ f(t) = f(x_1)f(t-x_1) = 0$, а значит $f\equiv 0$\\
    $f(x) = f(\frac x2)f(\frac x2) > 0$
    \item $\forall\,r \in \bb{Q}\ f(rx)=f(x)^r$\\
    \textbf{Доказательство}\\
    Если $r \in \bb{N}$, очевидно\\
    Если $r = -n, n \in \bb{N}$: $1 = f(0) = f(nx-nx) = f(nx)f(-nx) = f(nx)f(rx)$\\ Тогда $f(rx) = \frac 1{f(nx)}$\\
    Если $r = \frac 1n, n \in \bb{N}$: $f(x) = f(n\frac1n) = f(\frac xn)^n$\\ Тогда $f(\frac xn) = f(x)^{\frac 1n}$\\
    Если $r = \frac mn, \frac mn$ - несократимая дробь: $f(rx) = f(m\frac xn) = f(\frac xn)^m = f(x)^\frac mn$
    \item $f$ строго монотонна\\
    Пусть $a := f(1)$\\
    $a \neq 1$\\
    $a > 1 \Rightarrow f(x) \uparrow$\\
    $a < 1 \Rightarrow f(x) \downarrow$\\
    \textbf{Доказательство}\\
    Если $a = 1$, то $\forall\, r \in \bb{Q}\ f(r) = f(r-1)f(1) = f(r-1)$\\
    Тогда из непрерывности функция тождественна единице, что противоречит условию\\
    Пусть $a > 1$\\
    Тогда $\forall\,x>0\ f(x) > 1$:\\
    $f(1\cdot\frac mn) = a^\frac mn > 1, \frac mn$ - несократимая дробь - по свойствам степенной функции\\
    Тогда $\forall\,x>0\ f(x) \geq 1$(через предельный переход)\\
    Тогда $\forall\,x>0\ f(x) > 1$, т.к. $\forall\,x>0\ \exists\,r\in\bb{Q}:\ 0 < r < x$\\
    Тогда $f(x) = f(r)f(x-r)$. $f(r) > 1, f(x-r) \geq 1$. Отсюда $f(x) > 1$\\
    Т.о. $f(x)$ строго возрастает: $f(x+h)=f(x)f(h) > f(x)\cdot 1$\\
    Убывание аналогично
    \item Множество значений $f$ - это $(0, +\infty)$\\
    \textbf{Доказательство}\\
    $f$ строго монотонна и непрерывна. Тогда множество значений $f$ - $(\inf f, \sup f)$\\
    Из свойств $a^r, r \in \bb{Q}$: $\inf f = 0, \sup f = +\infty$
    \item Пусть $f, g$ - показательные функции. Тогда если $f(1) = g(1)$, то $f = g$
\end{enumerate}
\textbf{Теорема}\\
Пусть существует $f_0$ - показательная функция такая, что $\lim_{x\rightarrow 0} \frac{f(x)-1}{x} = 1$\\
\textbf{Доказательство}\\
Ниже
\textbf{Теорема}\\
Пусть $f$ - произвольная показательная функция\\
Тогда $\exists\,\alpha\in \bb{R}:\ \forall\,f(x)=f_0(\alpha x)$, где $f_0$ - функция из предыдущей теоремы\\
\textbf{Доказательство}\\
Множество значений $f_0$ - $(0, +\infty)$\\
$f(1) = a > 0, a \neq 1$\\
$\exists\,\alpha \neq 0:\ f_0(\alpha) = a$\\
Пусть $g(x) = f_0(\alpha x)$\\
$g$ - показательная функция\\
$g(1) = a$\\
Из свойства 5 $f(x) = g(x) = f_0(\alpha x)$, ч.т.д.\\
\textbf{Следствие 1}\\
Существует единственная $f_0$ из теоремы 2\\
\textbf{Доказательство}\\
Пусть $h$ - показательная функция из теоремы 2\\
Тогда $\exists\,\alpha:\ h(x) = f_0(\alpha x)$\\
По теореме 2: $1 \xleftarrow[x\rightarrow 0]{} \frac{h(x)-1}{x} = \frac{f_0(\alpha x) -1}{\alpha x}\alpha \xrightarrow[x\rightarrow 0]{} \alpha$\\
Отсюда $\alpha = 1, h = f_0$, ч.т.д.\\
\textbf{Определение}\\
$f_0$ - \textit{экспонента}\\
$f_0 = \exp$\\
$f_0(1) = e$\\
\underline{Обозначения} $\exp x$ и $e^x$ эквивалентны\\
\textbf{Следствие 2}\\
Для любого $a > 0, a \neq 1$ существует единственная показательная функция $f: f(1) = a$\\
Такую функцию будем обозначать $a^x$\\
\textbf{Доказательство}\\
Существование:\\
Для $a$ из условия $\exists!\,\alpha:\ f_0(\alpha) = a$\\
Тогда $f(x) = f_0(\alpha x)$\\
Единственность из свойства 5\\
\textbf{Следствие 3}\\
$\forall\,a >0, a\neq 1\ \forall\,x,y \in \bb{R}\ a^{xy} = (a^x)^y = (a^y)^x$\\
\textbf{Доказательство}\\
Если $x = 0$, все тривиально (хотя по определению справа не показательная функция)\\
Если $x \neq 0$:
$b:= a^x$. Из свойств функции $b > 0, b \neq 1$\\
Для $y \in \bb{Q}\ a^{xy} = (a^x)^y = b^y$ - из свойств\\
Для $y\in \bb{R}$ подберем $(r_k) \subset Q, r_k \rightarrow y $\\
$a^{xr_k} = (a^x)^{r_k}$\\
Тогда из непрерывности $a^{xr_k} \rightarrow a^{xy}, (a^x)^{r_k} \rightarrow (a^x)^y$\\
Отсюда $a^{xy} = (a^x)^y$
\subsection{Логарифм}
$a^x : \bb{R} \rightarrow (0, +\infty)$ - строго монотонна и непрерывна\\
Тогда существует обратная функция $\log_a x : (0, +\infty) \rightarrow \bb{R}$ - непрерывная и строго монотонная\\
\textbf{Свойства}\\
$a > 0, a \neq 1$
\begin{enumerate}
    \item $\log_a xy = \log_a x + \log_a y$    
    \item $\log_a b^x = x\log_a b$
    \item $\log_a x = \log_a c \log_c x$
\end{enumerate}
\textbf{Доказательство}\\
$u = v \Leftrightarrow a^u = a^v$
\begin{enumerate}
    \item $a^{\log_a xy} = xy = a^{\log_a x}a^{\log_a y} = a^{\log_a x + \log_a y}$
    \item $a^{\log_a b^x} = b^x = a^{x\log_a b}$
    \item $\log_a x = \log_a(c^{\log_c x}) = \log_a c \log_c x$
\end{enumerate}
\subsection{Степень с произвольным показателем}
Пусть $\sigma \in \bb{R}, x > 0$\\
$x^\sigma = e^{\sigma \ln x}$ - степенная функция\\
\textbf{Теорема}\\
Пусть $A \subset \bb{R}$ - \textit{рационально зависимое}, если $\exists\,x_1,\ldots,x_n \in A:\ \exists\, r_1, \ldots, r_n \in \bb{Q}$(не все нули)$: r_1x_1+\ldots+r_nx_n = 0$\\
Пусть $X$ - множество всех рационально независимых подмножеств $\bb{R}$\\
Введем в $X$ отношение частичного порядка $A \subset B$\\
//todo 11:30 12.12 аксиома выбора
\subsection{Тригонометрические функции}
\textbf{Утверждение}\\
При $0 < x < \frac\pi2\ \sin x < x < \tg x$ - из площадей\\
\textbf{Следствие}\\
$|\sin x| \leq |x|$\\
\textbf{Утверждение}\\
$\sin x, \cos x$ - непрерывны на $\bb{R}$\\
\textbf{Доказательство}\\
Докажем для $x_0$: $|\sin x - \sin x_0| = 2|\cos \frac{x+x_0}{2} \sin \frac{x-x_0}{2}| \leq 2\frac{|x-x_0|}{2} = |x-x_0|$\\
\textbf{Утверждение}\\
$\sin$ монотонен на $[-\frac{\pi}{2}, \frac{\pi}{2}] \Rightarrow \exists\,\arcsin: [-1, 1] \rightarrow [-\frac{\pi}{2}, \frac{\pi}{2}]$\\
$\cos$ монотонен на $[0, \pi] \Rightarrow \exists\, \arccos: [-1, 1] \rightarrow [0, \pi]$
\subsection{Асимптотические разложения}
\textbf{Определение}\\
$f,g:D\subset X \rightarrow \mathbb{R}$, $a$ - предельная точка $D$\\
Если существует $\phi: D \rightarrow \mathbb{R}$ и $\forall\,x\in D\setminus\{a\}\ f(x) = \phi(x)g(x)$
\begin{enumerate}
    \item $\phi$ ограничена на $V(a) \cap D$, то \textit{$f$ ограничена по сравнению с $g$ в $V(a)$}\\
    $f = O(g)$
    \item $\phi\xrightarrow[x\rightarrow a]{} 0$, то \textit{$f$ бесконечно мала относительно $g$ при $x\rightarrow a$}\\
    $f = o(g)$ 
    \item $\phi\xrightarrow[x\rightarrow a]{} 1$, то \textit{$f$ эквивалентна $g$ при $x\rightarrow a$}\\
    $f \sim g$
\end{enumerate}
\textbf{Аналогичные определения}\\
$f,g:D\subset \mathbb{R} \rightarrow \mathbb{R}$
\begin{enumerate}
    \item $\exists\,C > 0\ \forall\,x\in D\ |f(x)| \leq C|g(x)| \Leftrightarrow f=O(g)$ на $D$
    \item $f=O(g); g=O(f)$ - $f$ и $g$ \textit{асимптотически сравнимы} на $D$
\end{enumerate}
\textbf{Замечание}\\
$f=o(g); g \neq 0$ в $\overdot V(a) \cap D \Leftrightarrow \frac fg \xrightarrow[x\rightarrow a]{}0$\\
$f \sim g; g \neq 0$ в $\overdot V(a) \cap D \Leftrightarrow \frac fg \xrightarrow[x\rightarrow a]{}1$\\
\textbf{Примеры свойств}
\begin{enumerate}
    \item При $x \rightarrow a: f \sim g \Leftrightarrow f = g+o(g) = g+o(f)$
    \item $o(f) \pm o(f) = o(f)$
\end{enumerate}
\textbf{Эквивалентные функции при $x \rightarrow 0$}\\
$\begin{array}{ll}
    \sin x \sim x & \sin x = x + o(x)\\
    e^x -1 \sim x & e^x = 1 + x + o(x)\\
    (1+x)^\alpha - 1 \sim \alpha x & (1+x)^\alpha = 1 + \alpha x + o(x)\\
    \ln(1+x) \sim x & \ln(1+x) = x + o(x)\\
\end{array}$\\\\
\textbf{Теорема о замене на эквивалентные функции}\\
Пусть у нас есть функции $f,g,\widetilde{f}, \widetilde{g}: D \subset X \rightarrow Y$, $a$ - предельная точка $D$\\
$f \sim \widetilde{f}, g \sim \widetilde{g}$ при $x \rightarrow a$\\
Тогда
\begin{enumerate}
    \item $\exists\lim_{x\rightarrow a} f(x)g(x) \in \overline{\mathbb{R}} \Leftrightarrow \exists\lim_{x\rightarrow a} \widetilde{f}(x)\widetilde{g}(x) \in \overline{\mathbb{R}}$ и при существовании\\
    $\lim_{x\rightarrow a} f(x)g(x) = \lim_{x\rightarrow a} \widetilde{f}(x)\widetilde{g}(x)$
    \item $\exists\lim_{x\rightarrow a} \frac{f(x)}{g(x)} \in \overline{\mathbb{R}} \Leftrightarrow \exists\lim_{x\rightarrow a} \frac{\widetilde{f}(x)}{\widetilde{g}(x)} \in \overline{\mathbb{R}}$ и при существовании\\
    $\lim_{x\rightarrow a} \frac{f(x)}{g(x)} = \lim_{x\rightarrow a} \frac{\widetilde{f}(x)}{\widetilde{g}(x)}$, если $a$ - предельная точка $D'=D\cap \{x: g(x) \neq 0\}$
\end{enumerate}
\textbf{Доказательство}\\
$f(x) = \phi(x)\widetilde{f}(x), g(x) = \psi\widetilde{g}(x)$ в $U(a) \cap D$ и $\phi, \psi \xrightarrow[x\rightarrow a]{} 1$\\
Тогда
\begin{enumerate}
    \item $\lim_{x\rightarrow a} f(x)g(x) = \lim_{x\rightarrow a} \phi(x)\psi(x)\widetilde{f}(x)\widetilde{g}(x) = \lim_{x\rightarrow a} \widetilde{f}(x)\widetilde{g}(x)$\\
    $\exists\ \lim_{x\rightarrow a} f(x)g(x) \Rightarrow \exists\lim_{x\rightarrow a} \widetilde{f}(x)\widetilde{g}(x)$
    \item $\lim_{x\rightarrow a} \widetilde{f}(x)\widetilde{g}(x) = \lim_{x\rightarrow a} \frac1{\phi(x)}\frac1{\psi(x)}f(x)g(x) = \lim_{x\rightarrow a} f(x)g(x)$\\
    $\exists\lim_{x\rightarrow a} \widetilde{f}(x)\widetilde{g}(x) \Rightarrow \exists\ \lim_{x\rightarrow a} f(x)g(x)$
\end{enumerate}
\textbf{Определение}\\
$g_1,g_2,g_3,\ldots: D\subset X\rightarrow \mathbb{R}$, $a$ - предельная точка $D$\\
Пусть $\forall\,k\ g_{k+1}=o(g_k), x\rightarrow a$\\
Тогда набор функций $g_1,g_2,\ldots$ называют \textit{шкалой}\\
$f = c_1g_1+c_2g_2+\ldots+c_ng_n+o(g_n), x\rightarrow a$ - \textit{асимптотическое разложение по шкале $(g_k)$}\\\\
\textbf{Теорема о единственности асимптотического разложения}\\
$f,g_1,\ldots,g_n:D\subset X \rightarrow \mathbb{R}$, $a$ - предельная точка $D$\\
$g_1,\ldots,g_n$ - шкала асимптотического разложения при $x\rightarrow a$\\
$f = c_1g_1+c_2g_2+c_3g_3\ldots c_ng_n+o(g_n)$\\
$f = d_1g_1+d_2g_2+d_3g_3\ldots d_ng_n+o(g_n)$\\
Тогда $c_i = d_i, i = 1\ldots n$\\
\textbf{Доказательство}\\
Пусть $m := min\{k: c_k \neq d_k\}$\\
Тогда $f=c_1g_1+\ldots+c_mg_m+o(g_m)$\\
$f=d_1g_1+\ldots+d_mg_m+o(g_m)$\\
Отсюда $f-f=0=(c_m-d_m)g_m+o(g_m), x\rightarrow a$\\
$(d_m-c_m)g_m = o(g_m)$\\
Отсюда  $g_m = o(g_m)$, что невозможно, ч.т.д.\\
\textbf{Пример}\\
$f(x) = ax+b+o(1), x\rightarrow +\infty$ (для шкалы $x^1,x^0,x^{-1},\ldots$)\\
Тогда $y=ax+b$ - \textit{наклонная асимптота} к графику $y=f(x)$\\
$\lim_{x\rightarrow+\infty}\frac{f(x)}{x} = a$\\
$\lim_{x\rightarrow+\infty} f(x)-ax = b$\\\\
\textbf{Теорема (Формула Тейлора для многочленов)}\\
$f:\mathbb{R}\rightarrow\mathbb{R}$ - многочлен $\deg f = n$, $x_0\in \mathbb{R}$\\
Тогда $\forall\,x\in\mathbb{R}\\f(x)=f(x_0)+\frac{f'(x_0)}{1!}(x-x_0)+\frac{f''(x_0)}{2!}(x-x_0)^2+\ldots+\frac{f^{(n)}(x_0)}{n!}(x-x_0)^n$\\
\textbf{Доказательство}\\
Представим $f$ в виде $f(x)=b_0+b_1(x-x_0)+b_2(x-x_0)^2+\ldots+b_n(x-x_0)^n$\\
Тогда\\
$f'(x) = 1\cdot b_1+2\cdot b_2(x-x_0)+\ldots+n\cdot b_n(x-x_0)^{n-1}$\\
$f''(x) = 2\cdot1\cdot b_2+3\cdot2\cdot b_3(x-x_0)+\ldots+n\cdot (n-1)\cdot b_n(x-x_0)^{n-2}$\\
$\vdots$\\
$f^{(k)}(x) = k!\cdot b_k+\frac{(k+1)!}{2!}\cdot b_{k+1}(x-x_0)+\ldots+\frac{n!}{(n-k)!}\cdot b_n(x-x_0)^{n-k}$\\
Отсюда $f(x_0) = b_0, f'(x_0) = 1!\cdot b_1, \ldots f^{(k)}(x_0) = k!\cdot b_k$, из чего следует формула, ч.т.д.\\
\textbf{Теорема (Формула Тейлора)}\\
$f:\langle a, b \rangle \rightarrow \mathbb{R}, x_0\in \langle a, b \rangle$, $f$ - $m$ раз дифференцируема на $\langle a, b \rangle$\\
Тогда $\forall\,x\in\mathbb{R}\\f(x)=f(x_0)+\frac{f'(x_0)}{1!}(x-x_0)+\frac{f''(x_0)}{2!}(x-x_0)^2+\ldots+\frac{f^{(n)}(x_0)}{n!}(x-x_0)^n + o((x-x_0)^n)$\\
\subsection{Замечательные пределы}
\begin{enumerate}
    \item $\lim_{x\rightarrow 0} \frac{\sin x}{x} = 1$\\
    \textbf{Доказательство}\\
    При $x \in (0, \frac{\pi}{2})$ $\sin x < x < \tg x$\\
    $1 \xleftarrow[x \rightarrow 0]{} \cos x < \frac{\sin x}{x} < 1 \xrightarrow[x \rightarrow 0]{} 1$\\
    \textbf{Следствие}\\
    $\lim_{x\rightarrow 0} \frac{1-\cos x}{x^2} = \frac12$\\
    $\lim_{x\rightarrow 0} \frac{\tg x}{x} = 1$\\
    $\lim_{x\rightarrow 0} \frac{\arcsin x}{x} = 1$\\
    $\lim_{x\rightarrow 0} \frac{\arctg x}{x} = 1$\\
    \textbf{Следствие 2}\\
    $(\sin x_0)' = \lim_{x\rightarrow x_0} \frac{\sin x - \sin x_0}{x-x_0} = \lim_{x\rightarrow x_0} \frac{2\cos \frac{x+x_0}{2}\sin \frac{x-x_0}{2}}{x-x_0} = \cos x_0$
    \item $\lim_{x\rightarrow 0} \frac{e^x - 1}{x} = 1$ - из теоремы о существовании экспоненты\\
    \textbf{Следствие}\\
    $\lim_{x\rightarrow 0} \frac{a^x-1}{x} = \ln a$
    \item $\lim_{x\rightarrow 0} \frac{\ln (1+x)}{x} = 1$\\
    \textbf{Доказательство}\\
    Замена\\
    \textbf{Следствие}\\
    $(\ln x)' = \frac{1}{x}$
    \item $\lim_{x\rightarrow 0} (1+x)^{\frac{1}{x}} = e$\\
    \textbf{Доказательство}\\
    Экспонента от предыдущего предела\\
    \textbf{Следствие}\\
    $\lim_{n\rightarrow \infty} (1+\frac1n)^n = e$
    \item $\lim_{x\rightarrow 0} \frac{(1+x)^\alpha - 1}{x} = \alpha, \alpha \in \bb{R}$\\
    \textbf{Доказательство}\\
    Если $\alpha = 0$: $\lim_{x\rightarrow 0} \frac{(1+x)^\alpha - 1)}{x} = 0$\\
    Иначе: $f:= (1+x)^\alpha - 1$\\
    Заметим, что $\alpha \ln (1+x) = \ln(f+1)$\\
    Тогда $\lim_{x\rightarrow 0} \frac{(1+x)^\alpha - 1}{x} = \lim_{x\rightarrow 0} \frac{f(x)}{x} = \lim_{x\rightarrow 0}\frac{f(x)}{\ln (f(x)+1)} \cdot \frac{\alpha \ln (1+x)}{x} = \alpha$
\end{enumerate}
\section{Дифференциальное счисление}
\subsection{Производная}
\textbf{Определение}\\
Пусть $f: \lan a, b\ran \rightarrow \bb{R}, x_0 \in \lan a, b \ran$\\
Если $\exists\, A \in \bb{R}: f(x) = f(x_0) + A(x-x_0)+o(x-x_0), x\rightarrow x_0$, то $f$ - дифференцируема в $x_0$, $A = f'(x_0)$ - производная в $x_0$\\
$f'(x_0)$ однозначно определено по единственности асимптотического разложения\\
\textbf{Определение 2}\\
Пусть $f: \lan a, b\ran \rightarrow \bb{R}, x_0 \in \lan a, b \ran$\\
Если $\exists\, \lim_{x\rightarrow x_0} \frac{f(x)-f(x_0)}{x-x_0} = A \in \bb{R}$, то $f$ - дифференцируема, $A = f'(x_0)$ - производная\\\\
Определения 1 и 2 равносильны\\
\textbf{Замечание}
\begin{enumerate}
    \item $f'_\pm(x_0) = \lim_{x\rightarrow x_0\pm0} \frac{f(x)-f(x_0)}{x-x_0}$ - односторонняя производная\\
    Если существуют и равны $f'_+(x_0)$ и $f'_-(x_0)$, то $f$ дифференцируема в $x_0, f'(x_0) = f'_\pm(x_0)$
    \item Если $\lim_{x\rightarrow x_0} \frac{f(x)-f(x_0)}{x-x_0} = \infty$, то функция не считается дифференцируемой
    \item $f$ - дифференцируема $\Rightarrow f$ - непрерывна (для $f'(x_0) = \infty$ не действует)
\end{enumerate}
\textbf{Определение}\\
Пусть $f:\langle a, b \rangle \rightarrow \bb{R}, D$ - множество точек, где $f$ дифференцируема\\
$f'(x):D\rightarrow \bb{R}$\\
$f(x) = f(x_0) + f'(x_0)(x-x_0)+o(x-x_0)$\\
Рассмотрим в $\bb{R}^2$ прямую $y=f(x_0)+f'(x_0)(x-x_0)$ - \textit{касательную} к графику функции в $(x_0, f(x_0))$\\
\subsection{Правила дифференцирования}
\textbf{Теорема}\\
Пусть  $f,g: \langle a, b \rangle \rightarrow \bb{R}$\\
$f,g$ дифференцируемы в $x_0$\\
Производные следующих функций существуют и равны ...:
\begin{enumerate}
    \item $(f+g)' = f' + g'$
    \item $\forall\,\alpha \in \bb{R} (\alpha f)' = \alpha f'$
    \item $(fg)' = f'g+fg'$
    \item $g(x_0) \neq 0: \left(\frac fg \right)' = \frac{f'g-fg'}{g^2}$\\
    \textbf{Доказательство}\\
    $\lim_{h\rightarrow 0} \frac{\frac fg (x_0+h) - \frac fg (x_0)}h = \lim_{h \rightarrow 0} \frac{f(x_0+h)g(x_0)-f(x_0)g(x_0+h)}{hg(x_0+h)g(x_0)} = \\\lim_{h \rightarrow 0} \frac{\frac{f(x_0+h)-f(x_0)}hg(x_0) - f(x_0)\frac{g(x_0+h)-g(x_0)}{h}}{g(x_0+h)g(x_0)} = \frac{f'g-fg'}{g^2}$
\end{enumerate}
\textbf{Замечание без доказательств}\\
Возьмем $\left( \frac{x\sin x}{\ln x}\right)'$\\
Выберем какой-то $x$. Все остальные $x$ заменим на константу $x_0$ и выпишем производную такой функции. Сделаем так для всех $x$ и возьмем сумму от результатов. В получившемся выражении заменим $x_0$ обратно на $x$\\
$\left( \frac{x\sin x}{\ln x}\right)' = \frac{\sin x}{\ln x} + \frac{x}{\ln x}\cos x + x\sin\left(-\frac{1}{\ln^2 x} \cdot \frac 1x\right)$\\
В общем виде $\left(\frac{f}{g}\right) = \frac{f'}{g} + f\left(-\frac{g'}{g^2}\right)$\\
*TODO какие ограничения*\\
\textbf{Теорема}\\
Пусть $f: \langle a, b \rangle \rightarrow \langle c, d \rangle$, дифференцируема на $x \in \lan a, b \ran$\\
$g: \lan c, d \ran \rightarrow \bb{R}$, дифференцируема в $y = f(x)$\\
Тогда $g\circ f$ - дифференцируема в $x$ и $(g \circ f)' = g'(f(x))\cdot f'(x)$\\
\textbf{Доказательство}\\
$f(x+h) = f(x)+f'(x)h+h\alpha(h), \alpha(h)$ бесконечно малая при $h \rightarrow 0$\\
$g(y+k) = g(y)+g'(y)k + y\beta(k)$\\
Тогда $g(f(x+h)) = g(f(x)+f'(x)h+h\alpha(h))$. Заметим, что $f(x) = y, f'(x)h+h\alpha(h), \alpha(h)$ подходит под описание $k$\\
$g(f(x+h)) = g(f(x)) + g'(f(x))(f'(x)h+h\alpha(h)) + (f'(x)h+h\alpha(h))\beta(k) = \\
g(f(x)) + g'(f(x))f'(x)h + g'(f(x))h\alpha(h) + (f'(x)h+h\alpha(h))\beta(k)$\\
Заметим, что $g'(f(x))h\alpha(h) + f'(x)h\beta(k)+h\alpha(h)\beta(k) = o(h)$\\
Тогда $g(f(x+h)) = g(f(x))+g'(f(x))f'(x)h + o(h)$, ч.т.д.\\
\textbf{Замечание}\\
Можно считать, что $\alpha(0) = \beta(0) = 0$. Тогда $\alpha, \beta$ непрерывные, а значит мы считаем композицию непрерывных функций. Отсюда производная существует\\
\textbf{Теорема (о дифференцировании обратной функции)}\\
Пусть $f:\lan a, b \ran \rightarrow \bb{R}$, функция непрерывна на $\lan a, b \ran$, строго монотонна, дифференцируема в $x$, $f'(x) \neq 0$\\
Тогда $f^{-1}$ - дифференцируема в $f(x)$ и $(f^{-1})'(f(x)) = \frac{1}{f'(x)}$\\
Т.е. $(f^{-1})' = \frac{1}{f'\circ f^{-1}}$\\
\textbf{Доказательство}\\
Пусть $x = f^{-1}(y), h = f^{-1}(y+k)-f^{-1}(y)$\\
$\frac{f^{-1}(y+k) - f^{-1}(y)}k = \frac{h(k)}{f(x+k)-f(x)} = \frac{1}{\frac{f(x+h(k))-f(x)}{k(k)}} \xrightarrow[k\rightarrow 0]{} \frac{1}{f'(x)}$, ч.т.д.\\
\textbf{Таблица производных}
\begin{itemize}
    \item $(x^\alpha)' = \alpha x^{\alpha-1}$
    \item $(e^x)' = e^x$\\
    $(a^x)' = \ln a\cdot a^x$
    \item $\sin' = \cos$\\
    $\cos' = \sin$\\
    $\tg' = \frac 1{\cos^2} = \tg^2 + 1$
    \item $(\ln x)'= \frac 1x$
    \item $(\arcsin x)' = \frac{1}{\sqrt{1-x^2}}$\\
    $(\arccos x)' = -\frac{1}{\sqrt{1-x^2}}$\\
    $(\arctan x)' = \frac{1}{1+x^2}$
\end{itemize}
\subsection{Теорема о среднем}
\textbf{Лемма(о возрастании в точке)}\\
Пусть $f:\lan a, b\ran \rightarrow \bb{R}$, дифференцируема в $x_0 \in (a,b), f'(x_0) > 0$\\
Тогда $\exists\,\varepsilon > 0:\begin{array}{cc}
     \forall\,x\in (x_0, x_0+\varepsilon)\ f(x) > f(x_0)\\
     \forall\,x\in (x_0-\varepsilon, x_0)\ f(x) < f(x_0)
\end{array}$\\
\textbf{Доказательство}\\
$f'(x_0) = \lim_{h\rightarrow 0} \frac{f(x_0+h)-f(x_0)}{h} > 0$\\
При $h \rightarrow +0 (\Rightarrow h > 0)\ \exists\,\varepsilon\ f(x_0+h) - f(x_0) > 0$ (из предела)\\
Аналогично для $h \rightarrow -0$\\
\textbf{Теорема Ферма}\\
Пусть $f:\lan a, b \ran \rightarrow \bb{R}, x_0 \in (a,b), f(x_0) = \max_{\lan a, b \ran} f, f$ - дифференцируема в $x_0$\\
Тогда $f'(x_0) = 0$ (необходимое условие экстремума)\\
\textbf{Доказательство}\\
Очевидно из леммы\\
\textbf{Теорема Ролля}\\
Пусть $f:[a,b] \rightarrow \bb{R}$, непрерывная на $[a,b]$, дифференцируема на $(a,b)$, $f(a) = f(b)$\\
Тогда $\exists\,c \in (a,b):\ f(c) = 0$\\
\textbf{Доказательство}\\
$c$ - найдется среди точек максимума или минимума\\
По т. Вейерштрасса у этой функции существуют точки максимума или минимума\\
Если максимум и минимум достигаются только в $a$ и $b$, то $f = const$\\
Тогда $c$ - любая точка $(a,b)$\\
Иначе $c$ - любая точка максимума или минимума в $(a,b)$\\
\textbf{Обозначение}\\
$\nm{Ln}(x) = ((1-x^2)^n)^{(n)}$ - \textit{многочлен Лежандра}\\
\textbf{Пример-теорема}\\
Многочлен $\nm{Ln}(x)$ имеет $n$ различных вещественных корней\\
\textbf{Доказательство}\\
Пусть $f,g$ - многочлены\\
Введем понятие:\\
Если $f(x) = (x-a)^k g(x), g(a) \neq 0$\\
Тогда будем говорить, что $a$ - корень кратности $k$\\
Заметим, что $a$ - корень кратности $k-1$ у $f'(x)$:\\
$f(x)' = k(x-a)^{k-1}g(x) + (x-a)^kg'(x) = (x-a)^{k-1}(kg(x)-(x-a)g'(x))$\\
Теперь докажем пример\\
У $(1-x^2)^n$ - корни -1, 1 имеют кратность $n$. Больше у него корней нет, т.к. их не больше $2n$\\
Продифференцируем выражение\\
Тепер -1 и 1 имеют кратность $n-1$. По теореме Ролля в $(-1, 1)$ существует корень. Его кратность будет 1, т.к. всего корней $2n-1$\\
Продифференцируем выражение еще раз\\
Тепер -1 и 1 имеют кратность $n-2$. $c$ перестанет быть корнем. В $(-1, c)$ и $(c, 1)$ будут корни по теореме Ролля. Их кратность будет $1$\\
Тогда после $n-1$ дифференцирований корни -1 и 1 будут иметь кратность 1. Степень многочлена будет $n+1$\\ Аналогично предыдущим случаям будет $n-1$ корней кратности 1\\
После еще одного дифференцирования -1 и 1 перестанут быть корнями. Многочлен будет иметь $n$ корней кратности 1. Т.о. все они различны, ч.т.д.\\
\textbf{Теорема Лагранжа}\\
Пусть $f:[a,b] \rightarrow \bb{R}$\\
Функция непрерывна на отрезке $[a,b]$ и дифференцируема на $(a,b)$\\
Тогда $\exists\,c\in(a,b):\ \frac{f(b)-f(a)}{b-a} = f'(c)$\\
\textbf{Теорема Коши}\\
Пусть $f, g:[a,b] \rightarrow \bb{R}$\\
Функции непрерывны на отрезке $[a,b]$ и дифференцируемы на $(a,b)$\\
$g' \neq 0$ в $(a,b)$\\
Тогда $\exists\,c\in(a,b):\ \frac{f(b)-f(a)}{g(b)-g(a)} = \frac{f'(c)}{g'(c)}$\\
\textbf{Замечание}\\
Если $g(b)=g(a)$, то $g'(x)$ в какой-то момент будет 0 по теореме Ролля. Тогда $g(b) \neq g(a)$\\
\textbf{Доказательство}\\
Пусть $F(x) = f(x)-kg(x)$\\
Подберем $k: F(a) = F(b)$:\\
$k = \frac{f(b)-f(a)}{g(b)-g(a)}$\\
Т.к. $F(a) = F(b)$, то $\exists\,c\in (a,b)\ F'(c) = 0$, т.е. $f'(c) = k g'(c)$, ч.т.д.\\
\textbf{Следствие}\\
$f: \lan a, b \ran \rightarrow \bb{R}$, дифференцируема на $\lan a, b \ran$\\
Пусть $\exists\,M>0:\ \forall\,x\in \lan a,b \ran\ |f'(x)| \leq M$\\
Тогда $\forall\,x,x+h \in \lan a,b\ran\ |f(x+h)-f(x)| \leq M|h|$\\
\textbf{Следствие 2}\\
$f\in C[x_0, x_0+h]$, дифференцируема на $(x_0, x_0+h]$\\
Пусть $\exists\,\lim_{x\rightarrow x_0+0} f'(x) = k\in \overline{\bb{R}}$\\
Тогда $\exists\,f_+'(x_0) = k$\\
\textbf{Доказательство}\\
$f_+'(x_0) = \lim_{t\rightarrow +0} \frac{f(x_0+t)-f(x_0)}{t}$\\
По т. Лагранжа $\exists\,c\in(x_0, x_0+t):\ \frac{f(x_0+t)-f(x_0)}{t} = f'(c)$\\
Тогда $f_+'(x_0) = \lim_{t\rightarrow +0} f'(c) = k$\\
\textbf{Теорема Дарбу}\\
$f:[a,b] \rightarrow \bb{R}$, дифференцируемая на $[a,b]$\\
Тогда $\forall\, C: \min(f'(a), f'(b)) < C < \max(f'(a), f'(b))\ \exists\,c\in (a,b):\ f'(c) = C$\\
\textit{(При этом производная не является непрерывной)}\\
\textbf{Доказательство}\\
Пусть $g(x) = f(x) - Cx$\\
Тогда $g'(a)$ и $g'(b)$ разных знаков\\
Пусть $g'(a) > 0, g'(b) < 0$\\
По т. Вейерштрасса $\exists\,c: g(c) = \max_{[a,b]} g(x), c \neq a,b$ по лемме\\
Тогда $g'(c) = 0$\\
\textbf{Следствие}\\
Если $f$ дифференцируема на $\lan a,b \ran$, то $f'(\lan a, b \ran)$ - промежуток\\
\textbf{Следствие 2}\\
$f'$ не может иметь разрывов первого рода\\
\subsection{Производные высших порядков}
\textbf{Определение}\\
$f:\lan a,b \ran \rightarrow \bb{R}$, дифференцируемая на $\lan a,b \ran$\\
Тогда $f':\lan a, b\ran \rightarrow \bb{R}$\\
Если нашлась $x_0\in\lan a,b\ran: \exists\,(f')'(x_0)$, то говорят, что $(f')'(x_0)$ - это вторая производная в $f_0$\\
Аналогично далее\\
Аналогично для односторонних производных (заранее сужаем область определения до требуемой)\\
Пусть $E$ - промежуток на $\bb{R}$. За $C^n(E)$ будем обозначать множество функций, определенных на $E$, $n$ раз дифференцируемых на $E$, и $f^{(n)}$ - непрерывных на $E$\\
$C^\infty E = \bigcap_{n\in \bb{N}} C^n(E)$\\
\textbf{Замечание}\\
$C^E \supsetneqq C^1(E) \supsetneqq C^2(E) \supsetneqq \ldots \supset C^\infty(E)$\\\\
\textbf{Лемма}\\
Пусть $r: \lan a,b \ran \rightarrow \bb{R}, x_0 \in \lan a, b \ran$, \\
$r$ - $n-1$ раз дифференцируема на $\lan a, b \ran$, \\
$r$ - $n$ раз дифференцируема в $x_0$ и $r(x_0) = r'(x_0) = \ldots = r^{(n)}(x_0) = 0$\\
Тогда $r(x) = o((x-x_0)^n), x \rightarrow x_0$\\
\textbf{Доказательство}\\
Докажем по индукции
\begin{enumerate}
    \item База\\
    $r(x) = r(x_0) + r'(x_0)(x-x_0) + o(x-x_0)$\\
    Тогда $r(x) = o(x-x_0)$
    \item Переход (от n к n+1)\\
    Пусть $R(x)$ дифференцируема $n$ раз на $\lan a, b \ran$, $n+1$ раз - в $x_0$, $R(x_0) = \ldots = R^{(n+1)} (x_0) = 0$
    Тогда $r(x):= R'(x)$ - удовлетворяет предположению индукции\\
    Тогда $r(x) = o((x-x_0)^n)$\\
    Отсюда $\frac{R(x)}{(x-x_0)^{n+1}} = \frac{R(x)-R(x_0)}{x-x_0} \frac{1}{(x-x_0)^n}$\\
    По теореме Лагранжа для некоторой $c \in (\min(x, x_0), \max(x, x_0))$:\\
    $\frac{R(x)-R(x_0)}{x-x_0} \frac{1}{(x-x_0)^n} = \frac{R'(c)}{(x-x_0)^n}$\\
    $|\frac{R(x)}{(x-x_0)^{n+1}}| = \frac{|R'(c)|}{|x-x_0|^n} \leq \frac{|R'(c)|}{|c-x_0|^n} = \frac{|r(c)|}{|c-x_0|^n} \xrightarrow[x\rightarrow x_0]{} 0$
\end{enumerate}
\textbf{Формула Тейлора с остатком в форме Пеано}\\
$f: \lan a, b \ran \rightarrow \bb{R}$, $n-1$ раз дифференцируема на $\lan a, b\ran$, $n$ раз - в $x_0 \in \lan a, b \ran$\\
Тогда $f(x) = \sum_{k=0}^n \frac{f^{(k)}(x_0)}{k!}(x-x_0)^k + \underset{x \rightarrow x_0}o((x-x_0)^n)$\\
\textbf{Доказательство}\\
$r(x):= f(x) - \sum_{k=0}^n \frac{f^{(k)}(x_0)}{k!}(x-x_0)^k$\\
$r(x_0) = 0$\\
$r'(x_0) = f'(x_0) - \sum_{k=0}^{n-1} \frac{f^{(k+1)}(x_0)}{k!}(x-x_0)^k = 0$\\
$r^{(l)}(x_0) = f^{(l)}(x_0) - \sum_{k=l}^n \frac{f^{(k)}(x_0)}{(k-l)!}(x-x_0)^{k-l} = 0, l \leq n$\\
Из леммы $r(x) = \underset{x\rightarrow x_0}{o((x-x_0)^n)}$\\
\textbf{Обозначения}\\
$T_n(f, x_0)(x) = \sum_{k=0}^n \frac{f^{(k)}(x_0)}{k!}(x-x_0)^k$ - многочлен Тейлора $n$-ой степени функции $f$ в точке $x_0$\\
Формула Тейлора: $f(x) = T_n(f, x_0)(x) + R_n$, где $R_n$ - остаток в формуле Тейлора\\
$R_n = o((x-x_0)^n)$\\
\textbf{Теорема}\\
Пусть у нас есть рациональная функция $\frac{P(x)}{Q(x)}, P, Q$ - многочлены, $\deg P < \deg Q$\\
$Q(x) = (x-a_1)^{k_1}\cdot\ldots\cdot (x-a_n)^{k_n}$\\
Тогда существует n серий вещественных коэффициентов: $\alpha_1,\ldots,\alpha_{k_1}; \beta_1, \ldots, \beta_{k_2}; \ldots; \omega_1, \ldots, \omega_{k_n}$ таких, что\\
$\frac{P(x)}{Q(x)} = (\frac{\alpha_1}{x-a_1}+\ldots+\frac{\alpha_{k_1}}{(x-a_1)^{k_1}}) + \ldots + (\frac{\omega_1}{x-a_n}+\ldots+\frac{\omega_{k_n}}{(x-a_n)^{k_n}})$\\
\textbf{Доказательство}\\
Получим серию $\alpha$: $\frac{P}{Q} = \frac{1}{(x-a_1)^{k_1}}\cdot F_1, F_1 = \frac{P(x)}{(x-a_2)^{k_2}\cdot\ldots\cdot (x-a_n)^{k_n}}$\\
Заметим, что $F_1 \in C^{\infty}$\\
Отсюда \\
$\frac{P}{Q} = \frac{1}{(x-a_1)^{k_1}}\cdot (\alpha_{k_1} + \alpha_{k_1-1}(x-a_1)+ \ldots + \alpha_1(x-a_1)^{k_1-1} + \alpha_0(x-a_1)^{k_1} + o((x-a_1)^{k_1})) = \frac{\alpha_{k_1}}{(x-a_1)^{k_1}} + \ldots + \frac{\alpha_1}{(x-a_1)} + \alpha_0 + \frac{o((x-a_1)^{k_1})}{(x-a_1)^{k_1}}$\\
Наблюдение:\\
$\frac{P}{Q} - (\frac{\alpha_{k_1}}{(x-a_1)^{k_1}} + \ldots + \frac{\alpha_1}{(x-a_1)}) \xrightarrow[x\rightarrow a_1]{} \alpha_0$ - т.е. это конечный предел. Значит знаменатель $(x-a_1)$ полностью ушел из знаменателя (иначе бы предел был $\infty$)\\
Рассмотрим $R(x) = \frac{P}{Q} - (\frac{\alpha_{k_1}}{(x-a_1)^{k_1}} + \ldots + \frac{\alpha_1}{(x-a_1)}) - (\frac{\beta_{k_2}}{(x-a_2)^{k_2}} + \ldots + \frac{\beta_1}{(x-a_2)}) - \ldots - (\frac{\omega_{k_n}}{(x-a_n)^{k_n}} + \ldots + \frac{\omega_1}{(x-a_n)})$\\
$R(x)$ - рациональная дробь со знаменателем $Q$\\
По выше описанной логике в $R(x)$ полностью сократятся все знаменатели\\
Т.о. $R(x)$ - многочлен. При $x \rightarrow \infty\ R(x) \rightarrow 0$. Отсюда $R(x) = \const$\\
Т.е. $R(x) \equiv 0$\\
\textbf{Замечание}\\
Для нахождения числителей раскладываем $F_i$ в формулы Тейлора\\
//todo научиться это делать\\
\textbf{Теорема (формула Тейлора с остатком в форме Лагранжа)}\\
Пусть $f \in C^n(\lan a, b \ran)$, существует $f^{(n+1)}$ на $\lan a, b \ran, x_0, x \in \lan a, b \ran$\\
Тогда $\exists\,C \in (x_0, x)$ (или $(x, x_0)$)\\
$f(x) = f(x_0) + \ldots + \frac{f^{(n)}(x_0)}{n!}(x-x_0)^n + \frac{f^{(n+1)}(C)}{(n+1)!}(x-x_0)^{n+1}$\\
\textit{(Заметим, что $C$ зависит от $x$)}\\
\textbf{Доказательство}\\
$\phi(t) := f(x) - \sum_{k=0}^n \frac{f^{(k)}(t)}{k!}(x-t)^k, t \in [x_0, x]$(или наоборот)\\
$\phi(x) = 0$\\
$\phi(x_0) = f(x) - T_n(f, x_0)(x) = R_n(x)$ - остаток в формуле Тейлора\\
$\phi'(t) = -f'(t)-\sum_{k=1}^n(\frac{f^{(k+1)}(t)}{k!}(x-t)^k - \frac{f^{(k)}(t)}{(k-1)!}(x-t)^{k-1}) = -\frac{f^{(n+1)}(t)}{n!}(x-t)^n$\\\\
$\psi(t) = (x-t)^{n+1}$\\
$\psi(x) = 0, \psi(x_0) = (x-x_0)^{n+1}$\\
По теореме Коши:\\
$\frac{R_n(x)}{(x-x_0)^{n+1}} = \frac{\phi(x)-\phi(x_0)}{\psi(x)-\psi(x_0)} = \frac{\phi'(c)}{\psi'(c)} = \frac{-\frac{f^{(n+1)}(c)}{n!}(x-c)^n}{-(n+1)(x-c)^n}, c \in [x_0, x]$\\
Тогда $R_n(x) = \frac{f^{(n+1)}(c)}{(n+1)!}(x-x_0)^{n+1}$\\
\textbf{Замечание}
\begin{enumerate}
    \item Теорема эквивалентна следующему утверждению:\\
    $\exists\,\theta\in (0, 1):\ f(x) = T_n(f, x_0)(x) + \frac{f^{(n+1)}(x_0+\theta(x-x_0))}{(n+1)!}(x-x_0)^{n+1}$
    \item В доказательстве вместо $\psi$ можно взять функцию $(x-t)$\\
    Тогда $R_n(x) = \frac{f^{(n+1)}(x_0+\theta(x-x_0))}{(n+1)!}(x-x_0)^{n+1}(1-\theta)^n$
\end{enumerate}
\textbf{Метод Ньютона}\\
Пусть у нас есть дважды дифференцируемая функция $f(x)$ с неизвестным корнем $\xi$ и точка $x_1$. Сгенерируем последовательность $x_n$, приближающуюся к $\xi$\\
Пусть $x_{n+1}$ - точка пересечения OX и касательной к $f$ в точке $x_n$\\
$x_{n+1} = x_n - \frac{f(x_n)}{f'(x_n)}$\\
\textbf{Оценка}\\
Найдем разность $\xi - x_{n+1} = \xi - x_n + \frac{f(x_n)}{f'(x_n)} = \frac{f(x_n)+f'(x_n)(\xi - x_n)}{f'(x_n)}$\\
$0 = f(\xi) = f(x_n) + \frac{f'(x_n)}{1!}(\xi - x_n) + \frac{f''(c)}{2!}(\xi - x_n)^2$ по формуле Тейлора, $c$ - между $\varepsilon$ и $x_n$\\
Тогда $\xi - x_{n+1} = -\frac12 \frac{f''(c)}{f'(x_n)}(\xi-x_n)^2$\\
Пусть $m := \min_{\lan a, b\ran} |f'(x)|$\\
$M := \max_{\lan a, b \ran} |f''(x)|$\\
$|\xi - x_{n+1}| = \frac12 \frac{f''(c)}{f'(x_n)}|\xi-x_n|^2 \leq \frac{M}{2m}|\xi-x_n|^2 \leq \frac{M}{2m} \frac{M^2}{4m^2} |\xi - x_{n-1}|^4 \leq \ldots \leq |\xi - x_1| ^{2^n} \frac{M^{1+2+4+\ldots+2^{n-1}}}{(2m)^{1+2+4+\ldots+2^{n-1}}} = \frac{2m}{M}|\frac{M}{2m}(\xi - x_1)|^{2^n}$\\
Тогда при хорошем $x_1$ точность будет очень быстро увеличиваться с каждым шагом\\
\textbf{Следствие}\\
Пусть $f \in C^\infty \lan a,b \ran$ и $\exists\,M, A\ \forall\,t \in \lan a,b\ran\ \forall\,n\ f^{(n)}(t) \leq M\cdot A^n$\\
Тогда $\forall\,x, x_0 \in \lan a, b \ran\ T_n(f, x_0)(x-x_0) \xrightarrow[n \rightarrow +\infty]{} f(x)$\\
\textbf{Доказательство}\\
Из предыдущей теоремы $|f(x) - T_n(f, x_0)(x)| = |\frac{f^{(n+1)}(c)}{(n+1)!}(x-x_0)^n| \leq \frac{MA^{n+1}}{(n+1)!}|x-x_0|^{n+1} = MA\frac{|A(x-x_0)|^{n+1}}{(n+1)!} \xrightarrow[n\rightarrow \infty]{} 0$\\
\textbf{Таблица формул Тейлора}\\
При $x_0 = 0$\\
$e^x = 1+x+\frac{x^2}{2!}+\ldots + \frac{x^n}{n!} + o(x^n)$\\
$\sin x = x - \frac{x^3}{3!} + \frac{x^5}{5!} + \ldots + (-1)^n\frac{x^{2n+1}}{(2n+1)!} + o(x^{2n+2})$\\
$\cos x = 1 - \frac{x^2}{2!} + \frac{x^4}{4!} + \ldots + (-1)^n\frac{x^{2n}}{(2n)!} + o(x^{2n+1})$\\
$\ln(1+x) = x - \frac{x^2}{2} + \frac{x^3}{3} + \ldots + (-1)^{n-1}\frac{x^n}{n}+o(x^n)$\\
$(1+x)^\alpha = 1+\begin{pmatrix}\alpha\\1\end{pmatrix}x+\begin{pmatrix}\alpha\\2\end{pmatrix}x^2 + \ldots + \begin{pmatrix}\alpha\\n\end{pmatrix}x^n + o(x^n)$, где\\
$\alpha \in \bb{R}, \begin{pmatrix}\alpha\\n\end{pmatrix} = \frac{\alpha(\alpha-1)\cdot\ldots\cdot(\alpha-n+1)}{n!}$
\subsection{Равномерная непрерывность}
\textbf{Определение}\\
Пусть $f: X \rightarrow Y, X, Y$ - метрические пространства, $f$ - непрерывная на $X$\\
Если $\forall\,\varepsilon > 0\ \exists\,\delta > 0\ \forall\,x,x_0: \rho(x, x_0) < \delta\ \rho(f(x), f(x_0)) < \varepsilon$, то функция - \textit{равномерно непрерывная}\\
\textbf{Теорема Кантора}\\
$f: X \rightarrow Y$ - непрерывная на $X, X$ - компактно, $X, Y$ - метрическое пространство\\
Тогда $f$ - равномерно непрерывна на $X$\\
\textbf{Доказательство}\\
Докажем от противного\\
Докажем, что $\exists\,\varepsilon > 0: \forall\,n \in \bb{N}\ \exists\,x_n, x'_n \in X:\ \rho(x_n, x'_n) < \frac{1}{n}, \rho(f(x_n), f(x'_n)) \geq \varepsilon$\\
Т.к. $X$ компактно, $\exists\,n_k: x_{n_k} \rightarrow a \in X$\\
Отсюда $x'_{n_k} \rightarrow a$\\
Тогда $\rho(f(x_n), f(x'_n)) \rightarrow 0$ - противоречие\\
\textbf{Следствие}\\
$f:[a,b]\rightarrow \bb{R}$ - непрерывна\\
Тогда $f$ - равномерно непрерывная\\
\textbf{Замечание}\\
Если отображение равномерно непрерывное на двух множествах, то оно непрерывно и на их объединении\\
TODO проверить\\
\textbf{Минутка из теории игр}\\
Пусть у нас есть "прямоугольное" поле для игры в Hex, играют два игрока - белый и черный\\
Игроку выделены две противоположные стороны прямоугольника. Требуется, закрашивая клетки, провести путь между клетками\\
Утверждается, что в этой игре не бывает ничьих\\
\textbf{Доказательство}\\
Встанем в нижний угол и будем оттуда вести линию так, чтобы слева от линии были белые клетки, а справа - черные\\
Заметим, что мы можем построить такую линию\\
Заметим, что длина линии конечна\\
Тогда когда-то линия упрется куда-то\\
Линия не может зациклиться\\
Тогда линия всегда упрется в какую-то стенку\\
Тогда вдоль линии будет находиться выигрышный путь для черных или белых\\
\textbf{Теорема Брауэра о неподвижной точке}
\begin{enumerate}
    \item Пусть в $\bb{R}^m$ $B = B(0,1)$ и $f: B \rightarrow B$ - непрерывная\\
    Тогда $\exists\,x\in B: f(x)=x$
    \item Пусть $f:[0,1]^2 \rightarrow [0,1]^2$ - непрерывное\\
    Тогда $\exists\,x\in [0,1]^2: f(x) = x$\\
    \textbf{Доказательство}\\
    Будем задавать точку следующим образом: $x = (x_1, x_2)$\\
    $f(x) = (f_1(x_1, x_2), f_2(x_1, x_2))$\\
    Рассмотрим $\rho(x, y) := \max |x_1 - y_1|, |x_2 - y_2|$ - непрерывная на нашем квадрате (т.к. зажата между 0 и евклидовой метрикой)\\
    Для евклидовой метрики будем использовать $\|x-y\|$\\
    Пусть в квадрате нет неподвижных точек\\
    Тогда рассмотрим функцию $x\mapsto \rho(x, f(x))$. Эта функция положительна, непрерывна\\
    Тогда по т. Вейерштрасса существует минимум $\varepsilon := \max \rho(x, f(x)) > 0$\\
    $f$ по т. Кантора равномерно непрерывна, т.е. для $\varepsilon\ \exists\,\delta < \varepsilon:\ \forall\,x,x_0:\ \|x-x_0\| < \delta\sqrt2\ \|f(x)-f(x_0)\| < \varepsilon$\\
    Возьмем доску для игры в $\nm{Hex}(n, n), n: \frac {\sqrt{2}}{n} < \delta$\\
    Преобразуем ее, взяв центры ее клеток и соединив их. Тогда мы получим прямоугольную сетку с диагоналями. Покраска клеток теперь эквивалентна покраске ее центра\\
    Стороны первого игрока - левая и правая, второго - верхняя и нижняя\\
    Сожмем сетку до размеров $1\times1$. Теперь каждому узлу $(v_1, v_2)$ в сетке соответствует точка $(\frac{v_1}{n}, \frac{v_2}{n})$\\
    Теперь покрасим точки следующим образом: $\nm{color}(v) = \min(i: |f_i(\frac{v_i}{n}) - \frac{v_i}{n}| \geq \varepsilon)$ (хотя бы одна координата подходит по выше описанным причинам)\\
    По предыдущим рассуждениям существует одноцветный путь от нижней грани к верхней или от левой грани к правой\\
    Пронумеруем вершины в этом пути: $v^0, v^1, \ldots, v^N$\\
    Пусть путь имеет цвет 1 (путь - слева направо)\\
    Тогда $v^0_1 = 0$\\
    $f_1(\frac{v^0}{n}) \geq 0$ - по условию\\
    Т.к. цвет - 1, то $f_1(\frac{v^0}{n}) - \frac{v^0_1}{n} \geq \varepsilon$\\
    $v^N_1 = 1$\\
    $f_1(\frac{v^N}{n}) \leq 1$\\
    Т.к. цвет - 1, то $f_1(\frac{v^N}{n}) - \frac{v^N_1}{n} \leq -\varepsilon$\\
    Заметим, что в какой-то момент мы перейдем от $f_1(\frac{v^i}{n}) - \frac{v^i_1}{n} \geq \varepsilon$ к $f_1(\frac{v^i}{n}) - \frac{v^i_1}{n} \leq -\varepsilon$\\
    При переходе к следующему пункту значение $\frac{v_1^i}n$ меняеся на $\frac 1n < \delta$, $f_1(\frac{v^i}{n})$ - менее чем на $\varepsilon$\\
    Отсюда переход на $2\varepsilon$ невозможен, ч.т.д.\\
\end{enumerate}
\subsection{Монотонность и экстремумы}
\textbf{Теорема (критерий монотонности)}\\
$f \in C\lan a, b \ran$, диффереренцируема на $(a,b)$\\
Тогда\\
$f \uparrow$(нестрого) на $\lan a, b \ran \Leftrightarrow f'\geq 0$ на $(a,b)$\\
$f \downarrow$(нестрого) на $ \lan a, b \ran \Leftrightarrow f'\leq 0$ на $(a,b)$\\
\textbf{Доказательство $\Rightarrow$}\\
Из определения производной\\
\textbf{Доказательство $\Leftarrow$}\\
Из т. Лагранжа:\\
Возьмем $x_0 < x_1$\\
$f(x_1) - f(x_0) = f'(c) (x_1 - x_0) \geq 0$ для некоторого $c \in (a,b)$\\
\textbf{Следствие}\\
$f: \lan a, b \ran \rightarrow \bb{R}$\\
Тогда $f = \const \Leftrightarrow f\in C\lan a, b\ran \land f'\equiv 0$ на $(a,b)$\\
\textbf{Доказательство $\Rightarrow$} из определения\\
\textbf{Доказательство $\Leftarrow$}\\
Из теоремы $f$ нестрого возрастает и нестрого убывает\\
Тогда $f = \const$\\
\textbf{Следствие 2}\\
$f \in C\lan a, b \ran$, дифференцируема на $(a,b)$\\
Тогда $f$ строго возрастает $\Leftrightarrow \left\{\begin{array}{l}
     f'\geq 0 \text{ на } (a,b)\\
     f' = 0 \text{(не является тождественным 0 ни на каком интервале)}
\end{array}\right.$\\
\textbf{Доказательство $\Rightarrow$} по теореме и следствию 1\\
\textbf{Доказательство $\Leftarrow$}\\
Она нестрого возрастает\\
Но если есть промежутки, где она константа, то в этих промежутках $f'(x)$ - константа\\
Отсюда она строго возрастает, ч.т.д.\\
\textbf{Следствие 3}\\
Пусть $f, g \in C [a,b\ran$ и дифференцируемы на $(a,b)$\\
$f(a) \leq g(a)$\\
При $x\in(a,b)$ $f'(x) \leq g'(x)$\\
Тогда $f(x) \leq g(x)$\\
\textbf{Доказательство}\\
Рассмотрим $g(x) - f(x)$\\
Она неотрицательна и всегда возрастает\\
\textbf{Определение}\\
Пусть $f:X \rightarrow \bb{R}$\\
Тогда $x_0 \in X$ - \textit{точка локального максимума}, если $\exists\,U(x_0): \forall\,x\in U(x_0)\ f(x) \leq f(x_0)$\\
$x_0 \in X$ - \textit{точка строгого локального максимума}, если $\exists\,U(x_0): \forall\,x\in \overdot{U}(x_0)\ f(x) < f(x_0)$\\
\textit{Локальный экстремум} - локальный максимум или локальный минимум\\
\textbf{Определение}\\
Пусть $f:\lan a, b \ran \rightarrow \bb{R}$\\
Точка $x$ \textit{стационарная}, если $f'(x) = 0$\\\
\textbf{Определение}\\
$f:\lan a,b \ran \rightarrow \bb{R}$\\
Точка $x_0$ - точка строгого возрастания, если $\exists\,U(x_0):$\\
$\forall\,x \in U(x_0), x > x_0\ f(x) > f(x_0)$\\
$\forall\,x \in U(x_0), x < x_0\ f(x) < f(x_0)$\\
\textbf{Теорема о необходимом и достаточном условии экстремума}\\
$f: \lan a, b \ran \rightarrow \bb{R}$\\
$x_0 \in (a,b)$\\
\textit{(интервал!!!)}\\
Тогда
\begin{enumerate}
    \item Если $f$ - дифференцируема в $x_0$, $x_0$ - локальный экстремум. Тогда $f'(x_0) = 0$\\
    \textbf{Доказательство}\\
    По т. Ферма
    \item Пусть $f$ - $n$ раз дифференцируема в окрестности $x_0$\\
    $f'(x_0) = f''(x_0) = \ldots = f^{(n-1)}(x_0) = 0$\\
    $f^{(n)}(x_0) \neq 0$\\
    Если $f^{(n)}(x_0) > 0$:\\
    Если $n$ - четная, то $x_0$ - минимум\\
    Если $n$ - нечетная, то $x_0$ - не экстремум. $x_0$ - точка строгого возрастания\\
    Если $f^{(n)}(x_0) < 0$:\\
    Если $n$ - четная, то $x_0$ - максимум\\
    Если $n$ - нечетная, то $x_0$ - не экстремум. $x_0$ - точка строгого убывания\\
    \textbf{Доказательство}\\
    Распишем формулу Тейлора\\
    $f(x) = f(x_0) + 0 + \ldots + \frac{f^{(n)}}{n!}(x-x_0)^n + o((x-x_0)^n)$\\
    Тогда в некоторой окрестности $x_0$ знак $\frac{f^{(n)}}{n!}(x-x_0)^n$ совпадает со знаком $\frac{f^{(n)}}{n!}(x-x_0)^n + o((x-x_0)^n)$\\
    Отсюда свойства очевидны
\end{enumerate}
\end{document}