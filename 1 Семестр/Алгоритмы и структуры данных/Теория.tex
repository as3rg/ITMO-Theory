\documentclass[12pt]{article}
\usepackage{bbold}
\usepackage{amsfonts}
\usepackage{amsmath}
\usepackage{amssymb}
\usepackage{color}
\setlength{\columnseprule}{1pt}
\usepackage[utf8]{inputenc}
\usepackage[T2A]{fontenc}
\usepackage[english, russian]{babel}
\usepackage{graphicx}
\usepackage{hyperref}
\usepackage{mathdots}
\usepackage{xfrac}


\def\columnseprulecolor{\color{black}}

\graphicspath{ {./resources/} }


\usepackage{listings}
\usepackage{xcolor}
\definecolor{codegreen}{rgb}{0,0.6,0}
\definecolor{codegray}{rgb}{0.5,0.5,0.5}
\definecolor{codepurple}{rgb}{0.58,0,0.82}
\definecolor{backcolour}{rgb}{0.95,0.95,0.92}
\lstdefinestyle{mystyle}{
    backgroundcolor=\color{backcolour},   
    commentstyle=\color{codegreen},
    keywordstyle=\color{magenta},
    numberstyle=\tiny\color{codegray},
    stringstyle=\color{codepurple},
    basicstyle=\ttfamily\footnotesize,
    breakatwhitespace=false,         
    breaklines=true,                 
    captionpos=b,                    
    keepspaces=true,                 
    numbers=left,                    
    numbersep=5pt,                  
    showspaces=false,                
    showstringspaces=false,
    showtabs=false,                  
    tabsize=2
}

\lstset{extendedchars=\true}
\lstset{style=mystyle}

\newcommand\0{\mathbb{0}}
\newcommand{\eps}{\varepsilon}
\newcommand\overdot{\overset{\bullet}}
\DeclareMathOperator{\sign}{sign}
\DeclareMathOperator{\re}{Re}
\DeclareMathOperator{\im}{Im}
\DeclareMathOperator{\Arg}{Arg}
\DeclareMathOperator{\const}{const}
\DeclareMathOperator{\rg}{rg}
\DeclareMathOperator{\Span}{span}
\DeclareMathOperator{\alt}{alt}
\DeclareMathOperator{\Sim}{sim}
\DeclareMathOperator{\inv}{inv}
\DeclareMathOperator{\dist}{dist}
\newcommand\1{\mathbb{1}}
\newcommand\ul{\underline}
\renewcommand{\bf}{\textbf}
\renewcommand{\it}{\textit}
\newcommand\vect{\overrightarrow}
\newcommand{\nm}{\operatorname}
\DeclareMathOperator{\df}{d}
\DeclareMathOperator{\tr}{tr}
\newcommand{\bb}{\mathbb}
\newcommand{\lan}{\langle}
\newcommand{\ran}{\rangle}
\newcommand{\an}[2]{\lan #1, #2 \ran}
\newcommand{\fall}{\forall\,}
\newcommand{\ex}{\exists\,}
\newcommand{\lto}{\leftarrow}
\newcommand{\xlto}{\xleftarrow}
\newcommand{\rto}{\rightarrow}
\newcommand{\xrto}{\xrightarrow}
\newcommand{\uto}{\uparrow}
\newcommand{\dto}{\downarrow}
\newcommand{\lrto}{\leftrightarrow}
\newcommand{\llto}{\leftleftarrows}
\newcommand{\rrto}{\rightrightarrows}
\newcommand{\Lto}{\Leftarrow}
\newcommand{\Rto}{\Rightarrow}
\newcommand{\Uto}{\Uparrow}
\newcommand{\Dto}{\Downarrow}
\newcommand{\LRto}{\Leftrightarrow}
\newcommand{\Rset}{\bb{R}}
\newcommand{\Rex}{\overline{\bb{R}}}
\newcommand{\Cset}{\bb{C}}
\newcommand{\Nset}{\bb{N}}
\newcommand{\Qset}{\bb{Q}}
\newcommand{\Zset}{\bb{Z}}
\newcommand{\Bset}{\bb{B}}
\renewcommand{\ker}{\nm{Ker}}
\renewcommand{\span}{\nm{span}}
\newcommand{\Def}{\nm{def}}
\newcommand{\mc}{\mathcal}
\newcommand{\mcA}{\mc{A}}
\newcommand{\mcB}{\mc{B}}
\newcommand{\mcC}{\mc{C}}
\newcommand{\mcD}{\mc{D}}
\newcommand{\mcJ}{\mc{J}}
\newcommand{\mcT}{\mc{T}}
\newcommand{\us}{\underset}
\newcommand{\os}{\overset}
\newcommand{\ol}{\overline}
\newcommand{\ot}{\widetilde}
\newcommand{\vl}{\Biggr|}
\newcommand{\ub}[2]{\underbrace{#2}_{#1}}

\def\letus{%
    \mathord{\setbox0=\hbox{$\exists$}%
             \hbox{\kern 0.125\wd0%
                   \vbox to \ht0{%
                      \hrule width 0.75\wd0%
                      \vfill%
                      \hrule width 0.75\wd0}%
                   \vrule height \ht0%
                   \kern 0.125\wd0}%
           }%
}
\DeclareMathOperator*\dlim{\underline{lim}}
\DeclareMathOperator*\ulim{\overline{lim}}

\everymath{\displaystyle}

% Grath
\usepackage{tikz}
\usetikzlibrary{positioning}
\usetikzlibrary{decorations.pathmorphing}
\tikzset{snake/.style={decorate, decoration=snake}}
\tikzset{node/.style={circle, draw=black!60, fill=white!5, very thick, minimum size=7mm}}

\title{Алгоритмы и структуры данных. Теория}
\author{Александр Сергеев}
\date{}

\begin{document}
\maketitle
\section{Общие слова и Теоремы}
\textit{RAM-модель} - модель компьютера, в которой обращение к памяти происходит за $O(1)$\\\\
Асимптотика:\\
$f(n) = O(g(n))\ \Leftrightarrow\ \exists\,n_0 \in \mathbb{N}, C > 0:\ \forall\,n\,>\,n_0\ f(n) \leq g(n)\cdot C$\\
$f(n) = o(g(n))\ \Leftrightarrow\ \forall\,C>0\ \exists\,n_o\in\mathbb{N}:\forall\,n>n_0\ f(n)<g(n)\cdot C$\\
$f(n) = \Omega(g(n))\ \Leftrightarrow\ \exists\,n_0 \in \mathbb{N}, C > 0:\ \forall\,n\,>\,n_0\ f(n) \geq g(n)\cdot C$\\
$f(n) = \omega(g(n))\ \Leftrightarrow\ \forall\,C>0\ \exists\,n_o\in\mathbb{N}:\forall\,n>n_0\ f(n)>g(n)\cdot C$\\
$f(n) = \Theta(g(n))\ \Leftrightarrow\ f(n) = O(g(n)) \land f(n) = \Omega(g(n))$\\\\
\textit{Мастер-теорема}\\
Пусть есть функция, работающая от $T(n)$, $T(n)=a\cdot T(\frac{n}{b})+n^C$\\
Тогда $T(n)=
\left\{
\begin{array}{ll}
     & O(n^C), C > log_b\,a \\
     & O(n^Clog\,n), C = log_b\,a \\
     & O(n^{log_b\,a}), C < log_b\,a
\end{array}
\right.$\\\\\\
Время работы рандомизированного алгоритма - $\max\limits_{input} E(T(input))$, где $E$ - математическое ожидание. $T$ - время для данного $input$.\\\\
\textbf{Теорема о нижней оценке для сортировки сравнениями}\\
Любая сортировка сравнениями работает не быстрее, чем $O(n\log n)$.\\
\textbf{Доказательство}\\
Сопоставим сортировке сравнениями дерево, где вершиной будет являться операция сравнения, ребром - результат сравнения, а листом - конечная перестановка элементов\\
Рассмотрим все перестановки длины $n$. Всего их будет $n!$. Тогда в нашем дереве будет $n!$ листьев. Тогда глубина дерева $\log_2{n!} = \log_2 1 + \log_2 2 + \ldots + \log_2 n \geq \frac n2 \log_2 n$\\
Отсюда глубина дерева $\Omega(n\log n)$, т.е. необходимо совершить $\Omega(n\log n)$ сравнений, ч.т.д.\\\\
\textbf{Теорема}\\
Для проверки корректности работы сортировки достаточно проверить ее на всех возможных массивах из 0 и 1
\section{Структуры}
\subsection{Двоичная куча}
\begin{lstlisting}[language=Python]
    #Куча хранится в массиве arr, где для вершины v потомками являются вершины 2*v+1 и 2*v+2
    
    #Проверяет, что для вершины v и ее потомков выполнено условие кучи 
    #Иначе опускает значение v вниз по дереву, пока условие не выполнится
    #O(log n)
    def sift_down(v):
        while True:
            m = v
            if 2*v+1 < len(arr) and arr[2*v+1] < arr[v]:
                m = 2*v+1
            if 2*v+2 < len(arr) and arr[2*v+2] < arr[v] and arr[2*v+2] < arr[2*v+1]:
                m = 2*v+2
            if m == v: break
            swap(arr[v], arr[m])
            v = m
    

    #Проверяет, что для вершины v и ее родителя выполнено условие кучи 
    #Иначе поднимает значение v вверх по дереву, пока условие не выполнится
    #O(log n)
    def sift_up(v):
        while v != 0 and arr[(v-1)//2] > arr[v]:
            swap(arr[v], arr[(v-1)//2])
            v = (v-1)//2
    
    #Возвращает минимум из кучи и удаляет его
    #O(log n)
    def extract_min():
        res = arr[0]
        swap(arr[0], arr[-1])
        del arr[-1]
        sift_down(0)
        return res
        
    #Помещает v в кучу
    #O(log n)
    def insert(v):
        arr.append(v)
        sift_up(len(arr)-1)
\end{lstlisting}
Методы построения кучи:
\begin{enumerate}
    \item Последовательное добавление элементов в кучу через insert\\
    $O(n\log n)$
    \item Построение на том же массиве с начала\\
    $O(n\log n)$
\begin{lstlisting}[language=Python]
    for i in 1...n-1:
        sift_up(i)
\end{lstlisting}
    \item Построение на том же массиве с конца\\
    $O(n)$
\begin{lstlisting}[language=Python]
    for i in n-2...0:
        sift_down(i)
\end{lstlisting}
Доказательство асимптотики:
Рассмотрим каждый уровень нашего дерева. Для уровня $i$ операция sift\_down выполняется за $\log n-i$. Всего на уровне не более $2^i$ элементов. Отсюда количество операций $\sum\limits_{i=1}^{\log n} 2^i(\log n - i) = \sum\limits_{i=1}^{\log n} 2^{\log n - i}i = \sum\limits_{i=1}^{\log n} \frac{ni}{2^i} < n\sum\limits_{i=1}^\infty \frac{i}{2^i} = n\left(\begin{array}{ccccc}
    \frac12 & + & & & \\
    \frac14 & \frac14 & + & & \\
    \frac18 & \frac18 & \frac18 & + & \\
    \frac1{16} & \frac1{16} & \frac1{16} & \frac1{16} & + \\
    \ldots & \ldots & \ldots & \ldots & \ldots
\end{array}\right) = n\sum\limits_{i=1}^\infty (\frac{2}{2^i} \sum\limits_{i=1}^\infty \frac1{2_i}) = n\sum\limits_{i=1}^\infty \frac{2}{2^i}\cdot 1 = 2n\cdot1$, ч.т.д.

\end{enumerate}
\subsection{Персистентность}
Это способ хранения структуры, при котором мы можем получить любое состояние структуры во времени и создать на его основе новое (возможно множественное ветвление)
Самый простой способ - хранить копии структуры после каждого изменения\\
Также можно хранить ссылки на предыдущие состояния:\\
Пример для стека на односвязном списке: мы храним все состояния переменной-ссылки top во времени. При pop в новое состояние мы записываем ссылку на top->prev. push - добавляем вершину, ссылающуюся на top и в новое состояние записываем ссылку на этот элемент\\
Частично-персистентная структура - структура, хранящая все свои состояния, где все состояния кроме последнего - readonly\\
Для частично-персистентного двусвязного списка: в каждой вершине храним две пары (prev, next) и t. Если текущая версия меньше t, ходим по первой паре, иначе - по второй. Изначально вторая права не используется, t = $\infty$. Изменения происходят в момент модификации. Если мы хотим записать во вторую пару значения в момент, когда там уже что-то лежит, то мы копируем этот элемент в новую версию.
\section{Алгоритмы}
\subsection{Heap sort}
Суть алгоритма: последовательно вытаскиваем минимум из кучи, получая отсортированный массив\\
Время $O(n\log n)$\\
Дополнительная память $O(1)$
\subsection{Алгоритм Карацубы}
Рассмотрим многочлены\\
$P = a_nx^n+a_{n-1}x^{n-1}+\ldots+a_1x^1+a_0$\\
$Q = b_nx^n+b_{n-1}x^{n-1}+\ldots+b_1x^1+b_0$\\\\
Пусть\\
$P_0 = a_nx^{n-m}+a_{n-1}x^{n-m-1}+\ldots+a_m$\\
$P_1 = a_{m-1}x^{m-1}+a_{m-2}x^{m-2}+\ldots+a_1x^1+a_0$\\
Отсюда $P = P_0 x^m+P_1$\\\\
$Q_0 = b_nx^{n-m}+b_{n-1}x^{n-m-1}+\ldots+b_m$\\
$Q_1 = b_{m-1}x^{m-1}+b_{m-2}x^{m-2}+\ldots+b_1x^1+b_0$\\
Отсюда $Q = Q_0 x^m+Q_1$\\\\
Тогда\\
$P\cdot Q = (P_0 x^m+P_1)(Q_0\cdot x^m+Q_1) = P_0Q_0 x^{2m}+(P_1Q_0+P_0Q_1) x^m+P_1Q_1$\\
Заметим, что $P_1Q_0+P_0Q_1 = (P_0+P_1)(Q_0+Q_1) - P_0P_1 - Q_0Q_1$\\
Отсюда $PQ = P_0Q_0 x^{2m}+((P_0+P_1)(Q_0+Q_1) - P_0P_1 - Q_0Q_1) x^m+P_1Q_1$\\\\
Т.о. нам для подсчета $PQ$ нам достаточно посчитать $P_0P_1$, $Q_0Q_1$ и $(P_0+P_1)(Q_0+Q_1)$\\
Тогда при $m=\frac n2$: $T(n)=3T(\frac n2)$\\
Отсюда получаем время $O(n^{\log_2 3})$
\subsection{Quick sort}
\begin{lstlisting}[language=Python]
#sort [l,r)
#Перебрасываем все элементы меньше x налево
#Замечание: Нет поддержки одинаковых чисел
def sort(l,r):
    if r-l <= 1:
        return
    x = arr[(l+r)//2]
    m = l
    for i = l...r-1
        if arr[i] <= x:
            swap(arr[i],arr[m])
            m+=1
    sort(l,m)
    sort(m,r)
\end{lstlisting}
\begin{lstlisting}[language=Python]
#Рандомизированная версия
#Замечание: Нет поддержки одинаковых чисел
def sort(l,r):
    if r-l <= 1:
        return
    x = arr[randint(l,r)]
    m = l
    for i = l...r-1
        if arr[i] <= x:
            swap(arr[i],arr[m])
            m+=1
    sort(l,m)
    sort(m,r)
\end{lstlisting}
Представим, что с шансом $\frac13$ мы разбиваем массив на куски $\frac13$ и $\frac23$ и с вероятностью $\frac23$ длина не меняется.
Тогда мат.ожидание количества вызовов для получения "удачного" $1p+2(1-p)p+3(1-p)^2p+\ldots=\sum\limits_{i=1}^\infty i(1-p)^{i-1}p = 3$, где $p=\frac13$. Тогда мат.ожидание высоты дерева вызовов $3\log_{1.5} n$, а средняя асимптотика $O(n\log n)$.\\
В отличие от merge-sort, здесь $O(1)$ дополнительной памяти.
\begin{lstlisting}[language=Python]
#Замечание: Есть поддержка одинаковых чисел
def sort(l,r):
    if r-l <= 1:
        return
    x = arr[randint(l,r)]
    m1 = m2 = l
    for i = l...r-1
        if arr[i] > x: pass
        elif arr[i] == x:
            swap(arr[m2],arr[i]
            m2+=1
        else:
            swap(arr[i],arr[m2])
            swap(arr[m2],arr[m1])
            m1+=1
            m2+=1
    sort(l,m1)
    sort(m2,r)
\end{lstlisting}
\subsection{k-я порядковая статистика}
\textit{Цель} - вывести k-ый элемент отсортированного массива, не сортируя массив.
\begin{lstlisting}[language=Python]
#Замечание: Нет поддержки одинаковых чисел
def find(l,r,k):
    if r-l <= 1:
        return arr[l]
    x = arr[randint(l,r)]
    m = l
    for i = l...r-1
        if arr[i] <= x:
            swap(arr[i],arr[m])
            m+=1
    if k < m-l:
        return find(l,m,k)
    else:
        return find(m,r,k-(m-l))
\end{lstlisting}
Асимптотика $3(n+\frac23n+(\frac23)^2n+(\frac23)^3n+\ldots) = 3n\sum\limits_{i=1}^\infty (\frac23)^i = O(n)$
\begin{lstlisting}[language=Python]
#Замечание: Есть поддержка одинаковых чисел
def find(l,r,k):
    if r-l <= 1:
        return arr[l]
    x = arr[randint(l,r)]
    m1 = m2 = l
    for i = l...r-1
        if arr[i] > x: pass
        elif arr[i] == x:
            swap(arr[m2],arr[i]
            m2+=1
        else:
            swap(arr[i],arr[m2])
            swap(arr[m2],arr[m1])
            m1+=1
            m2+=1
    if k < m1-l:
        return find(l,m1,k)
    elif k == m1-l:
        return arr[k]
    else:
        return find(m2,r,k-(m2-l))
\end{lstlisting}
\subsection{k-я порядковая статистика за линейное время}
\begin{enumerate}
    \item Массив делится на группы по 5 элементов
    \item В каждой группе берется медиана
    \item Из медиан берется медиана $x$
    \item Массив рассекается по медиане $x$ на две части.
    \item Функция вызывается от одной из частей массива.
\end{enumerate}
Заметим, что каждая часть массива содержит не более $(\frac12+\frac25\cdot\frac12)n=\frac7{10}n$ элементов.\\
Тогда пусть $T(n)$ - время работы для массива длины $n$.\\
Разбиение на группы и поиск их медиан происходит за $n$\\
Нахождение медианы медиан происходит за $T(\frac{n}5)$\\
Вызов от одной части массива происходит за $T(\frac{7n}{10})$\\
$T(n) = n+T(\frac{n}5)+T(\frac{7n}{10})$.
Можно доказать по индукции, что $T(n) = O(n)$.
\subsection{Сортировка подсчетом}
\begin{enumerate}
    \item Пусть мы пытаемся отсортировать массив целых чисел $a$ длины $n$, где $m=\max\limits_{0\leq i\leq n} a[i]$\\
    Заведем массив $c$ размера $m+1$, где $c[i]$ - количество элементов, равных $i$\\
    Посчитаем $c[i]$, пройдясь по массиву $a$\\
    Теперь пройдемся по массиву $c$ слева направо и выпишем в результирующий массив элемент $i$ ровно $c[i]$ раз. В результате мы получим отсортированный массив\\
    Время работы алгоритма - $O(n)$
    \item Пусть мы хотим отсортировать массив $a$ по целому ключу $i \in [0, m]$\\
    Заведем массив $c$ размера $m+1$, где $c[i]$ - количество элементов с ключем $i$\\
    Посчитаем $c[i]$, пройдясь по массиву $a$\\
    Заведем результирущий массив $b$ и в нем выделим $c[i]$ элементов под элемент $i$. Для этого можно хранить массив индексов $p$ размера $m+1$, где $p[i]$ - первая свободная ячейка под элемент с ключем $i$\\
    По умолчанию $p[0] = 0, p[i] = p[i-1]+c[i-1]$\\
    Пройдемся по $a$ и будем записывать элемент с ключем $i$ в ячейку $b[p[i]]$ и увеличивать $p[i]$ на 1\\
    Время работы алгоритма - $O(n)$
\end{enumerate}
\subsection{Цифровая сортировка}
Пусть $0 \leq a_i < n^k$\\
Представим $a_i=d_{i0}\cdot n^{k-1} + d_{i1}\cdot n^{k-2} + \ldots + d_{i\,k-1}$.
\begin{enumerate}
    \item $a$ сортируем по $d_{k-1}$
    \item $a$ сортируем оп $d_i$ стабильно для $i \in k-2\ldots0$
\end{enumerate}
\subsection{Бинарный поиск}
\begin{lstlisting}[language=Python]
#Будем считать, что arr[-1] = -inf, arr[n] = +inf
#Поиск наименьшего числа не меньше данного
def binSearch(x):
    l = -1
    r = n
    while r-l > 1:
        m = (r+l)//2
        if arr[m] < x:
            l = m
        else:
            r = m
    #l - наибольшее <
    #r - наименьшее >=
    return r
    
#Будем считать, что arr[-1] = -inf, arr[n] = +inf
#Поиск наибольшего числа не больше данного
def binSearch(x):
    l = -1
    r = n
    while r-l > 1:
        m = (r+l)//2
        if arr[m] <= x:
            l = m
        else:
            r = m
    #l - наибольшее <=
    #r - наименьшее >
    return l
\end{lstlisting}
\subsection{Тернарный поиск}
Пусть есть функция, которая сначала возрастает, а потом убывает.
\begin{lstlisting}[language=Python]
#поиск максимума
def search(x):
    while r-l > e:
        m2 = l+(r-l)/3*2
        if f(m1) < f(m2):
            l = m1
        else:
            r = m2
    return r
\end{lstlisting}
Соптимизируем вычисление функции
\begin{lstlisting}[language=Python]
#поиск максимума
def search(x):
    while r-l > e:
        m2 = l+(r-l)/3*2
        if f(m1) < f(m2):
            l = m1
            m1 = m2
            m2 = r-p(r-l) #p - коэффициент золотого сечения
        else:
            r = m2
            m2 = m1
            m1 = l+p(r-l)
    return r
\end{lstlisting}
Также возможен поиск через бинарный поиск $x: f'(x)=0$
\end{document}