\documentclass[12pt]{article}
\usepackage{bbold}
\usepackage{amsfonts}
\usepackage{amsmath}
\usepackage{amssymb}
\usepackage{color}
\setlength{\columnseprule}{1pt}
\usepackage[utf8]{inputenc}
\usepackage[T2A]{fontenc}
\usepackage[english, russian]{babel}
\usepackage{graphicx}
\usepackage{hyperref}
\usepackage{mathdots}
\usepackage{xfrac}


\def\columnseprulecolor{\color{black}}

\graphicspath{ {./resources/} }


\usepackage{listings}
\usepackage{xcolor}
\definecolor{codegreen}{rgb}{0,0.6,0}
\definecolor{codegray}{rgb}{0.5,0.5,0.5}
\definecolor{codepurple}{rgb}{0.58,0,0.82}
\definecolor{backcolour}{rgb}{0.95,0.95,0.92}
\lstdefinestyle{mystyle}{
    backgroundcolor=\color{backcolour},   
    commentstyle=\color{codegreen},
    keywordstyle=\color{magenta},
    numberstyle=\tiny\color{codegray},
    stringstyle=\color{codepurple},
    basicstyle=\ttfamily\footnotesize,
    breakatwhitespace=false,         
    breaklines=true,                 
    captionpos=b,                    
    keepspaces=true,                 
    numbers=left,                    
    numbersep=5pt,                  
    showspaces=false,                
    showstringspaces=false,
    showtabs=false,                  
    tabsize=2
}

\lstset{extendedchars=\true}
\lstset{style=mystyle}

\newcommand\0{\mathbb{0}}
\newcommand{\eps}{\varepsilon}
\newcommand\overdot{\overset{\bullet}}
\DeclareMathOperator{\sign}{sign}
\DeclareMathOperator{\re}{Re}
\DeclareMathOperator{\im}{Im}
\DeclareMathOperator{\Arg}{Arg}
\DeclareMathOperator{\const}{const}
\DeclareMathOperator{\rg}{rg}
\DeclareMathOperator{\Span}{span}
\DeclareMathOperator{\alt}{alt}
\DeclareMathOperator{\Sim}{sim}
\DeclareMathOperator{\inv}{inv}
\DeclareMathOperator{\dist}{dist}
\newcommand\1{\mathbb{1}}
\newcommand\ul{\underline}
\renewcommand{\bf}{\textbf}
\renewcommand{\it}{\textit}
\newcommand\vect{\overrightarrow}
\newcommand{\nm}{\operatorname}
\DeclareMathOperator{\df}{d}
\DeclareMathOperator{\tr}{tr}
\newcommand{\bb}{\mathbb}
\newcommand{\lan}{\langle}
\newcommand{\ran}{\rangle}
\newcommand{\an}[2]{\lan #1, #2 \ran}
\newcommand{\fall}{\forall\,}
\newcommand{\ex}{\exists\,}
\newcommand{\lto}{\leftarrow}
\newcommand{\xlto}{\xleftarrow}
\newcommand{\rto}{\rightarrow}
\newcommand{\xrto}{\xrightarrow}
\newcommand{\uto}{\uparrow}
\newcommand{\dto}{\downarrow}
\newcommand{\lrto}{\leftrightarrow}
\newcommand{\llto}{\leftleftarrows}
\newcommand{\rrto}{\rightrightarrows}
\newcommand{\Lto}{\Leftarrow}
\newcommand{\Rto}{\Rightarrow}
\newcommand{\Uto}{\Uparrow}
\newcommand{\Dto}{\Downarrow}
\newcommand{\LRto}{\Leftrightarrow}
\newcommand{\Rset}{\bb{R}}
\newcommand{\Rex}{\overline{\bb{R}}}
\newcommand{\Cset}{\bb{C}}
\newcommand{\Nset}{\bb{N}}
\newcommand{\Qset}{\bb{Q}}
\newcommand{\Zset}{\bb{Z}}
\newcommand{\Bset}{\bb{B}}
\renewcommand{\ker}{\nm{Ker}}
\renewcommand{\span}{\nm{span}}
\newcommand{\Def}{\nm{def}}
\newcommand{\mc}{\mathcal}
\newcommand{\mcA}{\mc{A}}
\newcommand{\mcB}{\mc{B}}
\newcommand{\mcC}{\mc{C}}
\newcommand{\mcD}{\mc{D}}
\newcommand{\mcJ}{\mc{J}}
\newcommand{\mcT}{\mc{T}}
\newcommand{\us}{\underset}
\newcommand{\os}{\overset}
\newcommand{\ol}{\overline}
\newcommand{\ot}{\widetilde}
\newcommand{\vl}{\Biggr|}
\newcommand{\ub}[2]{\underbrace{#2}_{#1}}

\def\letus{%
    \mathord{\setbox0=\hbox{$\exists$}%
             \hbox{\kern 0.125\wd0%
                   \vbox to \ht0{%
                      \hrule width 0.75\wd0%
                      \vfill%
                      \hrule width 0.75\wd0}%
                   \vrule height \ht0%
                   \kern 0.125\wd0}%
           }%
}
\DeclareMathOperator*\dlim{\underline{lim}}
\DeclareMathOperator*\ulim{\overline{lim}}

\everymath{\displaystyle}

% Grath
\usepackage{tikz}
\usetikzlibrary{positioning}
\usetikzlibrary{decorations.pathmorphing}
\tikzset{snake/.style={decorate, decoration=snake}}
\tikzset{node/.style={circle, draw=black!60, fill=white!5, very thick, minimum size=7mm}}

\title{Теория вероятности. Теория}
\author{Александр Сергеев}
\date{}
\begin{document}
\maketitle
\section{Вероятностное пространство. Вероятность и ее свойство}
\textbf{Определение}\\
\textit{Алгебра событий}:\\
$\Omega$ -- множество элементарных исходов\\
$\mcA$ -- набор подмножеств $\Omega$\\
$\mcA$ -- алгебра, если
\begin{enumerate}
    \item $\Omega \in \mcA$
    \item $A \in \mcA \Rto \ol A = \Omega \setminus A \in \mcA$
    \item $A, B \in \mcA \Rto A \cup B = A + B \in \mcA$
\end{enumerate}
Элементы алегбры -- \textit{события}\\
\textbf{Операции с событиями}
\begin{enumerate}
    \item $A \cup B = A + B$
    \item $A \cap B = AB = \ol A + \ol B$
    \item $\ol A = \Omega \setminus A$
    \item $A \setminus B = A - B = A\ol B$
\end{enumerate}
\textbf{Определение}\\
$\sigma$-алгебра\\
$\mcA$ -- сигма-алгебра
\begin{enumerate}
    \item $\mcA$ -- алгебра
    \item $A_1, A_2, \ldots \in \mcA \Rto \bigcup A_i \in \mcA$
\end{enumerate}
\textbf{Определение}\\
События $A, B$ -- \textit{несовместные} $AB = \varnothing$\\
Набор несовместный, если события попарно несовместные\\
\textbf{Определение(вероятностное пространство)}\\
$\Omega$ -- множество элементарных исходов\\
$\mcA$ -- сигма-алгебра\\
$P: \mcA \rto \Rset$ -- вероятность, если
\begin{enumerate}
    \item $P(A) \geq 0$
    \item $P(\Omega) = 1$
    \item $P(\bigcup A_i) = \sum P(A_i)$
\end{enumerate} 
\textbf{Определение(вероятностное пространство в широком смысле)}
$\Omega$ -- множество элементарных исходов\\
$\mcA$ -- алгебра\\
$P: \mcA \rto \Rset$ -- вероятность, если
\begin{enumerate}
    \item $P(A) \geq 0$
    \item $P(\Omega) = 1$
    \item $\bigcup^\infty A_i \Rto P(\bigcup^\infty A_i) = \sum P(A_i)$
\end{enumerate}
\textbf{Теорема о продолжении меры}\\
$\lan \Omega, \mcA, P\ran$ -- вероятностное пространство в широком смысле\\
Тогда $\ex! Q: \sigma(\mcA) \rto \Rset$ -- вероятность, $Q\vl_{\mcA} = P$, где $\sigma(\mcA)$ -- сигма-алгебра, содержащая $\mcA$\\
\textbf{Определение}\\
$\mcA$ -- система интервалов на $\Rset$, замкнутая относительно конечного объединения и пересечения\\
$\mcB = \sigma(\mcA)$ -- борелевская сигма-алгебра\\
\textbf{Примеры вероятностных пространств}
\begin{enumerate}
    \item Модель классической вероятности\\
    $\Omega = \{\omega_1, \ldots, \omega_N\}$\\
    $\mcA = 2^\Omega$\\
    $P(\{\omega_i\}) = P(\omega_i) = \frac1N$\\
    $\mcA = \{\omega_{i_1}, \ldots, \omega_{i_M}\} \Rto P(\mcA) = \frac{M}{N}$
    \item $\Omega$ -- набор $\{0^i, 1\}, i \in \Nset_0 = \{0, 1, 2, 3, \ldots\}$\\
    $P(0^i1)=q^ip$
    \item Модель геометрической вероятности
    $\Omega$ -- ограниченное, измеримое по Лебегу множество\\
    $\mcA$ -- измеримое по Лебегу подмножество $\Omega$\\
    $P(A) = \frac{\lambda A}{\lambda \Omega}$
\end{enumerate}
\textbf{Теорема (свойство вероятности)}
\begin{enumerate}
    \item $A \subset B \Rto P(A) \leq P(B)$
    \item $P(A) \leq 1$
    \item $P(A) + P(\ol A) = 1$
    \item $P(A+B) = P(A) + P(B) - P(AB)$
    \item $P(\varnothing) = 0$
    \item $P(\bigcup A_i) \leq \sum P(A_i)$
\end{enumerate}
\textbf{Доказательство}
\begin{enumerate}
    \item $P(B) = P(A) + P(B - A)$
    \item $A \subset \Omega \Rto P(A) \leq 1$
    \item $A \sqcup \ol A = \Omega$
    \item $B = AB \sqcup (B \setminus AB)$
    \item $B_1 = A_1$
    $B_2 = A_2 \setminus A_1$\\
    $B_n = A_n \setminus (A_1 \cup \ldots A_{n-1})$
    $\bigsqcup B_i = \bigcup A_i$\\
    $B_i \subset A_i$\\
    $P(\bigcup A_i) = P(\bigsqcup B_i) = \sum P(B_i) \leq \sum P(A_i)$
\end{enumerate}
\textbf{Теорема (формула включения/исключения)}\\
$P(A_1 + \ldots + A_n) = \sum_i P(A_i) - \sum_{i < j} P(A_iA_j) + \sum_{i<j<k} P(A_iA_jA_k) + \ldots (-1)^{n+1} \sum_{i_1 < \ldots < i_n} P(A_{i_1} \ldots A_{i_n})$
\textbf{Доказательство}\\
Доказательство по индукции\\
\textbf{Теорема}\\
$\mcA$ -- алгебра(?) на $\Omega$, $p$ -- мера\\
Тогда равносильны
\begin{enumerate}
    \item $p$ -- счетно-аддитивно
    \item $p$ -- конечно-аддитивно+$\fall (B_n)_{n=1}^\infty: B_{n+1} \subset B_n, B = \bigcap B_n \Rto P(B_n) \rto P(B)$ -- непрерывность сверху
    \item $p$ -- конечно-аддитивно+$\fall (A_n)_{n=1}^\infty: A_{n+1} \supset A_n, A = \bigcup A_n \Rto P(A_n) \rto P(A)$ -- непрерывность снизу
\end{enumerate}
\textbf{Доказательство (непрерывность сверху) $\LRto$ (непрерывность снизу)}\\
$A(n): A_n \subset A_{n+1}; A = \bigcup A_n$\\
$B_n := \ol A_n, B := \ol A$\\
$B_n \supset B_{n+1}$\\
$B = \ol A = \ol{\bigcup A_n} = \cap \ol A_n = \cap B_n$\\
$p(B_n) = 1 - p(A_n) \rto 1 - p(A) = p(B)$\\
\textbf{Доказательство $1 \rto 2$}\\
$C_1 = B_1\ol B_2$\\
$C_2 = B_2\ol B_3$\\
$C_k = B_k\ol B_{k+1}$\\
$B_k = B \sqcup \bigsqcup_{j=k}^\infty C_j$\\
$p(B_k) = p(B) + \ub{\rto 0}{\sum_{j=k}^\infty p(C_j)}$\\
$p(B_k) \rto p(B)$\\
\textbf{Доказательство $2 \rto 1$}\\
$\sum_{k=1}^\infty p(C_k) = \lim_n \sum_{k=1}^n p(C_k) = \lim_n p(\bigsqcup_{k=1}^n C_k) = p(B)$
\section{Условная вероятность. Формуоа полной вероятности. Теорема Байеса}
\textbf{Определение (условная вероятность)}\\
$(\Omega, \mcA, p)$ -- вероятностное пространство\\
$B \in \mcA: p(B) > 0$\\
$p_B(A) = p(A|B) := \frac{p(AB)}{p(B)}$\\
\textbf{Замечание}\\
$p(AB) = p(A)p(B|A)$\\
\textbf{Теорема (формула произведения вероятностей)}\\
$p(A_1\ldots A_1) = p(A_1)p(A_2|A_1)\ldots p(A_n|A_1\ldots A_{n-1})$\\
\textbf{Доказательство}\\
Тривиально\\
\textbf{Определение}\\
$A_1, \ldots, A_n$ -- независимые, если $\fall A_{i_1},\ldots, A_{i_k}\ p(A_{i_1}\ldots A_{i_k}) = p(A_{i_1})\ldots p(A_{i_k})$\\
\textbf{Теорема (формула полной вероятности)}\\
$A \subset \bigsqcup_{k} B_k$ (как правило, $\bigsqcup B_k = \Omega)$\\
Тогда $p(A) = \sum_k p(A|B_k)p(B_k)$\\
\textbf{Доказательство}\\
$p(A)=p(A\cap \bigsqcup B_k) = p(\bigsqcup AB_k) = \sum_k p(AB_k) = \sum p(A|B_k)p(B_k)$\\
\textbf{Теорема Байеса}\\
Краткая форма:\\
$p(B|A) = \frac{p(A|B)p(B)}{p(A)}$\\
Полная:\\
$A\subset \bigsqcup B_k$\\
$\ub{\text{апостериорные; posterior}}{p(B_k|A)} = \frac{\ub{\text{likelyhood}}{p(A|B_k)}\ub{\text{априорные; prior}}{p(B_k)}}{\sum_j p(A|B_j)}p(B_j)$\\
\textbf{Доказательство краткой формы}\\
$p(B|A) = \frac{p(AB)}{p(A)} = \frac{p(A|B)p(B)}{p(A)}$\\\
\section{Схема Бернулли. Полиноминальная схема. Предельные теоремы, связь со схемой Бернулли}
\textbf{Определение (независимые испытания)}\\
$n \in \Nset$\\
$(\Omega_1, \mcA_1, p_1), \ldots (\Omega_n, \mcA_n, p_n)$ -- вероятностные пространтсва, описывающие виды экспериментов\\
$\Omega^{(n)} = \Omega_1 \times \ldots \times \Omega_n$\\
$\mcA^{(n)} = \mcA_1 \times \ldots \times \mcA_n$\\
$p^{(n)}: \mcA^{(n)} \rto \Rset$\\
$p^{(n)}(A_1\times \ldots \times A_n) = p_1(A_1)\cdot\ldots\cdot p_n(A_n)$\\
$(\Omega^{(n)}, \mcA^{(n)}, p^{(n)})$ -- описание $n$ независимых испытаний\\
\textbf{Замечание}\\
$\mcA^{(n)}$ может не быть сигма-алгеброй\\
$(\Omega^{(n)}, \mcA^{(n)}, p^{(n)})$ -- вероятностное пространство в широком смысле\\
$\sigma(\mcA^{(n)})$ -- сигма-алгебра\\
\textbf{Определение (схемы Бернулли)}\\
$n \in \Nset$\\
$p \in (0, 1)$ -- вероятность успеха\\
$q = 1 - p$ -- вероятность неудачи\\
$\Omega = \{0, 1\}$\\
$\mcA = 2^\Omega$\\
$\Omega^{(n)} = \{0, 1\}^n$\\
$w \in \Omega^{(n)}$ -- вектор из нуля и единиц\\
$p(w) = p^{\sum w_i}q^{n-\sum w_i}$\\
$(\Omega^{(n)}, \mcA^{(n)}, p^{(n)})$ -- схема Бернулли\\\\
\textbf{Теорема}\\
$S_n$ -- количество успехов в $n$ испытаниях\\
$p(S_n = k) = \binom{n}{k} p^k q^{n-k}$\\
\textbf{Теорема (о наиболее вероятном $k_*$)}\\
$p_k := p(S_n = k)$\\
$\frac{p_k}{p_{k+1}} = \frac{(k+1)q}{(n-k)p} \vee 1$\\
$(k+1)(1-p) \vee (n-k)p$\\
$k+1-pk-p \vee np - pk$\\
$k \vee (n+1)p - 1$
\begin{enumerate}
    \item $p(n+1) \in \Nset$\\
    Тогда $\ex k_0: k_0 = p(n+1) - 1$\\
    Если $k < k_0$, то $p_k < p_{k_0 + 1}$\\
    Если $k = k_0$, то $p_{k_0} = p_{k_0 + 1}$\\
    Если $k > k_0$, то $p_{k} > p_{k_0+1}$\\
    Т.о. $k_0, k_0 + 1$ -- наиболее вероятные
    \item $p(n+1) \not\in \Nset$\\
    Тогда $\ex k_1: p_{k_1} < p_{k_1+1}, p_{k_1+1} > p_{k_1+2}$\\
    Т.о. $k_1$ -- наиболее вероятные
\end{enumerate}
Тогда $k_* = \left\{\begin{array}{ll}
    \lceil p(n+1) - 1 \rceil, & p(n+1) \not\in \Nset\\
    p(n+1)-1, p(n+1), & p(n+1) \in \Nset
\end{array}\right.$\\
\textbf{Утверждение}\\
$p(S_n \geq 1) = 1 - q^n$\\
\textbf{Утверждение}\\
$\alpha \in (0, 1)$\\
$p \in (0, 1)$\\
$n: p(S_n \geq 1) \geq \alpha$\\
Ответ: $n = \lceil \log_q (1-\alpha) \rceil$\\
\textbf{Определение (полиноминальная схема)}\\
$n$ -- количество испытаний\\
$m$ -- количество возможных исходов\\
$p = (p_1, \ldots, p_m)$ -- вектор вероятностей исходов\\
$\sum p = 1$\\
$\Omega = \{ (i_1, \ldots, i_m)^T \in \{0,1\}^m: \sum_j i_j = 1\}$ -- множество столбцов с одной единицей\\
$\Omega^{(n)} = \{ A \in M_{m\times n}: A_{ij} \in \{0, 1\}, \sum_i A_{ij} = 1 \}$\\
$p(A) = p_1^{\sum_j A_{1,j}}\cdot \ldots \cdot p_m^{\sum_j A_{m,j}}$\\
\textbf{Теорема}\\
$S_{n, j}$ -- количество исходов типа $j$ в $n$ испытаниях\\
$p(S_{n,1}=k_1, \ldots, S_{n, m} = k_m) = \frac{n!}{k_1!\ldots k_m!} p_1^{k_1} \ldots p_m^{k_m}, \sum k_j = n$\\
\textbf{Теорема (закон больших чисел в форме Бернулли)}\\
$\fall \eps\ p(|\frac{S_n}{n} - p| > \eps) \xrto[n]{} 0$\\
\textbf{Теорема (теорема Пуассона)}\\
$p_n = \frac\lambda n + o(\frac1n), n \rto \infty$ -- схема серий из $n$ испытаний (вероятность успеха при $n$ испытаниях)\\
Тогда $p(S_n = k) \xrto[n\rto \infty]{} e^{-\lambda} \frac{\lambda^k}{k!}$\\
\textbf{Доказательство}\\
$p(S_n = k) = \frac{n!}{k!(n-k)!}(\frac{\lambda}{n} + o(\frac1n))^k(1-\frac\lambda n + o(\frac1n))^{n-k} =\\= \frac{1}{k!} \frac{\prod_{j=0}^{k-1}(n-j)}{n^k} (\lambda+o(\frac1n))^k \frac{(1-\frac\lambda n +o(\frac1n))^n}{(1-\frac\lambda n + o(\frac1n))^k} = \frac{\lambda^k}{k!}e^{-\lambda}$\\
\textbf{Замечание (о погрешности)}\\
$\lambda = np$\\
Тогда $|p(S_n = k) - e^{-\lambda}\frac{\lambda^k}{k!}| \leq np^2 = \frac{\lambda^2}{n}$\\
\textbf{Лемма 1}\\
$k \rto \infty, (n \geq k \Rto n \rto \infty)$\\
$p_* = \frac kn$\\
$H(x) = x \ln \frac xp + (1-x)\ln \frac{1-x}{1-p}$ -- энтропия\\
$n - k \rto \infty \Rto p(S_n = k) \sim \frac{1}{\sqrt{2\pi n p_* (1-p_*)}}\exp(-n H(p_*))$\\
\textbf{Доказательство}\\
$p(S_n = k) = \frac{n!}{k!(n-k)!}p^k (1-p)^{n-k} \sim \frac{\sqrt{2\pi n}n^n e^{-n} p^k (1-p)^{n-k}}{\sqrt{2\pi k} k^k e^{-k}\sqrt{2\pi (n-k)} (n-k)^{n-k} e^{-(n-k)}} = \frac{n^k n^{n-k}(1-p)^{n-k}p^k}{\sqrt{2\pi p_* n(1-p_*)} k^k (n-k)^{n-k}} = \frac{1}{\sqrt{\ldots}} \exp{\ln{\frac{n^kp^k}{k^k} \cdot \frac{n^{n-k}(1-p)^{n-k}}{(n-k)^{n-k}}}} \\= \frac1{\sqrt{\ldots}} \exp{k \ln \frac{p}{p_*} + (n-k)\ln \frac{n(1-p)}{n-k}} \\= \frac1{\sqrt{\ldots}}\exp(-n \ub{H(p_*)}{(p_* \ln \frac{p}{p_*} + (1-p_*)\ln \frac{1-p_*}{1-p})})$\\
\textbf{Лемма 2}\\
$H(p_*) = \frac1{2p(1-p)}(p-p_*)^2 + O((p-p_*)^3)$\\
При $p_* \rto p$\\
\textbf{Доказательство}\\
$H(p) = 0$\\
$H'(x) = \ln \frac xp - 1 - \ln \frac{1-x}{1-p} + 1; H'(p) = 0$\\
$H''(x) = \frac1x + \frac1{1-x} = \frac1{x(1-x)}; H''(p) = \frac1{p(1-p)}$\\
\textbf{Теорема (локальная предельная Муавра-Лапласа)}\\
Требуем Лемму 1 + ($k-np = o(n^{\frac23})$)\\
Тогда $p(S_n = k) \sim \frac1{\sqrt{2\pi np(1-p)}} \exp(-\frac{(k-np)^2}{2np(1-p)})$\\
\textbf{Доказательство}\\
$p_* - p = o(n^{-\frac13})$ -- из $k-np = o(n^{\frac23})$\\
Применим Л2\\
$\ub{\text{из Л1}}{\frac{1}{\sqrt{2\pi n p_* (1-p_*)}}\exp(-n H(p_*))} = \frac1{\sqrt{2\pi n p_*(1-p_*)}} \exp{-n(\frac{(p-p_*)^2}{2p(1-p)}+O((p-p_*)^3))} \sim\\ \frac1{\sqrt{2\pi np(1-p)}} \exp(-\frac{n(\frac kn - p)^2}{2p(1-p)} + nO(o(n^{-1}))) =\\ \frac1{\sqrt{2\pi np(1-p)}}\exp (-\frac{(k-np)^2}{2np(1-p)}+o(1))$\\\\\\\\
\textbf{Теорема(интегральная Муавра-Лапласа)}\\
$F_n(x) = p(\frac{S_n - np}{\sqrt{npq}} \leq x)$\\
$\Phi(x) = \int_{-\infty}^x \phi$\\
$\phi = \frac1{\sqrt{2\pi}} e^{-\frac {x^2}2}$\\
Тогда $\sup_{-\infty \leq x_1 \leq x_2 \leq +\infty} |F_n(x_2) - F_n(x_1) - \ub{\frac1{\sqrt{2\pi}}\int_{x_1}^{x_2} e^{-\frac {t^2}2}\df t}{(\Phi(x_2) - \Phi(x_1))}| \xrto[n\rto \infty]{} 0$\\
\textbf{Замечание}\\
$\sup_{x\in \Rset} |F_n(x) - \Phi(x)| \leq C \frac{p(1-p)^3 + (1-p)p^3}{(pq)^{\frac32} \sqrt n} = C \frac{(1-p)p((1-p)^2 + p^2)}{(pq)^{\frac32}\sqrt n} \leq C \frac1{\sqrt{pqn}}$\\
По последним оценкам $C < 1$
\section{Случайные величины. Распределение случайной величины}
% Давайте каждому элементарному исходу сопоставлять случайное число:\\
% $\Omega \rto \Rset$\\
% $\mcA \rto \mcB$\\
% $p \rto p_x$\\
\textbf{Определение}\\
Пусть $(\Omega, \mcA, P)$ -- вероятностное пространство\\
$\mcB$ -- борелевская сигма-алгебра (минимальная сигма-алгебра, содержащая все интервалы)\\
$X: \Omega \rto \Rset$ -- случайная величина, если $X$ -- измеримо, т.е. $\fall B \in \mcB$ -- борелевская сигма-алгебра $X^{-1}(B) \in \mcA$\\
$P_X: \mcB \rto \Rset$ -- распределение случайной величины, если $P_x(B) = P(\{\omega: X(\omega) \in B\})$ для всех $B \in \Bset$\\
\textbf{Замечание (обозначеиня)}\\
Случайные величины обозначаются большими латинскими буквами из конца алфавита($X,Y,U,W$) или маленькими греческими $\xi, \nu, \eta$\\
\textbf{Замечание}\\
$P_X$ удовлетворяет аксиомам вероятности\\
\textbf{Определение}\\
Функция распределенич $F_X(t) = P(X \leq t) = P(\{\omega: X(\Omega) \leq t\}), t \in \Rset$\\
\textbf{Теорема (свойства функции распределения)}
\begin{enumerate}
    \item $F$ не убывает
    \item $F(+\infty) = 1, F(-\infty) = 0$
    \item $F$ -- непрерывна справа
\end{enumerate}
\textbf{Доказательство}
\begin{enumerate}
    \item $t_1 < t_2$\\
    $F(t_2) = F(t_1) + P(\{\omega: t_1 < X(\Omega) \leq t_2\}) \geq F(t_1)$
    \item Пусть $t_n$ -- монотонно возрастает к $+\infty$\\
    $(-\infty, t_n] \subset (-\infty, t_{n+1}]$\\
    $\bigcup (-\infty,t_n] = \Rset$\\
    Тогда $F(t_n) \rto P(X \in \Rset) = 1$\\
    Пусть $t_n$ -- монотонно убывает к $-\infty$\\
    $(-\infty, t_n] \supset (-\infty, t_{n+1}]$\\
    $\bigcap (-\infty, t_n] = \varnothing$\\
    Тогда $F(t_n) \rto P(X \in \varnothing) = 0$
    \item Пусть $t_n$ -- монотонно стремится к 0\\
    $F(x_0 + t_n) \rto F(x_0)$\\
    $(-\infty, x_0 + t_{n+1}] \subset (-\infty, x_0 + t_n]$\\
    $\bigcap (-\infty, x_0 + t_{n}] = (-\infty, x_0]$\\
    $F(x_0+t_n) = P(X \leq x_0 + t_n) \rto P(X \leq x_0) = F(x_0)$
\end{enumerate}
\textbf{Замечание (непрерывность слева)}\\
Пусть $t_n$ -- монотонно стремится к 0\\
$F(x_0) - F(x_0 - t_n) = P(X \leq x_0) - P(X \leq x_0 - t_n) = P(x_0 - t_n < X \leq x_0)$\\
$(x_0 - t_n, x_0] \supset (x_0-t_{n+1}, x_0]$\\
$\bigcap (x_0-t_n, x_0] = \{x_0\}$\\
Т.о. $F(x_0) - F(x_0 - t_n) \rto P(X = x_0)$ -- иногда не 0\\
Т.о. нет непрерывности слева\\
\textbf{Лемма}\\
$\fall (B_n): B_{n+1}\subset B_n, B = \bigcap B_n\ P(B_n) \rto P(B)$\\
Равносильно $\fall (A_n): A_{n+1} \subset A_n, \bigcap A_n = \varnothing\ P(A_n) \rto 0$\\
\textbf{Доказательство $\Lto$}\\
$A_n = B_n \setminus B$\\
$A_{n+1} \subset A_n$\\
$\bigcap A_n = \varnothing$\\
$\ub{P(B_n) - P(B)}{P(A_n)} \rto 0$\\
\textbf{Теорема (о достаточности $F$ для описания вероятностного распределения)}\\
Пусть $F$ не убывает, $F(+\infty) = 1, F(-\infty) = 0, F$ -- непрерывно справа\\
Тогда $\ex (\Omega, \mcA, P), X$ -- случайная величина, такие что $F_X = F$\\
\textbf{Доказательство}\\
$\Omega := \Rset$\\
$\mcA$ -- система интервалов вида $(a,b]$ (+все лучи и $\Rset$), замкнутая относительно конечного числа $\sqcup$, т.е. $\mcA$ -- алгебра\\
$A := \bigsqcup_{k=1}^n (a_k, b_k], P(A) := \sum_{k=1}^n (F(b_k) - F(a_k))$\\
$P(\Rset) = F(+\infty) - F(-\infty) = 1$\\
Проверим счетную аддитивность\\
Пусть $(A_n): A_{n+1} \subset A_n, \bigcap A_n = \varnothing$\\
$A_n = \bigsqcup_{k=1}^{k_n} (a_{nk}, b_{nk}]$\\
Пусть все $A_n \subset [-M, M]$\\
Тогда $\ex (B_n): cl(B_n) \subset A_n$ и $P(A_n) - P(B_n) < \frac{\eps}{2^n}, cl(\bullet)$ -- замыкание\\
$B_n = \bigsqcup_{k=1}^{k_n} (a_{nk}+\delta, b_{nk}) \subset A_n$\\
$cl(B_n) = \bigsqcup_{k=1}^{k_n} [a_{nk}+\delta, b_{nk}) \subset A_n$\\
$P(A_n) - P(B_n) = \sum_{k=1}^{k_n} F(b_{nk})-F(a_{nk} - F(b_{nk}) + F(a_{nk}+\delta)) = \sum_{k=1}^{k_n} (F(a_{nk}+\delta) - F(a_{nk})) < \frac{\eps}{2^n}$ -- при правильном $\delta, k_n$\\
$cl(B_n) \subset A_n$\\
$\bigcap cl(B_n) \subset \varnothing$\\
$\bigcap cl(B_n) = \varnothing$\\
$\bigcup_{n=1}^\infty \ol{cl(B_n)} = \Rset \Rto \bigcup_{n=1}^\infty \ol{cl (B_n)} \supset [-M, M]$\\
$cl(B_n)$ -- открыто\\
$[-M, M]$ -- компакт\\
Тогда $\bigcup_{n=1}^N \ol{cl(B_n)} \supset [-M, M]$\\
Тогда $\bigcap_{n=1}^N cl(B_n) = \varnothing$\\
$\bigcap_{n=1}^N B_n = \varnothing$\\
$P(A_n) = P(A_N \setminus \bigcap_{n=1}^N B_n) + P(\bigcap_{n=1}^N B_n) = P(\bigcup_{n=1}^N (A_N \setminus B_n)) \leq \sum_{n=1}^N P(A_N \setminus B_n) \leq \sum_{n=1}^N P(A_n \setminus B_n) = \sum_{n=1}^N P(A_n) - P(B_n) \leq \sum_{n=1}^N \frac{\eps}{2^n} < \eps$\\
Тогда $P(A_n) \rto 0$\\
Теперь пусть $A_n$ -- не ограниченные\\
$\eps > 0$\\
$\ex [-M, M]: P(X \in \ol{[-M, M]}) < \eps$\\
$N: F(N) > 1 - \frac\eps2, F(-N) < \frac\eps2$\\
$F(N) = F(-N) + P((-N, N]) > 1 -\frac\eps2$\\
$F(N) = 1 - P((N, +\infty))$\\
$P((N, +\infty)) < \frac\eps2$\\
$P(A_n) = P(A_n\cap [-M, M]) + P(A_n \cap \ol{[-M, M]}) \leq P(\ol{[-M, M]}) < \eps$\\
Т.о. $P$ -- вероятность\\
$X(\omega) := \omega$\\
$F_X(t) = P((-\infty, t]) = F(t)$\\
\textbf{Замечание}\\
\section{Дискретные случайные величины и распределения}
\textbf{Определение}\\
$X, P_x$ -- дискретные, если существует не более чем счетное $E: P(X \in E) = 1$\\
$F(t) = P(X \leq t) = \sum_k p(k) \1(t \geq x_k), \1(cond) = (int)(cond)$\\
\textbf{Примеры}
\begin{enumerate}
    \item Вырожденное:\\
    $P(X=c) = 1$\\
    $X \sim I(c), I_c$
    \item Дискретное равномерное на $\{x_1, \ldots, x_n\}$\\
    $P(X = x_k) = \frac1n$\\
    $DU(x_1, \ldots, x_n)$
    \item Распределение Бернулли\\
    $Bern(p), p \in (0, 1)$\\
    $P(X = 0) = q = 1-p, P(X = 1) = p$
    \item Биномиальное; $Bin(n, p)$\\
    $P(X = k) = \binom{n}{k} p^k q^{n-k}, k \in \{0, \ldots, n\}$
    \item Полиномиальное; $Poly(n, p)$\\
    $p=(p_1, \ldots, p_m)$ -- вектор вероятностей\\
    $P(S_1 = k_1, \ldots, S_m = k_m) = \binom{n}{k_1, \ldots, k_m} p_1^{k_1}\ldots p_m^{k_m}$
    \item Геометрическое; $Geom(p)$\\
    $X$ -- количество неудач до первого успеха\\
    $P(X = k) = q^k p, k \in \Nset_0$\\
    \textbf{Альтернативная интерпретация геометрического распределения}\\
    $X$ -- номер первого успеха\\
    $P(X=k)=q^{k-1}p, k \in \Nset$
    \item Отрицательное биномиальное; $NB(r, p), p \in (0, 1), r > 0$
    \begin{itemize}
        \item $r \in \Nset:$\\
        $X \sim NB(r, p) \LRto X$ -- номер $r$-ого успеха (начало отсчета в $r$)\\
        $P(X=k) = P(\text{успех с номером $r$ случился на шаге $k+r$}) = \binom{k+r-1}{r-1} p^r q^{k}$
        \item Если $r \in \Rset$, то $P(X = k) = \frac{\Gamma(k+1)}{\Gamma(r)k!} p^r q^k$
    \end{itemize}
    \item Распределение Пуассона: $Pois(\lambda), \lambda > 0$\\
    $P(X = k) = e^{-\lambda}\frac{\lambda^k}{k!}, k \in \Nset_0$
    \item Гипергеометрическое\\
    $HG(M, N, K)$\\
    $M \in \Nset$ -- количество деталей\\
    $N \in [1:M]$ -- количество <<хороших>> деталей\\
    $K \in [1:M]$ -- количество деталей, которые мы вытаскиваем (без возвращения)\\
    $X$ -- количество <<хороших>> деталей, которые мы вытащили\\
    $P(X = j) = \frac{\binom{N}{j}\binom{M-N}{K-j}}{\binom{M}{K}}, \max(K + N - M, 0) \leq j \leq \min(N, K)$\\
    Пусть $M \rto \infty, \frac{N}{M} \rto p$\\
    $P(X=j) = \frac{N!(M-N)!K!(M-K)!}{j!(N-j)!(K-j)!(M-N-K+j)! M!} \\
    = \binom{K}{j}\frac{N(N-1)\ldots (N-j+1)}{M(M-1)\ldots (M-N + 1)} \frac{(M-K)\ldots (M - N - K + j + 1)}{} \rto \binom{K}{j} p^j(1-p)^{k-j}$\\\\
    $EX = \sum_{j = \max(0, N + K - M)}^{\min(N, K)} j \frac{\binom Nj \binom{M-N}{K-j}}{\binom{M}{K}} = \\
    \sum_{j = \max(1, N + K - M)}^{\min(N, K)} j \frac{\binom Nj \binom{M-N}{K-j}}{\binom{M}{K}} = \\
    N\sum_{j} \frac{(N-1)!}{(j-1)!(N-j)!}\frac{\binom{M-N}{K-j}}{\binom{M}{K}} = \\
    \frac {NK}{M}\sum_{j} \frac{\binom{N-1}{j-1}\binom{M-N}{K-j}}{\binom{M-1}{K-1}} = \\
    \frac{NK}{M} \ub{1}{\sum_{i = \max(0, N + K - M - 1)}^{\min(N-1, K-1)} \frac{\binom{N-1}{i}\binom{M-N}{K-i-1}}{\binom{M-1}{K-1}}} = \frac{NK}{M}$
\end{enumerate}
\section{Абсолютно непрерывные распределения}
\textbf{Определение}\\
$X, P_X$ -- абсолютно непрерывные, если $\ex p(x): \Rset \rto \Rset, p(x) \geq 0, p(x)$ -- интегрируемо по Лебегу на $\Rset$\\
$P(X \in B) = \int_B p(x)\df x, B \in \mcB$\\
$p$ -- плотность\\
\textbf{Теорема}
\begin{enumerate}
    \item $P(X = c) = 0$, т.к. $P(X = c) = \int_\varnothing \ldots = 0$
    \item \begin{enumerate}
        \item $F(x) = P(X \leq x) = \int_{-\infty}^x p(t) \df t$
        \item $F(x)$ -- непрерывна (равномерно непрерывная)
        \item $F'(x) = p(x)$ почти везде
    \end{enumerate}
    \item $\int_{-\infty}^\infty p(x)\df x = 1$
    \item $P(X \in (x_0, x_0 + h)) = F(x_0+h) - F(x_0) = p(x_0)h + o(h) \approx p(x_0)h$ при малых $h$
\end{enumerate}
\textbf{Определение (носитель)}\\
$E$ -- носитель ($\supp P_X$) $\LRto P(X \in E) = 1, E = \ol E, E$ наименьшее по включению\\
\textbf{Замечание}\\
$\lambda$ -- мера Лебега\\
$\nu$ -- другая мера, заданная на $\mcB, \nu(\Rset) < +\infty$\\
$\nu$ -- абсолютно непрерывна относительно $\lambda \LRto \lambda(A) = 0 \Rto \nu(A) = 0$\\
$\nu\ll \lambda$\\
\textbf{Теорема Радона-Никодима}\\
$\nu \ll \lambda \Rto \ex f(x): \nu(B) = \int_B f(x)\lambda(\df x)$\\
\textbf{Пример}
\begin{enumerate}
    \item $U[a,b]$: Равномерное на $[a,b]$\\
    $p(x) = \frac1{b-a} \1 (x\in [a,b])$\\
    $F(t) = \int_{-\infty}^t p(x)\df x = \left\{\begin{array}{cc}
        0,& t < a\\
        \frac{t-a}{b-a},& a \leq t \leq b\\
        1,& t > b
    \end{array}\right.$\\
    $X \sim U[a,b]$\\
    $Y = cX + d, c > 0$\\
    $Y \sim U[ca+d, cb+d]$\\
    $F_Y(t) = P(Y \leq t) = P(cX+d \leq t) = P(X \leq \frac{t-d}{c}) = F_X(\frac{t-d}c)$\\
    \item $N(\mu, \sigma^2), \mu \in \Rset, \sigma > 0$: Нормальное распределение\\
    $p(x) = \frac1{\sigma\sqrt{2\pi}}\exp(-\frac12 \frac{(x-\mu)^2}{\sigma^2})$\\
    $\mu$ -- высота графика, $\sigma$ -- ширина\\
    $N(0, 1)$ -- стандартный нормальный закон\\
    $\phi = \frac{1}{\sqrt{2\pi}}\exp(-\frac{x^2}{2}), \Phi(x) = \frac{1}{\sqrt{2\pi}}\int_{-\infty}^x e^{-\frac{t^2}2}$\\
    $\Phi(-x) = 1 - \Phi(x)$\\
    \textbf{Доказательство}\\
    $\Phi(x) = \int_{-\infty}^{-x} \ldots + \int_{-x}^x \ldots = \int_{-\infty}^{-x} \ldots + 2\int_0^x \ldots = \int_{-\infty}^{-x} \ldots + 2(\int_{-\infty}^x \ldots - \int_{-\infty}^0 \ldots) = \Phi(-x) + 2(\Phi(x) - \frac12)$\\
    $\Phi_0(x) = \frac{1}{\sqrt{2\pi}} \int_0^x e^{-\frac{t^2}2}\df t$ -- функция Лапласа; $\Phi_0(-x) = -\Phi_0(x)$\\
    $\Phi(x) = \frac12 + \Phi_0(x)$\\
    $\nm{erf}(x) = \frac{2}{\sqrt{\pi}} \int_0^x e^{-t^2}\df t$\\
    $\Phi_0(x) = \frac12 \nm{erf}(\frac{x}{\sqrt 2})$\\
    $y_{\text{изм}} = y_{\text{реал}} + \eps, \eps \sim N(0, \sigma^2)$\\
    $P(|y_{\text{изм}}-y_{\text{реал}}| > \delta) = P(|\eps| > \delta)$\\
    $X \sim N(\mu, \sigma^2)$\\
    $Y=aX + b$\\
    $F_Y(t) = P(aX+b \leq t) \us{a>0}= P(X \leq \frac{t-b}{a}) = F_x(\frac{t-b}{a})$\\
    $p_Y(t) = (F_X(\frac{t-b}{a}))'_t = p_X(\frac{t-b}{a})\frac1{a} = \ldots$\\
    $Y \sim N(a\mu + b, \sigma^2a^2)$\\
    Для $a < 0$ аналогично\\
    $X \sim N(\mu, \sigma^2), Y = aX+b \Rto Y \sim N(a\mu + b, a^2 \sigma^2)$\\
    $U=\frac{X-\mu}{\sigma} = \frac X\sigma - \frac\mu\sigma \Rto U\sim N(0, 1)$\\\\
    $X \sim N(\mu, \sigma^2)$\\
    $P(a \leq X \leq b) = P(\frac{a-\mu}{\sigma} \leq \frac{X-\mu}{\sigma} \leq \frac{b-\mu}{\sigma}) = \Phi(\frac{b-\mu}{\sigma}) - \Phi(\frac{a - \mu}\sigma)$\\
    $P(|X-\mu| < k\sigma) = P(-k\sigma < X - \mu < k\sigma) = P(\frac{X - \mu}{\sigma} \in (-k, k)) = \Phi(k) - \Phi(-k) = 2\Phi(k) - 1$
    \item $\nm{Exp}(\lambda), \lambda > 0$: Экспоненциальнле/показательное\\
    $p(x) = \lambda e^{-\lambda x} \1(x \geq 0)$\\
    $F(x) = (1-e^{-\lambda x}) \1(x \geq 0)$
    \item $\Gamma(\alpha, \beta), \alpha,\beta>0$: Гамма-распределение\\
    $p(x) = \frac{\beta^\alpha x^{\alpha-1}e^{-\beta x}}{\Gamma(\alpha)} \1(x\geq 0)$\\
    $\Gamma(1, \beta) = \nm{Exp}(\beta)$\\
    $\beta - rate$\\
    $\alpha \in \Nset \Rto $ распределение Эрланга порядка $k$
    \item $\nm{Cauchy}(x_0, \gamma)$: Распределение Коши\\
    $x_0 \in \Rset$ -- сдвиг\\
    $\gamma > 0$ -- scale\\
    $p(x) = \frac{1}{\pi \gamma} \frac{1}{1+(\frac{x-x_0}{\gamma}^2)}$\\
    $F_X(t) = \frac{1}{\pi\gamma} \int_{-\infty}^t \frac{\df x}{1 + (\frac{x-x_0}{\gamma})^2} = \left|\begin{array}{cc}
        \frac{x-x_0}{\gamma} = a\\
        \df x = \gamma \df u
    \end{array}\right| = \frac1\pi \int_{-\infty}^{\frac{t-x_0}{\gamma}} \frac{\df u}{1 + u^2} = \frac{1}{\pi} \arctg u \vl_{-\infty}^{\frac{t-x_0}{\gamma}} = \frac{1}{\pi} \arctg \frac{t-x_0}{\gamma} + \frac12$\\
    \textbf{Пример}\\
    $X \sim \nm{Cauchy}(0, 1)$\\
    $Y = \frac1X$\\
    $F_Y(t) = P(\frac1X \leq t) = P(\frac1X \leq t, X < 0) + P(\frac1X \leq t, X > 0)$\\
    $P(\frac1X \leq t, X < 0):$\\
    $t \geq 0 \Rto P(\ldots) = P(X < 0) = \frac12$\\
    $t < 0 \Rto P(X \leq 1, X < 0) = P(X \geq \frac1t, X < 0) = F(0) - F(\frac1t) = \frac12 - \frac12 - \frac1\pi \arctg \frac1t = -\frac1\pi\arctg \frac1t$\\
    $P(\frac1X \leq t, X > 0):$\\
    $t \leq 0 \Rto P(\ldots) = 0$\\
    $t > 0 \Rto P(X \geq \frac1t, X > 0) = P(X \geq \frac1t) = 1 - F(\frac1t) = \frac12 - \frac1\pi \arctg \frac1t$\\
    Т.о. $P(\frac1X \leq t, X > 0) = \left\{\begin{array}{ll}
        -\frac1\pi \arctg \frac1t,& t < 0\\
        \frac12, & t = 0\\
        1 - \frac1\pi\arctg \frac1t,& t > 0
    \end{array}\right.$
\end{enumerate}
\textbf{Теорема}\\
Пусть $p_x$ -- плотность с.в. $X$\\
$Y=g(X), g \in C^1$\\
$g'$ строго монотонна\\
Тогда $p_Y(y) = p_x(g^{-1}(y)) | \frac{\partial}{\partial y}(g^{-1}(y))| = p_x(g^{-1}(y)) \frac1{|g'_x (g^{-1}(y))}$\\
\textbf{Доказательство}\\
Пусть $g'\geq 0$\\
$F_Y(t) = P(g(X) \leq t) = P(X \leq g^{-1}(t))= F_X(g^{-1}(t))$\\
$p_Y(t) = (F_Y(t))'_t = p_X(y^{-1}(t)) \ppart{g^{-1}(t)}{y}$\\
Пусть $g' < 0$\\
$F_Y(t) = 1-F_X(g^{-1}(t))$\\
$p_Y(t) = -p_X(y^{-1}(t)) \ppart{g^{-1}(t)}{y}$
\section{Сингулярное распределение}
\textbf{Определение}\\
$X, P_X$ -- сингулярное $\LRto \supp P_X = E\ \&\ \lambda E = 0$ и $\fall x \in E\ P(X \in E) = 0$\\
\textbf{Замечание}\\
Сингулярное $\LRto \lambda\{x: F(x+\eps) - F(x - \eps) > 0\} = 0\ \fall \eps$\\
\textbf{Теорема Лебега о разложении $F$}\\
$F$ -- функция распределения\\
Тогда $\ex c_1, c_2, c_3 \geq 0: c_1 + c_2 + c_3 = 1, F(x) = c_1F_1(x) + c_2F_2(x) + c_3F_3(x), F_1$ -- функция распределения дискретной величины, $F_2$ -- функция распределения, абсолютно непрерывная, $F_3$ -- функция распределения сингулярного закона\\
\textbf{Замечание}\\
У $F$ могут быть только скачки, не более чем счетное количество
\section{Случайные векторы}
\textbf{Определение}\\
$X = \{X_1, \ldots, X_n\}$ -- случайный вектор (многомерная случайная величина), если $X_i$ -- случайная величина\\
$P_X(X_1 \in B_1, \ldots, X_n \in B_n)$ -- распределение случайного вектора (совместное распределение случайных величин $X_1, \ldots, X_n, B_i \in \mcB$)\\
$P_X: \sigma(\mcB^n)\rto \Rset$\\
\textbf{Определение}\\
$F(x_1, \ldots, x_n) = P(X_1 \leq x_1, \ldots, X_n \leq x_n)$ -- совместная функция распределения\\
\textbf{Свойства}
\begin{enumerate}
    \item $x_i \rto -\infty$ для некоторого $i \Rto F(x_1,\ldots, x_n) \rto 0$
    \item $x_i \rto +\infty$ для всех $i \Rto F(x_1,\ldots, x_n) \rto 1$
    \item $F$ -- непрерывна справа по каждой переменной
    \item $\Delta_i F(x_1, \ldots, x_i + \delta_i, \ldots, x_n) - F(x_1, \ldots, x_n)$\\
    $\Delta_1, \ldots, \Delta_n F \geq 0\ \fall x_1, \ldots, x_n\ \fall \delta_1, \ldots \delta_n \geq 0$
\end{enumerate}
\textbf{Теорема}\\
$F$ -- функция, для которой справедливы свойства $1-3$\\
Тогда существует вероятностное пространство и случайная величина с данной функцией распределения\\
\textbf{Определение}\\
$X_i, P_X$ -- дискретно, если $\supp P_X$ -- не более чем счетный\\
\textbf{Замечание}\\
$\supp P_X \neq \times_{i=1}^n \supp P_x$\\
Одномерное распределение = \textit{Маргинальное} распределение\\
\textbf{Пример}\\
Полиномиальное: $Poly(n, p)$\\
$n \in \Nset$\\
$p = (p_1, \ldots, p_n), \sum p_j = 1$\\
$S_n = (S_{n,1}, \ldots, S_{n,m})$\\
$P(S_{n,1}=k_1,\ldots, S_{nm}=k_m) = \frac{n!}{k_1!\ldots k_m!}p_1^{k_1}\ldots p_m^{k_m}$\\
$\sum k_i = n, k_i \geq 0$\\
$S_{nj} \sim Bin(n, p_j)$\\
\textbf{Определение}\\
$X, P_X$ -- абсолютно непрерывная $\LRto \ex p(x_1, \ldots, x_n) \geq 0$ -- плотность распределения\\
$P(X \in B) = \int_B p(x)\df x$\\
$F(x_1, \ldots, x_n) = \int_{-\infty}^{x_1}\ldots \int_{-\infty}^{x_n} p(t_1, \ldots, t_n) \df t_1\ldots \df t_n$\\
$p(t_1, \ldots, t_n) = \frac{\partial^n F}{\partial x_1 \ldots \partial x_n}$\\
$P(G(X) \in B) = \int_{x:F(x)\in B} p(x)\df x, G$ -- некая функция\\
\textbf{Определение (равномерное распределение на подмножестве)}\\
$U(E): p(x) = \frac{1}{\mu E}\1(x\in E), \mu$ -- мера\\
\textbf{Теорема (о плотности функции от непрерывного случайного вектора)}\\
Пусть $g:\Rset^n \rto \Rset^n, g$ -- гладкая биекция\\
Тогда $p_{g(U)}(t) = p_U(g^{-1}(t))|\det (g^{-1})'(t)| = \frac{p_U(g^{-1}(t))}{|\det g'(g^{-1}(t))|}$\\\\
\textbf{Определение (многомерное нормальное распределение/гауссовский вектор)}
\begin{enumerate}
    \item $X = (X_1, X_n)$ -- стандартный гауссовский вектор, если $p_X = \frac1{(\sqrt{2\pi})^n} \exp(-\frac12 \|x\|^2), \|x\|^2 = x^Tx = \sum x_i^2$\\
    $X \sim N(0, E_n), 0 \in \Rset^n$
    \item $X \sim N(0, E_n), Y = AX + b, A \in M_{m\times n}, b \in \Rset^m \Rto Y \sim N(b, AA^T)$
    \item $U \sim N(\mu, \Sigma)$, где $\mu \in \Rset^n, \Sigma \in M_{n\times n}$\\
    $\Sigma = \Sigma^T, \Sigma > 0$\\
    $\Lambda = \nm{diag}(\lambda_1, \ldots, \lambda_n), \lambda_i \geq 0$\\
    $\Lambda = L^T \Sigma L, L = (u_1, \ldots, u_n)$ -- ОНБ, $L^TL = E$\\
    $\Sigma = L\Lambda L^T = L = \ub{\sqrt{\Sigma}}{L \sqrt{\Lambda} L^T} \ub{\sqrt{\Sigma}}{L \sqrt{\Lambda}L^T}$\\
    $U = \sqrt{\Sigma}X + \mu, X \sim N(0, E)$\\
    $\Sigma > 0 \Rto p(x_1, \ldots, x_n) = \frac1{(\sqrt{2\pi})^n \sqrt{\det \Sigma}} \exp{-\frac12 (x-\mu)^T \Sigma^{-1} (x-\mu)}$\\
    $X=\sqrt{\Sigma}^{-1}(X-\mu)$\\
    $Y' = (\sqrt{\Sigma})^{-1}$\\
    $\det \Sigma = \det \sqrt{\Sigma}^2$\\
    $|\det \sqrt{\Sigma}| = \sqrt{\det \Sigma}$\\
    $p_U(t) = \frac{p_X(g^{-1}(t))}{|\det g'(g^{-1}(t))|} = \frac{1}{(\sqrt{2\pi})^n} \exp(-\frac12 (\sqrt{\Sigma})^{-1}(t-\mu)^T(\sqrt{\Sigma})^{-1}(t-\mu)) \frac{1}{|\det \sqrt{\Sigma}|}$\\
    $= \frac1{\sqrt{(2\pi)^n \det \Sigma}}\exp(-\frac12 (t-\mu)^T \Sigma^{-1} (t-\mu))$
\end{enumerate}
\section{Независимые случайные величины}
\textbf{Определение}\\
$X_1, \ldots, X_n$ -- независимые, если $\fall B_1, \ldots, B_n\ P(X_1 \subset B_1, \ldots, X_n \subset B_n) = P(X_1 \subset B_1) \ldots P(X_n \subset B_n)$\\
Последовательность независима, если любой ее префикс независимый\\
\textbf{Теорема (общий критерий независимости)}\\
Величины $X_1, \ldots, X_n$ независимы $\LRto F(x_1, \ldots, x_n) = F_1(x_1) \ldots F_n(x_n)$\\
\textbf{Доказательство $\Rto$} очевидно\\
\textbf{Доказательство $\Lto$}\\
Докажем для размерности 2\\
$P(X_1 \in (a_1, a_1 + \delta_1], X_2 \in (a2, a_2 + \delta_2]) =\\
\Delta_1 \Delta_2 F(a_1, a_2) = \Delta_1[F(a_1, a_2 + \delta_2) - F(a_1, a_2)] =\\
F(a_1 + \delta_1, a_2 + \delta_2) - F(a_1, a_2 + \delta_2) - F(a_1 + \delta_1, a_2) + F(a_1, a_2) =\\
F_1(a_1+\delta_1)F_2(a_2+\delta_2) - F_1(a_1)F_2(a_2+\delta_2) - F_1(a_1+\delta_1)F_2(a_2) + F_1(a_1)F_2(a_2) =\\
(F_1(a_1+\delta_1) - F_1(a_1))(F_2(a_2+\delta_2) - F_2(a_2)) = \\
P(X_1 \in (a_1, a_1+\delta_1])P(X_2 \in (a_2, a_2 + \delta_2])$\\
\textbf{Теорема (критерий независимости дискретных величин)}\\
Дискретные $X_1, \ldots, X_n$ -- независимые $\LRto P(X_1\in \{x_{1, 1}, \ldots, x_{1,m_1}\}, \ldots, X_n\in \{x_{n, 1}, \ldots, x_{n,m_n}\}) = \prod_{j=1}^n P(X_j\in \{x_{j, 1}, \ldots, x_{j,im_j}\})$\\
\textbf{Доказательство $\Rto$} очевидно\\
\textbf{Доказательство $\Lto$}\\
$P(X_1 \in B_1, \ldots, X_n \in B_n) = \sum_{i_1,\ldots, i_n} P(X_1 = x_{1, i_1}, \ldots, X_n=x_{n,i_n}) = \sum_j \prod_{i} P(X_j = x_{j, i}) =  \prod_{i}\sum_j P(X_j = x_{j, i}) = \prod_i P(X_i \in B_i)$\\
\textbf{Теорема (критерий независимости непрерывных величин)}\\
Непрерывные $X_1, \ldots, X_n$ -- независимые $\LRto p(x_1, \ldots, x_n) = \prod_i p_i(x_i)$\\
\textbf{Доказательство $\Rto$}\\
$F(x_1, x_2) = F_1(x_1) F_2(x_2) = (\int_{-\infty}^{x_1} p_1(x)\df x)(\int_{-\infty}^{x_2} p_2(y)\df y) = \int_{-\infty}^{x_1}\int_{-\infty}^{x_2} p_1(x)p_2(y)\df x \df y$\\
\textbf{Доказательство $\Lto$} очевидно (наверное)\\
\textbf{Пример}\\
$\Sigma = \nm{diag}(\sigma^2_1, \ldots, \sigma^2_n), \sigma_i > 0$\\
$p(x_1, \ldots, x_n) = \frac1{\sqrt{(2\pi)^n \sigma^2_1\ldots \sigma^2_n}} \exp(-\frac12 x^T(\sigma^2_1\ldots \sigma^2_n)x) = \prod_{i=1}^{n} \frac{1}{\sqrt{2\pi \sigma^2_i}}\exp(-\frac12 \frac{x_i^2}{\sigma_i^2})$\\
\textbf{Лемма}\\
$X,Y$ -- независимые целочисленные\\
Тогда $P(X+Y=K)= \sum_i P(X=i)P(Y=K-i) = \sum_i P(X=K-i)P(Y=i)$\\
\textbf{Доказательство}\\
$P(X+Y=K) = \sum_i P(X+Y=K|X=i)P(X=i) = \sum_i P(X=K-i)P(X=i)$\\
\textbf{Пример}
\begin{enumerate}
    \item $S_n \sim Bin(n, p)$\\
    $X_n = X_1 + \ldots + X_n, X_i = Bern(p), X_i$ -- независимые\\
    $P(S_{n+1}=k) = P(S_n=k)P(X_{n+1}=0)+P(S_n=k-1)P(X_{n+1}=1) = \binom nk p^k q^{n-k+1} + \binom n{k-1}p^kq^{n-k+1} = \binom{n+1}k p^k q^{n-k+1}$
    \item $X \sim Pois(\lambda)$\\
    $Y = Pois(\mu)$\\
    $X+Y \sim Pois(\lambda+\mu)$
    \item $X \sim NB(r_1, p), Y \sim NB(r_2, p)$\\
    Тогда $X+Y \sim NB(r_1+r_2, p)$\\
    $X_1+\ldots+X_n \sim NB(n, p), X_i \sim NB(1, p) = Geom(p)$\\
    $P(X_1+X_2=K) = \binom{(K+2)-1}1 q^kp^2$\\
    $P(S_n+X_{n+1}=K) = \binom{n+k}n q^kp^n$
\end{enumerate}
\textbf{Лемма}\\
$X,Y$ -- независимые\\
Тогда $p_{X+Y}(t) = \int_\Rset p_X(u)p_Y(t-u)\df u$\\
\textbf{Доказательство}\\
$F_{X+Y}(t) = P(X+Y \leq t) = \int_{-\infty}^\infty p_X(u) \int_{-\infty}^{t-u} p_Y(v)\df v$\\
\textbf{Пример}
\begin{enumerate}
    \item $X,Y$ -- независимые, $X\sim N(\mu_X, \sigma_X^2), Y \sim N(\mu_Y, \sigma_Y^2)$\\
    Тогда $X+Y \sim N(\mu_X + \mu_Y, \sigma_X^2 + \sigma_Y^2)$\\
    $X+Y = \sigma_X U_X + \mu_x + \sigma_Y U_Y + \mu_Y, U_X, U_Y \sim N(0, 1)$\\
    $\sigma_Y^2 u^2 + \sigma_X^2(t-u)^2 = (\ub{y}{\sqrt{\sigma_X^2 + \sigma_Y^2}u - \frac{\sigma_X^2 t}{\sqrt{\sigma_X^2 + \sigma_Y^2}}})^2 + \frac{t^2 \sigma_X^2 \sigma_Y^2}{\sigma_X^2 + \sigma_Y^2}$\\
    $P_{\sigma_X U_X + \mu_x + \sigma_Y U_Y + \mu_Y}(t) = \frac{1}{\sqrt{2\pi \sigma_X^2}}\frac1{\sqrt{2\pi \sigma_Y^2}} \int_{-\infty}^\infty \exp(-\frac12(\frac{u^2}{\sigma_X^2}+\frac{(t-u)^2}{\sigma_Y^2}))\df u = \\
     = \frac{1}{\sqrt{2\pi \sigma_X^2}}\frac1{\sqrt{2\pi \sigma_Y^2}} \int_{-\infty}^\infty \frac{1}{\sqrt{\sigma_X^2 + \sigma_Y^2}}\exp(-\frac1{2(\sigma_X^2 + \sigma_Y^2)}(y^2 + \frac{t^2\sigma_X^2\sigma_Y^2}{\sigma_X^2 + \sigma_Y^2}))\df u = \\
    = \frac{1}{\sqrt{2\pi(\sigma_X^2 + \sigma_Y^2)}} \exp (-\frac12 \frac{t^2}{\sigma_X^2 + \sigma_Y^2}) (\ub{1}{\frac{1}{2\pi\sigma_X^2\sigma_Y^2} \int_{-\infty}^{\infty} \exp(-\frac{y^2}{2\sigma_X^2\sigma_Y^2}) \df y})$\\
    $p_{U+\mu_x + \mu_Y} (t) = p_U(t-\mu_x-\mu_y) |t'| = \frac{1}{\sqrt{2\pi \sigma_X^2 \sigma_Y^2}} \exp(-\frac12 \frac{(t-\mu_X-\mu_Y)^2}{\sigma_X^2+\sigma_Y^2})$
\end{enumerate}
\textbf{Утверждение}\\
$X_1, \ldots, X_n$ -- независимые\\
$X_i \sim Exp(\lambda)$\\
Тогда $X_1+\ldots + X_n \sim \Gamma(n, \lambda)$ -- распределение Эранга порядка $n$\\
$P_{X_1 + X_2}(t) = \int_{-\infty}^{+\infty} p_1(x)p_2(t-x)\df x = \lambda^2 te^{-\lambda t}\1(t \geq 0)$\\
$P_{S_n + X_{n+1}} = \int_{-\infty}^{+\infty} \frac{\lambda^n x^{n-1}e^{-\lambda x}}{(n-1)!}\1(x \geq 0) \lambda e^{-\lambda (t-x)}\1(t-x \geq 0) \df x = \frac{\lambda^{n+1} x^n e^{-\lambda t}}{n!}\1(t \geq 0)$
\section{Симуляция (моделирование случайных величин)}
Пусть есть $rand(): U[0, 1]$ -- независимые\\
\begin{lstlisting}[language=python]
def Bern(p): 
    return rand() <= p


def Bin(n, p): 
    return sum(Bern(p) for _ in range(n))

def Geom(p): 
    res = 0
    while(!Bern(p)):
        res += 1
    return res

def Discrete(p_1, p_2, ..., p_n, x_1, ..., x_n):
    v = rand()
    if p_1 + ... + p_{n-1} < v: return x_n
    if p_1 + ... + p_{n-2} < v: return x_{n-1}
    ...
    if p_1 < v: return x_2
    return x_1

def Poly(1, p): ... #аналогично

def Poly(n, p):
    return sum(Poly(1, p) for _ in range(n))

def U(a, b):
    return a + (b - a) * rand()

def Exp(lmbda):
    return -ln(rand())/lmbda

def Gamma(n, lmbda):
    return sum(Exp(lmbda) for _ in range(n))

# Случайная величина с распределением F
def Generic(F):
    G = reversed(F) #обратная функция
    return G(rand())
\end{lstlisting}
\begin{lstlisting}[language=python]
    def N(0, 1):
        # n -- какое-то большое, p -- любое
        return (Bin(n, p) - n*p) / sqrt(n * p * (1 - p))
\end{lstlisting}

\textbf{Утверждение}\\
$X_1, \ldots, X_n \sim Exp(1)$ -- независимые\\
$X = \max\{n: X_1 + \ldots + X_n < \lambda\}, \lambda \geq 0$\\
Тогда $X \sim Pois(\lambda)$\\
\textbf{Доказательство}\\
$P(X \leq k) = P(\ub{\sim \Gamma(k+1, 1)}{X_1 + \ldots + X_{k+1}} \geq \lambda) = \int_\lambda^{+\infty} \frac{x^k e^{-x}}{k!}\df x = -\frac{x^k e^{-x}}{k!} \vl_\lambda^{+\infty} + \int_\lambda^{+\infty} x^{k+1}e^{-x}{(k-1)!}\df x = \frac{\lambda^k e^{-\lambda}}{k!} + \frac{\lambda^{k-1} e^{-\lambda}}{(k-1)!} + \ldots + \frac{\lambda^0 e^{-\lambda}}{0!}$\\
\section{Вероятностные интегралы}
$(\Omega, \mcA, P) \os{X}\mapsto (\Rset^n, \mcB, P_X)$\\
$P_X(B) = P(\{\omega: X(\omega) \in B\})$\\
$g \geq 0$ -- \textit{борелевская}, т.е. $\fall B \in \mcB\ g^{-1}(B) \in \mcB$\\
$\ub{\lim_{\nm{diam} V(\omega_k) \rto 0} \sum g(X(\omega_k)) P(V(\omega_k))}{\int_\Omega g(X(\omega)) P(\df \omega)} = \ub{\lim_{\nm{diam} V(x_k) \rto 0} \sum g(x_k) P(X \in V(x_k))}{\int_{\Rset^n} g(x) P_X(\df x)}, V(\cdot)$ -- какая-то окрестность\\
\textbf{Интеграл Стилтьеса}\\
$\int_\Rset g(x)\df F(x) = \lim_{x_{k+1}-x_k \rto 0, x_k^* \in [x_k, x_{k+1}]} \sum g(x_k^*) (F(x_{k+1}) - F(x_k)) =\\= \left\{\begin{array}{ll}
    \sum g(x_i) P(X = x_i),& \text{дискретный случай}\\
    \int g(x)p(x)\df x,& \text{непрерывный случай}
\end{array}\right.$\\\\
$g \geq 0 \Rto$ интеграл либо конечный, либо $+\infty$\\
$g = g_+ - g_i, I = \int_{\Rset^n} g_+(x) P(\df x) - \int_{\Rset^n} g_-(x) P(\df x)$\\
\begin{itemize}
    \item $I_+, I_- < +\infty \Rto I = I_+-I_-$
    \item $I_+ = +\infty, I_i < +\infty \Rto I = +\infty$
    \item $I_+ < +\infty, I_i = +\infty \Rto I = -\infty$
    \item $I_+, I_i = +\infty \Rto I$ не определено
\end{itemize}
\textbf{Свойства}
\begin{itemize}
    \item $\int af + bg = a\int f + b\int g$
    \item $\int_{A\sqcup B} =\int_A + \int_B$
    \item $f \leq g \Rto \int f \leq \int g$
\end{itemize}
\textbf{Теорема Фубини для независимых величин}\\
$X, Y$-- независимые $\Rto g(x,y) P(\df x, \df y) = \int_\Rset (\int_\Rset g(x, y) P_Y(\df y)) P_X(\df x)$\\
\textbf{Пример (свертка распределений)}\\
$X, Y$ -- независимые\\
$F_{X+Y}(t) = P(X + Y \leq t) = \int_{\Rset^2} \1(x + y \leq t)P(\df x, \df y) = \int_\Rset \int_\Rset \1(x+y \leq t) P_X(\df x) P_Y(\df y) = \int_\Rset P(X \leq t - y) P_Y(\df y) = \int_\Rset F_X(t-y)\df F_Y(y)$\\
Аналогично $F_{X+Y}(t) = \int_\Rset F_Y(t-x)\df F_X(x)$
\section{Математическое ожидание}
\textbf{Определение}\\
$EX = \int_\Rset x \df F(x) = \int x P_X(\df x)$\\
Дискретный случай: $EX = \sum x_k p_k$\\
Непрерывный случай: $EX = \int xp(x)\df x$\\
\textbf{Свойства}
\begin{enumerate}
    \item $Ef(X_1, \ldots, X_n) = \int_{\Rset^n} f(x_1, \ldots, x_n) P(\df x_1, \ldots, \df x_n) = \int_\Rset y \df F_Y(y), y = f(x_1, \ldots, x_n)$
    \item $E(aX) = aEx$
    \item $E(X+Y) = EX+EY$
    \item $X, Y$ -- независимые $\Rto E(XY) = (EX)(EY)$
    \item $\ub{P(X \geq 0) = 1}{X \geq 0} \Rto EX \geq 0$
    \item $\ub{P(X\geq Y) = 1}{X \geq Y} \Rto EX \geq EY$
    \item 
    \textbf{Теорема (неравенство Маркова)}\\
    $X \geq 0 \Rto P(X \geq C) \leq \frac{EX}{c}$\\
    \textbf{Доказательство}\\
    $EX = \int_0^{+\infty} d\df F(x) = \int_0^c x\df F(x) + \int_{c+\infty} x\df F(x) \geq 0 + c\int_c^{+\infty} \df F(x) = cP(X \geq c)$
    \item $X \geq 0, EX = 0 \Rto \ub{P(X = 0) = 1}{X=0}$\\
    \textbf{Доказательство}\\
    $P(X \geq c) \leq \frac0c = 0 \Rto P(X < c) = 1 \fall c \Rto P(X = 0) = 1$
    \item $\supp X = \Nset \Rto EX = \sum_{k=1}^{+\infty} P(X \geq k)$
    \item $\supp X = [0, +\infty)$\\
    $EX = \int_0^{+\infty} P(X \geq r) \df r$\\
    \textbf{Доказательство}\\
    $\int_0^{+\infty} P(X \geq r) \df r = \int_0^{+\infty} (1-F(r))\df r = \ub{0}{x(1-F(x))\vl_0^{+\infty}} + \int_0^{+\infty} x p(x)\df x$
\end{enumerate}
\textbf{Определение}\\
$X, Y$-- некоррелированные $\LRto E(XY)=(EX)(EY)$\\
\textbf{Определение}\\
$\Var X = E(X-EX)^2 = \left\{\begin{array}{c}
    \sum_k (x_k-EX)^2p_k\\
    \int (x-EX)^2 p(x)\df x
\end{array}\right.$\\
$Ef(x_1, \ldots, x_n) = \int_\Rset f(x_1, \ldots, x_n)P_X(\df x_1, \ldots, \df x_n)$\\
$\sqrt{\Var X}$ -- среднее квадратическое(стандартное) отклонение\\
\textbf{Свойства}
\begin{enumerate}
    \item $\Var X = EX^2 - (EX)^2$\\
    \textbf{Доказательство}\\
    $\Var X = E(X^2 - 2XEX + (EX)^2) = EX^2 - 2(EX)^2 + (EX)^2 = EX^2 - (EX)^2$
    \item $\Var X \geq 0$
    \item $\Var X = 0\Rto X = c$\\
    \textbf{Доказательство}\\
    $(X-EX)^2 = 0 \Rto X = EX$
    \item $\Var (aX + b) = a^2 \Var X$
    \item $\Var (X\pm Y) = \Var X + \Var Y \pm 2(E(XY) - EX\cdot EY)$\\
    \textbf{Доказательство}\\
    $\Var(X\pm Y) = E(X^2\pm 2XY + Y^2) - ((EX)^2 \pm 2EX\cdot EY + (EY)^2) = \Var X + \Var Y \pm 2(E(XY)-EX\cdot EY)$
    \item $\Var (X_1 + \ldots + V_n) = \sum_{i=1}^n \Var X_i + 2\sum_{i < j} (E(X_iX_j) - EX_iEX_j)$
\end{enumerate}
\textbf{Теорема (неравенство Чебышева)}\\
$P(|X-EX|\geq \eps) \leq \frac{\Var X}{\eps^2}$\\
\textbf{Свойства}
\begin{enumerate}
    \item $P(|X-EX| < \eps) \geq 1 - \frac{\Var X}{\eps^2}$
    \item $P(|X-EX| < k\sqrt{\Var X}) \geq 1 - \frac1{k^2}$
    \item ЗБЧ (слабый)\\
    $X_1,\ldots, X_n$ -- независимые\\
    $EX_i = \mc M$\\
    $\Var X_i = \sigma^2$\\
    $S_n = \sum_{i=1}^n X_i$\\
    Тогда $P(|\frac{S_n}{n} - \mc M| \geq \eps) \leq \frac{\sigma^2}{n\eps^2} \rto 0$\\
    $E \frac{S_n}{n} = \frac1n \sum EX_i = \mc M$\\
    $\Var \frac{S_n}{n} = \frac{n \Var X_1}{n^2} = \frac{\Var X_1}{n}$
\end{enumerate}
\textbf{Пример}\\
Дискретные:
\begin{itemize}
    \item $I(c)$\\
    $EX = 1c = c$\\
    $\Var X = 0$
    \item $Bern(p)$\\
    $EX = 1p + 0q = p$\\
    $EX^2 = p$\\
    $\Var X = p - p^2 = pq$
    \item $Bin(n, p)$\\
    $S_n = X_1 + \ldots + X_n$\\
    $X_i \sim Bern(p)$ -- независимые\\
    $ES_n = nEX_1 = np$\\
    $\Var S_n = n\Var X_1 = npq$
    \item $Pois(\lambda)$\\
    $EX = \sum_{k=1}^\infty ke^{-\lambda}\frac{\lambda^k}{k!} = e^{-\lambda}\lambda \sum_{k=1}^\infty \frac{\lambda^{k-1}}{(k-1)!} = e^{-\lambda+\lambda}\lambda = \lambda$\\
    $E(X(X-1)) = EX^2 - EX = EX^2 - (EX)^2  + (EX)^2 - EX = \Var X + (EX)^2 - EX \Rto \Var X = E(X(X-1)) - (EX)^2 + EX$\\
    $E(X(X-1))=\sum_{k=2}^\infty k(k-1)e^{-\lambda}\frac{\lambda^k}{k!} = \lambda^2$\\
    $\Var X = \lambda^2 - \lambda^2 + \lambda = \lambda$
    \item $Geom(p)$\\
    $EX = \sum_{k=1}^\infty P(X\geq k) = \sum_{k=1}^\infty \sum_{j=1}^k P(X=j) = \sum_{k=1}^\infty \sum_{j=k}^\infty q^jp= \sum_{k=1}^\infty \frac{q^k p}{1-q} = \sum_{k=1}^\infty q^k = \frac qp$\\
    $E(X(X-1)) = \sum_{k=2}^\infty k(k-1)q^kp = pq^2\sum_{k=2}^\infty k(k-1)q^{k-2} = pq^2 (\sum_{k=0}^\infty q^k)'' = pq^2 (\frac1{1-q})'' = \frac{2pq^2}{(1-q)^3} = 2\frac{q^2}{p^2}$
    Тогда $\Var X = 2\frac{q^2}{p^2} - \frac{q^2}{p^2} + \frac qp = \frac qp + (1 + \frac qp) = \frac{q}{p^2}$
    \item $S_n \sim NB(r, p), r \in \Nset$\\
    $X_1, \ldots, X_n$ -- независимые, $X_i \sim Geom(p)$\\
    Тогда $EX = \frac{rq}{p}, \Var S_n = \frac{rq}{p^2}$
\end{itemize}
Равномерные:
\begin{itemize}
    \item $U[0,1]$\\
    $EX = \frac12$\\
    $EX^2 = \frac13$\\
    $\Var X = \frac1{12}$
    \item $U[a,b]$\\
    $EX = \frac{a+b}{2}$\\
    $\Var X = \frac{(b-a)^2}{12}$
    \item $N(0, 1)$\\
    $EX = 0$\\
    $EX^2 = \frac{1}{\sqrt{2\pi}} \int_{-\infty}^\infty xxe^{-\frac{x^2}2}\df x = - \frac{1}{\sqrt{2\pi}}xe^{-\frac{x^2}2}\vl_{-\infty}^\infty + \frac{1}{\sqrt{2\pi}} \int_{-\infty}^\infty e^{-\frac{x^2}{2}}\df x = 1$\\
    \item $N(\mu, \sigma^2)$\\
    $X = \sigma N(0,1) + \mu$\\
    Тогда $EX = \mu, \Var X = \sigma^2$
    \item $\Gamma(\alpha, \lambda)$\\
    $p(x) = \frac{x^{\alpha-1}\lambda^\alpha e^{-\lambda x}}{\Gamma(\alpha)}\1(x \geq 0)$\\
    $EX = \frac{\alpha}{\lambda}\int_0^\infty \frac{x^\alpha\lambda^{\alpha+1} e^{-\lambda x}}{\Gamma(\alpha+1)} = \frac{\alpha}{\lambda}$\\
    $EX^2 = \frac{\alpha(\alpha+1)}{\lambda^2}\int_0^\infty \frac{x^{\alpha+1}\lambda^{\alpha+2} e^{-\lambda x}}{\Gamma(\alpha+2)} = \frac{\alpha(\alpha+1)}{\lambda^2}$\\
    $\Var X = \frac{\alpha(\alpha+1)}{\lambda^2} - \frac{\alpha^2}{\lambda^2} = \frac{\alpha}{\lambda^2}$
    \item $Exp(\lambda) = \Gamma(1, \lambda)$\\
    $EX = \frac1\lambda, \Var X = \frac1{\lambda^2}$
    \item $Cauchy(0, 1)$\\
    $EX = \int_{-\infty}^{+\infty} \frac{1}{\pi} \frac{x}{1+x^2}\df x \neq 0$ -- потому что интеграл расходится\\
    Матожидания не существует
\end{itemize}
\textbf{Определение (мода)}\\
Дискретный случай: $\mod X = x_k \LRto P(X = x_k)$ -- наибольшая\\
Непрерывный случай: $\mod X = x_* \LRto x_* = \argmax p(x)$\\
Унимодальность -- плотность имеет одну точку локального максимума\\
Бимодальность -- плотность имеет две точки локального максимума (но мода все равно одна)\\
\textbf{Определение (квантиль)}\\
$\alpha \in [0, 1]$\\
$q_\alpha$ -- квантиль порядка $\alpha \LRto P(X \geq q_\alpha) \geq 1 - \alpha, P(X \leq q_\alpha) \geq \alpha$\\
\textbf{Замечание}\\
$F$ строго монотонна $\Rto F(q_\alpha) = \alpha, F^{-1}(\alpha) = q_\alpha$\\
$F$ -- не строго монотонна\\
$F^{-1}(\alpha) := \left[\begin{array}{c}
    \sup \{x: F(x) \leq \alpha\}\\
    \inf \{x: F(x) \leq \alpha\}
\end{array}\right.$\\
\textbf{Определение}\\
$\med X = q_{\frac12}$\\
\textbf{Утверждение}\\
$\Var X = \min_{a \in \Rset} \ub{f(a)}{E(X-a)^2} \LRto EX = \argmin_{a\in \Rset} E(X-a)^2$\\
$f(a) = EX^2 - 2aEX + a^2$\\
$a_{\min} = EX$\\
\textbf{Утверждение}\\
$\med X = \argmin_{a\in \Rset} E|X-a|$\\
\textbf{Доказательство}\\
Н.у.о. $\med X = 0$\\
Проверим, что $E|X-c| \geq E|X|$\\
$c \geq 0: |X-C| - |X| = \left\{\begin{array}{cc}
    -c,& X > c\\
    c-2X,& 0 < X \geq c\\
    c,& X < 0
\end{array}\right.$\\
$E(|X-c| - |X|) =\\
E((|X-c| - |X|)\1(X > c)) + E((|X-c| - |X|)\1(0 < X \geq c)) + E((|X-c| - |X|)\1(X \geq 0)) =\\
-cE\1(X > c) + E((|X-c| - |X|)\1(0 < X \geq c)) + cE(\1(X \leq 0)) \geq\\
-cP(X > c) - cP(0 < X \leq c) + cP(X \leq 0) =\\
c(P(X \leq 0) - P(X > 0)) = c(2P(X \geq 0) - 1) \geq 0$\\\\
\textbf{Определение}\\
$EX^k$ -- $k$-ый момент\\
$E|X|^k$ -- $k$-ый абсолютный момент\\
$E(X-EX)^k$ -- $k$-ый абсолютный центральный момент\\
$E(X-EX)^k$ -- $k$-ый центральный момент\\
$\gamma = \frac{E(X-EX)^3}{\sigma^3}$ -- коэффициент асимметрии\\
$\gamma=0$ -- график симметричный\\
$\gamma > 0$ -- правый <<хвост>> длиннее левого\\
$\gamma < 0$ -- левый <<хвост>> длиннее правого\\
$\beta = \frac{E(X-EX)^4}{\sigma^4} - 3$ -- коэффициент эксцесса\\
(константа 3 для нормировки: $\beta_{N(0,1)} = 0$)\\
$\beta > 0$ -- плотности более сконцентрированы в центре\\
$\beta < 0$ -- плотности более распределены
\section{Числовые характеристики случайных величин}
\textbf{Определение}\\
$\Cov(X, Y) = E((X-EX)(Y-EY))$\\
\textbf{Свойства}
\begin{enumerate}
    \item $\Cov(X, Y) = E(XY)-EXEY$
    \item Билинейность
    \item $\Cov(X, c) = 0$
    \item $\Cov(X, X) = \Var X$
    \item $X,Y$ -- независимые $\Rto \Cov (X, Y) = 0$\\
    \textbf{Пример}\\
    $X \sim N(0, 1), Y = X^2$\\
    $\Cov(X, Y) = EX^2 - EXEX^2 = 0$, но $X, Y$ не независимые
\end{enumerate}
\textbf{Определение}\\
коэффициент корреляции $\rho(X,Y) = \frac{\Cov(X, Y)}{\sqrt{\Var X\Var Y}}$\\
\textbf{Замечание}\\
$X$ -- нерожденная\\
$P(Y = c) = 1$\\
$P(X \in B, Y = 1) = P(X \in B)P(Y = 1)$\\
$P(X \in B, Y = 0) = 0 = P(X \in B)P(Y = 1)$\\
Тогда они независимые\\
Тогда пусть $\rho(X, Y) = 0$\\
\textbf{Теорема}
\begin{itemize}
    \item $\rho(X, Y) \in [-1, 1]$
    \item $|\rho(X, Y)| = 1 \LRto Y = aX + b, \sign a = \sign \rho(X, Y)$
\end{itemize}
\textbf{Доказательство 1}\\
$\rho(X, Y) = \frac{\Cov(X, Y)}{\sqrt{\Var X \Var Y}} = \frac{\Cov(X - EX, Y - EY)}{\sqrt{\Var X\Var Y}} = \Cov \ot X, \ot Y$\\
$\ot X = \frac{X-EX}{\sqrt{\Var X}}$\\
$\ot Y = \frac{Y-EY}{\sqrt{\Var Y}}$\\
$E \ot X = 0 = E\ot Y$\\
$\Var \ot X = \frac{\Var \ot Y}{(\sqrt{\Var \ot Y})^2} = 1$\\
$\Var (\ot X + \ot Y) = \Var \ot X + \Var \ot Y + 2\Cov(\ot X, \ot Y) = 2(1 + \rho(X, Y)) \geq 0$\\
$\Var (\ot X - \ot Y) = 2(1 - \rho(X, Y)) \geq 0$\\
\textbf{Доказательство 2($\Rto$)}\\
$\rho(X, Y) = 1 \Rto \Var (\ot X - \ot Y) = 0 \Rto \ot X - \ot Y = c$\\
$\rho(X, Y) = -1 \Rto \ot X + \ot Y = c$\\
\textbf{Доказательство 2($\Lto$)}\\
$a \neq 0: \rho(aX+b, X) = \frac{\Cov(aX+b, X)}{\sqrt{\Var(aX + b)\Var X}} = \frac{a\Cov X,X}{\sqrt{a^2\Var^2 X}} = \frac{a}{|a|} = \sign a$
$a = 0$ -- очевидно\\
\textbf{Определение}\\
$X = (X_1, \ldots, X_n)^T$ -- случайный вектор\\
$EX = (EX_1, \ldots, EX_n)^T$ -- вектор математических ожиданий\\
\textbf{Замечание}\\
$X$ -- случайная матрица\\
$EX = (EX_{ij})$ -- матрица математических ожиданий\\
\textbf{Свойства}\\
$E(AX+b) = AEX + b$ -- линейность\\
\textbf{Определение}\\
$X = (X_1, \ldots, X_n)^T$ -- случайный вектор\\
$\Var X = (\Cov(X_i, X_j))_{i,j} = E((X-EX)(X-EX)^T)$\\
\textbf{Свойства}
\begin{enumerate}
    \item $\Var X = (\Var X)^T$
    \item $\Var X \geq 0$, т.е. $\fall t \in \Rset^n t^T \Var X t \geq 0$\\
    \textbf{Доказательство}\\
    $t^T \Var X t  = t^T(E(X-EX)(X-EX)^T)t = E t^T(X-EX)(X-EX)^Tt = E(t^T(X-EX))^2 \geq 0$
    \item $\Var (AX+b) = A\Var(X)A^T$
    \item $X, Y$ -- независимые $\Rto \Var X\pm Y = \Var X + \Var Y$\\
    \textbf{Доказательство}\\
    $\Cov (X_i \pm Y_i, X_j \pm Y_j) = \Cov(X_i, X_j) \ub{0}{\pm \Cov(X_i, Y_j) \pm \Cov(X_j, Y_i)} + \Cov(X_j, Y_j)$
\end{enumerate}
\textbf{Пример (гауссовский вектор)}\\
$X \sim N(\mu, \Sigma), \mu \in \Rset^n, \Sigma = \Sigma^T > 0$\\
Тогда $X = \sqrt{\Sigma}Y + \mu, Y \sim N(0, E)$ -- стандартный гауссовский вектор\\
$Y = (Y_1, \ldots, Y_n)$ -- стандартный гауссовский вектор $\LRto Y_i \sim N(0, 1), Y_i$ -- независимые\\
$EY = 0$\\
$\Var Y = E_n$\\
$EX = \sqrt{\Sigma}EY + \mu = \mu$ -- вектор математических ожиданий\\
$\Var X = \sqrt{\Sigma}\Var Y (\sqrt{\Sigma})^T = \Sigma$\\
\textbf{Пример}\\
$X$ -- гауссовский вектор\\
Покажем, что $X_i, X_j$ -- независимые $\LRto \Cov(X_i, X_j) = 0 \fall i, j$\\
$i := 1, j := 2$\\
$X = \begin{pmatrix}
    X_1\\X_2\\\vdots
\end{pmatrix} = \begin{pmatrix}
    \sqrt{\Sigma_{12}} & 0\\
    0 & \ddots
\end{pmatrix}\begin{pmatrix}
    Y_1\\\vdots\\Y_n
\end{pmatrix} + \begin{pmatrix}
    \mu_1\\\vdots\\\mu_n
\end{pmatrix}$\\
$\begin{pmatrix}
    X_1\\X_2\\
\end{pmatrix} = \begin{pmatrix}
    \sqrt{\Sigma_{12}} & \ldots\\
\end{pmatrix}\begin{pmatrix}
    Y_1\\\vdots\\Y_n
\end{pmatrix} + \begin{pmatrix}
    \mu_1\\\vdots\\\mu_n
\end{pmatrix} \Rto (X_1, X_2)^T \sim N((\mu_1, \mu_2)^T, \Sigma_{12})$\\
$(X_1, X_2)^T \sim N((\mu_1, \mu_2)^T, \nm{diag}(\sigma_1^2, \sigma_2^2))$\\
$\sigma_1^2 = \Var X_1$\\
$\sigma_2^2 = \Var X_2$\\
$p(X_1, X_2) = \frac{\exp(-\frac12 (\frac{(x_1 - \mu_1)^2}{\sigma_1^2} + \frac{(x_2 - \mu_2)^2}{\sigma_2^2}))}{\sqrt{2\pi}^2 \sigma_1 \sigma_2}$\\\\
$\begin{pmatrix}
    X\\Y
\end{pmatrix} = \begin{pmatrix}
    \sqrt{\Sigma}_X & \sqrt{\Sigma}_{XY}\\
    \sqrt{\Sigma}_{XY} & \sqrt{\Sigma}_Y
\end{pmatrix}\begin{pmatrix}
    U_1\\U_2
\end{pmatrix} + \begin{pmatrix}
    \mu_X\\\mu_Y
\end{pmatrix}$\\
$X = \sqrt{\Sigma}_X U_1 + \sqrt{\Sigma}_{XY}U_2 + \mu_X$\\
$\Var X = \begin{matrix}
    \sqrt{\Sigma}_X & \sqrt{\Sigma}_{XY}
\end{matrix}\begin{matrix}
    \sqrt{\Sigma}_X\\
    \sqrt{\Sigma}_{XY}
\end{matrix} = \sqrt{\Sigma_X}^2 + \sqrt{\Sigma_{XY}}^2$\\
$\begin{pmatrix}
    \sqrt{\Sigma}_X & \sqrt{\Sigma}_{XY}\\
    \sqrt{\Sigma}_{XY} & \sqrt{\Sigma}_Y
\end{pmatrix}\begin{pmatrix}
    \sqrt{\Sigma}_X & \sqrt{\Sigma}_{XY}\\
    \sqrt{\Sigma}_{XY} & \sqrt{\Sigma}_Y
\end{pmatrix} = \begin{pmatrix}
    \sqrt{\Sigma_X}^2 + \sqrt{\Sigma_{XY}}^2 & \sqrt{\Sigma}_{XY}(\sqrt{\Sigma}_X + \sqrt{\Sigma}_Y)\\
    \sqrt{\Sigma}_{XY}(\sqrt{\Sigma}_X + \sqrt{\Sigma}_Y) & \sqrt{\Sigma_Y}^2 + \sqrt{\Sigma_{XY}}^2
\end{pmatrix}$
\section{Условное распределение числовых характеристик}
\textbf{Определение}\\
$(X, Y)$ -- случайный вектор\\
Дискретный случай: $P(X = x_i|Y = y_j) = \frac{P(X=x_i, Y = y_j)}{P(Y = y_j)}$\\
$P(Y = y_j) = \sum_k P(Y = y_j | X = x_i)P(X = x_i)$\\
$P(X = x_i|Y = y_j) = \frac{P(X=x_i, Y = y_j)}{\sum_k P(Y = y_j | X = x_i)P(X = x_i)}$ -- теорема Байеса\\
Непрерывный случай: $p_{X|Y}(x|y) = \frac{p_{X,Y}(x, y)}{p_Y(y)}$\\
$p_Y(y) = \int_\Rset p_{Y|X}(y|x)p_X(x) \df x$\\
$p_{X|Y}(x|y) = \frac{p_{X,Y}(x, y)}{\int_\Rset p_{Y|X}(y|x)p_X(x) \df x}$ -- теорема Байеса\\
\textbf{Определение}\\
$E(X | Y = y) = E(X|Y) = \left\{\begin{array}{c}
    \sum_i x_iP(X=x_i|Y=y)\\
    \int_\Rset x p(x|y)\df x
\end{array}\right.$\\
Условное математическое ожидание -- случайная величина\\
\textbf{Свойства}
\begin{enumerate}
    \item $E(aX+bY|U) = aE(X|U) + bE(Y|U)$
    \item $E(X|Y) = \argmin_{f(Y)} E(X-f(Y))^2$
    \item $E(X|X) = X$
    \item $X, Y$ -- независимые $\Rto E(X|Y) = E(X)$
    \item $EX = E_Y(E_{X|Y}(X|Y))$\\
    \textbf{Доказательство}\\
    $EX = \sum_k x_k P(X=x_k) = \sum_k x_k \sum_j P(X=x_k | Y = y_j) P(Y_j) = \sum_j P(Y=y_j) \sum_k x_k P(X_k | Y=y_k) = \sum_j P(Y=y_j)E(X|Y=y_j) = E_Y(E(X|Y))$
\end{enumerate}
\textbf{Пример}\\
$(X, Y) \sim N(\begin{pmatrix}
    \mu_X\\\mu_Y
\end{pmatrix}, \begin{pmatrix}
    \sigma_X^2 & \sigma_{XY}\\
    \sigma_{XY} & \sigma_Y^2
\end{pmatrix})$\\
Тогда $E(Y|X) = aX  + b$\\
$E(Y|X) = E(Y-Ax + AX|X) = E(Y-aX|X) + E(aX|X) = E(Y - \frac{\sigma_{XY}}{\sigma_{X}^2}) + \frac{\sigma_{XY}}{\sigma_X^2}X = (X-\mu_X)\frac{\sigma_{XY}}{\sigma_X^2} + \mu_X$\\
$\begin{pmatrix}
    Y-aX\\ X
\end{pmatrix} \sim N(\cdot, \cdot)$\\
$0 = \Cov(Y-aX, X) = \Cov(Y, X) - a\Cov(X, X) \Rto a = \frac{\sigma_{XY}}{\sigma_X^2}$\\
\textbf{Определение}\\
$\Var(X|Y) = E((X-E(X|Y))|Y) = E(X^2|Y) - (E(X|Y))^2$\\
$\Var X = \Var E(X|Y) + E \Var(X|Y)$
\section{Сходимости}
$X_n \xrto{a.s.} X \LRto P(\{\omega: X_n(\omega) \rto X(\pi)\}) = 1$ -- почти наверное\\
$X_n \xrto{p} X \LRto \fall \eps > 0\ P(|X_n-X| > \eps) \rto 0$ -- по вероятности\\
$X_n \xrto{Lp} \LRto E|X_n-X|^p \rto 0, \geq 1$\\
$X_n \xrto{d} X \LRto F_n(x) \rto F(x), x \in C(F)$ -- по распределению\\
$F_n \Rto F \LRto \fall f$ -- непр. $\int f\df F_n(x) \rto \int f\df F(x)$ -- слабая\\
\textbf{Теорема}\\
$X_n \xrto{a.s.} X \Rto X_n \xrto{d} X$\\
$X_n \xrto{Lp} X \Rto X_n \xrto{d} X$\\\\
% $X_n \xrto{d} X \LRto F_n \rto F$\\\\
$\mc F$ -- множество функций распределения\\
$\mc G = \{G: \Rset \rto \Rset| G \text{ -- монот. возр, непр. справа}, G(-\infty) \geq 0, G(+\infty \leq 1)\}$ -- расширенные распределения\\
Распределения:\\
$P(X \in \Rset) = G(+\infty) - G(-\infty)$\\
$P(X = -\infty) = G(-\infty)$\\
$P(X = +\infty) = 1 - G(+\infty)$\\
$P(X_n = n) = P(X_n = -n) = \frac12$\\
$F_n(x) = \left\{\begin{array}{cc}
    0, & x < -n\\
    \frac12, & -n \leq x < n\\
    1, & x \geq n
\end{array}\right.$\\
$F_n(x) \xrto{d} \frac12$\\
$F_n(x) \not \Rto \frac12$\\
\textbf{Замечание}
\begin{enumerate}
    \item $\xrto{d} \Rto$ можно задать на $\mc G$
    \item $G_n, G \in \mc G\ (G_n \Rto G) \Rto G_n \xrto{d} G$\\
    В обратную сторону не действует
\end{enumerate}
\textbf{Теорема (Хелли)}\\
$G_n \in G \Rto \ex G_{n_k}, G \in G: G_{n_k} \Rto G$\\
(без доказательства)\\
\textbf{Следствие}\\
Если всякая слабо сходящаяся подпоследовательность сходится к одной и той же точке $G$, то $G_n \Rto G$\\
% \textbf{Доказательство}\\
% Пусть $G_n \xrto{d} G$\\
% Тогда $G_n \not\Rto G$\\
% $\ex t \in C(G): G_n(t) \not\rto G(t)$\\
% $\ex G_{n_k}: (G_{n_k} \Rto G) \Rto G_{n_k}(t) \rto G(t)$\\
% Тогда $G_n \xrto{d} G$
\textbf{Определение (плотные распределения)}\\
$F_n \in \mc G$\\
$\{F_n$\} -- плотная, если $\fall \eps > 0\ex M: \inf_n \ub{P(M\leq X_n \leq M)}{(F_n(M) - F_n(-M-0))} > 1-\eps$\\
Т.е. <<хвосты>> равномерно маленькие\\
\textbf{Определение}\\
Пусть $L$ -- подмножество непрерввных и ограниченных функций\\
$L$ -- определеяет распределение $\LRto F \in \mc F, G \in \mc G, \fall f \in L\ \int f\df F = \int f \df G \Rto F = G$\\
\textbf{Теорема (Критерий существования слабого предела из $\mc F$)}\\
$F_n \in \mc F, L$ -- определяет распределение
Тогда $\ex F \in \mc F: F_n \Rto F \LRto$
\begin{enumerate}
    \item $\{F_n\}$ -- плотное\\
    \item $\ex \lim_n \int f\df F_n \fall f \in L$
\end{enumerate}
\textbf{Доказательство $\Rto$}\\
2 выполнено\\
$\int f\df F_n \rto \int f\df F\ \fall f$\\
$F_n \xrto{d} F$\\
$\ex  M > 0: M, -M \in C(F), F_n(+M) - F_n(-M-0) \rto F(M) - F(-M) > 1-\eps$\\
\textbf{Доказательство $\Lto$}\\
$\ex F_{n_k}: F_{n_k} \Rto G \in \mc G$\\
Тогда $F_{n_k} \xrto{d} G$\\
$F_{n_j} \Rto F \in \mc F$\\
$\ex \lim \int f \df F_n$\\
$\int f \df F_{n_k} \Rto A$\\
$\int f \df F_{n_k} \Rto \int f\df G$\\
$\int f \df F_{n_j} \Rto A$\\
$\int f \df F_{n_j} \Rto \int f\df F$\\
Тогда $G = F$\\
\textbf{Следствие}\\
$L$ -- определяет распределение\\
$\int f \df F_n \rto f\df F$\\
Пусть верно одно из трех условий
\begin{enumerate}
    \item $\{F_n\}$ -- плотное
    \item $F \in \mc F$
    \item $1 \in L$
\end{enumerate}
Тогда $F_n \Rto F \in \mc F$\\
\textbf{Доказательство 1}\\
По теореме\\
\textbf{Доказательство 2}\\
Тогда выполнено 1\\ 
\textbf{Доказательство 3}\\
$\ub{1}{\int \df F_n} \rto \int \df F$\\
Тогда $\int \df F = 1$\\
$\int \df F = F(+\infty) - F(-\infty)$\\
Тогда $F(+\infty) = 1, F(-\infty) = 0$\\
Тогда выполнено условие 2\\
\textbf{Теорема}\\
Пусть $F_n \xrto{d} F \in \mc F$\\
Т.е. $F_n(x) \rto F(x)\ \fall x \in C(F)$\\
Покажем, что $\fall f \in \ot C^0_1(\Rset)\ \int_{-\infty}^{\infty} f \df F_n(x) \rto \int_{-\infty}^\infty f\df F(x)$\\
\textbf{Доказательство}\\
$\int_{-\infty}^\infty f\df F(x) = \ub{0}{f F\vl_{-\infty}^\infty} - \int_{-\infty}^\infty F \df f(x)$\\
$\int_{-\infty}^{\infty} f \df F_n(x) = \ub{0}{f F_n\vl_{-\infty}^\infty} - \int_{-\infty}^\infty F_n \df f(x)$\\
Тогда $F_n \Rto F$\\
\end{document}