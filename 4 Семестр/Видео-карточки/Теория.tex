\documentclass[12pt]{article}
\usepackage{bbold}
\usepackage{amsfonts}
\usepackage{amsmath}
\usepackage{amssymb}
\usepackage{color}
\setlength{\columnseprule}{1pt}
\usepackage[utf8]{inputenc}
\usepackage[T2A]{fontenc}
\usepackage[english, russian]{babel}
\usepackage{graphicx}
\usepackage{hyperref}
\usepackage{mathdots}
\usepackage{xfrac}


\def\columnseprulecolor{\color{black}}

\graphicspath{ {./resources/} }


\usepackage{listings}
\usepackage{xcolor}
\definecolor{codegreen}{rgb}{0,0.6,0}
\definecolor{codegray}{rgb}{0.5,0.5,0.5}
\definecolor{codepurple}{rgb}{0.58,0,0.82}
\definecolor{backcolour}{rgb}{0.95,0.95,0.92}
\lstdefinestyle{mystyle}{
    backgroundcolor=\color{backcolour},   
    commentstyle=\color{codegreen},
    keywordstyle=\color{magenta},
    numberstyle=\tiny\color{codegray},
    stringstyle=\color{codepurple},
    basicstyle=\ttfamily\footnotesize,
    breakatwhitespace=false,         
    breaklines=true,                 
    captionpos=b,                    
    keepspaces=true,                 
    numbers=left,                    
    numbersep=5pt,                  
    showspaces=false,                
    showstringspaces=false,
    showtabs=false,                  
    tabsize=2
}

\lstset{extendedchars=\true}
\lstset{style=mystyle}

\newcommand\0{\mathbb{0}}
\newcommand{\eps}{\varepsilon}
\newcommand\overdot{\overset{\bullet}}
\DeclareMathOperator{\sign}{sign}
\DeclareMathOperator{\re}{Re}
\DeclareMathOperator{\im}{Im}
\DeclareMathOperator{\Arg}{Arg}
\DeclareMathOperator{\const}{const}
\DeclareMathOperator{\rg}{rg}
\DeclareMathOperator{\Span}{span}
\DeclareMathOperator{\alt}{alt}
\DeclareMathOperator{\Sim}{sim}
\DeclareMathOperator{\inv}{inv}
\DeclareMathOperator{\dist}{dist}
\newcommand\1{\mathbb{1}}
\newcommand\ul{\underline}
\renewcommand{\bf}{\textbf}
\renewcommand{\it}{\textit}
\newcommand\vect{\overrightarrow}
\newcommand{\nm}{\operatorname}
\DeclareMathOperator{\df}{d}
\DeclareMathOperator{\tr}{tr}
\newcommand{\bb}{\mathbb}
\newcommand{\lan}{\langle}
\newcommand{\ran}{\rangle}
\newcommand{\an}[2]{\lan #1, #2 \ran}
\newcommand{\fall}{\forall\,}
\newcommand{\ex}{\exists\,}
\newcommand{\lto}{\leftarrow}
\newcommand{\xlto}{\xleftarrow}
\newcommand{\rto}{\rightarrow}
\newcommand{\xrto}{\xrightarrow}
\newcommand{\uto}{\uparrow}
\newcommand{\dto}{\downarrow}
\newcommand{\lrto}{\leftrightarrow}
\newcommand{\llto}{\leftleftarrows}
\newcommand{\rrto}{\rightrightarrows}
\newcommand{\Lto}{\Leftarrow}
\newcommand{\Rto}{\Rightarrow}
\newcommand{\Uto}{\Uparrow}
\newcommand{\Dto}{\Downarrow}
\newcommand{\LRto}{\Leftrightarrow}
\newcommand{\Rset}{\bb{R}}
\newcommand{\Rex}{\overline{\bb{R}}}
\newcommand{\Cset}{\bb{C}}
\newcommand{\Nset}{\bb{N}}
\newcommand{\Qset}{\bb{Q}}
\newcommand{\Zset}{\bb{Z}}
\newcommand{\Bset}{\bb{B}}
\renewcommand{\ker}{\nm{Ker}}
\renewcommand{\span}{\nm{span}}
\newcommand{\Def}{\nm{def}}
\newcommand{\mc}{\mathcal}
\newcommand{\mcA}{\mc{A}}
\newcommand{\mcB}{\mc{B}}
\newcommand{\mcC}{\mc{C}}
\newcommand{\mcD}{\mc{D}}
\newcommand{\mcJ}{\mc{J}}
\newcommand{\mcT}{\mc{T}}
\newcommand{\us}{\underset}
\newcommand{\os}{\overset}
\newcommand{\ol}{\overline}
\newcommand{\ot}{\widetilde}
\newcommand{\vl}{\Biggr|}
\newcommand{\ub}[2]{\underbrace{#2}_{#1}}

\def\letus{%
    \mathord{\setbox0=\hbox{$\exists$}%
             \hbox{\kern 0.125\wd0%
                   \vbox to \ht0{%
                      \hrule width 0.75\wd0%
                      \vfill%
                      \hrule width 0.75\wd0}%
                   \vrule height \ht0%
                   \kern 0.125\wd0}%
           }%
}
\DeclareMathOperator*\dlim{\underline{lim}}
\DeclareMathOperator*\ulim{\overline{lim}}

\everymath{\displaystyle}

% Grath
\usepackage{tikz}
\usetikzlibrary{positioning}
\usetikzlibrary{decorations.pathmorphing}
\tikzset{snake/.style={decorate, decoration=snake}}
\tikzset{node/.style={circle, draw=black!60, fill=white!5, very thick, minimum size=7mm}}

\title{Видео-карточки. Теория}
\author{Александр Сергеев}
\date{}
\begin{document}
\maketitle
\section{Общие штуки}
Платформы(Intel, AMD, NVIDIA) содержат девайсы(видяха 1, видяха 2)
\begin{lstlisting}{language=cpp}
clGetPlatformIDs(NULL, 0, &sz); //return memory size needed in sz
clGetPlatformIDs(buffer, buffer_size, NULL); //return platform list in buffer
\end{lstlisting}
Возвращает переменное количество аргументов
\begin{lstlisting}{language=cpp}
#include <CL/cl.h>  //minimal needed header
\end{lstlisting}
Виды функций в cl:
\begin{enumerate}
    \item возвращает код ошибки
    \item функции \textit{clCreate*}: возвращает объект, код ошибки по указателю
\end{enumerate}
\begin{lstlisting}{language=cpp}
clGetPlatformInfo(...); //get platform info
clGetDeviceIDs(platform, ...);
clGetDeviceInfo(device, ...);
\end{lstlisting}
\section{Создание контекста}
\begin{lstlisting}{language=cpp}
clCreateContext(...);
id = clCreateProgramWithSource(...);    //load files in context
err = clBuildProgram(id, device_list, build_options, ...);   //compile file $id for devices from $device_list and link it to program $id, build_options = "...", not NULL
clGetProgramBuildInfo(...);
\end{lstlisting}
\section{Код}
\begin{lstlisting}{language=cl}
kernel void add(global const int *a, global const int *b, global int *c) {
    size_t x = get_global_id(0);      //0 -- номер координаты
    c[x] = a[x] + b[x];
}
\end{lstlisting}
\textit{kernel} -- точка входа
\begin{lstlisting}{language=cpp}
clCreateKernel(...);
\end{lstlisting}
\section{Память}
size\_t на девайсе != size\_t на хосте
\begin{lstlisting}{language=cpp}
cl_mem buf = clCreateBuffer(...);
clCreateCommandQueue(device, flags); //flags: profiling_info -- enable stats, out_of_order_execution_enable -- do not use it
clSetKernelArg(id, arg_n, buf, buf_size);
clEnqueueWriteBuffer(buf, data, data_size, ...);  //flag: blocking_write. If not, 
clEnqueueNDRangeKernel(dimentions, &global_work_size, offset=NULL, local_work_size=NULL);             //dimentions = 1, global_work_size = 1, local_work_size=NULL -- автоматическое разбиение
clEnqueueReadBuffer(...);  //flag: blocking_read
clReleaseMemObject(buf);
\end{lstlisting}
Можно сделать запись и исполнение неблокирующими, а чтение -- блокирующим\\
Блокирующие операции запускают очередь\\
Т.к. действия выполняются последовательно, то мы заблокируемся до конца исполнения\\
CreateBuffer -- ленивый, т.е. память создается в момент использования
\section{Понятия}
Work Item -- логический исполнитель\\
SINT -- single instruction, multiple threads\\
Ядра в видяхи $\sim$ конвейеры в процессоре: умеют считать, но не более\\
Пачка тредов(Warp) исполняется с единым IP\\
Если треды наткнулись на if, то исполняется и if, и else, но результат применяется только в тех тредах, для которых if актуален (остальные треды простаивают)\\
Если ни одному треду не надо входить в if, то ок\\
В видяхах есть кэш, который используется, чтобы уменьшить передачу данных и сократить энергопотребление (дешевле, чем жирная шина)\\
У видеокарт есть новый тип памяти: shared\\
Эта память доступна для всего исполнителя\\
Ее больше, чем регистров, но меньше, чем оперативки\\
Локальная группа -- это набор тредов, исполняемых на одном исполнителе\\
Локальная группа состоит из нескольких warp'ов. Из них в каждый момент времени исполняется только один\\
Смена warp'ов происходит в момент обращения к памяти и прочей тяжелой фигне\\
(хотя на одном исполнителе может быть несколько локальных групп)\\
Все треды, относящиеся к одной локальной группе, будут исполняться на одном локальном исполнителе\\
Shared память запрашивается в момент запуаска: если попросил слишком много, то локальная группа даже не запустится\\
Заметим, что в видяхах память оптимизирована на запись, а не на чтение, поэтому оптимальнее иметь больше warp'ов на исполнителе\\
Заметим, что регистры 32битные регистры, а значит использование 64битных значений съедает регистры (ценный ресурс так-то)\\
В OpenCL регистры -- приватная память\\
Заметим, что регистры общие на исполнитель. Если мы <<съедаем>> слишком много, то вместо регистров начинает использоваться память, а она медленная\\
Поэтому регистры надо экономить\\
Нумерация тредов может быть 1-3 мерной\\
$Occupancy$ -- параметр, равный отношению количества загруженных ядер к количеству ядер вообще\\
При низком occupancy и высоком 
\section{Продвинутое произведение матриц}
Написав примитивное произведение матриц, мы получим маленькие flops\\
Это нормально\\
Давайте оптимизировать\\
В тупом кернеле каждый тред будет читать по столбику и строке\\
Т.е. требуется $\Omega(n)$ памяти для вычисления $n$ столбцов\\
Давайте кэшировать строки и столбцы, чтобы соседние треды использовали этот кэш\\
Поместим их в локалькую память перед использованием\\
Локальной памяти $\geq 32KB$ с версии 1.2\\
Однако этого может не хватить даже на строку\\
Поэтому давайте кэшировать блоками(TILE). Сначала первые $m$ элементов строки и столбца, потом следующие $m$ блоков
\begin{enumerate}
    \item Разобьем треды на локальные группы\\
    Удобнее всего делать группу квадратной
    \item Задефайним размер  в девайс-коде (дефайн удобно, т.к. можно менять при компиляции)
    \item Теперь наш внутренний цикл по $k$ разобьется на два: читаем данные в кэш, потом вычисляем
    \item Переносить из глобальной памяти в локальную можно всей локальной группы. Если завести буфер размера TILE * TILE, то каждый поток перенесет ровно по одному элементу
    \item Не забываем ставить барьер после переноса
\end{enumerate}
В OpenCL требуется, чтобы размер глобальной группы был кратен размеру локальной группы\\
Если размеры матрицы не кратны размеру локальной группы, то можно расширить матрицу\\
Другой вариант -- написать if при переносе из глобальной памяти в локальную\\
Теперь оптимизируем сами обращения\\
Вспоминаем, что при запросе к памяти обычно возвращается не байтик, а сразу пачка\\
Если несколько соседних тредов в варпе посылают запросы на соседние ячейки, то они объединяются в один запрос\\
В OpenCL треды с соседними $x$ обычно расположены рядом\\
Для локальной памяти справедливы те же соображения\\
Локальная память состоит из банков памяти\\
Они чередуются где-то через 4 байта\\
Лучше ходить в разные банки\\
Отсюда следует, что лучше читать сплошные куски памяти (т.к. в таком случае распределение будет равномернее)\\
Еще одна причина не итерироваться по $y$ (шанс попасть в один банк)\\
Однако если несколько тредов делают запрос в одну ячейку, то это все еще один запрос\\\\
//todo:
\begin{enumerate}
    \item zero cost copy
    \item host memory
    \item const memory
\end{enumerate}
\section{Векторные типы данных}
$float4$ -- векторный тип данных из 4 флотов\\
Нотации: \begin{enumerate}
    \item $b.x, b.y, b.z, b.w$
    \item $b.r, b.g, b.b, b.a$
    \item $b.s0, b.s1, b.s2, b.s3$
\end{enumerate}
Классический способ обращения по индексам:
\begin{lstlisting}[language=c]
    union {
        float4 vect;
        float[4] arr;
    }
\end{lstlisting}
Есть поддержка операторов:
\begin{lstlisting}[language=c]
    float4 + float4; //vector sum
    float4 + float = float + float4; //add float to each coord
\end{lstlisting}
Не забываем ставить $f$ в конце констант! Double работают очень медленно, не используй их\\
\begin{lstlisting}[language=c]
    float4.xy == float2; //first 2 coords
    float4.zxy;
    float.xxz;
\end{lstlisting}
Аналогично для остальных типов\\
Можно вызывать арифметические операции, sin\\
При вызове <=> на float4 получаем int4, где char.si = 0 при a.si <=> b.si == false и -1 при true\\
При сравнении double4 получаем long4 и т.д., т.е. sizeof(x) == sizeof(x <=> x)\\
На видеокарточке эти типы нужны чисто для удобства: выигрыша в производительности нет\\
На процессоре эти операции действительно могут преобразоваться в векторные инструкции(не факт)\\
Замечание: по стандарту алайнмент элемента должен быть кратен размеру\\
Т.е. float* -> float4* -- ub\\
Поэтому при перемещении данных из глобальной в локалькую память надо использовать специальные методы: vloadn, vstoren\\

Еще замечание: в opencl есть 3 адресных пространства: private, global, local\\
Поэтому и указатели бывают 3 типов\\

\end{document}