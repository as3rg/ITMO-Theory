\documentclass[12pt]{article}
\usepackage{bbold}
\usepackage{amsfonts}
\usepackage{amsmath}
\usepackage{amssymb}
\usepackage{color}
\setlength{\columnseprule}{1pt}
\usepackage[utf8]{inputenc}
\usepackage[T2A]{fontenc}
\usepackage[english, russian]{babel}
\usepackage{graphicx}
\usepackage{hyperref}
\usepackage{mathdots}
\usepackage{xfrac}


\def\columnseprulecolor{\color{black}}

\graphicspath{ {./resources/} }


\usepackage{listings}
\usepackage{xcolor}
\definecolor{codegreen}{rgb}{0,0.6,0}
\definecolor{codegray}{rgb}{0.5,0.5,0.5}
\definecolor{codepurple}{rgb}{0.58,0,0.82}
\definecolor{backcolour}{rgb}{0.95,0.95,0.92}
\lstdefinestyle{mystyle}{
    backgroundcolor=\color{backcolour},   
    commentstyle=\color{codegreen},
    keywordstyle=\color{magenta},
    numberstyle=\tiny\color{codegray},
    stringstyle=\color{codepurple},
    basicstyle=\ttfamily\footnotesize,
    breakatwhitespace=false,         
    breaklines=true,                 
    captionpos=b,                    
    keepspaces=true,                 
    numbers=left,                    
    numbersep=5pt,                  
    showspaces=false,                
    showstringspaces=false,
    showtabs=false,                  
    tabsize=2
}

\lstset{extendedchars=\true}
\lstset{style=mystyle}

\newcommand\0{\mathbb{0}}
\newcommand{\eps}{\varepsilon}
\newcommand\overdot{\overset{\bullet}}
\DeclareMathOperator{\sign}{sign}
\DeclareMathOperator{\re}{Re}
\DeclareMathOperator{\im}{Im}
\DeclareMathOperator{\Arg}{Arg}
\DeclareMathOperator{\const}{const}
\DeclareMathOperator{\rg}{rg}
\DeclareMathOperator{\Span}{span}
\DeclareMathOperator{\alt}{alt}
\DeclareMathOperator{\Sim}{sim}
\DeclareMathOperator{\inv}{inv}
\DeclareMathOperator{\dist}{dist}
\newcommand\1{\mathbb{1}}
\newcommand\ul{\underline}
\renewcommand{\bf}{\textbf}
\renewcommand{\it}{\textit}
\newcommand\vect{\overrightarrow}
\newcommand{\nm}{\operatorname}
\DeclareMathOperator{\df}{d}
\DeclareMathOperator{\tr}{tr}
\newcommand{\bb}{\mathbb}
\newcommand{\lan}{\langle}
\newcommand{\ran}{\rangle}
\newcommand{\an}[2]{\lan #1, #2 \ran}
\newcommand{\fall}{\forall\,}
\newcommand{\ex}{\exists\,}
\newcommand{\lto}{\leftarrow}
\newcommand{\xlto}{\xleftarrow}
\newcommand{\rto}{\rightarrow}
\newcommand{\xrto}{\xrightarrow}
\newcommand{\uto}{\uparrow}
\newcommand{\dto}{\downarrow}
\newcommand{\lrto}{\leftrightarrow}
\newcommand{\llto}{\leftleftarrows}
\newcommand{\rrto}{\rightrightarrows}
\newcommand{\Lto}{\Leftarrow}
\newcommand{\Rto}{\Rightarrow}
\newcommand{\Uto}{\Uparrow}
\newcommand{\Dto}{\Downarrow}
\newcommand{\LRto}{\Leftrightarrow}
\newcommand{\Rset}{\bb{R}}
\newcommand{\Rex}{\overline{\bb{R}}}
\newcommand{\Cset}{\bb{C}}
\newcommand{\Nset}{\bb{N}}
\newcommand{\Qset}{\bb{Q}}
\newcommand{\Zset}{\bb{Z}}
\newcommand{\Bset}{\bb{B}}
\renewcommand{\ker}{\nm{Ker}}
\renewcommand{\span}{\nm{span}}
\newcommand{\Def}{\nm{def}}
\newcommand{\mc}{\mathcal}
\newcommand{\mcA}{\mc{A}}
\newcommand{\mcB}{\mc{B}}
\newcommand{\mcC}{\mc{C}}
\newcommand{\mcD}{\mc{D}}
\newcommand{\mcJ}{\mc{J}}
\newcommand{\mcT}{\mc{T}}
\newcommand{\us}{\underset}
\newcommand{\os}{\overset}
\newcommand{\ol}{\overline}
\newcommand{\ot}{\widetilde}
\newcommand{\vl}{\Biggr|}
\newcommand{\ub}[2]{\underbrace{#2}_{#1}}

\def\letus{%
    \mathord{\setbox0=\hbox{$\exists$}%
             \hbox{\kern 0.125\wd0%
                   \vbox to \ht0{%
                      \hrule width 0.75\wd0%
                      \vfill%
                      \hrule width 0.75\wd0}%
                   \vrule height \ht0%
                   \kern 0.125\wd0}%
           }%
}
\DeclareMathOperator*\dlim{\underline{lim}}
\DeclareMathOperator*\ulim{\overline{lim}}

\everymath{\displaystyle}

% Grath
\usepackage{tikz}
\usetikzlibrary{positioning}
\usetikzlibrary{decorations.pathmorphing}
\tikzset{snake/.style={decorate, decoration=snake}}
\tikzset{node/.style={circle, draw=black!60, fill=white!5, very thick, minimum size=7mm}}

\title{Теория вероятности. Теория}
\author{Александр Сергеев}
\date{}
\begin{document}
\maketitle
\section{Вероятностное пространство. Вероятность и ее свойство}
\textbf{Определение}\\
\textit{Алгебра событий}:\\
$\Omega$ -- множество элементарных исходов\\
$\mcA$ -- набор подмножеств $\Omega$\\
$\mcA$ -- алгебра, если
\begin{enumerate}
    \item $\Omega \in \mcA$
    \item $A \in \mcA \Rto \ol A = \Omega \setminus A \in \mcA$
    \item $A, B \in \mcA \Rto A \cup B = A + B \in \mcA$
\end{enumerate}
Элементы алегбры -- \textit{события}\\
\textbf{Операции с событиями}
\begin{enumerate}
    \item $A \cup B = A + B$
    \item $A \cap B = AB = \ol A + \ol B$
    \item $\ol A = \Omega \setminus A$
    \item $A \setminus B = A - B = A\ol B$
\end{enumerate}
\textbf{Определение}\\
$\sigma$-алгебра\\
$\mcA$ -- сигма-алгебра
\begin{enumerate}
    \item $\mcA$ -- алгебра
    \item $A_1, A_2, \ldots \in \mcA \Rto \bigcup A_i \in \mcA$
\end{enumerate}
\textbf{Определение}\\
События $A, B$ -- \textit{несовместные} $AB = \varnothing$\\
Набор несовместный, если события попарно несовместные\\
\textbf{Определение(вероятностное пространство)}\\
$\Omega$ -- множество элементарных исходов\\
$\mcA$ -- сигма-алгебра\\
$P: \mcA \rto \Rset$ -- вероятность, если
\begin{enumerate}
    \item $P(A) \geq 0$
    \item $P(\Omega) = 1$
    \item $P(\bigcup A_i) = \sum P(A_i)$
\end{enumerate} 
\textbf{Определение(вероятностное пространство в широком смысле)}
$\Omega$ -- множество элементарных исходов\\
$\mcA$ -- алгебра\\
$P: \mcA \rto \Rset$ -- вероятность, если
\begin{enumerate}
    \item $P(A) \geq 0$
    \item $P(\Omega) = 1$
    \item $\bigcup^\infty A_i \Rto P(\bigcup^\infty A_i) = \sum P(A_i)$
\end{enumerate}
\textbf{Теорема о продолжении меры}\\
$\lan \Omega, \mcA, P\ran$ -- вероятностное пространство в широком смысле\\
Тогда $\ex! Q: \sigma(\mcA) \rto \Rset$ -- вероятность, $Q\vl_{\mcA} = P$, где $\sigma(\mcA)$ -- сигма-алгебра, содержащая $\mcA$\\
\textbf{Определение}\\
$\mcA$ -- система интервалов на $\Rset$, замкнутая относительно конечного объединения и пересечения\\
$\mcB = \sigma(\mcA)$ -- борелевская сигма-алгебра\\
\textbf{Примеры вероятностных пространств}
\begin{enumerate}
    \item Модель классической вероятности\\
    $\Omega = \{\omega_1, \ldots, \omega_N\}$\\
    $\mcA = 2^\Omega$\\
    $P(\{\omega_i\}) = P(\omega_i) = \frac1N$\\
    $\mcA = \{\omega_{i_1}, \ldots, \omega_{i_M}\} \Rto P(\mcA) = \frac{M}{N}$
    \item $\Omega$ -- набор $\{0^i, 1\}, i \in \Nset_0 = \{0, 1, 2, 3, \ldots\}$\\
    $P(0^i1)=q^ip$
    \item Модель геометрической вероятности
    $\Omega$ -- ограниченное, измеримое по Лебегу множество\\
    $\mcA$ -- измеримое по Лебегу подмножество $\Omega$\\
    $P(A) = \frac{\lambda A}{\lambda \Omega}$
\end{enumerate}
\textbf{Теорема (свойство вероятности)}
\begin{enumerate}
    \item $A \subset B \Rto P(A) \leq P(B)$
    \item $P(A) \leq 1$
    \item $P(A) + P(\ol A) = 1$
    \item $P(A+B) = P(A) + P(B) - P(AB)$
    \item $P(\varnothing) = 0$
    \item $P(\bigcup A_i) \leq \sum P(A_i)$
\end{enumerate}
\textbf{Доказательство}
\begin{enumerate}
    \item $P(B) = P(A) + P(B - A)$
    \item $A \subset \Omega \Rto P(A) \leq 1$
    \item $A \sqcup \ol A = \Omega$
    \item $B = AB \sqcup (B \setminus AB)$
    \item $B_1 = A_1$
    $B_2 = A_2 \setminus A_1$\\
    $B_n = A_n \setminus (A_1 \cup \ldots A_{n-1})$
    $\bigsqcup B_i = \bigcup A_i$\\
    $B_i \subset A_i$\\
    $P(\bigcup A_i) = P(\bigsqcup B_i) = \sum P(B_i) \leq \sum P(A_i)$
\end{enumerate}
\textbf{Теорема (формула включения/исключения)}\\
$P(A_1 + \ldots + A_n) = \sum_i P(A_i) - \sum_{i < j} P(A_iA_j) + \sum_{i<j<k} P(A_iA_jA_k) + \ldots (-1)^{n+1} \sum_{i_1 < \ldots < i_n} P(A_{i_1} \ldots A_{i_n})$
\textbf{Доказательство}\\
Доказательство по индукции\\
\end{document}