\documentclass[12pt]{article}
\usepackage{bbold}
\usepackage{amsfonts}
\usepackage{amsmath}
\usepackage{amssymb}
\usepackage{color}
\setlength{\columnseprule}{1pt}
\usepackage[utf8]{inputenc}
\usepackage[T2A]{fontenc}
\usepackage[english, russian]{babel}
\usepackage{graphicx}
\usepackage{hyperref}
\usepackage{mathdots}
\usepackage{xfrac}


\def\columnseprulecolor{\color{black}}

\graphicspath{ {./resources/} }


\usepackage{listings}
\usepackage{xcolor}
\definecolor{codegreen}{rgb}{0,0.6,0}
\definecolor{codegray}{rgb}{0.5,0.5,0.5}
\definecolor{codepurple}{rgb}{0.58,0,0.82}
\definecolor{backcolour}{rgb}{0.95,0.95,0.92}
\lstdefinestyle{mystyle}{
    backgroundcolor=\color{backcolour},   
    commentstyle=\color{codegreen},
    keywordstyle=\color{magenta},
    numberstyle=\tiny\color{codegray},
    stringstyle=\color{codepurple},
    basicstyle=\ttfamily\footnotesize,
    breakatwhitespace=false,         
    breaklines=true,                 
    captionpos=b,                    
    keepspaces=true,                 
    numbers=left,                    
    numbersep=5pt,                  
    showspaces=false,                
    showstringspaces=false,
    showtabs=false,                  
    tabsize=2
}

\lstset{extendedchars=\true}
\lstset{style=mystyle}

\newcommand\0{\mathbb{0}}
\newcommand{\eps}{\varepsilon}
\newcommand\overdot{\overset{\bullet}}
\DeclareMathOperator{\sign}{sign}
\DeclareMathOperator{\re}{Re}
\DeclareMathOperator{\im}{Im}
\DeclareMathOperator{\Arg}{Arg}
\DeclareMathOperator{\const}{const}
\DeclareMathOperator{\rg}{rg}
\DeclareMathOperator{\Span}{span}
\DeclareMathOperator{\alt}{alt}
\DeclareMathOperator{\Sim}{sim}
\DeclareMathOperator{\inv}{inv}
\DeclareMathOperator{\dist}{dist}
\newcommand\1{\mathbb{1}}
\newcommand\ul{\underline}
\renewcommand{\bf}{\textbf}
\renewcommand{\it}{\textit}
\newcommand\vect{\overrightarrow}
\newcommand{\nm}{\operatorname}
\DeclareMathOperator{\df}{d}
\DeclareMathOperator{\tr}{tr}
\newcommand{\bb}{\mathbb}
\newcommand{\lan}{\langle}
\newcommand{\ran}{\rangle}
\newcommand{\an}[2]{\lan #1, #2 \ran}
\newcommand{\fall}{\forall\,}
\newcommand{\ex}{\exists\,}
\newcommand{\lto}{\leftarrow}
\newcommand{\xlto}{\xleftarrow}
\newcommand{\rto}{\rightarrow}
\newcommand{\xrto}{\xrightarrow}
\newcommand{\uto}{\uparrow}
\newcommand{\dto}{\downarrow}
\newcommand{\lrto}{\leftrightarrow}
\newcommand{\llto}{\leftleftarrows}
\newcommand{\rrto}{\rightrightarrows}
\newcommand{\Lto}{\Leftarrow}
\newcommand{\Rto}{\Rightarrow}
\newcommand{\Uto}{\Uparrow}
\newcommand{\Dto}{\Downarrow}
\newcommand{\LRto}{\Leftrightarrow}
\newcommand{\Rset}{\bb{R}}
\newcommand{\Rex}{\overline{\bb{R}}}
\newcommand{\Cset}{\bb{C}}
\newcommand{\Nset}{\bb{N}}
\newcommand{\Qset}{\bb{Q}}
\newcommand{\Zset}{\bb{Z}}
\newcommand{\Bset}{\bb{B}}
\renewcommand{\ker}{\nm{Ker}}
\renewcommand{\span}{\nm{span}}
\newcommand{\Def}{\nm{def}}
\newcommand{\mc}{\mathcal}
\newcommand{\mcA}{\mc{A}}
\newcommand{\mcB}{\mc{B}}
\newcommand{\mcC}{\mc{C}}
\newcommand{\mcD}{\mc{D}}
\newcommand{\mcJ}{\mc{J}}
\newcommand{\mcT}{\mc{T}}
\newcommand{\us}{\underset}
\newcommand{\os}{\overset}
\newcommand{\ol}{\overline}
\newcommand{\ot}{\widetilde}
\newcommand{\vl}{\Biggr|}
\newcommand{\ub}[2]{\underbrace{#2}_{#1}}

\def\letus{%
    \mathord{\setbox0=\hbox{$\exists$}%
             \hbox{\kern 0.125\wd0%
                   \vbox to \ht0{%
                      \hrule width 0.75\wd0%
                      \vfill%
                      \hrule width 0.75\wd0}%
                   \vrule height \ht0%
                   \kern 0.125\wd0}%
           }%
}
\DeclareMathOperator*\dlim{\underline{lim}}
\DeclareMathOperator*\ulim{\overline{lim}}

\everymath{\displaystyle}

% Grath
\usepackage{tikz}
\usetikzlibrary{positioning}
\usetikzlibrary{decorations.pathmorphing}
\tikzset{snake/.style={decorate, decoration=snake}}
\tikzset{node/.style={circle, draw=black!60, fill=white!5, very thick, minimum size=7mm}}

\title{Математический анализ. Теория}
\author{Александр Сергеев}
\date{}

\begin{document}
\maketitle
\section{Интеграл}
\subsection{Неопределенный интеграл}
\textbf{Определение}\\
$f,F: \lan a, b \ran \rto \Rset$\\
$F$ -- \textit{первообразная} функции $f$, если $F$ дифференцируема на $\an ab$ и $\fall\,x\in \lan a, b\ran\ F'(x) = f(x)$\\
\textbf{Теорема 1}\\
Если $f$ непрерывна на $\lan a, b \ran$, то первообразная существует\\
\textbf{Теорема 2}\\
Пусть $F$ - первообразная $f$ на $\lan a, b\ran$\\
Тогда
\begin{enumerate}
    \item $\fall c \in \Rset\ F+c$ - тоже первообразная
    \item Если $G$ - первообразная, то $G-F = \const$
\end{enumerate}
\textbf{Определение}\\
\textit{Неопределенный} интеграл на $\lan a, b \ran$ -- множество всех первообразных\\
$\int f(x)\df x = \{ F: F'=f \}$\\
\textbf{Таблица первообразных}\\
$\int x^P \df x = \frac{x^{P+1}}{P+1} + C, P \neq -1$\\
$\int \frac1x \df x = \ln |x| + C$\\
$\int e^x\df x = e^x + C$\\
$\int a^x\df x = \frac{a^x}{\ln a}$\\
$\int \sin x \df x = -\cos x + C$\\
$\int \cos x \df x = \sin x + C$\\
$\int \frac1{\cos^2 x}\df x = \tg x + C$\\
$\int \frac1{\sin^2 x}\df x = -\ctg x + C$\\
$\int \frac{\df x}{x^2 + 1} = \arctg x + C$\\
$\int \frac{dx}{1-x^2} = \frac12 \ln |\frac{1+x}{1-x}| + C = \nm{arcth} x + C$\\
$\int \frac{\df x}{\sqrt{1-x^2}} = \arcsin x + C$\\
$\int \frac{\df x}{\sqrt{1+x^2}} = \ln |x+\sqrt{1+x^2}| + C= \nm{arcsh} x + C$ -- "длинный логарифм"\\\\
\textbf{Гиперболические функции}\\
$e^{ix} = \cos x + i\sin x$ - из ряда Тейлора\\
$\cos x = \frac{e^{ix}+e^{-ix}}2$\\
$\sin x = \frac{e^{ix}-e^{-ix}}{2i}$\\
$\ch x = \frac{e^{x}+e^{-x}}2$ -- гиперболический косинус\\
$\sh x = \frac{e^{x}-e^{-x}}2$ -- гиперболический синус\\
$\ch^2 x - \sh^2 x = 1$\\
$\sh 2x = 2\sh x \ch x$\\\\
\textbf{Теорема о свойствах неопределенного интеграла}\\
Пусть $f, g$ - имеют первообразные на $\lan a, b \ran$\\
Тогда
\begin{enumerate}
    \item $\int f + g = \int f + \int g$\\
    $\int af = a\int f$
    \item Пусть $\phi: \lan p, q \ran \rto \lan a, b \ran$\\
    $\int f(\phi(t))\phi'(t)\df t = \int f(x)\df x |_{x:= \phi(t)} = F(\phi(t)) + C$\\
    \textbf{Замечание}\\
    Пусть $\phi$ обратима\\
    Тогда $F(x) = \int f(\phi(t))\phi'(t)\df t |_{t:= \phi^{-1}(x)}$
    \item $\fall \alpha \neq 0, \beta \in \Rset$\\
    $\int f(\alpha x + \beta)\df x = \frac1\alpha F(\alpha x + \beta) + C$
    \item $f, g$ -- дифференцируемы и $f'g$ имеет первообразную\\
    Тогда $fg'$ имеет первообразную\\
    $\int fg' = fg - \int f'g$
\end{enumerate}
\textbf{Определение}\\
Дифференциал $\df \phi(x) = \phi'(x)\df x$\\
\subsection{Правило Лопиталя}
\textbf{Лемма об ускоренной сходимости}\\
$f, g: D \subset \Rset \rto \Rset, a\in \Rex$ - предельная точка $D$\\
Пусть $\ex U(a):\ f,g\neq 0$ в $\overdot{U}(a)$\\
$\lim_{x\rto a} f(x) = 0, \lim_{x\rto a} g(x) = 0$\\
Тогда $\fall x_k: \begin{array}{l}
     x_k \rto a\\
     x_k \in D\\
     x_k \neq a
\end{array} \ex y_k: \begin{array}{cc}
     y_k \rto a\\
     y_k \in D\\
     y_k \neq a\\
     \lim_{k \rto \infty} \frac{f(y_k)}{f(x_k)} = 0\\
     \lim_{k \rto \infty} \frac{f(y_k)}{g(x_k)} = 0
\end{array}$\\
\textbf{Доказательство}\\
Выберем $y_k$ как подпоследовательность $x_k$\\
$\fall k \frac{f(x_l)}{f(x_k)},\frac{f(x_l)}{g(x_k)} \xrto[l\rto \infty]{}$\\
Тогда $\ex l_0: |\frac{f(x_{l_0})}{f(x_k)}|,|\frac{f(x_{l_0})}{g(x_k)}| < \frac1k$\\
Отсюда $y_k := x_{l_0}$\\\\
\textbf{Правило Лопиталя}\\
$f,g: \lan a, b \ran \rto \Rset$ -- дифференцируемы на $(a,b), a \in \Rex$\\
$\lim_{x\rto a} \frac{f(x)}{g(x)}$ - неопределенность\\
Пусть $\lim_{x\rto a}\frac{f'(x)}{g'(x)} = A \in \Rex$\\
Тогда $\ex \lim_{x\rto a} \frac{f(x)}{g(x)} = A$\\
\textbf{Доказательство}\\
По Гейне\\
Возьмем $x_k: \begin{array}{l}
     x_k \rto a\\
     x_k \in (a, b)\\
     x_k \neq a
\end{array}$\\
Из леммы берем $y_k: \begin{array}{l}
     y_k \rto a\\
     y_k \in (a, b)\\
     y_k \neq a
\end{array}$\\
По т. Коши $\frac{f(x_k)-f(y_k)}{g(x_k)-g(y_k)} = \frac{f'(\xi_k)}{g'(\xi_k)}, \xi$ -- между $x_k$ и $y_k$ (т.е. $\xi_k \rto a$)\\
$\frac{f(x_k)}{g(x_k)} = \frac{f(y_k)}{g(x_k)} + \frac{f'(\xi_k)}{g'(\xi_k)}(1-\frac{g(y_k)}{g(x_k)})$\\
$\frac{f(x_k)}{g(x_k)} \xrto[k\rto \infty]{} \frac{f'(\xi_k)}{g'(\xi_k)} = A$\\
\textbf{Замечание}\\
\underline{Работает только на неопределенностях}\\
\textbf{Следствие}\\
$\frac{x}{\ln x} \xrto[x\rto+\infty]{} +\infty$\\
$\frac{x^a}{e^x} = (\frac{x}{e^{\frac{x}{n}}})^a \xrto[x\rto +\infty]{} 0$\\\\
\textbf{Теорема Штольца}\\
$x_n, y_n \rto 0, y_n$ -- строго монотонная\\
$\lim_{n \rto \infty} \frac{x_n-x_{n-1}}{y_n-y_{n-1}} = a \in \Rex$\\
Тогда $\frac{x_n}{y_n} \rto a$\\
\textbf{Замечание}\\
При $a = 0$ требуем монотонность $x_n$\\
\textbf{Замечание}\\
При $x_n, y_n \rto \infty$ теорема тоже верна\\
\textbf{Доказательство}
\begin{enumerate}
    \item $a > 0, a \in \Rset$\\
    \textbf{Утверждение}\\
    $p < \frac ab,\frac cd < q \Rightarrow p < \frac{a + c}{b + d} < q$\\\\
    $\fall 0 < \eps < a\ \ex N_1\ \fall n > N \geq N_1\ a-\eps < \frac{x_{N+1} - x_N}{y_{N+1} - y_N} < a + \eps$\\
    $\fall 0 < \eps < a\ \ex N_1\ \fall n > N \geq N_1\ a-\eps < \frac{x_{N+2} - x_{N+1}}{y_{N+2} - y_{N+1}} < a + \eps$\\
    $\vdots$\\
    $\fall 0 < \eps < a\ \ex N_1\ \fall n > N \geq N_1\ a-\eps < \frac{x_n - x_{n-1}}{y_n - y_{n-1}} < a + \eps$\\
    Отсюда $\fall 0 < \eps < a\ \ex N_1\ \fall n > N \geq N_1\ a-\eps < \frac{x_n - x_N}{y_n - y_N} < a + \eps$\\
    Устремим $n$ к $\infty$\\
    Тогда $a-\eps < \frac{x_N}{y_N} < a + \eps$, т.е. $\frac{x_N}{y_N} \rto a$
    \item $a < 0$ -- аналогично
    \item $a = \pm\infty$\\
    Аналогично $\frac{x_N}{y_N} \rto a$
    \item $a = 0$ (потребуем монотонность $x_n$)\\
    Пусть $\frac{x_n-x_{n-1}}{y_n-y_{n-1}} \rto +\infty$\\
    Перевернем дробь. Через доказанное выше
\end{enumerate}
\textbf{Упражнение}\\
Посчитаем $1^k + 2^k + \ldots + n^k$\\
Рассмотрим функцию $f(x) = 1 + x + x^2 + \ldots + x^n = \frac{x^{n+1}-1}{x-1}$\\
$x\frac{\df}{\df x} f(x) = x + 2x^2 + \ldots + nx^n$\\
$(x\frac{\df}{\df x})^2 f(x) = x + 2^2x^2 + \ldots + n^2x^n$\\
$(x\frac{\df}{\df x})^k f(x) = x + 2^kx^2 + \ldots + n^kx^n$\\
Отсюда $1^k + 2^k + \ldots + n^k = ((x\frac{\df}{\df x})^k f)(1)$\\
Заметим, что в результате дифференцирования мы получим дробь, знаменатель которой при подстановке 1 обращается в 0\\
Но данный дефект возникает лишь из-за формы записи. На самом деле функция непрерывна, поэтому можно взять ее предел в 1\\
$1^k + 2^k + \ldots + n^k = \lim_{x \rto 1} (x\frac{\df}{\df x})^k f$\\
Применим правило Лопиталя\\
$1^k + 2^k + \ldots + n^k = (\frac1{(k+1)!}(\frac{\df}{\df x})^k (x-1)^{k+1}(x\frac{\df}{\df x})^k \frac{x^{n+1}-1}{x-1}) (1)$
\subsection{Определенный интеграл}
\textbf{Определение}\\
Пусть $\eps$ - множество ограниченных плоских фигур\\
$\sigma: \eps \rto [0, +\infty)$ - \textit{площадь}, если
\begin{enumerate}
    \item Аддитивность -- $A = A_1 \sqcup A_2 \Rto \sigma(A) = \sigma(A_1)+\sigma(A_2)$, где $\sqcup$ -- дизъюнктное объединение (объединение непересекающихся фигур)
    \item Нормировка -- $\sigma([a,b]\times[c,d]) = (b-a)(d-c)$
\end{enumerate}
\textbf{Замечание}
\begin{enumerate}
    \item $A \subset B \in \eps \Rto \sigma A \leq \sigma B$\\
    \textbf{Доказательство}\\
    $B = A \sqcup (B \setminus A) \Rto \sigma(B) = \sigma(A) + \sigma(B\setminus A) \geq \sigma(A)$
    \item $A$ - вертикальный отрезок $\Rto \sigma(A) = 0$\\
    \textbf{Доказательство}\\
    Впишем отрезок в прямоугольник с шириной $\eps$\\
    Для любого $\eps$ это возможно. Отсюда площадь стремится к 0
    \item Мы не доказываем, что такая функция существует
\end{enumerate}
\textbf{Определение}\\
$\sigma: \eps \rto [0, +\infty)$ - \textit{ослабленная площадь}, если
\begin{enumerate}
    \item Монотонность:\\
    $E \subset D \Rto \sigma E \leq \sigma D$
    \item Нормировка
    \item Ослабленная аддитивность:\\
    Если $A = A_1 \cup A_2, A_1 \cap A_2 \subset$ вертикальный отрезок, $A_1, A_2$ лежат в разных полуплоскостях, то $\sigma A = \sigma A_1 + \sigma A_2$\\
    \textbf{UPD}\\
    Позже предыдущее утверждение было заменено на следующее:\\
    Если вертикальная прямая $l$ делит фигуру на $A$ на части $A_r$ и $A_r$(части могут иметь общие точки на $l$), то $\sigma A = \sigma A_l + \sigma A_r$
\end{enumerate}
\textbf{Примеры}
\begin{enumerate}
    \item $\sigma A := \inf \{ \sum_{k=1}^n S(P_k): A \subset \bigcup_{k=1}^n P_k\}$, где $P_k$ - прямоугольник
    \item $\sigma A := \inf \{ \sum_{k=1}^\infty S(P_k): A \subset \bigcup_{k=1}^\infty P_k\}$, где $P_k$ - прямоугольник\\
    \textbf{Эти площади разные}\\
    К примеру, рассмотрим $C = [0,1]^2 \cap \Qset \times \Qset$\\
    $\sigma_1(C) = 1$\\
    $\sigma_2(C) = 0$
\end{enumerate}
\textbf{Определение}\\
$f: \lan a,b\ran \rto \Rset$\\
Положительная срезка $f^+ = \max(f,0)$\\
Отрицательная срезка $f^- = \max(-f, 0)$\\
$f = f^+ - f^-$\\
$|f| = f^+ + f^-$\\
\textit{Подграфик} $(F,E) = \{ (x,y) \in \Rset^2: x \in E, 0 \leq y \leq f(x)\}$\\
\textbf{Определенный интеграл}\\
$f:[a,b] \rto \Rset$ -- непрерывная\\
Тогда \textit{определенный интеграл} f по $[a,b]$ - $\int_a^b f(x)\df x = \sigma(\text{ПГ}(f^+, [a,b]))-\sigma(\text{ПГ}(f^-, [a,b]))$, где $\sigma$ -- ослабленная площадь\\
\textbf{Замечания}
\begin{enumerate}
    \item $f\geq 0 \Rto \int_a^b f \geq 0$
    \item $f \equiv C \Rto \int_a^b f = C(b-a)$
    \item $\int_a^b -f = -\int_a^bf$
    \item Можно считать, что $\int_a^a f = 0$
\end{enumerate}
\textbf{Свойства}
\begin{enumerate}
    \item Аддитивность по промежутку\\
    $\int_a^b f = \int_a^c f + \int_c^b f$
    \item Монотонность\\
    $f \us{[a,b]}\leq g [a,b] \Rto \int_a^bf \leq \int_a^bg$
    \item $(b-a) \us{[a,b]}\min f \leq \int_a^b f \leq (b-a)\us{[a,b]}\max f$
    \item $|\int_a^b f| \leq \int_a^b |f|$
\end{enumerate}
\textbf{Теорема о среднем}\\
Пусть $f \in C[a,b]$\\
Тогда $\ex c \in [a,b]: \int_a^b f = f(c)(b-a)$\\
\textbf{Доказательство}\\
Если $a = b$ -- тривиально\\
Иначе по утверждению 3: $\min f \leq \frac{1}{b-a}\int_a^b f \leq \max f$\\
Т.к. $f$ принимает все значения между минимумом и максимумом, то $\ex c: f(c) = \frac{1}{b-a}\int_a^b f$\\\\
\textbf{Определение}\\
Пусть $f \in C[a,b]$\\
$\Phi: [a,b] \to \Rset, \Phi(x) = \int_a^x f$ -- интеграл с переменным верхним пределом\\
$\Psi: [a,b] \to \Rset, \Psi(x) = \int_x^b f$ -- интеграл с переменным нижним пределом\\
\textbf{Теорема Барроу}\\
$f\in C[a,b], \Phi$ -- интеграл с переменным верхним пределом\\
Тогда $\Phi$ дифференцируема на $[a,b]$ и $\Phi'(x) = f(x)$\\
\textbf{Доказательство}\\
Пусть $x \in (a,b), y > x$\\
$\Phi_+' = \lim_{y \rto x+0} \frac{\Phi(y) - \Phi(x)}{y-x} = \lim_{y\rto x+0}\frac{\int_x^y f}{y-x} = \lim_{y\rto x+0}f(c), c\in [x,y] = f(x)$\\
Аналогично $\Phi_-' = f(x)$\\
Тогда $\Phi'(x) = f(x)$\\
\textbf{Замечание}\\
$\Psi'(x) = -f(x)$\\
\textbf{Теорема (Формула Ньютона-Лейбница)}\\
Пусть $F$ -- первообразная $f$ на $[a,b], f \in C[a,b]$\\
Тогда $\int_a^b f = F(b)-F(a)$\\
\textbf{Доказательство}\\
По т. Барроу $\Phi$ -- первообразная\\
Тогда $F = \Phi + C$\\
$\int_a^b f = \Phi(b) = \Phi(b) - \Phi(a) = (F(b) + C) - (F(a) + C) = F(b) - F(a)$\\
\textbf{Замечание}\\
Теорема Барроу -- это теорема о существовании первообразной у непрерывной функции\\
Также площадь под графиком непрерывной функции не зависит от $\sigma$\\
\textbf{Соглашение}\\
При $c > d\ \int_c^d f := - \int_d^c f$\\
\textbf{Свойства}
\begin{enumerate}
    \item Линейность\\
    $f, g \in C[a,b], \alpha, \beta \in \Rset$\\
    $\int_a^b (\alpha f + \beta g) = \alpha \int_a^b f + \beta \int_a^b$\\
    \textbf{Пример (неравенство Чебышева)}\\
    $f, g \in C[a,b]$ -- монотонно возрастающие/убывающие\\
    $I_f := \frac1{b-a}\int_a^b f$ -- \textit{среднее значение функции}\\
    Тогда $I_f I_g \leq I_{fg}$\\
    \textbf{Доказательство}\\
    $x,y \in [a,b]$\\
    Тогда $(f(x)-f(y))(g(x)-g(y)) \geq 0$ -- из монотонности\\
    $f(x)g(x) - f(y)g(x)-f(x)g(y)+f(y)g(y) \geq 0$\\
    Проинтегрируем по $x$ по $[a,b]$:\\
    $\int_a^b f(x)g(x) - f(y)\int_a^b g(x) - f(y)g(y)(b-a) \geq 0$\\
    Поделим на $b-a$:\\
    $I_{fg} - f(y)I_g - I_f g(y) + f(y)g(y) \geq 0$\\
    Проинтегрируем по $y$ по $[a,b]$ и поделим на $b-a$:\\
    $I_{fg} - I_fI_g-I_fI_g + I_{fg} \geq 0$\\
    $I_{fg} \geq I_fI_g$
    \item Интегрирование по частям\\
    $f, g \in C^1[a,b]$\\
    $\int_a^b fg' = fg\vl_a^b - \int_a^b gf'$\\
    \textbf{Пример}\\
    $H_n = \frac{1}{n!}\int_{-\frac\pi2}^{\frac\pi2}(\frac{\pi^2}4 - x^2)^n\cos x \df x = F_n(\pi^2)$ -- какой-то многочлен степени $n$\\
    % $H_n = \frac{1}{n!}\int_{-\frac\pi2}^{\frac\pi2}(\frac{\pi^2}4 - x^2)^n\cos x \df x = \begin{bmatrix}
    % f = (\frac{\pi^2}4 - x^2)^n && g' = \cos x\\
    % f' = n(\frac{\pi^2}4 - x^2)^{n-1}(-2x) && g = \sin x
    % \end{bmatrix} = \frac{1}{n!}(\frac{\pi^2}4 - x^2)^{n^2}\sin x \vl_{-\frac\pi2}^{\frac\pi2} + \frac{2}{(n-1)!}\int_{-\frac\pi2}^{\frac\pi2}(\frac{\pi^2}4 - x^2)^{n-1}x\sin x \df x = \begin{bmatrix}
    %     g' = \sin x && f = (\frac{\pi^2}4 - x^2)^{n-1}x\\
    %     g = -\cos x && f' = 2(n-1)(\frac{\pi^2}4 - x^2)^{n-2}x^2 + (\frac{\pi^2}4 - x^2)^{n-1)
    % \end{bmatrix}$
    Пример -- полный треш. Если вам надо, смотрите видео:\\
    \url{https://www.youtube.com/live/7ZQr_OKhuq4?feature=share&t=7020}\\
    Таймкод: 1:57:00
    \item Замена переменных в интеграле\\
    $f \in C\an ab$\\
    $\phi: \an ab \rto \an ab, \phi \in C^1$\\
    $[p,q] \in \an ab$\\
    Тогда $\int_p^q f(\phi(t))\phi'(t)\df t = \int_{\phi(p)}^{\phi(q)} f(x)\df x$\\
    \textbf{Доказательство}\\
    $F$ -- первообразная $f$\\
    $F(\phi(t))$ -- первообразная $f(\phi(t))\phi'(t)$\\
    $\int_p^q f(\phi(t))\phi'(t)\df t = F(\phi(q)) - F(\phi(p)) = \int_{\phi(p)}^{\phi(q)} f(x)\df x$\\
    \textbf{Замечание}
    \begin{enumerate}
        \item Может оказаться, что $\phi(p) > \phi(q)$
        \item $\phi[p,q]$ может быть крупнее $[\phi(p), \phi(q)]$
    \end{enumerate}
    \textbf{Пример (Формула Тейлора с остатком в интегральной форме)}\\
    $a,b \in \Rex, a < b, f \in C^{n+1}\an ab, x, x_0 \in \an ab$\\
    Тогда $f(x) = \sum_{k=0}^n \frac1{k!}f^{(k)}(x_0)(x-x_0)^k + \frac1{n!}\int_{x_0}^x (x-t)^n f^{(n+1)}(t)\df t$\\
    \textbf{Доказательство}\\
    Индукция по $n$:
    \begin{enumerate}
        \item $n = 0$:\\
        $f(x) = f(x_0) = \int_{x_0}^x f'(t)\df t$
        \item Интегрирование по частям\\
        Пусть доказано для $n$\\
        $\frac1{n!}\int_{x_0}^x (x-t)^n f^{(n+1)}(t)\df t = \begin{bmatrix}
        f = f^{(n+1)} && g' = (x-t)^n\\
        f' = f^{(n+2)} && g = -\frac{(x-t)^{n+1}}{n+1}
        \end{bmatrix} = -\frac1{n!}\frac{(x-t)^{n+1}}{n+1}f^{(n+1)}(t) \vl_{t=x_0}^{t=x} + \frac1{(n+1)!}\int_{x_0}^x (x-t)^{n+1}f^{(n+2)}(t)\df t = \frac{(x-x_0)^{n+1}}{(n+1)!}f^{(n+1)}(x_0) + \frac1{(n+1)!}\int_{x_0}^x (x-t)^{n+1}f^{(n+2)}(t)\df t$
    \end{enumerate}
    \textbf{Замечание}\\
    Формулу Тейлора можно интегрировать\\
    $F$ -- первообразная $f$\\
    Проинтегрируем слагаемое:\\
    $\int_{x_0}^x \frac{f^{(k)}(x_0)(x-x_0)^k}{k!} = \frac{f^{(k)}(x_0)(x-x_0)^{k+1}}{(k+1)!}\vl_{x=x_0}^{x=x} = \frac{f^{(k)}(x_0)(x-x_0)^{k+1}}{(k+1)!} = \frac{F^{(k+1)}(x_0)(x-x_0)^{k+1}}{(k+1)!}$\\
    \textbf{Пример}\\
    $\frac1{1+x^2}=1-x+x^2-\ldots + (-1)^nx^{2n} + o(x^{2n})$\\
    Тогда $\arctan x = F(0) + x - \frac{x^3}{3} + \frac{x^5}{5} - \ldots + o(x^{2n+1})$\\
\end{enumerate}
\textbf{Утверждение}\\
$\pi$ -- иррациональное (даже $\pi^2$ -- иррациональное)\\
\textbf{Доказательство}\\
Пусть $\pi^2 = \frac km$\\
Тогда $m^n F(\frac km)$ -- целое число, где $F$ -- из примера к интегрированию по частям\\
Отсюда $m^n F(\frac km) =  \frac{m^n}{n!}\int_{-\frac\pi2}^{\frac\pi2}(\frac{\pi^2}4 - x^2)^n\cos x \df x$ -- положительное целое число\\
Отсюда выражение $\geq 1$\\
$|\frac{m^n}{n!}\int_{-\frac\pi2}^{\frac\pi2}(\frac{\pi^2}4 - x^2)^n\cos x \df x| \leq \frac{m^n}{n!}\int_{-\frac\pi2}^{\frac\pi2}|(\frac{\pi^2}4 - x^2)^n\cos x| \df x \leq \frac{m^n 3^n \pi}{n!} \xrto[n \rto +\infty]{} 0$\\
\subsection{Продолжение свойств интеграла}
\textbf{Определение}
\begin{enumerate}
    \item $f: [a,b] \rto \Rset$ -- кусочно непрерывная, если функция непрерывна всюду, кроме конечного числа точек, где у нее разрыв 1 рода
    \item $f$ -- кусочно непрерывная функция\\
    $x_0 = a < x_1 < \ldots < x_n = b$ -- точки разрыва ($a,b$ могут и не быть разрывными)\\
    $\int_a^b f(x)\df x := \sum_{i=1}^n\int_{x_{i-1}}^{x_i} f$
    \item Пусть $f$ -- кусочно непрерывная на $[a,b]$\\
    $F:[a,b] \rto \Rset$ -- \textit{почти первообразная} функции $f$, если
    \begin{enumerate}
        \item $F$ -- непрерывна на $[a,b]$
        \item $F'(x) = f(x)$ при $x \in [a,b]$, кроме конечного числа точек
    \end{enumerate}
    Если $F_i$ -- первообразная $f$ на $[x_{i-1}, x_i]$\\
    Тогда $F(x) = \left[\begin{array}{ll}
         F_1(x),& x \in [x_0, x_1]\\
         F_2(x) + c_1,& x \in [x_1, x_2]\\
         \vdots\\
         F_n(x) + c_{n-1}, & x\in [x_{n-1}, x_n]
    \end{array}\right.$,\\
    где $c_i = F_i(x_i) - F_{i+1}(x_i)$
\end{enumerate}
\textbf{Утверждение}\\
Если $f$ -- кусочно непрерывная на $[a,b]$\\
$F$ -- первообразная\\
Тогда $\int_a^b f = F(b) - F(a)$\\
\textbf{Доказательство}\\
$\int_a^b f(x)\df x := \sum_{i=1}^n\int_{x_{i-1}}^{x_i} f = \sum_{i=1}^n F_i(x_i) - F_i(x_{i-1}) = \sum_{i=1}^n F(x_i) - F(x_{i-1}) = F(b) - F(a)$\\
\textbf{Пример}\\
Пусть $a_1 \leq a_2 \leq \ldots \leq a_n, b_1 \leq b_2 \leq \ldots \leq b_n$\\
Тогда $\frac{a_1 + a_2 + \ldots + a_n}{n}\frac{b_1 + \ldots + b_n}{n} \leq \frac{a_1b_1 + \ldots + a_nb_n}{n}$ -- неравенство Чебышева (ч.с.)\\
\textbf{Доказательство}\\
Определим функции как $F_a(x) = \sum_{i=1}^{\lfloor x \rfloor} a_i, F_b(x) = \sum_{i=1}^{\lfloor x \rfloor} b_i$\\
Тогда неравенство выполняется по неравенству Чебышева\\
\textbf{Замечание}\\
Утверждается, что все доказанные свойства интеграла выполняются и для кусочно-непрерывных функций\\
\subsection{Приложение определенного интеграла}
\textbf{Общая схема}\\
Пусть фиксирован $\an ab$\\
Обозначения: $\nm{Segm}\an ab = \{ [p,q]: [p,q] \subset \an ab\}$\\
\textbf{Определение}\\
Отображение $\Phi: \nm{Segm}\an ab \rto \Rset$ -- функция промежутка\\
$\Phi$ -- \textit{Аддитивная функция промежутка}, если $\fall c \in (p,q)\ \Phi[p,q] = \Phi[p,c] + \Phi[c,q]$\\
\textbf{Определение}\\
$\Phi: \nm{Segm}\an ab \rto \Rset$ -- аддитивная функция промежутка\\
$f: \an ab \rto \Rset$ -- плотность а.ф.п. $\Phi$, если $\fall \delta \in \nm{Segm}\an ab\ |\delta|\cdot\inf_\delta f \leq \Phi(\delta) \leq |\delta|\cdot\sup_\delta f$\\
\textbf{Основной пример}\\
$\Phi[p,q] := \int_a^b f(x)\df x$\\
Тогда $f$ -- плотность\\
\textbf{Теорема 1 (о вычислении а.ф.п. по плотности)}\\
$\Phi: \nm{Segm}\an ab \rto \Rset$ -- а.ф.п\\
$f: \an ab \rto \Rset$ -- плотность $\Phi$, непрерывна\\
Тогда $\fall [p,q] \in \nm{Segm}\an ab\ \Phi[p,q] = \int_p^q f$\\
\textbf{Доказательство}\\
Пусть $F(x) = \left[\begin{array}{ll}
     0, & x = p \\
     \Phi[p,x] & p < x \leq q
\end{array}\right.$\\
Докажем, что $F$ -- первообразная $f$\\
$F'_+(x) = \lim_{h \rto 0+0} \frac{F(x+h)-F(x)}{h} = \lim_{h \rto 0+0} \frac{\Phi[p, x+h]-\Phi[p,c]}h = \lim_{h\rto 0+0} \frac{\Phi[x, x+h]}{h}$\\
$\inf_{[p,q]} \leq \frac{1}{q-p}\Phi[p,q] \leq \sum_{[p,q]} f$\\
Отсюда $F'_+(x) = \lim_{h\rto 0+0} f(x+\theta h), \theta \in [0,1] = f(x)$\\
Аналогично $F'_-(x) = f(x)$\\
$\Phi[p,q] = F(q)-F(p) = \int_p^q f(x)\df x$\\
\textbf{Пример}\\
Пусть $r = f(\phi)$ -- функция в полярных координатах\\
$\phi \in \an{\phi_0}\phi$\\
Пусть $\Phi: \nm{Segm}\an {\phi_0}\phi \rto \Rset$ -- площадь сектора $(f, \an{\phi_0}\phi)$\\
Т.е. $\Phi$ -- Отображение $[\alpha, \beta] \mapsto \sigma($Сектор$(f, \an\alpha\beta))$\\
Это аддитивная функция промежутка\\
\textbf{Теорема}\\
$f: \an{\phi_0}{\phi_1} \rto \Rset_+$ -- непрерывна, $\an{\phi_0}{\phi_1} \subset [0, 2\pi]$\\
Тогда $\sigma($Сектор$(f, \an\alpha\beta)) = \frac12 \int_\alpha^\beta f^2(\phi)\df \phi$\\
($[\alpha, \beta] \in \an{\phi_0}{\phi_1}$)\\
\textbf{Доказательство}\\
Проверим, что $\frac12 \int_\alpha^\beta f^2(\phi)\df \phi$ -- плотность а.ф.п. $\Phi$\\
Т.е. проверим неравенство $\fall [\alpha, \beta]\ \min_{\phi \in [\alpha, \beta]}(\frac12 f^2(\phi))(\beta - \alpha) \leq \Phi([\alpha, \beta]) \leq \max_{\phi \in [\alpha, \beta]}(\frac12 f^2(\phi))(\beta - \alpha)$\\
Рассмотрим произвольный сектор\\
Круговой сектор $(\min f, [\alpha, \beta])\subset $ Сектор$(f, [\alpha, \beta])\subset$ Круговой сектор $(\max f, [\alpha, \beta])$ из геометрических соображений\\
Отсюда $\frac12(\beta-\alpha) (\min f)^2 \leq \Phi([\alpha, \beta]) \leq \frac12(\beta - \alpha)(\max f)^2$\\
\textbf{Пример}\\
$x = r(t-\sin t)$\\
$y = r(1 + \cos t$ -- циклоида (координата точки на поверхности катящегося колеса)\\
Найдем площадь арки(периода) циклоиды\\
Происходят геометрические рассуждения, которые плохо конвертируются в конспект, поэтому оставляю ссылку: \url{https://www.youtube.com/live/CaG68GvecLw?feature=share&t=6813}\\
Посчитаем площадь через интеграл\\
$S = \int_0^{2\pi r}y(x)\df x = \int_0^{2\pi} r(1-\cos t)\df (r(t-\sin t)) = \int_0^{2\pi} r(1-\cos t)r(1-\cos t)\df t = r^2\int_0^{2\pi} (1 - 2\cos t + \cos^2t) \df t = 2\pi r^2 + r^2\int_0^{2\pi} \frac{\cos 2t + 1}2 \df t = 2\pi r^2 + \pi r^2 = 3\pi r^2$\\
\textbf{Пример 2}\\
Пусть задана кривая $(x(t), y(t))$ -- путь\\
Научимся считать площадь сектора $[t_0, t_1]$\\
Перейдем в полярные координаты (считая, что $\phi_0 \leq \phi_1$):\\
$S = \frac12 \int_{\phi_0}^{\phi_1} r^2(\phi)\df \phi = \begin{bmatrix}
    \phi = \arctan \frac{y(t)}{x(t)}\\
    r = \sqrt{x^2(t)+y^2(t)}
\end{bmatrix} = \frac12 \int_{t_0}^{t_1} (x^2+y^2)\frac1{1 + \frac{y^2}{x^2}} \frac{x'y-xy'}{x^2} \df t = \frac12 \int_{t_0}^{t_1} (x^2+y^2)\frac{x'y-xy'}{x^2+y^2}\df t = \frac12\int_{t_0}^{t_1}(x'y-xy')\df t$\\
\textbf{Пример 3 (Изопериметрическое неравенство)}\\
Пусть $G \subset \Rset^2$ -- замкнутая, ограниченная, выпуклая фигура\\
Пусть $\nm{diam}(G) \leq 1$, где $\nm{diam}(G) = \sup (\rho(x,y): x,y \in G)$\\
(Из компактности $G$, а значит $\nm{diam}(G) = \max (\rho(x,y): x,y \in G)$)\\
Тогда $\sigma(G) \leq \frac\pi4$\\
\textbf{Доказательство}\\
Зададим фигуру в полярных координатах\\
Граница фигуры -- выпуклая функция\\
Выберем на поверхности точку $A$, где функция дифференцируема(точек недифференцируемости не более чем счетное множество). Проведем в этой точке касательную и повернем фигуру, чтобы касательная стала вертикальной, а фигура была расположена в правой полуплоскости\\
Зададим функцию $f(\phi), \phi \in (-\frac\pi2, \frac\pi2)$ следующим образом:\\
Проведем из точки $A$ прямую под углом $\phi$\\
Она пересечет границу в точке $B$\\
Тогда $f(\phi) = |AB|$\\
$\sigma G = \frac12 \int_{-\frac\pi2}^{\frac\pi2} f^2(\phi)\df \phi = \frac12 \int_0^{\frac\pi2} f^2(\phi)\df \phi + \frac12 \int_{-\frac\pi2}^0 f^2(\phi)\df \phi = \frac12 \int_0^{\frac\pi2} f^2(\phi)\df \phi + \frac12 \int_0^{\frac\pi2} f^2(\phi-\frac\pi2)\df \phi = \frac12 \int_0^{\frac\pi2} (f^2(\phi)+f^2(\phi-\frac\pi2)) \df \phi$\\
$(f^2(\phi)+f^2(\phi-\frac\pi2))$ -- квадрат длины некоторой хорды в $G$\\
Отсюда $\sigma G = \frac12 \int_0^{\frac\pi2} (f^2(\phi)+f^2(\phi-\frac\pi2)) \df \phi \leq \frac12 \int_0^{\frac\pi2} 1 \df \phi = \frac\pi4$
\subsection{Интегральные суммы}
\textbf{Определение}\\
Пусть $[a,b]$ -- отрезок суммирования\\
Разделим отрезок конечным набором точек $x_0, \ldots, x_n$\\
$a = x_0 < x_1 < \ldots < x_n = b$\\
$i$-ый отрезок -- $[x_{i-1}, x_i]$\\
$\max |x_i - x_{i-1}|$ = ранг дробления = мелкость\\
Оснащение -- $\xi_1, \ldots, \xi_n$ -- набор точек таких, что $\xi_i \in [x_{i-1}, x_i]$\\
$f: [a,b] \rto \Rset$\\
Тогда $\sum_{i=1}^n f(\xi_i)(x_i-x_{i-1})$ -- Риманова сумма\\
\textbf{Теорема об интеграле как пределе Римановых сумм}\\
$f \in C[a,b]$\\
Тогда $\fall \eps > 0 \ex \delta > 0 \fall$ дроблений $a = x_0 < x_1 < \ldots < x_n = b$, у которых ранг $< \delta |\int_a^b f(x)\df x - \sum_{i=1}^n f(x_i)(x_i - x_{i-1})| < \eps$\\
\textbf{Доказательство}\\
Зафиксируем $\eps$\\
Для этого $\eps\ \ex \delta > 0 \fall x_1, x_2 \in [a,b]: |x_2-x_1| < \delta |f(x_1)-f(x_2)| < \frac\eps{b-a}$ -- по т. Кантора\\
Отсюда $|\int_a^b f-\sum_{i=1}^n f(x_i)(x_i-x_{i-1})| = |\sum_{i=1}^n (\int_{x_{i-1}}^{x_i} f(x)\df x - f(x_i)(x_i-x_{i-1}))| = |\sum_{i=1}^n (\int_{x_{i-1}}^{x_i} f(x)\df x - \int_{x_{i-1}}^{x_i} f(x_i)\df x)| = |\sum_{i=1}^n \int_{x_{i-1}}^{x_i} (f(x)-f(x_i))\df x| \leq \sum_{i=1}^n |\int_{x_{i-1}}^{x_i} (f(x)-f(x_i))\df x| \leq \sum_{i=1}^n \int_{x_{i-1}}^{x_i} |f(x)-f(x_i)|\df x < \sum_{i=1}^n \int_{x_{i-1}}^{x_i} \frac\eps{b-a}\df x = \int_a^b \frac\eps{b-a}\df x = \eps$\\
\textbf{Пример}\\
$\int_0^1 x\df x$\\
Разобъем отрезок $[0,1]$ на отрезки по $\frac1n$\\
Т.е. $\int_0^1 x\df x = \lim_{n\rto \infty} \frac in \frac1n = \lim_{n\rto \infty} \frac{n+1}{2n} = \frac12$\\
\textbf{Замечание}\\
Пусть $|f'(x)| \leq M$ на $[a,b]$\\
Разделим отрезок на части $\frac{b-a}n$\\
$x_i = a + \frac{b-a}ni$\\
Тогда $|\int_a^b f - \sum_{i=1}^n f(a+\frac{b-a}ni)\frac{b-a}n| < $ по т. Лагранжа $ < \sum_{i=1}^n \int_{x_{i-1}}^{x_i} |f'(\ol x_i)|(x_i-x) \df x \leq \sum_{i=1}^n M\int_{x_{i-1}}^{x_i} (x_i-x)\df x = \sum_{i=1}^n M\frac{(x_i-x_{i-1})^2}2 = \frac M2 (\frac{b-a}n)^2 n$\\
\textbf{Обобщенная теорема о плотности}\\
Пусть $f: \an ab \rto \Rset$ -- непрерывная\\
$\Phi: \nm{Segm}\an ab \rto \Rset$\\
$\fall \Delta \in \nm{Segm}\an ab$ заданы $m_\Delta, M_\Delta$ -- неточные минимум и максимум:
\begin{enumerate}
    \item $m_\Delta\cdot|\Delta| \leq \Phi(\Delta) \leq M_\Delta\cdot|\Delta|$
    \item $\fall x \in \Delta\ m_\Delta \leq f(t) \leq M_\Delta$
    \item $\fall x \in \an ab\ \fall \Delta \ni x: |\Delta| \rto 0\ M_\Delta-m_\Delta \rto 0$\\
\end{enumerate}
Тогда $\fall [p,q] \in \nm{Segm}\an ab\ \Phi[p,q] = \int_p^q f$\\
\textbf{Доказательство}\\
$F(x) = \left.\begin{array}{ll}
   \Phi[p,x],  & p < x \leq q \\
    0, & x=p
\end{array}\right.$\\
$\Delta := [x,x+h]$\\
$m_\Delta \leq \frac{F(x+h)-F(x)}{h} = \frac{\Phi(\Delta)}{|\Delta|} \leq M_\delta$\\
$m_\Delta \leq f(x) \leq M_\Delta$\\
Тогда $| \frac{F(x+h)-F(x)}h - f(x)| \leq M_\Delta - m_\Delta \xrto[h\rto 0]{} 0$\\
Т.е. $F_+'(x) = f(x)$\\
Аналогично $F_-'(x) = f(x)$\\
Т.о. $\Phi[p,q] = F(q)-F(p) = \int_p^q f$\\
\textbf{Пример}\\
Пусть $a > 0$\\
$f: \an ab \rto \Rset, f \geq 0$\\
$\Phi_x, \Phi_y: \nm{Segm}\an ab \rto \Rset$\\
Пусть $\Phi_x[p,q] = V_{\Omega^x}$\\
$\Phi_y[p,q] = V_{\Omega^y}$\\
$\Omega^x = \{(x,y,z) \in \Rset^3: x \in [p,q], \sqrt{y^2+z^2} \leq f(x)\}$ -- фигура вращения вокруг $OX$\\
$\Omega^y = \{(x,y,z) \in \Rset^3: p \leq \sqrt{x^2+z^2} \leq q, 0 \leq y \leq f(\sqrt{x^2+y^2}) \}$ -- фигура вращения вокруг $OY$\\
$\Phi_x, \Phi_y$ -- а.ф.п.\\
\textbf{Теорема}\\
$f\in C[p,q], f \geq 0$\\
Тогда $\Phi_x[p,q] = \pi\int_p^q f^2(x)\df x$\\
$\Phi_y[p,q] = 2\pi\int_p^q xf(x)\df x$\\
\textbf{Доказательство}\\
Для $\Phi_x$ -- очевидно (из 1 теоремы о плотности)\\
Для $\Phi_y$:\\
Проверим, что $2\pi xf(x)$ -- плотность $\Phi_y$\\
Из геометрических соображений:\\
$V(\Omega^y[\alpha, \beta]) \leq \pi(\beta^2 - \alpha^2)\max_{[\alpha, \beta]}f = (\beta-\alpha)\pi (\beta+\alpha)\max_{[\alpha, \beta]} f \leq (\beta - \alpha) \pi \max_{[\alpha, \beta]} 2x \max_{[\alpha, \beta]} f$\\
$V(\Omega^y[\alpha, \beta]) \geq \pi(\beta^2 - \alpha^2)\min_{[\alpha, \beta]}f = (\beta-\alpha)\pi (\beta+\alpha)\min_{[\alpha, \beta]} f \geq (\beta - \alpha) \pi \min_{[\alpha, \beta]} 2x \min_{[\alpha, \beta]} f$\\
Отсюда $M_\Delta := \pi \max_\Delta 2x \max_\Delta f(x)$\\
$m_\Delta := \pi \min_\Delta 2x \min_\Delta f(x)$\\
Т.о. условие 1 выполнено\\
$m_\Delta \leq 2\pi x f(x) \leq M_\Delta$ -- условие 2 выполнено\\
$max_\Delta 2x \max_\Delta f(x) - \min_\Delta 2x\min_\Delta f(x) \rto 0$ по непрерывности $f$ и $2x$ -- условие 3 выполнено\\
\textbf{Пример}\\
Найдем объем тора с радиусом сечения $r$ и радиусом кольца $R$\\
Формула прямой -- $y = \sqrt{r^2 - (x-R)^2}$\\
Отсюда $V = 4\pi \int_{R-r}^{R+r} x\sqrt{r^2 - (x-R)^2} \df x = 4\pi \int_{R-r}^{R+r} (x-R)\sqrt{r^2-(x-R)^2}\df x + 4\pi R \int_{R-r}^{R+r} \sqrt{r^2-(R-x)^2}\df x = 0$(из симметричности относительно $R$)$ + 4\pi R\frac12 \pi r^2 = 2\pi R\pi r^2$\\
\subsubsection{Длина пути}
$\gamma: [a,b] \rto \Rset^m$, непрерывная -- путь\\
$\gamma(a)$ -- начало пути, $\gamma(b)$ -- конец пути\\
$t \mapsto \begin{pmatrix}
\gamma_1(t)\\
\vdots\\
\gamma_m(t)
\end{pmatrix}$, где $\gamma_i: [a,b] \rto \Rset$ -- $i$-ая координатная функция пути\\
Путь \textit{гладкий}, если $\fall i\ \gamma_i \in C^1$\\
$v(t) = \begin{pmatrix}
    \gamma_1'(t)\\
    \vdots\\
    \gamma_m'(t)
\end{pmatrix}$ -- вектор скорости пути в точке $t$\\
$\lim_{h\rto 0}\frac{\gamma(t+h)-\gamma(t)}{h} = v(t)$ -- считается покоординатно\\
\textit{Носитель пути} -- $\gamma([a,b])$\\
\textbf{Определение}\\
$l$ -- функция на множестве гладких путей называется длиной пути, если
\begin{enumerate}
    \item $l \geq 0$
    \item $l$ -- аддитивна:\\
    $\fall \gamma: [a,b] \rto \Rset^m\ \fall c \in [a,b] l(\gamma) = l(\gamma|_{[a,c]})+l(\gamma|_{[c,b]})$
    \item $\gamma, \ol\gamma$ -- два пути в $\Rset^m, C_\gamma, C_{\ol\gamma}$ -- их носители\\
    Пусть $\ex T: C_\gamma \rto C_{\ol \gamma}$ -- сжатие:\\
    $\fall x,y \rho(T(x), T(y)) \leq \rho(x,y)$\\
    Тогда $l(\ol \gamma) \leq l(\gamma)$
    \item $\gamma(t) = \vect A+t\cdot\vect v \Rto l(\gamma) = \rho(\gamma(a), \gamma(b)$ -- длина прямолинейного пути
\end{enumerate}
\textbf{Замечание 1}
\begin{enumerate}
    \item Длина дуги $\geq$ длина хорды\\
    (по свойству 3: ортогонально проецируем дугу на хорду)
    \item При "расширении" длина дуги растет
    \item При движении пространства (изометрии) длина пути не меняется
\end{enumerate}
\textbf{Теорема о длине гладкого пути}\\
Пусть $\gamma: [a,b] \rto \Rset^m, \gamma \in C^1$\\
$l(\gamma) = \int_a^b \|\gamma'(t)\|\df t$\\
\textbf{Доказательство}\\
Пусть $\gamma$ -- инъекция. Т.е. путь не проходит через одну точку два раза. Если это не так, разобъем путь на несколько и посчитаем их длины отдельно\\
Проверим, что $\|\gamma'(t)\|$ -- плотность а.ф.п. $\ub{\subset [a,b]}{[p,q]} \mapsto l(\gamma|_{[p,q]})$\\
$m_i(\Delta) = \min_{t\in\Delta}\gamma_i'(t)$\\
$M_i(\Delta) = \max_{t\in\Delta}\gamma_i'(t)$\\
$m_\Delta = \sqrt{\sum_{i=1}^m m_i^2(\Delta)}$\\
$M_\Delta = \sqrt{\sum_{i=1}^m M_i^2(\Delta)}$\\
Тогда свойства 2 ($m_\Delta \leq \|\gamma'(t)\| \leq M_\Delta)$) и 3 ($M_\Delta - m_\Delta \xrto[|\Delta| \rto 0]{} 0$) очевидны, т.к. $\|\gamma'(t)\| = \sqrt{\sum_{i=1}^m (\gamma_i'(t))^2}$\\
Докажем, что $m_\Delta |\Delta| \leq l(\gamma|_\Delta) \leq M_\Delta |\Delta|$\\
Зафиксируем $\Delta = [t_0, t_1]$\\
$\letus \vec M = (M_1(\Delta), \ldots, M_m(\Delta))$\\
$\ot \gamma(t): \Delta \rto \Rset^m$\\
$\ot \gamma(t) := \vec M t$\\
Проверим $T: C_\gamma \rto C_{\ot \gamma}: \gamma(t) \mapsto \ot \gamma(t)$ -- растяжение\\
Пусть $p < q$\\
$\rho(\gamma(p), \ot \gamma(q)) = \sqrt{\sum_{i=1}^m (\gamma_i(p) - \ot\gamma_i(q))^2} = \sqrt{\sum_{i=1}^m (\gamma_i'(\ol p)(p-q))^2} \leq |p-q|\sqrt{\sum_{i=1}^m (M_i[p,q])^2} = \|\vec M[p,q]\||p-q| = l(\ot\gamma|_{[p,q]}) = \rho(\ot\gamma(p), \ot\gamma(q))$\\
Т.е. $l(\gamma|_{[p,q]}) \leq l(\ot\gamma|_{[p,q]}) = \|\vec M_{[p,q]}\||p-q| = \|\vec M_{\Delta}\||\Delta|$\\
Аналогично $\|\vec m_{\Delta}\||\Delta| \leq l(\gamma|_{\Delta})$\\
$l(\gamma) = \int_a^b \|\gamma'(t)\|\df t$, ч.т.д.\\
\textbf{Пример}\\
Посчитаем длину дуги эллипса\\
$\frac{x^2}{a^2} + \frac{y^2}{b^2}=1$\\
Пусть $a > b$\\
Параметризуем его: $(x, y) = (a\sin t, b\cos t)$\\
$\gamma' = (a\cos t, -b\sin t)$\\
$\|\gamma'\| = a^2\cos^2 t + b^2 \sin^2 t = a^2(1 -\eps^2 \sin^2 t), t = \frac{\sqrt{a^2 - b^2}}{a}$\\
Тогда $L[0,T] = a\int_0^T\sqrt{1-\eps^2 \sin^2 t}\df t$ -- не берется(\\
Формула -- Эллиптический интеграл $II$ рода\\
\textbf{Следствие}\\
$f:[a,b] \rto \Rset, f\in C^1$\\
Тогда $l(\Gamma(f, [a,b])) = \int_a^b \sqrt{1 + f'(x)^2}\df x$\\
\textbf{Доказательство}\\
$\Gamma(f, [a,b])$ -- носитель пути $x \mapsto (x, f(x))$\\
$\gamma(x) = (x, f(x)), \gamma'=(1,f'), \|\gamma'\| = \sqrt{1 + f'^2}$\\
\textbf{Следствие 2}\\
$r: [\alpha, \beta] \rto \Rset, r \in C^1$ -- функция в полярных координатах\\
$\gamma(\phi) = (r(\phi)\cos \phi, r(\phi)\sin\phi)$\\
Тогда $l(\phi) = \int_a^b \sqrt{r^2 + r'^2}\df \phi$\\
\textbf{Определение (способ определения длины пути)}\\
Разобъем кривую на $n$ частей "точками" $\alpha = t_0 < t_1 < \ldots < t_n = \beta$\\
Тогда $l(\gamma) = \sup_{n, (t_i)} \sum_{i=1}^n \rho(\gamma(t_i), \gamma(t_{i-1}))$\\
\textbf{Определение}\\
$f:[a, b] \rto \Rset^n$\\
Тогда \textit{вариация} $f$ на $[a,b]$ $\nm{Var}_a^b f = \sup_{n, (t_i)} \sum_{i=1}^n |f(t_i)-f(t_{i-1})|$\\
Если $f \in C^1, \nm{Var}_a^b f = $ длина пути $ = \int_a^b |f'|$\\
\textbf{Лемма}\\
$f:[a,b] \rto \Rset, \nm{Var}_a^b f$ -- ограничена\\
Тогда $\ex p,q: [a,b]\rto \Rset$ -- монотонные такие, что $f\equiv p-q$\\
\textbf{Доказательство}\\
$f(x)-f(a)=p(x)-q(x)$, где\\
$2p(x) = \nm{Var}_a^x f(x)+(f(x)-f(a))$\\
$2q(x) = \nm{Var}_a^x f(x) - (f(x) - f(a))$\\
Докажем, что $p,q$ -- возрастяют\\
$|f(y)-f(x)| \leq \nm{Var}_x^y f$\\
Отображение $\Delta \mapsto \nm{Var}_\Delta f$ -- а.ф.п.\\
Для $x < y\ 2(p(y)-p(x))=\nm{Var}_x^y f + f(y)-f(x) \geq 0$\\
Т.е. $p(y) \geq p(x)$, ч.т.д.\\
Кстати, $p(x) + q(x) = \nm{Var}_a^x f$\\
\subsection{Конечные $\eps$-сети}
\textbf{Упражнение}\\
Пусть $(X, \rho)$ -- метрическое пространство, $K \subset X$ -- компактно $\LRto K$ -- секвенциально компактно\\
\textbf{Определение}\\
Пусть $(X, \rho)$ -- метрическое пространство, $D \subset X, \eps > 0$\\
Множество $N \subset X$ -- $\eps$-сеть множества $D$, если $\fall x \in D \ \ex y \in N: \rho(x, y) < \eps$\\
Если $N$ -- конечное, то $N$ -- конечная $\eps$-сеть\\
\textbf{Определение}\\
$D$ -- сверхограниченное, если в $X$, если $\fall \eps > 0 \ \ex$ конечная $\eps$-сеть $N \subset X$\\
\textbf{Лемма 1}\\
$D$ -- сверхограниченно в $X \LRto D$ -- сверхограниченно в $D$ (в себе)\\
\textbf{Доказательство $\Lto$} -- тривиально\\
\textbf{Доказательство $\Rto$}\\
Возьмем $\eps > 0$\\
Берем $\frac\eps2$ в $X: N = \{ x_1, \ldots, x_n\}$\\
$\fall i B(x_i, \frac\eps2)$ выберем какую-нибудь $y_i \in D$ (если такая есть)\\
Тогда $\{ y_1, y_2, \ldots, y_n\}$ -- $\eps$-сеть $D$\\
\textbf{Лемма 2}\\
$f: D \rto Y, D$ -- сверхограниченное\\
$f$ -- равномерно непрерывно\\
Тогда $f(D)$ -- сверхограниченно\\
\textbf{Доказательство}\\
Равномерная непрерывность $\fall \eps > 0\ \ex \delta > 0: \fall x_1, x_2: \rho(x_1, x_2) < \delta\ \rho(f(x_1), f(x_2)) < \eps$\\
Зафиксируем $\eps$\\
Возьмем конечную $\delta$-сеть в $D =: N$\\
$f(N)$ -- конечная $\eps$-сеть в $Y$\\
\textbf{Лемма 3}\\
$D$ -- сверхограниченно $\LRto$ любая последовательность в $D$ содержит фундаментальную подпоследовательность\\
\textbf{Доказательство $\Rto$}\\
Возьмем последовательность $(x_n)$\\
$\ex$ конечная 1-сеть $\{y_1, \ldots, y_k\}$\\
Тогда $\ex i: B(y_i, 1)$ содержит бесконечно много членов последовательности\\
Пусть $x_k \in B(y_i, 1)$\\
Тогда $n_1 := k$\\
$\ex$ конечная $\frac12$-сеть $D$, а значит конечная $\frac12$-сеть $D \cap B(y_i, 1)$\\
Повторим действия\\
Получившаяся последовательность $y_i$ фундаментальна\\
\textbf{Доказательство $\Lto$}\\
Докажем от противного\\
Пусть $\ex \eps > 0:$ не существует конечной $\eps$-сети\\
Возьмем $x_1$\\
Т.к. сети не существует, то $\ex x_2 \in D \setminus B(x_1, \eps)$\\
$\ex x_3 \in D \setminus \bigcup_{i=1}^2 B(x_i, \eps)$\\
И т.д.\\
Построена последовательность $x_n \in D: \fall k, l\ \rho(x_k, x_l) \geq \eps$ -- не фундаментальная\\
Отсюда противоречие\\
\textbf{Теорема}\\
$D$ -- метрическое пространство\\
Тогда эквивалентны следующие утверждения:
\begin{enumerate}
    \item $D$ -- компактно
    \item $D$ -- сверхограниченное и полное
\end{enumerate}
\textbf{Доказательство $1 \Rto 2$}\\
$D$ -- компактно\\
Если $D$ -- неполное, то $\ex (x_n) \in D$ -- фундаментальная, но не имеющая предела\\
Тогда $\fall (n_k)\ x_{n_k}$ -- не имеет предела\\
(Т.к. фундаментальная последовательность сходится, если ее подпоследовательность сходится)\\
Это противоречит секвенциальной компактности\\
Если $D$ -- не сверхограниченное\\
Тогда $\ex (x_n)$, не содержащей фундаментальную\\
Но тогда нет и сходящейся, что противоречит секвенциальной компактности\\
\textbf{Доказательство $2 \Rto 1$}\\
$D$ -- сверхограниченное и полное\\
Тогда любая последовательность содержит фундаментальную подпоследовательность (в силу полноты -- сходящуюся)\\
Это секвенциальная компактность\\
\textbf{Следствие}\\
$X$ -- полное метрическое пространство $D \subset X$\\
Тогда эквивалентны следующие утверждения
\begin{enumerate}
    \item $D$ -- компактно
    \item $D$ -- сверхограниченное и замкнутое
\end{enumerate}
\textbf{Теорема о формулах центральных прямоугольников и трапеций}\\
Пусть $f \in C^2[a,b], a=x_0 < x_1 < \ldots < x_n = b, \delta = \max_i (x_i - x_{i-1})$\\
$t = \frac{x_i + x_{i-1}}2$\\
Тогда выполняются две формулы
\begin{enumerate}
    \item $| \int_a^b f(x)\df x - \sum_{i=1}^n f(t_i)(x_i-x_{i-1})| < \frac{\delta^2}8 \int_a^b |f''(x)|\df x$
    \item $| \int_a^b f(x)\df x - \sum_{i=1}^n \frac{f(x_i) + f(x_{i-1})}2 (x_i - x_{i-1})| < \frac{\delta^2}8 \int_a^b |f''(x)| \df x$
\end{enumerate}
При равномерном дроблении $\delta = \frac{b-a}n:$\\
$M:= \max |f''(x)|, \frac{\delta^2}8 \int_a^b |f''(x)| \df x \leq \frac{M(b-a)^3}{n^2}$\\
\textbf{Доказательство (только 2)}\\
Пусть $\df g:= g'(x) \df x$\\
$\int_{x_{i-1}}^{x_i} f(x)\df x = \int_{x_{i-1}}^{x_i} f(x) \df (x-t_i) = f(x)(x-t_i)\vl_{x=x_{i-1}}^{x=x_i} - \int_{x_{i-1}}^{x_i} f'(x)(x-t_i)\df x = \frac{f(x_i)+f(x_{i-1})}2(x_i-x_{i-1}) + \frac12\int_{x_{i-1}}^{x_i} f'(x)\df \psi$, где $\psi(x) = (x-x_{i-1})(x_i-x), x \in [x_{i-1}, x_i]$\\
$\int_{x_{i-1}}^{x_i} f(x)\df x = \frac{f(x_i)+f(x_{i-1})}2(x_i-x_{i-1}) + \ub{0}{\frac12 f'\psi\vl_{x_{i-1}}^{x_i}} - \frac12 \int_{x_{i-1}}^{x_i} f''\psi \df x$\\
Тогда $|\int_a^b \ldots - \sum_{i=1}^n \ldots| = |\sum_{i=1}^n (\int_{x_{i-1}}^{x_i} f(x)\df x - \frac{f(x_i)+f(x_{i-1})}2(x_i-x_{i-1}))| = |\sum_{i=1}^n -\frac12 \int_{x_{i-1}}^{x_i} f''\psi| = \frac12\int_a^b |f''|\ub{\leq \frac{\delta^2}4}\psi \df x \leq \frac{\delta^2}8\int_a^b |f''|$\\
\textit{Подсказка: для прямоугольников $\psi = \left\{\begin{array}{cc}
    (x-x_{i-1})^2, & x \in [x_{i-1}, t_i]\\
    (x-x_i)^2, & x \in [t_i, x_{i}]\\
\end{array}\right.$}\\
\textbf{Формула Эйлера-Маклорена}\\
Пусть $f\in C^2[m,n], m,n\in \Zset$\\
Тогда $\int_m^n f(x)\df x = {\sum_{i=m}^n}^* f(i) - \frac12 \int_m^n f''(x)\{x\} (1-\{x\})\df x$\\
* -- два крайних слагаемых -- с коэффициентом $\frac12$\\
Т.е. ${\sum_{i=m}^n}^* f(i) = \frac12 f(m) + \sum_{i=m+1}^{n-1} f(i) + \frac12 f(n)$\\
\textbf{Доказательство}\\
Из доказательства формулы для трапеции, где $\psi = (x-k)(k + 1 - x) = \{x\}(1 - \{x\})$\\
\textbf{Пример}\\
$p > -1, f(x) = x^p$\\
Тогда $1^p + \ldots + n^p = \int_1^n x^p\df x + \frac12 + \frac{n^p}2 + \frac{p(p-1)}2 \int_1^n x^{p-2} \{x\}(1-\{x\})\df x = \frac{n^{p+1}}{p+1} - \frac1{p+1} + \frac12 + \frac{n^p}2 + O(\max(1, n^{p-1}))$\\
Пояснение:\\
$0 \leq \int_1^n x^{p-2} \{x\}(1-\{x\})\df x \leq \int_1^n x^{p-2} \df x = \frac{x^{p-1}}{p-1}\vl_1^n = \frac{n^{p-1}}{p-1} - \frac1{p-1}$\\
При $p < 1\ \frac{n^{p-1}}{p-1} - \frac1{p-1} = O(1)$\\
При $p > 1\ \frac{n^{p-1}}{p-1} - \frac1{p-1} = O(n^{p-1})$\\
\textit{Замечание: при $p < -1$ слагаемые будут ограниченными, т.е. вся сумма будет $O(1)$}\\
\textbf{Пример 2}\\
$f(x) = \frac1x$\\
$1 + \ldots + \frac1n = \ln n + \frac12 + \frac1{2n} + \ub{=:y_n, \text{возрастающая последовательность}}{\int_1^n \frac1{x^3} \{x\}(1-\{x\})\df x}$\\
$y_n$ -- возрастает\\
$y_n$ -- ограниченная\\
$y_n \leq \frac14 \int_1^n \frac1{x^3}\df x = \frac18 (-\frac1{x^2})\vl_1^n = \frac18 - \frac1{8n^2} < \frac18$\\
Тогда $1 + \ldots + \frac1n = \ln n + \ub{\text{имеет предел} \gamma \in [\frac12, \frac58]}{\frac12 + \frac1{2n} + y_n} = \ln n + \gamma +o(1)$\\
$\gamma \approx 0.57\ldots$ -- постоянная Эйлера\\
\textbf{Пример 3}\\
$f(x) = \ln x$\\
$\ln 1 + \ldots + \ln n = \int_1^n \ln x \df x + \frac{\ln n}2 \ub{x_n}{- \frac12 \int_1^n \frac1{x^2} \{x\}(1-\{x\}) \df x} = n\ln n - n + \frac{\ln n}2 + x_n$\\
$x_n$ монотонная и ограниченная\\
Тогда $x_n \rto C$\\
Отсюда $n! = n^n e^{-n}\sqrt n e^{C+o(1)} \rto C_1 n^n e^{-n}\sqrt n, C_1 = e^C$\\
Найдем $C_1$\\
\textbf{Формула Валлиса}\\
$I_n = \int_0^{\frac\pi2} \sin^2 x \df x = \int_0^{\frac\pi2} \sin^{n-1} x \sin x \df x = \ub0{-\sin^{n-1}\cos x\vl_0^{\frac\pi2}} + (n-1)\int_0^{\frac\pi2} \sin^{n-2} x(1-\sin^2 x)\df x = (n-1)I_{n-2} - (n-1)I_n$\\
$I_n = \frac{n-1}n I_{n-2}$\\
$I_0 = \frac\pi2, I_1 = \int_0^{\frac\pi2} \sin x\df x = 1$\\
$I_n = \left\{\begin{array}{ll}
    \frac{(n-1)!!}{n!!}\frac\pi2, & n \text{-- четное}\\
    \frac{(n-1)!!}{n!!}, & n \text{-- нечетное}
\end{array}\right.$\\\\
Рассмотрим на $[0,\frac\pi2]:\ \sin^{2k+1} x \leq \sin^{2k} x \leq \sin^{2k-1} x$\\
Проинтегрируем: $\frac{(2k)!!}{(2k+1)!!} \leq \frac{(2k-1)!!}{(2k)!!} \leq \frac{(2k-2)!!}{(2k-1)!!}$\\
$\ub{\alpha_k}{(\frac{(2k)!!}{(2k-1)!!})^2 \frac1{2k+1}} \leq \frac\pi2 \leq \ub{\beta_k}{\frac1{2k}(\frac{(2k)!!}{(2k-1)!!})^2}$\\
$\beta_k - \alpha_k = (\frac{(2k)!!}{(2k-1)!!})^2 (\frac1{2k}-\frac1{2k+1})=\ub{\leq \frac\pi2}{(\frac{(2k)!!}{(2k-1)!!})^2 \cdot\frac1{2k+1}}\cdot\frac1{2k} \leq \frac\pi2 \frac1{2k} \xrto[k\rto \infty]{} 0$\\
Тогда $\lim_{k\rto +\infty} (\frac{(2k)!!}{(2k-1)!!})^2\frac1{2k} = \frac\pi2$\\
\textbf{Замечание}\\
$2b_k:= (4k+3)(\frac{(2k)!!}{(2k+1)!!})^2$\\
$2c_k:= \frac4{4k+1}(\frac{(2k)!!}{(2k-1)!!})^2$\\
Тогда $\alpha_k < \frac{b_k}2 < \frac{c_k}2 < \beta_k$\\
При этом $b_k \uto, c_k \dto$\\
$c_k - b_k = c_k \frac1{4(2k+1)^2}$\\\\
$\pi = \frac{\prod_{k=1}^{+\infty} (1 + \frac1{4k^2-1})}{\prod_{k=1}^{+\infty} \frac1{4k^2-1}}$\\\\
По формуле Валлиса $\sqrt \pi = \lim_{k\rto + \infty} \frac1{\sqrt k}\frac{2\cdot 4 \cdot \ldots \cdot (2k)}{1\cdot 3 \cdot \ldots \cdot (2k-1)} = \lim_{k\rto + \infty} \frac1{\sqrt k} \frac{(2^k k!)^2}{(2k)!} = \lim_{k\rto +\infty} \frac1{\sqrt k} \frac{(2^k k^k e^{-k}\sqrt{k})^2 C_1^2}{(2k)^{2k}e^{-2k}\sqrt {2k}C_1} = \frac{C_1}{\sqrt{2}}$\\
Отсюда $C_1 =\sqrt{2\pi}$\\
Тогда $n! = n^n e^{-n}\sqrt n e^{2\pi}$ -- формула Стирлинга\\
\section{Выпуклость}
Множество $A \subset \Rset^m$ называется выпуклым, если\\
$\fall x,y \in A\ [x,y] \subset A$,\\
где $[x,y] = \{ x + t(y-x), y \in [0,1]\} = \{ \alpha x + (1-\alpha)y, \alpha \in [0,1] \}$ -- отрезок прямой, содержащей $x,y$\\
\textbf{Определение}\\
$f: \lan a, b \ran \rto \Rset$\\
Если $\fall x,y \in \lan a, b \ran\ \fall \alpha \in [0,1]\ f(\alpha x + (1-\alpha) y) \leq \alpha f(x) + (1-\alpha) f(y)$, то $f$ -- выпуклая (выпуклая вниз) на $\lan a, b \ran$\\
Если $\fall x,y \in \lan a, b \ran\ \fall \alpha \in [0,1]\ f(\alpha x + (1-\alpha) y) \geq \alpha f(x) + (1-\alpha) f(y)$, то $f$ -- вогнутая (выпуклая вверх) на $\lan a, b \ran$\\
\textbf{Примеры}\\
$e^x$ -- выпуклая\\
$x^2$ -- выпуклая\\
\textbf{Замечание}\\
$f$ -- выпуклая $\LRto$ любая хорда(отрезок, соединяющий две точки графика) расположена нестрого выше графика $\LRto$ надграфик выпуклый\\
\textit{Надграфиком} $f$ на $\lan a, b \ran$ называется $\{ (x,y): x \in \lan a, b\ran, y \geq f(x)\}$\\
\textbf{Определение}\\
$f: \lan a, b \ran \rto \Rset$\\
Если $\fall x,y \in \lan a, b \ran\ \fall \alpha \in (0,1)\ f(\alpha x + (1-\alpha) y) \lessgtr \alpha f(x) + (1-\alpha) f(y)$, то $f$ -- строго выпуклая/вогнутая на $\lan a, b \ran$\\
\textbf{Лемма о трех хордах}\\
$f: \lan a, b \ran \rto \Rset$\\
Тогда эквивалентны:
\begin{enumerate}
    \item $f$ выпуклая на $\lan a, b \ran$
    \item $\fall x_1 < x_2 < x_3 \in \an ab\ \frac{f(x_2)-f(x_1)}{x_2-x_1} \leq \frac{f(x_3)-f(x_1)}{x_3-x_1} \leq \frac{f(x_3)-f(x_2)}{x_3-x_2}$
\end{enumerate}
\textbf{Доказательство $\Rto$}\\
Докажем первое неравенство\\
$x_2 = \frac{x_3-x_2}{x_3-x_1}x_1 + \frac{x_2-x_1}{x_3-x_1}x_3$\\
Тогда неравенство 1 $\LRto f(x_2) \leq f(x_1) \frac{x_3-x_2}{x_3-x_1} + f(x_3)\frac{x_2-x_1}{x_3-x_1}$\\
При $\alpha = \frac{x_3-x_2}{x_3-x_1}, 1 - \alpha = \frac{x_2-x_1}{x_3-x_1}$ получаем знакомое неравенство\\
Второе неравенство аналогично\\
\textbf{Доказательство $\Lto$}\\
Очевидно из предыдущего доказательства\\
\textbf{Следствие}\\
$f$ строго выпукла $\LRto$ строгое неравенство в теореме\\
\textbf{Замечание}\\
Если $f,g$ -- выпуклые на $\lan a, b \ran$, то $f+g$ -- выпуклая\\
$f$ -- выпуклая, то $-f$ -- вогнутая\\
\textbf{Теорема об односторонней дифференцируемости выпуклой функции}\\
$f$ -- выпуклая на $\lan a, b \ran$\\
Тогда $\fall x \in (a,b)\ \ex f_+'(x), f_-'(x)$ -- конечные\\
и $\fall x_1, x_2 \in (a,b), x_1 \leq x_2\ f_-'(x_1) \leq f_+'(x_1) < \frac{f(x_2)-f(x_1)}{x_2-x_1} \leq f_-'(x_2) \leq f_+'(x_2)$\\
\textbf{Доказательство}\\
Сделать предельный переход в лемме о 3 хордах\\
$g(\xi) := \frac{f(\xi)-f(x_1)}{\xi - x_1}, \xi \in (a,b) \setminus \{x_1\}$\\
$g(\xi) \uto$ по лемме о 3 хордах\\
При $\xi \in (x_1, x_2)$\\
Т.к. функция монотонна, $\ex \lim_{\xi \rto x_1 + 0} g(\xi)$\\
Возьмем $\xi_0 \leq \xi_1 < x_1 < \xi_2 < \xi_3 < x_2 < \xi_4 \leq \xi_5$\\
Из лемм о трех хордах(средние неравенства) и монотонности(крайние неравенства):\\
$\frac{f(\xi_0)-f(x_1)}{\xi_0 - x_1} \leq \frac{f(\xi_1)-f(x_1)}{\xi_1 - x_1} \leq \frac{f(\xi_2)-f(x_1)}{\xi_2 - x_1} \leq \frac{f(x_2) -f(x_1)}{x_2-x_1} \leq \frac{f(\xi_3)-f(x_2)}{\xi_3 - x_2} \leq \frac{f(\xi_4)-f(x_2)}{\xi_4-x_2} \leq \frac{f(\xi_5)-f(x_2)}{\xi_5-x_2}$\\
Пусть $\xi_1 \rto x_1 - 0, \xi_2 \rto x_1 + 0, \xi_3 \rto x_2 - 0, \xi_4 \rto x_2 + 0$\\
Отсюда $C_0 \leq f_-'(x_1) \leq f_+'(x_1) \leq \frac{f(x_2) -f(x_1)}{x_2-x_1} \leq f_-'(x_2) \leq f_+'(x_2) \leq C_5$, где $C_0 = \frac{f(\xi_0)-f(x_1)}{\xi_0-x_1}, C_5 = \frac{f(\xi_5)-f(x_2)}{\xi_5-x_2}$\\
Отсюда производные конечные(ограничены $C_0$ и $C_5$)\\
\textbf{Воспоминания о прошлом семе}\\
Если $\ex f_+'(x_0)$, то $f$ -- непрерывна справа в $x_0$\\
\textbf{Следствие}\\
$f$ -- выпуклая на $\lan a, b \ran$\\
Тогда она непрерывна на $(a,b)$\\
(т.к. у нее есть левосторонние и правосторонние производные, то она непрерывна слева и справа, т.е. непрерывна)\\
\textbf{Контр-пример для $[a,b]$}\\
$f(1) = 2$\\
$f(-1) = 2$\\
$f(x) = x^2, x \in (a,b)$\\
Функция выпукла, но не непрерывна на $[a,b]$\\
\textbf{Теорема}\\
$f$ -- дифференцируема на $\an ab$\\
Тогда $f$ -- выпуклая вниз на $\lan a,b\ran \LRto$ график расположен не ниже любой касательной, т.е. $\fall x_0, x \in \lan a,b\ran f(x) \geq f'(x_0)(x-x_0)+f(x_0)$\\
\textbf{Доказательство $\Rto$}\\
Требуемое неравенство есть в предыдущей теореме\\
\textbf{Доказательство $\Lto$}\\
Возьмем $x_1 < x_0 < x_2$\\
Тогда $\frac{f(x_1)-f(x_0)}{x_1-x_0} \leq f'(x_0) \leq \frac{f(x_2)-f(x_0)}{x_2-x_0}$\\
По лемме о трех хордах $f$ -- выпуклая\\
\textbf{Определение}\\
Пусть имеется выпуклая фигура $A \subset \Rset^2$\\
$b \in A$ - граничная точка\\
Прямая $l: b \in l$ -- опорная прямая, если $A$ полностью содержится в одной из полуплоскостей, образованных прямой\\
\textbf{Утверждение}\\
$f$ -- выпуклая на $\an ab$\\
Тогда $\fall x \in \an ab$ через $(x, f(x))$ можно провести опорную прямую к надграфику $f$\\
(для $x \in (a,b)$ есть односторонняя дифференцируемость, можем провести одностороннюю касательную)\\
(для $x = a$ и $x = b$ можем провести вертикальные прямые или касательные, если односторонние производные есть и конечны)\\
\textbf{Утверждение 2}\\
Если $A \subset \Rset^2$ -- замкнутая и ограниченная выпуклая фигура, то через каждую граничную точку можно провести опорную прямую\\
\textbf{Доказательство}\\
Возьмем точку $b$\\
Докажем, что для нее можно провести опорную прямую\\
Для этого выберем оси $X, Y$ так, чтобы проекция $b$ на $X$ была внутренней точкой проекции фигуры на $X$\\
Теперь определим функцию $f(x) := \inf (y: (x,y) \in A)$ -- нижнюю часть границы фигуры\\
Данная функция выпуклая, а значит в точке $b$ к ней можно провести опорную прямую\\
\textbf{Замечание}\\
$f$ -- выпуклая на $\an ab$\\
Тогда $f$ дифференцируема на $\an ab$ всюду, кроме не более чем счетного множества точек (возможно, пустого)\\
\textbf{Доказательство}\\
Пусть $E$ -- множество точек, где не существует производной\\
Но существуют $f_-'(x) < f_+'(x)$\\
Тогда $\fall x_1 < x_2 \in E\ f_-'(x_1) < f_+'(x_1) \leq f_-'(x_2) < f_+'(x_2)$\\
Тогда для $x \in E$ построим отображение $q(x) \in (f_-'(x), f_+'(x)) \cap \Qset$\\
$q: E \rto \Qset$ -- инъекция\\
Отсюда $E$ не более чем счетно\\
\textbf{Теорема (дифференциальный критерий выпуклости)}
\begin{enumerate}
    \item $f$ -- непрерывна на $\an ab$, дифференцируема на $(a,b)$\\
    Тогда $f$ -- выпукла (строго выпукла) $\LRto f'$ возрастает (строго возрастает)\\
    \textbf{Доказательство $\Rto$}\\
    По теореме об односторонней дифференцируемости $f'_-(x_1) \leq f'_-(x_2)$ при $x_1 < x_2$\\
    ($f_-' = f'$ из дифференцируемости)\\
    Знак строгий, если $f$ строго выпукла(смотри доказательство теоремы)\\
    \textbf{Доказательство $\Lto$}\\
    Проверим лемму о трех хордах\\
    $x_1 < x_2 < x_3$\\
    Тогда $\frac{f(x_1)-f(x_2)}{x_1-x_2} = f'(\xi_1), x_1 < \xi_1 < x_2$ -- по т. Лагранжа\\
    $\frac{f(x_2)-f(x_3)}{x_2-x_3} = f'(\xi_2), x_2 < \xi_2 < x_3$\\\\
    Из возрастания $f'(\xi_1) \leq f'(\xi_2)$ (при строгом возрастании знак $<$)
    \item $f$ -- непрерывна на $\an ab$, дважды дифференцируема на $(a,b)$\\
    Тогда $f$ -- выпукла $\LRto f'' \geq 0$\\
    \textbf{Доказательство}\\
    $f'$ - возрастает $\LRto f'' \geq 0$
\end{enumerate}
\textbf{Пример 1}\\
При $x \in (0, \frac \pi2)\ \sin x \geq \frac2\pi x$\\
При $x = 0 \lor x = \frac\pi2$ достигается равенство\\
\textbf{Доказательство}\\
$(\sin x)' = \cos x$ -- строго убывает на промежутке\\
Тогда функция строго вогнутая на промежутке\\
Отсюда значение функции строго выше хорды, соединяющей $(0, 0)$ и $(\frac\pi2, 1)$ (ее уравнение $y = \frac2\pi x$)
\section{Верхний и нижний предел}
\textbf{Определение}\\
Частичный предел последовательности = предел вдоль подпоследовательности\\
\textbf{Пример}\\
$x_n = (-1)^n$\\
$-1, 1$ -- частичные пределы $x_n$\\
\textbf{Пример}\\
$\fall a \in [-1, 1]\ \ex n_k: \sin n_k \rto a$\\
\textbf{Определение 2}\\
$x_n$ -- вещественная последовательность\\
$y_n = \sup (x_n, x_{n+1}, \ldots) \in \Rex$ -- верхняя огибающая\\
$z_n = \inf (x_n, x_{n+1}, \ldots) \in \Rex$ -- нижняя огибающая\\
Верхний предел $\ulim_{n\rto \infty} x_n = \limsup_{n\rto \infty} x_n = \lim_{n\rto \infty} y_n$\\
Нижний предел $\dlim_{n\rto \infty} x_n = \liminf_{n\rto \infty} x_n = \lim_{n\rto \infty} z_n$\\
\textbf{Замечание}
\begin{enumerate}
    \item $y_n \geq y_{n+1} \geq \ldots, z_n \leq z_{n+1} \leq \ldots$
    \item $\fall n\ z_n \leq x_n \leq y_n$
    \item При изменении конечного числа $x_n$ изменяется конечное число $y_n, z_n$
\end{enumerate}
\textbf{Пример}
\begin{enumerate}
    \item $x_n = (-1)^n$\\
    $\ulim x_n = 1, \dlim x_n = -1$
    \item $x_n = (1+(-1)^n)n$\\
    $\ulim x_n = +\infty, \dlim x_n = 0$
\end{enumerate}
\textbf{Свойства}
\begin{enumerate}
    \item $\dlim x_n \leq \ulim x_n$
    \item $x_n \leq \ot x_n$\\
    Тогда $\ulim x_n \leq \ulim \ot x_n$\\
    $\dlim x_n \leq \dlim \ot x_n$
    \item $\fall \lambda \geq 0\ (0\cdot \infty =: 0)$\\
    $\ulim \lambda x_n = \lambda \ulim x_n$\\
    $\dlim \lambda x_n = \lambda \dlim x_n$
    \item $\fall \lambda < 0 $\\
    $\ulim \lambda x_n = -\lambda \dlim x_n$\\
    $\dlim \lambda x_n = -\lambda \ulim x_n$
    \item $\ulim (x_n + \ot x_n) \leq \ulim x_n + \ulim \ot x_n$\\
    $\dlim (x_n + \ot x_n) \geq \dlim x_n + \dlim \ot x_n$\\
    (если сумма в правой части имеет смысл)
    \item $t_n \rto l \in \Rset$\\
    Тогда $\ulim (x_n + t_n) = \ulim x_n + l$\\
    \textbf{Доказательство}\\
    $\fall \eps > 0\ \ex N_0\ \fall k > N_0\ x_k + l - \eps < x_k + t_k < x_k + l + \eps$\\
    Возьмем $\sup_{k \geq N}$ для некого $N > N_0$\\
    $y_N + l - \eps < \sup_{k \geq N} (x_k + t_k) < y_n + l + \eps$\\
    Возьмем предел $N \rto +\infty$\\
    $\ulim x_n + l - \eps \leq \ulim (x_k + t_k) \leq \ulim x_n + l - \eps$
    \item $t_n \rto l>0, l \in \Rset$\\
    Тогда $\ulim t_nx_n = l\ulim x_n$
\end{enumerate}
\textbf{Техническое описание верхнего предела}
\begin{enumerate}
    \item $\ulim x_n = +\infty \LRto x_n$ -- не ограничено сверху
    \item $\ulim x_n = -\infty \LRto x_n \rto -\infty$
    \item $\ulim x_n = l \in \Rset \LRto A + B$\\
    $A: \fall \eps > 0\ \ex N:\ \fall n > N\ x_n < l + \eps$\\
    $B: \fall \eps > 0\ \fall N\ \ex n > N: l-\eps < x_n$
\end{enumerate}
\textbf{Доказательство 1}\\
Очевидно в обе стороны\\
\textbf{Доказательство 2 $\Rto$}\\
$x_n \leq y_n \rto -\infty \Rto x_n \rto -\infty$\\
\textbf{Доказательство 2 $\Lto$}\\
$x \rto -\infty \LRto \fall A\ \ex N:\ \fall n>N\ x_n < A$ (а значит $y_n \leq A)$\\
Отсюда $y_n \rto -\infty$\\
\textbf{Доказательство 3 $\Rto$}\\
$y_n \dto, l = \lim y_n = \inf y_n$\\
Возьмем $\eps > 0$\\
Тогда $\ex N: y_N < l+\eps \Rto \fall n > N x_n < l+\eps$\\
Отсюда $A$ -- доказано\\
$\fall N\ y_n > l-\eps \Rto \ex n > N: l-\eps < x_n \leq y_N$\\
\textbf{Доказательство 3 $\Lto$}\\
$A \Rto \fall \eps > 0\ \ex N: \fall n \geq N\ y_n \leq l+\eps$\\
$B \Rto \fall \eps > 0\ \ex N: \fall n \geq N\ l - \eps \leq y_n$\\
Отсюда $y_n \rto l$\\
\textbf{Теорема}\\
$(x_n) \in \Rset$\\
Тогда $\ex \lim x_n \in \Rex \LRto \ulim x_n = \dlim x_n (= \lim x_n)$\\
\textbf{Доказательство $\Rto$}
\begin{enumerate}
    \item $\lim x_n = +\infty \Rto \dlim x_n = +\infty \leq \ulim x_n$
    \item $\lim x_n = -\infty \Rto \ulim x_n = -\infty \geq \dlim x_n$
    \item $\lim x_n = l \in \Rset$\\
    Тогда из A и B $\ulim x_n = \dlim x_n = l$
\end{enumerate}
\textbf{Доказательство $\Lto$}\\
$z_n \leq x_n \leq y_n$\\
Тогда $\dlim x_n \leq \lim x_n \leq \ulim x_n$\\
\textbf{Теорема о характеризации частичных пределов}\\
$(x_n) \in \Rset$
\begin{enumerate}
    \item Если $l$ -- частичный предел $x_n$ (существует по принципу выбора Больцано-Вейерштрасса), то $\dlim x_n \leq l \leq \ulim x_n$\\
    \textbf{Доказательство}\\
    $z_{n_k} \leq x_{n_k} \leq y_{n_k}, k \rto +\infty$\\
    $\dlim x_n \leq l \leq \ulim x_n$
    \item $\ex x_{n_k} \rto \ulim x_n, \ex x_{m_k} \rto \dlim x_n$\\
    \textbf{Доказательство для $\ulim x_n$}\\
    Если $\ulim x_n = +\infty$, то $x_n$ не ограничена сверху\\
    Если $\ulim x_n = -\infty$, то $\lim x_n = -\infty$\\
    Если $\ulim x_n = l$, то по A, B:\\
    $\fall k \in \Nset\ \ex n_k: l-\frac1k < x_{n_k} < l+\frac1k$\\
    Будем выбирать $n_{k+1} > n_k$\\
    Тогда $x_{k_k} \rto l$
\end{enumerate}
\textbf{Пример}\\
$\ulim \sin n = 1, \dlim \sin n = -1$\\
$\fall l \in (-1, 1)\ \ex n_k: \sin n_k \rto l$\\
\textbf{Замечание}\\
И множество $\{\sin n, n \in \Nset\}$ плотно на $[-1, 1]$\\
\textbf{Доказательство}\\
$\pi \not\in \Qset \Rto \fall m,n \in \Nset, m\neq n\ \sin n \neq \sin m$\\
Т.е. невозможно $n = m + 2\pi k, \pi - m + 2\pi k, k \in \Zset$\\
Будем двигаться по окружности с шагом $l_1 = 1$\\
Движение с шагом $6l_1$ равносильно движению с шагом $l_2 := |6l_1 - 2\pi|$ в противоположную сторону\\
Т.о. мы научились двигаться с шагом $l_2$\\
Будем по индукции уменьшать $l_i$\\
Пусть $|ml_i| < 2\pi, |(m+1)l_i| > 2\pi, m\in \Nset$\\
Тогда $l_{i+1} := \min (2\pi - ml_i, (m+1)l_i - 2\pi)$\\
Заметим, что т.к. $2\pi \in (ml_i, (m+1)l_i)$, то $l_{i+1} \leq \frac{l_i}2$\\
Рассмотрим отрезок в $[-1, 1]$\\
Ему соответствует отрезок $[a,b], a,b \in [0,2\pi)$ на окружности\\
Пусть $l = b-a$\\
Подберем $l_k < l$\\
Тогда для некоторого $q \in \Nset\ ql_k\in [a,b]$\\
Т.о. $\sin ql_k$ будет лежать в нашем отрезке. Отсюда $\sin n$ плотно в $[-1, 1]$\\
Докажем, что $\fall \alpha\in [-1, 1]\ \ex q_i: \lim q_i \rto \alpha$\\
Возьмем некоторую окрестность $\alpha$\\
Будем генерировать последовательность, лежащую в этой окрестности, и сдвигать границу
\section{Несобственный интеграл}
\textbf{Определение}
\begin{enumerate}
    \item $f: [a, b) \rto \Rset, -\infty < a < b \leq +\infty$ -- допустимая, если $\fall A \in (a, b)\ f\vl_{[a,A]}$ -- кусочно непрерывная
    \item $\Phi(A) = \int_a^A f(x)\df x$\\
    Если $\ex \lim_{A \rto b-0} \Phi(A) \in \Rex$, то он называется несобственным интегралом $f$ на $[a,b)$\\
    Отозначение: $\int_a^{\rto b} f(x)\df x$
\end{enumerate}
Если $\not\ex \lim \Phi(A)$ -- несобственный интеграл не существует\\
Если $\lim_{A \rto b-0} \Phi(A)$ -- конечный, то интеграл сходится\\
Если $\lim_{A \rto b-0} = \infty$ -- интеграл расходится\\
Аналогично определяем $\int_{\rto a}^b f(x)\df x$\\
$\int_{-\infty}^{+\infty} f := \int_{-\infty}^c f + \int_c^{+\infty} f$\\
\textbf{Пример}\\
$\int_1^{+\infty} \frac1x \df x = \lim_{A \rto +\infty} \int_1^A \frac1x\df x = \lim_{A \rto +\infty} \ln A = +\infty$\\
\textbf{Пример}\\
$\int_{\rto 0}^1 \frac1x \df x = \lim_{A\rto +0} \ln A = +\infty$\\\\
$f: (a,b) \rto \Rset, -\infty \leq a < b \leq +\infty$\\
$x_1 < \ldots < x_n \in (a,b), n$ -- нечетное\\
Пусть $f$ допустимо на каждом из промежутков $(a, x_1], [x_1,x_2), (x_2, x_3], [x_3, x_4), \ldots, [x_n, b)$\\
Тогда $\int_a^b f = \int_{\rto a}^{x_1} f + \int_{x_1}^{\rto x_2} f + \int_{\rto x_2}^{x_3} f + \ldots + \int_{x_n}^{\rto b} f$\\
Если хотя бы один интеграл не существует или сумма некорректная (есть $+\infty$ и $-\infty$), то интеграл расходится\\
\textbf{Пример}\\
$\int_{-1}^1 \frac1x \df x = \ub{-\infty}{\int_{-1}^{\rto 0}\frac1x \df x} + \ub{+\infty}{\int_{\rto 0}^1 f}$\\
Данный интеграл расходится\\
\textbf{Свойства}
\begin{enumerate}
    \item Критерий Больцано-Коши о сходимости несобственного интеграла\\
    $f$ -- допустимая на $[a,b)\ -\infty < a < b \leq +\infty$\\
    Тогда $\int_a^{\rto b} f$ -- сходится $\LRto \fall \eps > 0\ \ex \delta \in (a,b):\ \fall A,B \in (\delta, b)\ |\int_A^B f| < \eps$\\
    \textbf{Доказательство}\\
    $\ex \lim_{R\rto b-0}\Phi(R) \in \Rset \LRto \fall \eps > 0\ \ex \delta \in (a,b)\ \fall A,B \in (\delta, b)\ |\Phi(A)-\Phi(B)| < \eps$ -- критерий Больцано-Коши\\
    $\int f$ -- расходится $\Rto \ex A_n, B_n \rto b-0: \int_{A_n}^{B_n} f \not\rto 0$\\
    \textbf{Пример}\\
    $\int_1^{+\infty} \frac1x \df x, A_n = n, B_n = 2n$\\
     Тогда $\int_n^{2n}\frac1x\df x \geq \frac1{2n}n = \frac12$ -- расходится\\
     \textbf{Пример}\\
     $\int_{-1}^1 \frac1x \df x, A_n = \frac1{2n}, B_n = \frac1n$\\
     $\int_{\frac1{2n}}^{\frac1n} \frac1x \df x \geq 2n\frac1{2n} = 1$
     \item Аддитивность по промежутку\\
     $f$ -- допустима $[a,b), c\in (a,b)$\\
     Тогда $\int_a^{\rto b} f, \int_c^{\rto b} f$ сходятся/расходятся одновременно\\
     И в случае сходимости $\int_a^{\rto b} f = \int_a^c f + \int_c^{\rto b} f$\\
     \textbf{Следствие}\\
     Если $\int_a^{\rto b}f$ -- сходится, то $\int_A^{\rto b}f \xrto[A\rto b-0]{} 0$
     \item $f, g$ -- допустимы на $[a, b), \int_a^{\rto b} f, \int_a^{\rto b} g$ -- сходятся, $\lambda \in \Rset$\\
     Тогда $\lambda f, f\pm g$ -- допустимы $\int_a^{\rto b} \lambda f = \lambda\int_a^{\rto b}, \int_a^{\rto b} (f + g) = \int_a^{\rto b}f + \int_a^{\rto b} g$
     \item $\int_a^{\rto b} f, \int_a^{\rto b} g$ -- существуют в $\Rex, f \leq g$\\
     Тогда $\int_a^{\rto b} f \leq \int_a^{\rto b} g$
     \item $f, g$ -- дифференцируемые на $[a,b)$\\
     $f', g'$ -- допустимые на $[a,b)$\\
     Тогда* $\int_a^{\rto b} fg' = fg\vl_a^{\rto b}-\int_a^{\rto b} f'g$, где $fg\vl_a^{\rto b} = (\lim_{B \rto b-0} fg(b)) - f(a)$\\
     * -- если здесь существуют два предела из трех, то существует и третье и равенство выполняется
     \item $\phi: [\alpha, \beta) \rto \an AB, \phi \in C^1$\\
     Пусть $\ex \phi(\beta-0) \in \Rex$\\
    $f \in C\an AB$\\
    Тогда* $\int_\alpha^\beta (f \circ \phi) \phi' = \int_{\phi(\alpha)}^{\rto \phi(\beta-0)} f$\\
    * -- если существует один интеграл, то существует и другой\\
    \textbf{Замечание}\\
    Это означает, что мы можем начать вычислять интеграл, не зная, существует ли он. Тогда если посчитать его удалось, то он существует
\end{enumerate}
Т.к. свойства собственных и несобственных интегралов одинаковые, то с этого момента стрелочку писать не будем\\
\textbf{Лемма об интегрировании асимптотических равенств}\\
$f, g \in C[a,b), g \geq 0, \int_a^b g(x)\df x = +\infty$\\
$F(x):= \int_a^x f, G(x) := \int_a^x g, x\in [a,b)$\\
Тогда при $x\rto b-0$ из $f=O(g) / f=o(g) / f \sim g$\\
Следует $F=O(G) / F=o(G) / F \sim G$\\
\textbf{Доказательство}
\begin{enumerate}
    \item $f = O(g)$\\
    $\ex M\ \ex x_0: \fall x\in[x_0, b)\ |f(x)| \leq M|g(x)|$\\
    Пусть $\int_a^{x_0} |f| = C_1$\\
    Выберем $x_1 \in (x_0, b)$ такую, что $\int_{x_0}^{x_1} |g| = \alpha > 0$\\
    При $x > x_1 \ |F(x)| = | \int_a^x|f| | = \int_a^{x_0} |f| + \int_{x_0}^{x_1} |f| \leq C_1 + M\int_{x_0}^x g \leq \frac{C_1}\alpha \int_{x_0}^{x_1} g + M\int_{x_0}^xg \leq (\frac{C_1}\alpha + M)\int_a^x g = (\frac{C_1}\alpha + M) G(x)$
    \item Аналогично
    \item Из эквивалентности $F(x) \rto +\infty$\\
    Тогда $\lim_{x\rto b-0} \frac FG=\lim_{x\rto b-0} \frac fg = 1$
\end{enumerate}
\textbf{Лемма}\\
Пусть $\phi_n$ -- шкала асимптотического разложения при $x\rto b-0$ на $[a,b)$\\
$\phi_n \in C[a,b), \phi_n \geq 0$\\
$\Phi_n:=\int_x^b \phi_n(x)$ (считаем, что $\fall n$ -- сходится)\\
Тогда
\begin{enumerate}
    \item $\Phi_n$ -- шкала
    \item $f\in C[a,b), F(x) = \int_x^b f$ -- сходится\\
    $f = \sum_{k=1}^n \alpha_k \phi_k(x)+o(\phi_n)$\\
    Тогда $F = \sum_{k=1}^n \alpha_k \Phi_k + o(\Phi_n)$
\end{enumerate}
\textbf{Доказательство}
\begin{enumerate}
    \item По правилу Лопиталя\\
    $\fall n \Phi_n(x) \xrto[x\rto b-0]{} 0$\\
    $\lim \frac{\Phi_{n+1}}{\Phi_n} = [\frac00] = \lim \frac{\Phi'_{n+1}}{\Phi'_n} = \lim \frac{-\phi_{n+1}}{-\phi_n} = 0$
    \item $\lim_{x\rto b-0} \frac{F(x) - \sum_{k=1}^n \alpha_k \Phi_k}{\Phi_n(x)} = [\frac00] = \lim_{x\rto b-0} \frac{f(x) - \sum_{k=1}^n \alpha_k \phi_k}{\phi_n(x)} = 0$
\end{enumerate}
\subsection{Признаки сходимости несобственного интеграла}
\textbf{Теорема}\\
$f \geq 0$
\begin{enumerate}
    \item $f$ -- допустима на $[a,b), f \geq 0, \Phi(A) = \int_a^A f$\\
    Тогда $\int_a^b f$ -- сходится $\LRto \Phi$ -- ограничена на $[a,b)$\\
    \textbf{Доказательство}\\
    $\lim_{A\rto b-0} \Phi(A) = [\Phi \uto] = \sup \Phi(A)$
    \item \textit{Признак сравнения}\\
    $f, g$ -- допустимы на $[a,b), f,g \geq 0$
    \begin{enumerate}
        \item Пусть $f \leq q$\\
        Тогда если $\int_a^b g$ -- сходится, то и $\int_a^b f$ -- сходится\\
        Если $\int_a^b f$ -- расходится, то и $\int_a^b g$ -- расходится
        \item Пусть $\lim_{x\rto b-0}\frac fg = l < +\infty$\\
        Тогда если $\int_a^b g$ -- сходится, то и $\int_a^b f$ -- сходится\\
        \item Пусть $\lim_{x\rto b-0}\frac fg = m > 0$ (вероятно $m = +\infty$)\\
        Тогда если $\int_a^b f$ -- сходится, то и $\int_a^b g$ -- сходится\\
    \end{enumerate}
    \textbf{Замечание}\\
    Если $\lim_{x\rto b-0}\frac fg \in (0, \infty)$, то интегралы $f,g$ сходятся одновременно\\
    \textbf{Доказательство}
    \begin{enumerate}
        \item $\Phi(A) = \int_a^A f, \Psi (A) = \int_a^A g$\\
        $f \leq g \Rto \Phi(A) \leq \Psi(A)$, обе $\uto$\\
        $\int_a^b g$ -- сходится $\Rto \Psi$ -- ограничена $\Rto \Phi$ ограничена $\Rto \int_a^b f$ -- сходится\\
        $\int_a^b f$ -- расходится $\Rto \Phi$ -- не ограничена $\Rto \Psi$ не ограничена $\Rto \int_a^b g$ -- расходится
        \item $\frac fg \rto l < L < +\infty$\\
        Тогда $\ex c\in[a,b):\ f\leq Lg$ на $[c,b)$\\
        Т.е. $\int_a^b g$ -- сходится $\Rto \int_a^b Lg$ -- сходится $\Rto \int_a^b f$ -- сходится
        \item Аналогично
    \end{enumerate}
\end{enumerate}
\textbf{Пример}\\
$\int_1^{+\infty} \frac1{x^p}\df x$ -- при $p>1$ сходится, иначе расходится\\
$\int_0^1 \frac1{x^p}\df x$ -- при $p<1$ сходится, иначе расходится\\
(Чтобы не путаться, можно посмотреть, что происходит в $p=1$)\\
\textbf{Метод "удавить логарифм"}\\
Пусть есть интеграл $\int^{+\infty} \frac{\df x}{x^\alpha(\ln x)^\beta}$\\
(Нижняя граница не важна)\\
Определим, сходится ли он
\begin{enumerate}
    \item $\alpha > 1, \alpha = 1 + 2a, a > 0$\\
    Тогда $\frac1{x^\alpha(\ln x)^\beta} = \frac1{x^{1+a}}\frac1{x^a(\ln x)^\beta}$\\
    Утверждается, что $x^a (\ln x)^\beta \xrto[x\rto +\infty]{} 0$ (упражнение на правило Лопиталя)\\
    Тогда $\ex x_0: \fall x > x_0 \ x^a(\ln x)^\beta > 1$\\
    А значит с некоторого места $\frac1{x^\alpha(\ln x)^\beta} \leq \frac1{x^{1+a}}$\\
    Тогда выражение сходится по признаку сходимости
    \item $\alpha < 1, \alpha = 1 - 2a, a > 0$\\
    Тогда $\frac1{x^\alpha(\ln x)^\beta} = \frac1{x^{1-a}}\frac1{x^{-a}(\ln x)^\beta} > \frac1{x^{1-a}}$\\
    $x^{-a}(\ln x)^\beta \rto +0$\\
    $\frac1{x^\alpha(\ln x)^\beta} > \frac1{x^{1-a}}$\\
    Тогда выражение расходится по признаку сходимости
    \item $\alpha = 1$\\
    $\int^{+\infty} \frac{\df x}{x^\alpha(\ln x)^\beta} = \int^{+\infty}\frac{\df y}{y^\beta}$\\
    При $b > 1$ -- сходится\\
    При $b \leq 1$ -- расходится
\end{enumerate}
\textbf{Определение}\\
Гамма-функция Эйлера $\Gamma(t) = \int_0^{+\infty} x^{t-1}e^{-x}\df x, t > 0$\\
$\letus\ g(t,x) = x^{t-1}e^{-x}$
\begin{enumerate}
    \item Определим сходимость $g(t, x)$ в $x \rto +0$\\
    $x^{t-1}e^{-x} \sim x^{t-1}$ при $x\rto +0$ -- сходится при $t > 0$\\
    Определим сходимость $g(t, x)$ в $x \rto +\infty$\\
    $x^{t-1}e^{-x} = \ub{\xrto[x\rto +\infty]{} 0}{x^{t-1}e^{-\frac x2}}e^{-\frac x2} \leq e^{-\frac x2} \xrto[x\rto +\infty]{} 0$
    \item $\Gamma(t)$ выпукла на $(0, +\infty):$\\
    $\fall x>0\ g(t,x)$ -- выпуклая\\
    $g(\alpha t_1 + (1-\alpha)t_2, x) \leq \alpha g(t_1, x) + (1-\alpha)g(t_2, x)$\\
    $\int_0^{+\infty} g(\alpha t_1 + (1-\alpha)t_2, x) \leq \alpha \int_0^{+\infty} g(t_1, x) + (1-\alpha)\int_0^{+\infty} g(t_2, x)$\\
    Отсюда $\Gamma(\alpha t_1 + (1-\alpha)t_2) \leq \alpha \Gamma(t_1) + (1-\alpha)\Gamma(t_2)$\\
    Следствие -- $\Gamma$ дифференцируема на $(0, +\infty)$\\
    $\Gamma(t+1) = t\Gamma(t)$\\
    $\int_0^{+\infty} x^t e^{-x}\df x =-x^t e^{-x}\vl_{x=0}^{x=+\infty} + t \int_0^{+\infty} x^{t-1}e^{-x}\df x$
    \item $\Gamma(1) = 1$\\
    Тогда $\Gamma(n+1) = n!$\\
    \textbf{Следствие}\\
    $\Gamma(t) = \frac{\Gamma(t+1)}{t} \us{t\rto +0}\sim \frac1t$
    \item $\Gamma(\frac12) = \int_0^{+\infty} x^{-\frac12}e^{-x}\df x \us{x=y^2}= 2\ub{\text{интеграл Эйлера-Пуассона}, \frac{\sqrt \pi}2}{\int_0^{+\infty} e^{-y^2} \df y} = \sqrt\pi$\\
    \textbf{Лемма}\\
    $\int_0^{+\infty} e^{-y^2} \df x = \frac{\sqrt\pi}2$\\
    \textbf{Доказательство}\\
    $1-y^2 \leq e^{-y^2} \leq \frac1{1+y^2}$ -- переписанное $e^t \geq 1 + t$\\
    $\int_0^1 (1-y^2)^n \df y \leq \int_0^1 e^{-ny^2}\df y \leq \int_0^{+\infty} e^{-ny^2}\df y \leq \int_0^{+\infty} \frac{\df y}{(1+y^2)^n}$\\
    $\int_0^1 (1-y^2)^n \df y \us{y=\cos t}= \int_0^{\frac\pi2}(\sin t)^{2n+1} \df t =: w_{2n+1} = \frac{(2n)!!}{(2n+1)!!}$\\
    $\int_0^{+\infty} \frac{\df y}{(1+y^2)^n} \us{y=\tan t}= \int_0^{\frac\pi2} (\cos t)^{2n-2}\df t = w_{2n-2} = \frac{(2n-3)!!}{(2n-2)!!}\frac\pi2$\\
    $\frac{(2n)!!}{(2n+1)!!}\sqrt n \leq \sqrt n\int_0^{+\infty} e^{-ny^2}\df y \leq \frac{(2n-3)!!}{(2n-2)!!}\frac\pi2\sqrt n$\\
    По формуле Валлиса $\frac{(2k)!!}{(2k-1)!!}\frac1{\sqrt n} \rto \sqrt \pi$\\
    $\frac{(2n)!!}{(2n+1)!!}\sqrt n = \frac{(2n)!!}{(2n-1)!!}\frac1{\sqrt n}\frac n{2n+1} \rto \frac{\sqrt\pi}2$\\
    $\frac{(2n-3)!!}{(2n-2)!!}\frac\pi2\sqrt n = \frac1{\frac{(2n-2)!!}{(2n-3)!!}\frac1{\sqrt {n-1}}}\frac{\sqrt n}{\sqrt{n-1}}\frac\pi2 \rto \frac{\sqrt\pi}2$\\
    $\sqrt n\int_0^{+\infty} e^{-ny^2}\df y \us{z=\sqrt n y}= \int_0^{+\infty}e^{-z^2}\df z \rto \frac{\sqrt\pi}2$
\end{enumerate}
//todo continue \url{https://www.youtube.com/live/tYOFdAXEMx4?feature=share&t=4738}
\section{Несколько классических неравенств}
\textbf{Неравенство Йенсена}\\
$f$ -- выпуклая на $\an ab$\\
Тогда $\fall a_1, \ldots, a_n \in \an ab\ \fall \alpha_1,\ldots,\alpha_n \geq 0: \sum_i \alpha_i = 1\ f(\sum_i \alpha_i x_i) \leq \sum_i \alpha_i f(x_i)$\\
\textbf{Доказательство}\\
$x^* := \sum_i \alpha_i x_i$\\
Тогда $x^* \leq \sum_i \alpha_i (\max_i x_i) = \max_i x_i$\\
Аналогично $x^* \geq \min_i x_i$\\
Тогда $x^* \in \an ab$\\
Проведем в $x^*$ опорную прямую $y = kx+b$\\
$f(x^*) = kx^*+b = k\sum_i \alpha_i x_i + b\sum_i \alpha_i = \sum_i \alpha_i(kx_i + b) \leq \sum_i \alpha_if(x_i)$ -- из выпуклости\\
Заметим, что в $a, b$ последний переход может не выполняться, если опорная прямая вертикальная. Но тогда $x^* = \max_i x_i = \min_i x_i \Rto \fall x_i: \alpha_i = 0 \lor x^* = x_i$, что доказывается тривиально\\
\textbf{Пример}\\
Неравенство Коши\\
$\fall a_1, \ldots, a_n > 0\ \frac{a_1 + \ldots + a_n}n \geq \sqrt[n]{a_1\ldots a_n}$\\
\textbf{Доказательство}\\
$\ln (\frac1n a_1 + \ldots + \frac1n a_n) \geq \frac1n (\ln a_1 + \ldots + \ln a_n)$\\
Применим неравенство для вогнутых функций\\
\textbf{Интегральное неравенство Йенсена}\\
$f$ -- выпуклая на $\an AB$\\
$\phi: [a,b] \rto \an AB$ -- непрерывная\\
$\lambda: [a,b] \rto [0, \infty), \int_a^b \lambda(x)\df x = 1$ -- непрерывная\\
Тогда $f(\int_a^b \lambda(x)\phi(x)\df x) \leq \int_a^b \lambda(x)f(\phi(x))$\\
\textbf{Доказательство}\\
Докажем для случая $\lambda > 0$ в силу сложности доказательства в общем случае\\
$x^* = \int_a^b \lambda(x)\phi(x)\df x \leq \max \phi \int_a^b \lambda(x)\df x, \geq \min \phi \int_a^b \lambda(x)\df x$\\
Рассмотрим $y=kx+l$ -- опорную прямую в $x^*$\\
$f(x^*) = kx^*+l = \int_a^b \lambda(k\phi + l) \leq \int_a^b \lambda(x)f(\phi(x)) \df x$\\
Тут существует проблема, аналогичная предыдущей теореме. Но выкинуть все точки, где $\lambda = 0$ мы не можем, т.к. получившееся множество будет сложным\\
\textbf{Пример (Продолжение)}\\
$\frac1{b-a}\int_a^b f(x)\df x$ -- среднее арифметическое $f$ на $[a,b]$\\
Тогда cреднее геометрическое -- это $\exp (\frac1{b-a}\int_a^b \ln f(x) \df x)$\\
\textbf{Теорема}\\
$\phi \in C[a,b], \phi > 0$\\
Тогда $\ln (\frac1{b-a}\int_a^b \phi(x)\df x) \geq \frac1{b-a}\int_a^b \ln \phi(x) \df x$\\
\textbf{Доказательство}\\
$f(t) = \ln t$ -- вогнутая\\
Применим неравенство Йенсена: $\phi$ -- это $\phi$\\
$\lambda(x) = \frac1{b-a}$\\
\textbf{Неравенство Гельдера}\\
Пусть $a_i, b_i > 0$\\
Заметим, что $\fall p > 1\ \ex q > 1: \frac1p + \frac1q = 1, q$ -- \textit{сопряженный}\\
$\sum_{i=1}^n a_ib_i \leq (\sum_i a_i^p)^{\frac1p}(\sum_i b_i^q)^{\frac1q}$\\
\textbf{Доказательство}\\
$f(x) = x^p, p > 1$ -- выпуклая при $x > 0$\\
По неравенству Йенсена $(\sum_i \alpha_i x_i)^p \leq \sum_i \alpha_i x_i^p$\\
$\alpha_i = \frac{b_i}{\sum_j b_j^q}$. Тогда $\alpha_i > 0, \sum_i \alpha_i = 1$\\
$x_i := a_ib_i^{-\frac1{p-1}} (\sum_j b_j^q)$\\
$(\sum_i \alpha_i x_i)^p = (\sum_i a_i b_i^{q-\frac1{p-1}})^p = (\sum_i a_i b_i)^p$\\
$\sum_i \alpha_ix_i^p = \sum_i \frac{b_i^q}{\sum_j b_j^q} a_i^p b_i^{-\frac{p}{p-1}}(\sum_j b_j)^p = \sum_i (a_i^p (\sum_j b_j^q)^{p-1}) = (\sum_i a_i^p)(\sum_j b_j^q)^{p-1}$\\
$(\sum_i a_i b_i)^p \leq (\sum_i a_i^p)(\sum_j b_j^q)^{p-1}$\\
Возведем в степень $\frac1p$\\
$\sum_i a_ib_i \leq (\sum_i a_i^p)^{\frac1p}(\sum_j b_j^q)^{\frac1q}$\\
\textbf{Замечание}\\
В неравенстве Йенсена равенство достигается при $x_1 = \ldots = x_n$\\
Отсюда в неравенстве Гельдера = достигается при $\fall i,j\ x_i = x_j \LRto x_i^p = x_j^p = \lambda \LRto a_i^pb_i^{-q} = a_j^pb_j^{-q} = \lambda \LRto a_i^p = \lambda b_i^q$\\
Т.е. вектора $\vec{(a_i^p)_i} \| \vec{(b_j^q)_j}$\\
\textbf{Замечание}\\
$|\sum_i a_i b_i| \leq \sum_i |a_i b_i| \leq (\sum_i |a_i|^p)^{\frac1p}(\sum_i |b_i|^q)^{\frac1q}$ -- общий вид неравенства Гельдера\\
$a_1, \ldots, a_n, b_1 \ldots, b_n \in \Rset$\\
Равенство при $\vec{(a_i^p)_i} \| \vec{(b_j^q)_j}$\\
\textbf{Интегральное неравенство Гельдера}\\
$p,q > 1, \frac1p + \frac1p = 1, f,g \in C[a,b]$\\
Тогда $\int_a^b |fg| \leq (\int_a^b |f|^p)^{\frac1p}(\int_a^b |g|^q)^{\frac1q}$\\
\textbf{Доказательство}\\
$x_k := a+k\frac{b-a}n, k = 0\ldots n$\\
$a_k := f(x_k) (\frac{b-a}n)^{\frac1p}, b_k = g(x_k)(\frac{b-a}n)^{\frac1q}$\\
$\sum_k a_k b_k = \sum_k |f(x_k)g(x_k)|\frac{b-a}n \xrto[n\rto +\infty]{} \int_a^b |fg|$\\
$(\sum a_k^p)^{\frac1p} = (\sum_k |f(x_k)|^p \frac{b-a}n)^{\frac1p} \xrto[n\rto +\infty]{} (\int_a^b |f|^p)^\frac1p$\\
$(\sum_k b_k^q)^{\frac1q} \rto (\int_a^b |g|^q)^{\frac1q}$\\
\textbf{Замечание}\\
При $p=2$ неравенство Гельдера = КБШ\\
\textbf{Неравенство Минковского}\\
$p\geq 1, a_1, \ldots, a_n, b_1, \ldots, b_n \in \Rset$\\
Тогда $(\sum_i |a_i+b_i|^p)^{\frac1p} \leq (\sum_i |a_i|^p)^{\frac1p} + (\sum_i |b_i|^p)^{\frac1p}$\\
Это утверждение о том, что $(a_1, \ldots, a_n) \mapsto (\sum_i |a_i+b_i|^p)^{\frac1p}$ -- норма в $\Rset^n$\\
\textbf{Доказательство}\\
Если $p = 1$, очевидно\\
Если $p > 1$:\\
Пусть $a_i, b_i > 0$\\
Тогда $\sum_i a_i (a_i + b_i)^{p-1} \leq (\sum_i a_i^p)^{\frac1p}(\sum_i (a_i + b_i)^p)^{\frac1q}$\\
$\sum_i b_i (a_i + b_i)^{p-1} \leq (\sum_i b_i^p)^{\frac1p}(\sum_i (a_i + b_i)^p)^{\frac1q}$\\
Тогда $(\sum_i (a_i+b_i)^p)^1 \leq ((\sum_i a_i^p)^{\frac1p} + (\sum_i b_i^p)^{\frac1p})(\sum_i (a_i + b_i)^p)^{\frac1q}$\\
$(\sum_i (a_i+b_i)^p)^{\frac1p} \leq (\sum_i a_i^p)^{\frac1p} + (\sum_i b_i^p)^{\frac1p}$\\
Для произвольных $a_i, b_i$ заметим, что $(\sum_i |a_i+b_i|^p)^{\frac1p} \leq (\sum_i (|a_i|+|b_i|)^p)^{\frac1p}$\\
\textbf{Неравенство Минковского -- интегральный вид}\\
$f,g \in C[a,b], p\geq 1$\\
Тогда $(\int_a^b |f+g|^p)^{\frac1p} \leq (\int_a^b |f|^p)^{\frac1p} + (\int_a^b |g|^p)^{\frac1p}$\\
\textbf{Доказательство}: самостоятельно
\end{document}